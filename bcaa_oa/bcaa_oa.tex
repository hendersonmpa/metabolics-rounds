% Created 2018-01-22 Mon 16:09
\documentclass[presentation, smaller]{beamer}
\usepackage[utf8]{inputenc}
\usepackage[T1]{fontenc}
\usepackage{fixltx2e}
\usepackage{graphicx}
\usepackage{grffile}
\usepackage{longtable}
\usepackage{wrapfig}
\usepackage{rotating}
\usepackage[normalem]{ulem}
\usepackage{amsmath}
\usepackage{textcomp}
\usepackage{amssymb}
\usepackage{capt-of}
\usepackage{hyperref}
\hypersetup{colorlinks,linkcolor=white,urlcolor=blue}
\usepackage{textpos}
\usepackage{textgreek}
\usepackage[version=4]{mhchem}
\usepackage{chemfig}
\usepackage{siunitx}
\usepackage{gensymb}
\usepackage[usenames,dvipsnames]{xcolor}
\usepackage[T1]{fontenc}
\usepackage{lmodern}
\usepackage{verbatim}
\usepackage{tikz}
\usetikzlibrary{shapes.geometric,arrows,decorations.pathmorphing,backgrounds,positioning,fit,petri}
\usetheme{Hannover}
\usecolortheme{whale}
\author{Matthew Henderson, PhD, FCACB}
\date{\today}
\title{Branched Chain Organic Acidurias}
\institute[NSO]{Newborn Screening Ontario | The University of Ottawa}
\titlegraphic{\includegraphics[height=1cm,keepaspectratio]{../logos/NSO_logo.pdf}\includegraphics[height=1cm,keepaspectratio]{../logos/cheo-logo.png} \includegraphics[height=1cm,keepaspectratio]{../logos/UOlogoBW.eps}}
\hypersetup{
 pdfauthor={Matthew Henderson, PhD, FCACB},
 pdftitle={Branched Chain Organic Acidurias},
 pdfkeywords={},
 pdfsubject={},
 pdfcreator={Emacs 25.2.1 (Org mode 8.3.4)}, 
 pdflang={English}}
\begin{document}

\maketitle
%\logo{\includegraphics[width=1cm,height=1cm,keepaspectratio]{../logos/NSO_logo_small.pdf}~%
%    \includegraphics[width=1cm,height=1cm,keepaspectratio]{../logos/UOlogoBW.eps}%
%}

\vspace{220pt}
\beamertemplatenavigationsymbolsempty
\setbeamertemplate{caption}[numbered]
\setbeamerfont{caption}{size=\tiny}
% \addtobeamertemplate{frametitle}{}{%
% \begin{textblock*}{100mm}(.85\textwidth,-1cm)
% \includegraphics[height=1cm,width=2cm]{cat}
% \end{textblock*}}

\tikzstyle{chemical} = [rectangle, rounded corners, text width=5em, minimum height=1em,text centered, draw=black, fill=none]
\tikzstyle{hardware} = [rectangle, rounded corners, text width=5em, minimum height=1em,text centered, draw=black, fill=gray!30]
\tikzstyle{ms} = [rectangle, rounded corners, text width=5em, minimum height=1em,text centered, draw=orange, fill=none]
\tikzstyle{msw} = [rectangle, rounded corners, text width=7em, minimum height=1em,text centered, draw=orange, fill=none]
\tikzstyle{label} = [rectangle,text width=8em, minimum height=1em, text centered, draw=none, fill=none]
\tikzstyle{hl} = [rectangle, rounded corners, text width=5em, minimum height=1em,text centered, draw=black, fill=red!30]
\tikzstyle{box} = [rectangle, rounded corners, text width=5em, minimum height=5em,text centered, draw=black, fill=none]
\tikzstyle{arrow} = [thick,->,>=stealth]
\tikzstyle{hl-arrow} = [ultra thick,->,>=stealth,draw=red]

\section{Introduction}
\label{sec:orgheadline9}
\begin{frame}[label={sec:orgheadline1}]{BCAAs}
\centering
\chemname{\chemfig[][scale=.75]{^{+}H_3N-C(-[2]COO^{-})(-[6]CH(-[7]CH_3)(-[5]CH_3))-H}}{\small valine}
\chemname{\chemfig[][scale=.75]{^{+}H_3N-C(-[2]COO^{-})(-[6]CH_2-[6]CH(-[7]CH_3)(-[5]CH_3))-H}}{\small leucine}
\chemname{\chemfig[][scale=.75]{^{+}H_3N-C(-[2]COO^{-})(-[6]CH(-CH_3)-[6]CH_2-[6]CH_3)-H}}{\small isoleucine}
\end{frame}

\begin{frame}[label={sec:orgheadline2}]{BCAA Catabolism}
\centering
\includegraphics[height=0.85\textheight]{./figures/bcaa.png}
\end{frame}

\begin{frame}[label={sec:orgheadline3}]{BCAA Catabolism}
\begin{itemize}
\item First two steps are shared
\begin{description}
\item[{BCAT}] reversible transammination
\item[{BCKDC}] irreversible oxidative decarboxylation and thioesterification
\begin{itemize}
\item CoA derivatives committed to oxidative phosphorylation
\item Regulated step
\begin{itemize}
\item products
\item phosporylation
\item transcription
\end{itemize}
\end{itemize}
\end{description}
\item Escape first pass hepatic metabolism
\item Metabolized in skeletal muscle and adipose tissue
\end{itemize}
\end{frame}

\begin{frame}[label={sec:orgheadline4}]{Branched Chain \(\alpha\)-Ketoacid-Dehdrogenase Complex}
\begin{itemize}
\item multi-enzyme complex in the IMM
\item shares structural and biochemical features with pyruvate and
\(\alpha\)-ketoglutarate dehydrogenase
\end{itemize}
\begin{block}{Feedback Inhibition}
\begin{itemize}
\item product inhibition by branched chain acyl-CoAs
\item \(\uparrow\) NADH:NAD\(^{\text{+}}\) ratio
\end{itemize}
\end{block}
\begin{block}{Phosphorylation}
\begin{itemize}
\item Inhibited by phosphorylation of E1\(\alpha\) Ser 293
\item \(\uparrow\) protein diet, adrenaline and glucogon dephosphorylate BCKD
\item Phenylbutyrate prevents dephosphorylation
\end{itemize}
\end{block}
\begin{block}{Gene Expression}
\begin{itemize}
\item three subunits differentially regulated
\end{itemize}
\end{block}
\end{frame}

\begin{frame}[label={sec:orgheadline5}]{Amino Acid Transport}
\begin{itemize}
\item Large Neutral Amino Acids
\begin{itemize}
\item Phe, Trp, Met, Tyr, His, Thr, \textbf{Leu, Ile, Val}
\end{itemize}
\item Transported to the brain and other organs primarily by the
L1-neutral amino acid transporter
\item Relative [ ] \(\to\) competition \(\to\) neurotransmitters
\end{itemize}
\end{frame}
\begin{frame}[label={sec:orgheadline6}]{Leucine}
nnn
\end{frame}
\begin{frame}[label={sec:orgheadline7}]{Maple Syrup Urine Disease}
\begin{itemize}
\item Only BCAA disorder detected by plasma amino acids.
\item \(\Uparrow\) \(\Uparrow\) Leucine, also isoleucine and valine
\item BCKA = \(\alpha\)-ketoisocaproate
\end{itemize}
\end{frame}


\begin{frame}[label={sec:orgheadline8}]{Blocks Distal to BCKD}
\begin{itemize}
\item Do no accumulate AAs or 2-oxo-acids
\item Lead to accumulation of intermediates proximal to the block
\item Plasma Acylcarnitines or Urine Organic Acids
\end{itemize}

\begin{columns}
\begin{column}{0.33\columnwidth}
\begin{block}{Leucine}
\begin{itemize}
\item isovaleric acidemia
\item 3-MCC carboxylase
\item 3-Methyl-glutaconic aciduria
\item 3-OH-2-methylglutaryl-CoA lyase
\end{itemize}
\end{block}
\end{column}

\begin{column}{0.33\columnwidth}
\begin{block}{Isoleucine}
\begin{itemize}
\item mitochondrial acetoacetyl-CoA thiolase
\item short/branched chain acyl-CoA dehydrogenase
\item 2-methyl-3-OH-butyryl-CoA dehydrogenase
\end{itemize}
\end{block}
\end{column}

\begin{column}{0.33\columnwidth}
\begin{block}{Valine}
\begin{itemize}
\item isobutyryl-CoA dehydrogenase
\item 3-OH-isobutyryl-CoA hydrolase
\item 2-OH-isobutyric aciduria
\item methylmalonic semialdehyde dehydrogenase
\end{itemize}
\end{block}
\end{column}
\end{columns}
\end{frame}
\end{document}

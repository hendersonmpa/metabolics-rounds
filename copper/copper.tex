% Created 2019-05-02 Thu 12:58
% Intended LaTeX compiler: pdflatex
\documentclass[presentation, smaller]{beamer}
\usepackage[utf8]{inputenc}
\usepackage[T1]{fontenc}
\usepackage{graphicx}
\usepackage{grffile}
\usepackage{longtable}
\usepackage{wrapfig}
\usepackage{rotating}
\usepackage[normalem]{ulem}
\usepackage{amsmath}
\usepackage{textcomp}
\usepackage{amssymb}
\usepackage{capt-of}
\usepackage{hyperref}
\hypersetup{colorlinks,linkcolor=white,urlcolor=blue}
\usepackage{textpos}
\usepackage{textgreek}
\usepackage[version=4]{mhchem}
\usepackage{chemfig}
\usepackage{siunitx}
\usepackage{gensymb}
\usepackage[usenames,dvipsnames]{xcolor}
\usepackage[T1]{fontenc}
\usepackage{lmodern}
\usepackage{verbatim}
\usepackage{tikz}
\usepackage{wasysym}
\usetikzlibrary{shapes.geometric,arrows,decorations.pathmorphing,backgrounds,positioning,fit,petri}
\usetheme{Hannover}
\usecolortheme{whale}
\author{Matthew Henderson, PhD, FCACB}
\date{\today}
\title{Copper Transport Disorders}
\institute[NSO]{Newborn Screening Ontario | The University of Ottawa}
\titlegraphic{\includegraphics[height=1cm,keepaspectratio]{../logos/NSO_logo.pdf}\includegraphics[height=1cm,keepaspectratio]{../logos/cheo-logo.png} \includegraphics[height=1cm,keepaspectratio]{../logos/UOlogoBW.eps}}
\hypersetup{
 pdfauthor={Matthew Henderson, PhD, FCACB},
 pdftitle={Copper Transport Disorders},
 pdfkeywords={},
 pdfsubject={},
 pdfcreator={Emacs 26.1 (Org mode 9.1.9)}, 
 pdflang={English}}
\begin{document}

\maketitle

%\logo{\includegraphics[width=1cm,height=1cm,keepaspectratio]{../logos/NSO_logo_small.pdf}~%
%    \includegraphics[width=1cm,height=1cm,keepaspectratio]{../logos/UOlogoBW.eps}%
%}

\vspace{220pt}
\beamertemplatenavigationsymbolsempty
\setbeamertemplate{caption}[numbered]
\setbeamerfont{caption}{size=\tiny}
% \addtobeamertemplate{frametitle}{}{%
% \begin{textblock*}{100mm}(.85\textwidth,-1cm)
% \includegraphics[height=1cm,width=2cm]{cat}
% \end{textblock*}}

\section{Introductions}
\label{sec:orgf8eb4c1}
\begin{frame}[label={sec:org2a95cf6}]{Introduction}
\begin{itemize}
\item Copper is essential for cellular metabolism and toxic at [\(\Uparrow\)]
\begin{itemize}
\item superoxide dismutase
\item > 90\% of circulating copper is bound to ceruloplasmin
\end{itemize}
\item 2 mg of copper absorbed each day from the intestine
\item removed from portal circulation by hepatocytes
\item excretion of copper by liver into bile is only method of removal
\item Copper transporter CTR1 responsible for copper uptake in enterocytes and hepatocytes
\item Intra-cellular transport done by two related ATPases
\begin{itemize}
\item ATP7A : absorption : enterocytes : Menkes
\item ATP7B : excretion : hepatocytes : Wilson
\end{itemize}
\end{itemize}
\end{frame}

\begin{frame}[label={sec:org6653d3c}]{Copper Metabolism}
\begin{figure}[htbp]
\centering
\includegraphics[width=0.9\textwidth]{./figures/copper.PNG}
\caption[copper]{\label{fig:orgb2773d4}
Copper Metabolism}
\end{figure}
\end{frame}

\section{Wilson Disease}
\label{sec:org82955cf}
\begin{frame}[label={sec:orgfc2d2ad}]{Clinical Presentation}
\begin{itemize}
\item hepatic symptoms 8-20 years
\item neurological symptoms 2nd-3rd decade
\item Suspect in patients with:
\begin{itemize}
\item liver disease w no obvious cause
\item movement disorder
\end{itemize}
\item Occasionally isolated:
\begin{itemize}
\item \(\uparrow\) transaminases
\item Kayser-Fleischer rings
\item hemolysis
\end{itemize}
\item Diagnosis often made in siblings of patient
\end{itemize}
\end{frame}

\begin{frame}[label={sec:org260f9d2}]{Metabolic Derangement}
\begin{itemize}
\item Defect in trans-Golgi protein ATP7B
\begin{itemize}
\item required for excretion of copper and incorporation into ceruloplasmin
\end{itemize}
\item \(\downarrow\) t\(_{\text{1/2}}\) of ceruloplasmin with out bound copper
\item rare patients with excretion defect and normal ceruloplasmin binding
\item \(\downarrow\) excretion of copper into bile
\item accumulation of copper in liver
\begin{itemize}
\item secondary accumulation in brain, kidney and eyes
\end{itemize}
\end{itemize}
\end{frame}

\begin{frame}[label={sec:orgfc3a191}]{Genetics}
\begin{itemize}
\item AR, mutations in ATP7B, 1:30,000 may be higher, \(\sim\) 1/90 carriers
\item ATP7b has 6 copper binding domains
\item expressed predominantly in liver and kidney
\item > 500 mutations in ATP7B described
\begin{itemize}
\item most patients are compound heterozygotes
\end{itemize}
\item loss of function mutations \(\to\) early hepatic presentation
\item mutations with residual activity \(\to\) late neurological presentation
\end{itemize}
\end{frame}

\begin{frame}[label={sec:org7f18729}]{Diagnostic tests}
\begin{itemize}
\item \(\downarrow\) serum ceruloplasmin
\item \(\downarrow\) serum copper
\item \(\uparrow\) urine copper
\item \(\uparrow\) liver copper
\item \(\uparrow\) free copper
\begin{itemize}
\item 1 mg ceruloplasmin contains 3.4 ug copper
\end{itemize}
\item results should be taken together, there is a scoring system \footnote{Clinical Practice Guidelines: Wilson's Disease, J Hepatol 56:671-685}
\item genetic analysis in family

\item possible candidate for NBS
\end{itemize}
\end{frame}

\begin{frame}[label={sec:org566055f}]{Treatment}
\begin{itemize}
\item excellent prognosis if treated before severe damage
\item penicillamine chelates copper and is excreted in urine
\item oral zinc induces metallothionein synthesis
\begin{itemize}
\item metallothionein binds copper preferentially to zinc
\item fecal excretion
\end{itemize}
\item Trien (triethylenetetramine) is a chelator, used in patients who don't tolerate penicillamine
\item combination therapy should be staggered - don't chelate treatment!
\end{itemize}
\end{frame}


\section{Menkes Disease}
\label{sec:orgc5980b2}

\begin{frame}[label={sec:orgae61583}]{Clinical Presentation}
\begin{itemize}
\item male infants 2-3 months
\item neurodegeneration manifests as:
\begin{itemize}
\item seizures, hypotonia, loss of milestones
\end{itemize}
\item non-specific signs at birth:
\begin{itemize}
\item prematurity, large cephalhaematomas, skin laxity, hypothermia
\item hair breaks easily, sandpaper feel
\end{itemize}
\end{itemize}
\end{frame}

\begin{frame}[label={sec:org1ff563b}]{Metabolic Derangement}
\begin{itemize}
\item Defect in ATP7A
\item normal copper uptake, can not be exported from enterocytes into circulation
\item insufficient copper for incorporation into \textasciitilde{}20 cuproenzymes
\begin{itemize}
\item lysloxidase : collagen cross-linking
\item tyrosinase : melanin formation
\item dopamine \(\beta\)-hydroxylase : catacholamin biosynthesis
\item peptidyl glycine monooxygenase : neuropeptide precursors
\item cytochrome c-oxidase : ETC
\end{itemize}
\end{itemize}
\end{frame}

\begin{frame}[label={sec:org3c7fe19}]{Genetics}
\begin{itemize}
\item ATB7A, XR, 1:250,000 , 1/3 \emph{de novo} mutations
\item expressed in all tissues except liver
\end{itemize}

\begin{quotation} %% :B\(_{\text{quotation}}\):
If the reproductive fitness of a male affected with an X-linked
recessive disorder is low or nil, then in a population \alert{one-third of}
\alert{all affected X chromosomes will be removed from the gene pool every
generation}. If the incidence of the disease is constant, then
one-third of cases must be due to mutations arising de novo in a
family.
\end{quotation}

\begin{columns}
\begin{column}{0.5\columnwidth}
\[
p^2 + 2pq + q^2 = 1 
\]
\[
q^2 \sim 0
\]
\end{column}

\begin{column}{0.5\columnwidth}
\begin{center}
\begin{tabular}{|c|c|c|}
 & X & Y\\
\hline
X & XX & XY\\
\hline
x & xX & \alert{xY}\\
\end{tabular}
\end{center}
\end{column}
\end{columns}
\end{frame}


\begin{frame}[label={sec:org9d66e01}]{The Haldane Hypothesis}
\begin{itemize}
\item Applies to X-linked recessive traits
\begin{itemize}
\item A study of fertility rates in hemophillia
\end{itemize}

\item In a large population of 2N (N \male{} and N \female)
\item (1 - f)xN genes removed per generation
\begin{itemize}
\item x = proportion of affected males in the polulation
\item f = effective fertility
\end{itemize}

\item Each of N \female{} has 2X/cell, and each of N \male{} has 1X/cell
\item The mean mutation rate per X-chromosome per generation is: \footnote{Haldane JB. The rate of spontaneous mutation of a human gene. 1935. J Genet 2004;83:235-44.}
\end{itemize}

\[
u = 1/3(1 - f)x  
\]
\end{frame}


\begin{frame}[label={sec:org6018370}]{Diagnostic Tests}
\begin{itemize}
\item \(\downarrow\) serum copper (< 11 umol/L)
\item \(\downarrow\) serum ceruloplasmin (< 200 mg/L)
\item not specific in 0-3 months of life
\item plasma dopamine/norepinephrine
\item copper retention in cultured fibroblasts
\end{itemize}
\end{frame}

\begin{frame}[label={sec:orga2f0119}]{Treatment}
\begin{itemize}
\item often fatal < 3 years
\begin{itemize}
\item infection or vascular complications
\end{itemize}
\item Parenteral treatment should bypass ATP7A
\begin{itemize}
\item disappointing results
\item near normal intellectual and motor development only possible with
residual ATP7A activity
\end{itemize}
\end{itemize}
\end{frame}
\end{document}
% Created 2017-09-28 Thu 09:50
\documentclass[presentation, smaller]{beamer}
\usepackage[utf8]{inputenc}
\usepackage[T1]{fontenc}
\usepackage{fixltx2e}
\usepackage{graphicx}
\usepackage{grffile}
\usepackage{longtable}
\usepackage{wrapfig}
\usepackage{rotating}
\usepackage[normalem]{ulem}
\usepackage{amsmath}
\usepackage{textcomp}
\usepackage{amssymb}
\usepackage{capt-of}
\usepackage{hyperref}
\hypersetup{colorlinks,linkcolor=white,urlcolor=blue}
\usepackage{textpos}
\usepackage[version=4]{mhchem}
\usepackage{chemfig}
\usepackage[usenames,dvipsnames]{xcolor}
\usepackage[T1]{fontenc}
\usepackage{lmodern}
\usepackage{verbatim}
\usetheme[height=20pt]{Boadilla}
\usecolortheme[RGB={170,160,80}]{{structure}}
\author{Matthew Henderson, PhD, FCACB}
\date{\today}
\title{Newborn Screening for Urea Cycle Disorders}
\institute[NSO]{Newborn Screening Ontario | The University of Ottawa}
\titlegraphic{\includegraphics[height=1cm,keepaspectratio]{../logos/NSO_logo.pdf} \includegraphics[height=1cm,keepaspectratio]{../logos/UOlogoBW.eps}}
\hypersetup{
 pdfauthor={Matthew Henderson, PhD, FCACB},
 pdftitle={Newborn Screening for Urea Cycle Disorders},
 pdfkeywords={},
 pdfsubject={},
 pdfcreator={Emacs 25.2.1 (Org mode 8.3.4)}, 
 pdflang={English}}
\begin{document}

\maketitle
\logo{\includegraphics[width=1cm,height=1cm,keepaspectratio]{../logos/NSO_logo_small.pdf}~%
    \includegraphics[width=1cm,height=1cm,keepaspectratio]{../logos/UOlogoBW.eps}%
}

\vspace{220pt}}
\beamertemplatenavigationsymbolsempty
\setbeamertemplate{caption}[numbered]
\setbeamerfont{caption}{size=\tiny}

% \addtobeamertemplate{frametitle}{}{%
% \begin{textblock*}{100mm}(.85\textwidth,-1cm)
% \includegraphics[height=1cm,width=2cm]{cat}
% \end{textblock*}}

\section{Background}
\label{sec:orgheadline7}
\begin{frame}[label={sec:orgheadline1}]{Objectives}
\begin{itemize}
\item Review of the Urea cycle
\item Newborn Screening for UCD
\end{itemize}
\end{frame}



\begin{frame}[label={sec:orgheadline2}]{The Urea Cycle}
\begin{itemize}
\item Free ammonium ion, generated from the glutaminase and glutamate
dehydrogenase reactions,
\begin{itemize}
\item Condensed with bicabonate and converted to urea for excretion.
\end{itemize}
\item All required enzymes are expressed in the periportal cells of the liver lobule.
\end{itemize}

\includegraphics[width=.9\linewidth]{./figures/liver_lobule.png}
\end{frame}

\begin{frame}[label={sec:orgheadline3}]{The Urea Cycle}
\begin{block}{Mitochondrial enzymes:}
\begin{itemize}
\item Carbamoylphosphate synthetase I (CPS1, AR)
\begin{itemize}
\item rate-limiting reaction of the urea cycle
\item N-acetylglutamate is an obligate activator
\end{itemize}
\item N-acetylglutamate synthetase (NAGS, AR)
\item Ornithine transcarbamylase (OTC, X-linked)
\end{itemize}
\end{block}
\begin{block}{Cytoplasmic enzymes:}
\begin{itemize}
\item Argininosuccinic acid synthetase (ASS1, AR)
\item Argininosuccinic acid lyase (ASL, AR)
\item Arginase (ARG1, AR)
\end{itemize}
\end{block}

\begin{block}{Transporters:}
\begin{itemize}
\item Ornithine translocase (SLC25A15, AR)
\end{itemize}
\end{block}
\end{frame}

\begin{frame}[label={sec:orgheadline4}]{The Urea Cycle: Ancillary Enzymes}
\begin{itemize}
\item Carbonic Anhydrase Va (CAVA)
\item Citrin (SLC25A13, AR)
\item \(\Delta\)-pyrroline-5 carboxylate synthetase (P5CS)
\end{itemize}
\end{frame}


\begin{frame}[label={sec:orgheadline5}]{The Urea Cycle}
\includegraphics[width=.9\linewidth]{./figures/ucd-overview-Image001.jpg}
\end{frame}

\begin{frame}[label={sec:orgheadline6}]{Incidence of UCDs}
\begin{center}
\begin{tabular}{llll}
Urea Cycle Disorder & Estimated Incidence & RUSP & NSO\\
\hline
NAGS deficiency & <1:2,000,000 & No & No\\
CPS1 deficiency & 1:1,300,000 & No & No\\
\textbf{OTC deficiency} & \textbf{1:56,500} & \textbf{No} & \textbf{No}\\
ASS1 deficiency & 1:250,000 & Yes & Yes\\
ASL deficiency & 1:218,750 & Yes & Yes\\
ARG1 deficiency & 1:950,000 & No & No\\
Ornithine translocase deficiency & Unknown & No & No\\
Citrin deficiency & 1:100,000-1:230,000 in Japan 1 & No & No\\
\end{tabular}
\end{center}
\end{frame}

\section{NBS for UCDs}
\label{sec:orgheadline18}
\begin{frame}[label={sec:orgheadline8}]{NBS Challenges}
\begin{itemize}
\item Biochemical 
\begin{itemize}
\item Sensitivity of NBS methods and markers for the mitochondrial UCDs.
\end{itemize}
\item Clinical 
\begin{itemize}
\item Severely affected patients - very early onset of disease before NBS results are available
\item Patients with mild disease - pre-symptomatic detection of disease remains controversial
\end{itemize}
\end{itemize}
\end{frame}

\begin{frame}[label={sec:orgheadline9}]{Markers for NBS for Mitochondrial UCDs}
\begin{itemize}
\item Measurement of ammonia is not feasible
\begin{itemize}
\item post-collection elevation, no methods in lit
\end{itemize}
\item Glutamine is another metabolite that is generally elevated in UCDs.
\begin{itemize}
\item spontaneous glutamate and pyroglutamate formation.
\end{itemize}
\item Orotic acid is often elevated in the urine of patients with
OTC deficiency, cannot be used as a marker in blood.
\end{itemize}
\end{frame}

\begin{frame}[label={sec:orgheadline10}]{NBS for Mitochondrial UCDs}
\begin{itemize}
\item Mitochondrial enzymes: NAGS, CPS1 and OTC
\end{itemize}

\begin{block}{Biomarkers}
\begin{itemize}
\item Low or decreased plasma levels of citrulline and arginine
\item Urine orotic acid is often elevated in OTCD
\end{itemize}
\end{block}
\end{frame}

\begin{frame}[label={sec:orgheadline11}]{NBS for Cytoplasmic UCDs}
\begin{itemize}
\item Cytoplasmic enzymes: ASS1, ASL, ARG1
\end{itemize}

\begin{block}{Biomarkers}
\begin{itemize}
\item ASSD: \(\uparrow\) citrulline
\item ASLD: \(\uparrow\) argininosuccinate, \(\uparrow\) citrulline
\item ARG1D: \(\uparrow\) arginine
\end{itemize}
\end{block}
\end{frame}

\begin{frame}[label={sec:orgheadline12}]{Transporter Defects}
\begin{itemize}
\item Membrane bound transporters : ORNT1, Citrin
\end{itemize}

\begin{block}{Biomarkers}
\begin{itemize}
\item Hyperammonemia-hyperornithinemia-homocitrullinuria syndrome (ORNT1): \(\uparrow\) ornithine
\begin{itemize}
\item Ornithine not elevated in newborns
\end{itemize}
\item Citrullinemia type II (Citrin): \(\uparrow\) citrulline
\end{itemize}
\end{block}
\end{frame}

\begin{frame}[label={sec:orgheadline13}]{NBS for UCDs in the US}
\begin{itemize}
\item Current newborn screening panels in the United States using tandem
mass spectrometry detect abnormal concentrations of analytes
associated with ASS1 deficiency, and ASL deficiency in all states.

\item Other disorders are screened for in some states only:
\begin{itemize}
\item CPS1 deficiency is screened for in Florida, Maine, Massachusetts,
Mississippi, New Hampshire, Pennsylvania, Rhode Island, and
Vermont.
\item OTC deficiency is screened for in Connecticut, Maine,
Massachusetts, New Hampshire, Rhode Island, and Vermont, and is
likely to be detected in Kentucky and Utah.
\item Arginase deficiency is screened for in 35 states and likely to be
detected in four more.
\item Citrin deficiency is screened for in 36 states and likely to be
detected in 13 more.
\end{itemize}
\end{itemize}
\end{frame}

\begin{frame}[label={sec:orgheadline14}]{Newborn Screening for UCD: OTC}
\begin{itemize}
\item The sensitivity and specificity of a low citrulline level as a
marker for OTC deficiency in NBS has been questioned.
\begin{itemize}
\item common causes of low citrulline in premature infants or in sick
babies such as those with pathological conditions involving the
small intestine, i.e. short-bowel syndrome
\end{itemize}
\item The detection of OTC deficiency on NBS may be improved by using
Collaborative Laboratory Integrated Reports (CLIR) which includes
glutamine, glutamate, and amino acid ratios in the analysis.
\end{itemize}
\end{frame}

\begin{frame}[label={sec:orgheadline15}]{Newborn Screening for UCD in Ontario}
\begin{itemize}
\item Screen for ASA and ASL
\item Primary marker is citrulline
\item Secondary markers are:
\begin{itemize}
\item ASA
\item CIT/ORN
\item ASA/ARG
\end{itemize}
\end{itemize}
\end{frame}

\begin{frame}[label={sec:orgheadline16}]{Quantitative FIA-MS/MS}
\begin{itemize}
\item Amino acids in the DBS eluate are esterified as butyl esters with butanol-hydrogen chloride
\end{itemize}

\centering
\schemedebug{false}
\schemestart
\chemname{\chemfig[][scale=.33]{H_2N-[::30,,2,](=[::60]O)-[::-60]NH-[::60]-[::-60]-[::60]-[::-60](<[::-60]NH_3^+)-[::60](=[::60]O)-[::-60]OH}}{\tiny citrulline 175 Da}
\+
\chemname{\chemfig[][scale=.33]{HO-[::30]-[::-60]-[::60]-[::-60]}}{\tiny n-butanol 74 Da}
\arrow{-U>[][{\tiny \ce{H2O}}]}
\chemname{\chemfig[][scale=.33]{H_2N-[::30,,2,](=[::60]O)-[::-60]NH-[::60]-[::-60]-[::60]-[::-60](<[::-60]NH_3^+)-[::60](=[::60]O)-[::-60]O-[::60]-[::-60]-[::60]-[::-60]}}{\tiny 232 Da}
\schemestop

\begin{itemize}
\item Citrulline contains a labile amino group that fragments together with butyl formate.
\item CID results in net neutral fragmentation of butyl formate (102 Da) plus \ce{NH3} (17 Da)
\item \href{https://en.wikipedia.org/wiki/Selected_reaction_monitoring}{SRM} Citrulline-Bu 232.15 Da \(\to\) 113 Da , loss of 119 Da
\end{itemize}

\centering
\schemedebug{false}
\schemestart
\chemname{\chemfig[][scale=.33]{H_2N-[::60]-[::-60]-[::60]-[::-60]-[::60]N=O=C}}{\tiny 113 Da}
\+
\chemname{\chemfig[][scale=.33]{H-[::60](=[::60]O)-[::-60]O-[::60]-[::-60]-[::60]-[::-60]}}{\tiny 102 Da}
\+
\chemname{\chemfig[][scale=.43]{NH_3}}{\tiny 17 Da}
\schemestop
\end{frame}


\begin{frame}[label={sec:orgheadline17}]{Screening Thresholds}
\begin{quote} %% :B\(_{\text{quote}}\):
CIT \(\ge\) 70 OR \\
[CIT \(\ge\) 40 AND (ASA≥2.5 OR CIT/ARG\(\ge\) 6.61 OR CIT/ORN\(\ge\) 2.40 OR ASA/ORN\(\ge\) 0.10 OR ASA/ARG\(\ge\) 0.12)]
\end{quote}

\begin{itemize}
\item Citrulline > 70 \textmu{}mol/L is confirmed

\item Citrulline > 100 \textmu{}mol/L also prompt measurement of ASA

\begin{itemize}
\item Daughter ions of 459 Da
\end{itemize}
\end{itemize}
\end{frame}


\section{Clinical Challenges}
\label{sec:orgheadline21}
\begin{frame}[label={sec:orgheadline19}]{Severe Mitochondrial UCDs}
\begin{itemize}
\item First symptoms occur soon after birth between 12 and 72 h of age
\item Although results were late, some patients still benefited from the availability of results shortly after presentation.
\end{itemize}
\end{frame}

\begin{frame}[label={sec:orgheadline20}]{Mild UCDs}
\begin{itemize}
\item there are patients described with a late-onset of disease with only a single, few or even absence of symptom(s) and only a biochemical phenotype.
\item case of ASSD, described as suffering from mild citrullinemia type 1,
\begin{itemize}
\item a condition allelic to classical citrullinemia type 1 but much milder and with less, if any need for medical intervention.
\end{itemize}
\item Such patients were often identified in neonatal screening programs and it has been discussed whether their mild phenotype would result from early detection
\end{itemize}
and initiation of treatment or from a relevant residual enzyme or transporter function. 
\begin{itemize}
\item Metabolites and/or mutation analysis may help to identify attenuated patients to avoid medicalization of non-diseases
\end{itemize}
\end{frame}
\end{document}

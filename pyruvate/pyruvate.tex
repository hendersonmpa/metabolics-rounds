% Created 2019-02-27 Wed 15:49
% Intended LaTeX compiler: pdflatex
\documentclass[presentation, smaller]{beamer}
\usepackage[utf8]{inputenc}
\usepackage[T1]{fontenc}
\usepackage{graphicx}
\usepackage{grffile}
\usepackage{longtable}
\usepackage{wrapfig}
\usepackage{rotating}
\usepackage[normalem]{ulem}
\usepackage{amsmath}
\usepackage{textcomp}
\usepackage{amssymb}
\usepackage{capt-of}
\usepackage{hyperref}
\hypersetup{colorlinks,linkcolor=white,urlcolor=blue}
\usepackage{textpos}
\usepackage{textgreek}
\usepackage[version=4]{mhchem}
\usepackage{chemfig}
\usepackage{siunitx}
\usepackage{gensymb}
\usepackage[usenames,dvipsnames]{xcolor}
\usepackage[T1]{fontenc}
\usepackage{lmodern}
\usepackage{verbatim}
\usepackage{tikz}
\usetikzlibrary{shapes.geometric,arrows,decorations.pathmorphing,backgrounds,positioning,fit,petri}
\usetheme{Hannover}
\usecolortheme{whale}
\author{Matthew Henderson, PhD, FCACB}
\date{\today}
\title{Disorders of Pyruvate Metabolism}
\institute[NSO]{Newborn Screening Ontario | The University of Ottawa}
\titlegraphic{\includegraphics[height=1cm,keepaspectratio]{../logos/NSO_logo.pdf}\includegraphics[height=1cm,keepaspectratio]{../logos/cheo-logo.png} \includegraphics[height=1cm,keepaspectratio]{../logos/UOlogoBW.eps}}
\hypersetup{
 pdfauthor={Matthew Henderson, PhD, FCACB},
 pdftitle={Disorders of Pyruvate Metabolism},
 pdfkeywords={},
 pdfsubject={},
 pdfcreator={Emacs 26.1 (Org mode 9.1.9)}, 
 pdflang={English}}
\begin{document}

\maketitle

%\logo{\includegraphics[width=1cm,height=1cm,keepaspectratio]{../logos/NSO_logo_small.pdf}~%
%    \includegraphics[width=1cm,height=1cm,keepaspectratio]{../logos/UOlogoBW.eps}%
%}

\vspace{220pt}
\beamertemplatenavigationsymbolsempty
\setbeamertemplate{caption}[numbered]
\setbeamerfont{caption}{size=\tiny}
% \addtobeamertemplate{frametitle}{}{%
% \begin{textblock*}{100mm}(.85\textwidth,-1cm)
% \includegraphics[height=1cm,width=2cm]{cat}
% \end{textblock*}}

\tikzstyle{chemical} = [rectangle, rounded corners, text width=5em, minimum height=1em,text centered, draw=black, fill=none]
\tikzstyle{hardware} = [rectangle, rounded corners, text width=5em, minimum height=1em,text centered, draw=black, fill=gray!30]
\tikzstyle{ms} = [rectangle, rounded corners, text width=5em, minimum height=1em,text centered, draw=orange, fill=none]
\tikzstyle{msw} = [rectangle, rounded corners, text width=7em, minimum height=1em,text centered, draw=orange, fill=none]
\tikzstyle{label} = [rectangle,text width=8em, minimum height=1em, text centered, draw=none, fill=none]
\tikzstyle{hl} = [rectangle, rounded corners, text width=5em, minimum height=1em,text centered, draw=black, fill=red!30]
\tikzstyle{box} = [rectangle, rounded corners, text width=5em, minimum height=5em,text centered, draw=black, fill=none]
\tikzstyle{arrow} = [thick,->,>=stealth]
\tikzstyle{hl-arrow} = [ultra thick,->,>=stealth,draw=red]

\section{Introduction}
\label{sec:orgc3432d4}
\begin{frame}[label={sec:org10ec973}]{Disorders of Pyruvate Metabolism}
\begin{itemize}
\item Pyruvate Carboxylase Deficiency
\item Phospoenolpyruvate Carboxykinase Deficiency
\item Pyruvate Dehydrogenase Complex Deficiency
\begin{itemize}
\item Pyruvate Transporter Defect
\end{itemize}
\item Dihydrolipoamide Dehydrogenase Deficiency
\end{itemize}
\end{frame}

\begin{frame}[label={sec:orgbd1d733}]{Tricarboxylic Acid Cycle}
\begin{figure}[htbp]
\centering
\includegraphics[width=0.7\textwidth]{./figures/tca.png}
\caption[TCA]{\label{fig:orgafdecd9}
Tricarboxylic Acid Cycle}
\end{figure}
\end{frame}

\begin{frame}[label={sec:org95f450b}]{Tricarboxylic Acid Cycle}
\begin{figure}[htbp]
\centering
\includegraphics[width=0.7\textwidth]{./figures/pyruvate_disorders.png}
\caption[TCA]{\label{fig:org8f1c4b2}
Tricarboxylic Acid Cycle Disorders}
\end{figure}
\end{frame}

\section{Pyruvate Carboxylase Deficiency}
\label{sec:orge347ffa}
\begin{frame}[label={sec:org4ce7072}]{Pyruvate Carboxylase}
\begin{itemize}
\item PC is a biotinylated mitochondrial matrix enzyme.
\end{itemize}
\begin{quotation} %% :B\(_{\text{quotation}}\):
pyruvate + \ce{CO2 ->[PC]} oxaloacetate
\end{quotation}

\begin{itemize}
\item important role in:
\begin{itemize}
\item gluconeogenesis
\begin{itemize}
\item urea cycle indirectly
\end{itemize}
\item anaplerosis
\begin{itemize}
\item \(\downarrow\) 2-ketoglutarate \(\to\) \(\downarrow\) glutamate
\end{itemize}
\item lipogenesis
\begin{itemize}
\item oxaloacetate + acetyl-CoA \(\to\) citrate
\end{itemize}
\end{itemize}
\end{itemize}
\end{frame}

\begin{frame}[label={sec:org6ff2822}]{Clinical Presentation}
\begin{itemize}
\item French phenotype (type B), most severe
\begin{itemize}
\item acute illness 3-48h after birth
\item hypothermia, hypotonia, lethargy, vomiting
\item severe neurological dysfunction
\item death prior to 5 months
\end{itemize}
\item North American phenotype (type A)
\begin{itemize}
\item severe illness between 2 and 5 months of age
\item progressive hypotonia
\item acute vomiting, dehydration, tachypnoea, metabolic acidosis
\item severe intellectual disability
\item progressive with death in infancy
\end{itemize}
\item Benign phenotype (type c)
\begin{itemize}
\item rare
\item acute episodes of lactic acidosis and ketoacidosis
\item near normal cognitive and motor development
\end{itemize}
\end{itemize}
\end{frame}
\begin{frame}[label={sec:org84ff3a5}]{Genetics}
\begin{itemize}
\item Autosomal recessive with incidence of 1 in 250000
\item PC is a homo-tetramer
\item PC protein and mRNA absent in 50\% of French phenotype
\item American and Benign phenotypes have cross-reacting material
\item Mosaicism has been observed with prolonged survival
\end{itemize}
\end{frame}

\begin{frame}[label={sec:org7cbf8f6}]{Diagnostic Tests}
\begin{itemize}
\item PC deficiency should be considered in any child presenting with lactic acidosis and neurological abnormalities
\begin{itemize}
\item with hypoglycemia, hyperammonemia, or ketosis
\end{itemize}

\item \(\uparrow\) L/P with \(\downarrow\) BHB/acetoacetate  in severely affected patients
\begin{itemize}
\item pathognomonic in neonates
\end{itemize}

\item post-prandial ketosis, hypercitrullinemia, hyperammonemia, low glutamine

\item CSF lactate, alanine and L/P are elevated, glutamine decreased

\item PC activity in cultured skin fibroblasts
\begin{itemize}
\item can not distinguish severity
\end{itemize}
\end{itemize}
\end{frame}

\begin{frame}[label={sec:orgb71ecaf}]{Treatment}
\begin{itemize}
\item Currently, no treatment.
\end{itemize}
\end{frame}

\section{Phospoenolpyruvate Carboxykinase Deficiency}
\label{sec:orgeca828c}
\begin{frame}[label={sec:org8454c12}]{Phospoenolpyruvate Carboxykinase Deficiency}
\begin{itemize}
\item PEPCK has cytosolic and mitochondria isoforms
\item Cytosolic PEPCK deficiency is secondary to hyperinsulinism
\begin{itemize}
\item insulin represses expression of the cytosolic form
\end{itemize}
\item Mitochondrial PEPCK deficiency has not been clearly demonstrated
\end{itemize}
\end{frame}

\section{Pyruvate Dehydrogenase Complex Deficiency}
\label{sec:org9033998}
\begin{frame}[label={sec:orgaee470b}]{Pyruvate Dehydrogenase Complex Deficiency}
\begin{itemize}
\item PDHC deficiency provokes conversion of pyruvate to lactate and alanine rather than acetly-CoA
\item Glucose \(\to\) lactate, produces 1/10 ATP compared complete oxidation via TCA and ETC
\item Impairs production of NADH but not oxidation
\item NADH/NAD\{\^{}+\} is normal, \(\therefore\) normal L/P
\begin{itemize}
\item ETC deficiencies \(\to\) \(\uparrow\) L/P
\end{itemize}
\item 
\end{itemize}
\end{frame}
\begin{frame}[label={sec:orgfef5b2f}]{Pyruvate Dehydrogenase Complex}
\begin{itemize}
\item PDHC, KDHC and BCKD have similar structure and mechanism
\item Composed of:
\begin{itemize}
\item E1 \(\alpha\)-ketoacid dehydrogenase
\item E2 dihydrolipoamide acyltransferase
\item E3 dihydrolipoamide dehydrogenases
\end{itemize}
\item E1 is specific to each complex
\begin{itemize}
\item Composed of E1\(\alpha\) and E1\(\beta\)
\end{itemize}
\item In E1 is the rate limiting step in PDHC
\begin{itemize}
\item regulated by phosphorylation
\end{itemize}
\end{itemize}

\begin{figure}[htbp]
\centering
\includegraphics[width=0.6\textwidth]{./figures/pdhe1_phos.png}
\caption[pdhe1]{\label{fig:orgd2d1e60}
Activation/deactivation of PDHE1}
\end{figure}
\end{frame}


\begin{frame}[label={sec:orga89f503}]{Pyruvate Dehydrogenase Complex}
\begin{figure}[htbp]
\centering
\includegraphics[width=0.9\textwidth]{./figures/pdhc.png}
\caption[pdhc]{\label{fig:org67c947d}
Pyruvate Dehydrogenase Complex}
\end{figure}
\end{frame}


\begin{frame}[label={sec:org913e587}]{Clinical Presentation: PDHE1\(\alpha\)}
\begin{itemize}
\item Majority of cases involve the X encoded to \(\alpha\)-subunit of the dehydrogenase (E1)
\begin{itemize}
\item PDHE1\(\alpha\) deficiency
\item developmental delay, hypotonia, seizures and ataxia
\end{itemize}

\item Common presentations in hemizygous males:
\begin{enumerate}
\item neonatal lactic acidosis
\begin{itemize}
\item most severe
\end{itemize}
\item Leigh's encephalopathy
\begin{itemize}
\item most common
\item present in first 5 years
\end{itemize}
\item intermittent ataxia
\begin{itemize}
\item rare
\item ataxia after carbohydrate rich meals \(\to\) Leigh's
\end{itemize}
\end{enumerate}

\item Females with PDHE1\(\alpha\), uniform presentation, variable severity
\begin{itemize}
\item dismorphic features
\item moderate to severe intellectual disability
\item seizures common
\item severe neonatal lactic acidosis can be present
\end{itemize}
\end{itemize}
\end{frame}

\begin{frame}[label={sec:orgaff1d3e}]{Clinical Presentation: PDHE1\(\beta\)}
\begin{itemize}
\item Only a few cases
\item similar to PDHE1\(\alpha\)
\end{itemize}
\end{frame}

\begin{frame}[label={sec:org9b5f6dc}]{Genetics}
\begin{itemize}
\item All components of PDHC are encoded by nuclear genes
\item Autosomal except E1\(\alpha\) on Xp22.11
\begin{itemize}
\item \therefor most PDHC deficiency is X-linked
\end{itemize}
\item No null E1\(\alpha\) identified except in a mosaic state
\begin{itemize}
\item suggests E1\(\alpha\) is essential
\end{itemize}
\end{itemize}
\end{frame}

\begin{frame}[label={sec:org3dca4e9}]{Diagnostic Tests}
\begin{itemize}
\item Lactate and pyruvate in blood and CSF
\item CSF lactate is generally \(\uparrow\) compared to blood
\item Urine organic acids
\begin{itemize}
\item lactate, pyruvate,
\end{itemize}
\item plasma amino acids
\begin{itemize}
\item alanine
\end{itemize}
\item L/P ratio is usually normal
\begin{itemize}
\item 
\end{itemize}

\item Skin fibroblasts for PDHC

\begin{itemize}
\item also lymphocytes, separated from EDTA <2days
\end{itemize}

\item PDHE1\(\alpha\) genotype in girls is useful
\end{itemize}
\end{frame}
\begin{frame}[label={sec:org269954e}]{Treatment}
\begin{itemize}
\item Early adoption of ketogenic diet may have a benefit
\item Thiamine
\item DCA is a pyruvate analog, inhibits E1 kinase, keeps E1 dephosphorylated (active)
\end{itemize}
\end{frame}

\begin{frame}[label={sec:org74fce1c}]{Pyruvate Transport Defect}
\begin{itemize}
\item MPC1 mutations have been described in 5 patients
\item mediates the proton symport of pyruvate across the IMM.
\item \(\therefore\) metabolic derangement similar to PDHC deficiency

\item No treatment
\end{itemize}
\end{frame}

\begin{frame}[label={sec:orgd1d05f8}]{Next time}
\begin{itemize}
\item TCA disorder
\begin{itemize}
\item 2-Ketoglutarate Dehydrogenase Complex Deficiency
\item Succinate Dehydrogenase Deficiency
\item Fumarase Deficiency
\end{itemize}
\end{itemize}
\end{frame}
\end{document}
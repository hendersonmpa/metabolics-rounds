% Created 2019-01-10 Thu 12:56
\documentclass[presentation, smaller]{beamer}
\usepackage[utf8]{inputenc}
\usepackage[T1]{fontenc}
\usepackage{fixltx2e}
\usepackage{graphicx}
\usepackage{grffile}
\usepackage{longtable}
\usepackage{wrapfig}
\usepackage{rotating}
\usepackage[normalem]{ulem}
\usepackage{amsmath}
\usepackage{textcomp}
\usepackage{amssymb}
\usepackage{capt-of}
\usepackage{hyperref}
\hypersetup{colorlinks,linkcolor=white,urlcolor=blue}
\usepackage{textpos}
\usepackage{textgreek}
\usepackage[version=4]{mhchem}
\usepackage{chemfig}
\usepackage{siunitx}
\usepackage{gensymb}
\usepackage[usenames,dvipsnames]{xcolor}
\usepackage[T1]{fontenc}
\usepackage{lmodern}
\usepackage{verbatim}
\usepackage{tikz}
\usetikzlibrary{shapes.geometric,arrows,decorations.pathmorphing,backgrounds,positioning,fit,petri}
\usetheme{Ilmenau}
\usecolortheme{whale}
\author{Matthew Henderson, PhD, FCACB}
\date{\today}
\title{Niemann-Pick Disease}
\subtitle{Sphingolipid and Cholesterol Degradation}
\institute[NSO]{Newborn Screening Ontario | The University of Ottawa}
\titlegraphic{\includegraphics[height=1cm,keepaspectratio]{../logos/NSO_logo.pdf}\includegraphics[height=1cm,keepaspectratio]{../logos/cheo-logo.png} \includegraphics[height=1cm,keepaspectratio]{../logos/UOlogoBW.eps}}
\hypersetup{
 pdfauthor={Matthew Henderson, PhD, FCACB},
 pdftitle={Niemann-Pick Disease},
 pdfkeywords={},
 pdfsubject={},
 pdfcreator={Emacs 25.2.1 (Org mode 8.3.4)}, 
 pdflang={English}}
\begin{document}

\maketitle
%\logo{\includegraphics[width=1cm,height=1cm,keepaspectratio]{../logos/NSO_logo_small.pdf}~%
%    \includegraphics[width=1cm,height=1cm,keepaspectratio]{../logos/UOlogoBW.eps}%
%}

\vspace{220pt}
\beamertemplatenavigationsymbolsempty
\setbeamertemplate{caption}[numbered]
\setbeamerfont{caption}{size=\tiny}
% \addtobeamertemplate{frametitle}{}{%
% \begin{textblock*}{100mm}(.85\textwidth,-1cm)
% \includegraphics[height=1cm,width=2cm]{cat}
% \end{textblock*}}

\tikzstyle{chemical} = [rectangle, rounded corners, text width=5em, minimum height=1em,text centered, draw=black, fill=none]
\tikzstyle{hardware} = [rectangle, rounded corners, text width=5em, minimum height=1em,text centered, draw=black, fill=gray!30]
\tikzstyle{ms} = [rectangle, rounded corners, text width=5em, minimum height=1em,text centered, draw=orange, fill=none]
\tikzstyle{msw} = [rectangle, rounded corners, text width=7em, minimum height=1em,text centered, draw=orange, fill=none]
\tikzstyle{label} = [rectangle,text width=8em, minimum height=1em, text centered, draw=none, fill=none]
\tikzstyle{hl} = [rectangle, rounded corners, text width=5em, minimum height=1em,text centered, draw=black, fill=red!30]
\tikzstyle{box} = [rectangle, rounded corners, text width=5em, minimum height=5em,text centered, draw=black, fill=none]
\tikzstyle{arrow} = [thick,->,>=stealth]
\tikzstyle{hl-arrow} = [ultra thick,->,>=stealth,draw=red]

\section{Introduction}
\label{sec:orgheadline10}

\begin{frame}[label={sec:orgheadline1}]{Niemann-Pick Disease}
\begin{itemize}
\item There are two distinct diseases called Niemann-Pick
\begin{itemize}
\item Types A and B (acid sphingomyelinase deficiency)
\item Type C (cholesterol recycling)
\end{itemize}
\end{itemize}
\end{frame}

\begin{frame}[label={sec:orgheadline2}]{Niemann-Pick A \& B}
\begin{itemize}
\item Incidence of Niemann-Pick A among Ashkenazi Jews \textasciitilde{} 1:40,000.
\item Incidence of both Niemann–Pick A and B in all other populations \textasciitilde{} 1:250,000.
\item Niemann-Pick Type A and B are caused by deficiency of acid sphingomyelinase (ASM).
\begin{itemize}
\item sphingomyelinase is required to metabolize sphingomyelin.
\item results in progressive accumulation of sphingomyelin in systemic organs
\begin{itemize}
\item brain accumulation in neuronal forms
\end{itemize}
\item There is growing evidence that NPA \& NPB represent opposite ends of a continuum.
\begin{itemize}
\item NPA generally have little or no ASM production (less than 1\% of normal).
\item NPB have approximately 10\% of normal level of ASM.
\end{itemize}
\end{itemize}
\end{itemize}
\end{frame}

\begin{frame}[label={sec:orgheadline3}]{Sphingomyelin and Sphingomyelinase}
\begin{columns}
\begin{column}{0.5\columnwidth}
\begin{figure}[htb]
\centering
\includegraphics[width=0.8\textwidth]{./figures/sphingomyelin.png}
\caption{\label{fig:}
Sphingomyelin}
\end{figure}
\end{column}

\begin{column}{0.5\columnwidth}
\begin{figure}[htb]
\centering
\includegraphics[width=0.8\textwidth]{./figures/sphingomyelinase.png}
\caption{\label{fig:}
Sphingomyelinase}
\end{figure}
\end{column}
\end{columns}
\end{frame}


\begin{frame}[label={sec:orgheadline4}]{Sphingolipid degradation}
\begin{figure}[htb]
\centering
\includegraphics[width=0.6\textwidth]{./figures/sl_degradation.png}
\caption[deg]{\label{fig:sld}
Sphingolipid degradation}
\end{figure}
\end{frame}

\begin{frame}[label={sec:orgheadline5}]{Niemann-Pick C}
\begin{itemize}
\item a fatal, neuro-degenerative disease that affects \textasciitilde{} 1:150,000
\begin{itemize}
\item sometimes referred to as Childhood Alzheimer’s
\item extremely heterogeneous
\item biochemically, genetically and clinically distinct from Niemann-Pick A and B.
\end{itemize}
\item Accumulation of unesterified cholesterol, sphingomyelin, glycolipids in systemic organs
\item GM2 and GM3 accumulate in brain
\begin{itemize}
\item no increase in cholesterol
\end{itemize}
\item NPC has two sub types NP-C1 (95\%) and NP-C2 (5\%)
\end{itemize}
\end{frame}

\begin{frame}[label={sec:orgheadline6}]{Niemann-Pick C}
\begin{columns}
\begin{column}{0.5\columnwidth}
\begin{itemize}
\item LDL cholesterol enters cells via endocytosis at the LDL receptor.
\item delivered to the late-stage endosomes and lysosomes
\item hydrolyzed and released as free cholesterol.
\item Unesterified cholesterol is transported to the plasma membrane and the ER for recycling.

\item In NP-C, the LDL-cholesterol is trapped in lysosomes
\end{itemize}
\end{column}

\begin{column}{0.5\columnwidth}
\begin{figure}[htb]
\centering
\includegraphics[width=0.8\textwidth]{./figures/cholesterol1.jpg}
\label{fig:}
\end{figure}
\end{column}
\end{columns}
\end{frame}


\begin{frame}[label={sec:orgheadline7}]{NPC1 \& NPC2}
\begin{figure}[htb]
\centering
\includegraphics[width=0.65\textwidth]{./figures/Niemann-Pick-C-Brown-and-Goldstein.png}
\label{fig:}
\end{figure}

\footnotesize
\begin{itemize}
\item NPC1 is a lysosomal membrane protein involved in transport in the endosomal-lysosomal system
\item NPC2 acts in cooperation with the NPC1
\item The disruption of transport results in accumulation of cholesterol and glycolipids in lysosomes.
\end{itemize}
\end{frame}

\begin{frame}[label={sec:orgheadline8}]{Lysosomal Protein Trafficking}
\begin{figure}[htb]
\centering
\includegraphics[width=0.65\textwidth]{./figures/lysosome_trafficking.jpeg}
\caption[traf]{\label{fig:traf}
Lysosomal protein trafficking receptors}
\end{figure}

\footnotesize
\begin{itemize}
\item lysosomal trafficking of acid sphingomyelinase is mediated by sortilin and mannose 6-phosphate receptor.
\item MPR alone is sufficient to transport NPC2 to the endo/lysosomal compartment
\item Sorting of LMPs from Golgi/PM to endosomal system is mediated by
signals in the cytosolic domain
\end{itemize}
\end{frame}

\begin{frame}[label={sec:orgheadline9}]{Genetics}
\begin{block}{Niemann-Pick A \& B}
\begin{itemize}
\item Mutations in SMPD1
\item Good phenotype-genotype correlation
\end{itemize}
\end{block}
\begin{block}{Niemann-Pick C}
\begin{itemize}
\item Autosomal recessive inheritance,
\item Mutations in NPC1 (95\%) and NPC2 (5\%)
\end{itemize}
\end{block}
\end{frame}



\section{Clinical Findings}
\label{sec:orgheadline14}

\begin{frame}[label={sec:orgheadline11}]{Niemann-Pick A \& B symptoms}
\begin{block}{Niemann-Pick A symptoms}
\begin{itemize}
\item hepatosplenomegaly by age 3 months
\item Failure to thrive
\item Psychomotor regression at age 1
\begin{itemize}
\item progressive loss of abilities – mental and physical
\end{itemize}
\item Interstitial lung disease resulting in lung infections and lung failure
\item Cherry-red spot identified with eye examination (100\%)
\end{itemize}
\end{block}

\begin{block}{Niemann-Pick B symptoms}
\begin{itemize}
\item Symptoms outlined under NPA (but less severe)
\item Thrombocytopenia
\item Short stature
\item Cherry-red spot identified with eye examination (50\%)
\end{itemize}
\end{block}
\end{frame}

\begin{frame}[label={sec:orgheadline12}]{Niemann-Pick C symptoms}
\begin{itemize}
\item onset of the disease can happen at any age.
\begin{itemize}
\item often school age children.
\item also adults
\end{itemize}

\item Symptoms may include:
\begin{itemize}
\item Jaundice at birth or shortly afterwards
\item Hepatosplenomegaly
\item Vertical supranuclear gaze palzy
\item Ataxia
\item Dystonia
\item Dysarthria
\item Cognitive dysfunction/dementia
\item Cataplexy
\item Tremors accompanying movement
\item Seizures
\item Dysphagia
\end{itemize}
\end{itemize}
\end{frame}

\begin{frame}[label={sec:orgheadline13}]{Niemann-Pick C neurological forms}
\small
\begin{itemize}
\item Classification by neurological form is widely used
\item correlation between age at neurological onset and course of disease
and lifespan has been established
\end{itemize}

\begin{columns}
\begin{column}{0.5\columnwidth}
\begin{block}{Early infantile}
\begin{itemize}
\item pre-existing hepatosplenomegaly
\item delay in motor milestones 9m-2yrs
\item survival <6 years
\end{itemize}
\end{block}

\begin{block}{Late-infantile}
\begin{itemize}
\item classic NPC, 60-70\% of cases
\item language delay
\item Ataxia, 3-5 yrs
\item Cognitive dysfunction, 6-12 yrs
\end{itemize}
\end{block}
\end{column}


\begin{column}{0.5\columnwidth}
\begin{block}{Adult}
\begin{itemize}
\item diagnosis 15->60yrs.
\item insidious presentation
\item ataxia, dystonia, dysarthria, movement disorders
\item variable cognitive dysfunction
\item Vertical gaze palzy common
\end{itemize}
\end{block}
\end{column}
\end{columns}
\end{frame}

\section{Laboratory Investigations}
\label{sec:orgheadline21}
\begin{frame}[label={sec:orgheadline15}]{Newborn Screening}
\begin{itemize}
\item New York state is conducting a pilot newborn screening program for four lysosomal storage disorders.
\item Pompe, Gaucher, Niemann-Pick A/B, Fabry, and MPS 1

\item 4 years, 65,605 infants participated, representing an overall consent rate of 73\%.
\begin{itemize}
\item Sixty-nine infants were screen-positive.
\item Twenty-three were confirmed true positives, all of whom were predicted to have late-onset phenotypes.
\item Six of the 69 currently have undetermined disease status.
\end{itemize}
\end{itemize}
\end{frame}

\begin{frame}[label={sec:orgheadline16}]{Biomarkers: oxysterols}
\begin{columns}
\begin{column}{0.5\columnwidth}
\begin{itemize}
\item Plasma oxysterols
\begin{itemize}
\item oxysterols cholestane-3\(\beta\), 5\(\alpha\), 6\(\beta\)-triol
\item 7-ketocholesterol
\end{itemize}
\end{itemize}
\end{column}



\begin{column}{0.5\columnwidth}
\begin{figure}[htb]
\centering
\includegraphics[width=0.7\textwidth]{./figures/biomarkers.jpg}
\caption{\label{fig:}
Klinke, G. Clin Biochem 2015}
\end{figure}
\end{column}
\end{columns}
\end{frame}




\begin{frame}[label={sec:orgheadline17}]{Biomarkers: lysosphingomylin}
\begin{columns}
\begin{column}{0.5\columnwidth}
\begin{itemize}
\item Plasma and DBS
\begin{itemize}
\item lysosphingomylin
\item lysosphingomylin-509
\end{itemize}
\end{itemize}
\end{column}



\begin{column}{0.5\columnwidth}
\begin{figure}[htb]
\centering
\includegraphics[width=0.8\textwidth]{./figures/biomarkersII.jpg}
\caption{\label{fig:}
Kuckar, L. Anal Biochem. 2017}
\end{figure}
\end{column}
\end{columns}
\end{frame}

\begin{frame}[label={sec:orgheadline18}]{Enzymology}
\begin{block}{Niemann-Pick A \& B}
\begin{itemize}
\item Deficient ASM activity in leukocytes or cultured cells.
\begin{itemize}
\item use of native or radio-labelled substrate preferred to fluorescent substrate
\begin{itemize}
\item 6-hexadecanoylamino-4-methylumbelliferyl-phosphorylcholine
\item Does not detect Q292K mutation
\end{itemize}
\end{itemize}
\end{itemize}
\end{block}
\end{frame}

\begin{frame}[label={sec:orgheadline19}]{Pathology: Niemann-Pick A \& B}
\begin{figure}[htb]
\centering
\includegraphics[width=0.45\textwidth]{./figures/foam_cells.png}
\caption{\label{fig:}
Foam cells in bone marrow}
\end{figure}
\end{frame}

\begin{frame}[label={sec:orgheadline20}]{Niemann-Pick C}
\begin{itemize}
\item Filipin test
\begin{itemize}
\item Streptomyces filipinensis - anti-fungal
\item culture fibroblasts in an LDL-enriched medium
\item pathognomonic free cholesterol accumulation in lysosomes
\item fluorescence microscopy after filipin staining
\item unequivocal results in \textasciitilde{} 85\% of patients
\end{itemize}
\end{itemize}

\begin{figure}[htb]
\centering
\includegraphics[width=0.5\textwidth]{./figures/filipin.png}
\caption{\label{fig:}
Filipin staining (red:filipin, green:CellMask)}
\end{figure}
\end{frame}

\section{Treatment}
\label{sec:orgheadline25}
\begin{frame}[label={sec:orgheadline22}]{Treatment: Niemann-Pick A\&B}
\begin{itemize}
\item No approved treatments
\item Olipudase alfa, a recombinant human acid sphingomyelinase (ASM), is
an enzyme replacement therapy for the treatment of nonneurologic
manifestations of acid sphingomyelinase deficiency (ASMD).
\item ongoing, open-label, long-term study (NCT02004704) assessed safety
and efficacy of olipudase alfa following 30 months of treatment in
five adult patients with ASMD.
\item There were no deaths, serious or severe events, or discontinuations
during 30 months of treatment.
\item Chitotriosidase in serum and lyso-sphingomyelin in dried blood spots
decreased with olipudase alfa treatment
\end{itemize}
\end{frame}

\begin{frame}[label={sec:orgheadline23}]{Treatment: Niemann-Pick C}
\begin{itemize}
\item substrate reduction therapy
\begin{itemize}
\item miglustat approved for treatment of neurological manifestations
\item miglustat is an iminosugar, a synthetic analogue of D-glucose
\end{itemize}
\end{itemize}
\end{frame}


\begin{frame}[label={sec:orgheadline24}]{Next time}
\begin{itemize}
\item Disorders of sphingolipid degradation continued\ldots{}
\begin{itemize}
\item Krabbe and Metachromatic Leukodystrophy
\end{itemize}
\end{itemize}
\end{frame}
\end{document}

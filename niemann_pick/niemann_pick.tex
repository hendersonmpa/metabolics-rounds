% Created 2019-01-08 Tue 12:58
\documentclass[presentation, smaller]{beamer}
\usepackage[utf8]{inputenc}
\usepackage[T1]{fontenc}
\usepackage{fixltx2e}
\usepackage{graphicx}
\usepackage{grffile}
\usepackage{longtable}
\usepackage{wrapfig}
\usepackage{rotating}
\usepackage[normalem]{ulem}
\usepackage{amsmath}
\usepackage{textcomp}
\usepackage{amssymb}
\usepackage{capt-of}
\usepackage{hyperref}
\hypersetup{colorlinks,linkcolor=white,urlcolor=blue}
\usepackage{textpos}
\usepackage{textgreek}
\usepackage[version=4]{mhchem}
\usepackage{chemfig}
\usepackage{siunitx}
\usepackage{gensymb}
\usepackage[usenames,dvipsnames]{xcolor}
\usepackage[T1]{fontenc}
\usepackage{lmodern}
\usepackage{verbatim}
\usepackage{tikz}
\usetikzlibrary{shapes.geometric,arrows,decorations.pathmorphing,backgrounds,positioning,fit,petri}
\usetheme{Ilmenau}
\usecolortheme{whale}
\author{Matthew Henderson, PhD, FCACB}
\date{\today}
\title{Sphingolipid Degradation: Niemann-Pick}
\institute[NSO]{Newborn Screening Ontario | The University of Ottawa}
\titlegraphic{\includegraphics[height=1cm,keepaspectratio]{../logos/NSO_logo.pdf}\includegraphics[height=1cm,keepaspectratio]{../logos/cheo-logo.png} \includegraphics[height=1cm,keepaspectratio]{../logos/UOlogoBW.eps}}
\hypersetup{
 pdfauthor={Matthew Henderson, PhD, FCACB},
 pdftitle={Sphingolipid Degradation: Niemann-Pick},
 pdfkeywords={},
 pdfsubject={},
 pdfcreator={Emacs 25.2.1 (Org mode 8.3.4)}, 
 pdflang={English}}
\begin{document}

\maketitle
%\logo{\includegraphics[width=1cm,height=1cm,keepaspectratio]{../logos/NSO_logo_small.pdf}~%
%    \includegraphics[width=1cm,height=1cm,keepaspectratio]{../logos/UOlogoBW.eps}%
%}

\vspace{220pt}
\beamertemplatenavigationsymbolsempty
\setbeamertemplate{caption}[numbered]
\setbeamerfont{caption}{size=\tiny}
% \addtobeamertemplate{frametitle}{}{%
% \begin{textblock*}{100mm}(.85\textwidth,-1cm)
% \includegraphics[height=1cm,width=2cm]{cat}
% \end{textblock*}}

\tikzstyle{chemical} = [rectangle, rounded corners, text width=5em, minimum height=1em,text centered, draw=black, fill=none]
\tikzstyle{hardware} = [rectangle, rounded corners, text width=5em, minimum height=1em,text centered, draw=black, fill=gray!30]
\tikzstyle{ms} = [rectangle, rounded corners, text width=5em, minimum height=1em,text centered, draw=orange, fill=none]
\tikzstyle{msw} = [rectangle, rounded corners, text width=7em, minimum height=1em,text centered, draw=orange, fill=none]
\tikzstyle{label} = [rectangle,text width=8em, minimum height=1em, text centered, draw=none, fill=none]
\tikzstyle{hl} = [rectangle, rounded corners, text width=5em, minimum height=1em,text centered, draw=black, fill=red!30]
\tikzstyle{box} = [rectangle, rounded corners, text width=5em, minimum height=5em,text centered, draw=black, fill=none]
\tikzstyle{arrow} = [thick,->,>=stealth]
\tikzstyle{hl-arrow} = [ultra thick,->,>=stealth,draw=red]

\section{Introduction}
\label{sec:orgheadline10}

\begin{frame}[label={sec:orgheadline1}]{Niemann-Pick Disease}
\begin{itemize}
\item There are three types of Niemann-Pick Disease:
\item Types A and B (ASMD or Acid Sphingomyelinase Deficiency)
\begin{itemize}
\item Type A and B (ASMD or Acid Sphingomyelinase Deficiency) are known as Type 1
\end{itemize}
\item Niemann-Pick Disease Type C (NPC).
\begin{itemize}
\item Type C (C1 \& C2) is known as Type 2
\end{itemize}
\end{itemize}
\end{frame}

\begin{frame}[label={sec:orgheadline2}]{Niemann-Pick A \& B}
\begin{itemize}
\item Niemann-Pick Type A (NPA) and Type B or (ASMD), is caused by the deficiency of acid sphingomyelinase (ASM).
\begin{itemize}
\item required to metabolize sphingomyelin.
\item progressive accumulation of sphingomyelin in systemic organs
\begin{itemize}
\item brain accumulation in neuronal forms
\end{itemize}
\item There is growing evidence that NPA \& NPB represent opposite ends of a continuum.
\begin{itemize}
\item NPA generally have little or no ASM production (less than 1\% of normal).
\item NPB have approximately 10\% of normal level of ASM.
\end{itemize}
\end{itemize}
\end{itemize}
\end{frame}

\begin{frame}[label={sec:orgheadline3}]{Sphingomyelinase}
\begin{columns}
\begin{column}{0.5\columnwidth}
\begin{figure}[htb]
\centering
\includegraphics[width=0.8\textwidth]{./figures/sphingomyelin.png}
\label{fig:}
\end{figure}
\end{column}

\begin{column}{0.5\columnwidth}
\begin{figure}[htb]
\centering
\includegraphics[width=0.8\textwidth]{./figures/sphingomyelinase.png}
\label{fig:}
\end{figure}
\end{column}
\end{columns}
\end{frame}


\begin{frame}[label={sec:orgheadline4}]{Niemann-Pick C}
\begin{itemize}
\item a fatal, neuro degenerative disease that affects 1 in 100,000
\begin{itemize}
\item sometimes referred to as Childhood Alzheimer’s
\item extremely heterogeneous
\end{itemize}
\item Accumulation of unesterified cholesterol, sphingomyelin, glycolipids in systemic organs
\item GM2 and GM3 accumulate in brain
\begin{itemize}
\item no increase in cholesterol
\end{itemize}
\item NPC has two sub types NP-C1 (95\%) and NP-C2 (5\%)
\end{itemize}
\end{frame}


\begin{frame}[label={sec:orgheadline5}]{Niemann-Pick C}
\begin{columns}
\begin{column}{0.5\columnwidth}
\begin{itemize}
\item LDL cholesterol enters cells via endocytosis at the LDL receptor.
\item delivered to the late-stage endosomes and lysosomes
\item hydrolyzed and released as free cholesterol.
\item Unesterified cholesterol is transported to the plasma membrane and the ER for recycling.

\item In NP-C, the LDL-cholesterol is trapped in lysosomes
\end{itemize}
\end{column}

\begin{column}{0.5\columnwidth}
\begin{figure}[htb]
\centering
\includegraphics[width=0.8\textwidth]{./figures/cholesterol1.jpg}
\label{fig:}
\end{figure}
\end{column}
\end{columns}
\end{frame}


\begin{frame}[label={sec:orgheadline6}]{NPC1 \& NPC2}
\begin{figure}[htb]
\centering
\includegraphics[width=0.8\textwidth]{./figures/Niemann-Pick-C-Brown-and-Goldstein.png}
\label{fig:}
\end{figure}
\end{frame}


\begin{frame}[label={sec:orgheadline7}]{Sphingolipid degradation}
\begin{figure}[htb]
\centering
\includegraphics[width=0.6\textwidth]{./figures/sl_degradation.png}
\caption[deg]{\label{fig:sld}
Sphingolipid degradation}
\end{figure}
\end{frame}


\begin{frame}[label={sec:orgheadline8}]{Lysosomal Protein Trafficking}
\begin{figure}[htb]
\centering
\includegraphics[width=0.8\textwidth]{./figures/lysosome_trafficking.jpeg}
\caption[traf]{\label{fig:traf}
Lysosomal protein trafficking receptors}
\end{figure}

\footnotesize
\begin{itemize}
\item \(\beta\)-galactosidase, hexoaminidase A and B require the M6P-receptor
\item GM2 activator protein - sortilin
\end{itemize}
\end{frame}

\begin{frame}[label={sec:orgheadline9}]{Genetics}
\begin{block}{Niemann-Pick A \& B}
\begin{itemize}
\item Mutations in SMPD1
\end{itemize}
\end{block}
\begin{block}{Niemann-Pick C}
\begin{itemize}
\item Autosomal recessive inheritance
\end{itemize}
\end{block}
\end{frame}

\section{Clinical Findings}
\label{sec:orgheadline14}

\begin{frame}[label={sec:orgheadline11}]{Niemann-Pick A \& B symptoms}
\begin{block}{Niemann-Pick A symptoms}
\begin{itemize}
\item hepatosplenomegaly by age 3 months
\item Failure to thrive
\item Psychomotor regression at age 1
\begin{itemize}
\item progressive loss of abilities – mental and physical
\end{itemize}
\item Interstitial lung disease resulting in lung infections and ultimate lung failure
\item Cherry-red spot identified with eye examination (all affected children)
\end{itemize}
\end{block}

\begin{block}{Niemann-Pick B symptoms}
\begin{itemize}
\item Symptoms outlined under NPA (but less severe)
\item Thrombocytopenia
\item pulmonary infiltration
\item Short stature
\item Cherry-red spot identified with eye examination (⅓ of affected children)
\end{itemize}
\end{block}
\end{frame}


\begin{frame}[label={sec:orgheadline12}]{Niemann-Pick C symptoms}
\begin{itemize}
\item onset of the disease can happen at any age.
\begin{itemize}
\item Often school age children.
\item also been found in adults
\end{itemize}

\item Symptoms May Include:
\begin{itemize}
\item Jaundice at Birth or Shortly Afterwards
\item Hepatosplenomegaly
\item Vertical Supranuclear Gaze Palzy
\item Ataxia
\item Dystonia
\item Dysarthria
\item Cognitive Dysfunction/Dementia
\item Cataplexy
\item Tremors Accompanying Movement
\item Seizures
\item Dysphagia
\end{itemize}
\end{itemize}
\end{frame}

\begin{frame}[label={sec:orgheadline13}]{Niemann-Pick C classification}
\begin{itemize}
\item Classification by neurological form is widely used
\item correlation between age at neurological onset and course of disease
and lifespan has been established
\end{itemize}

\begin{block}{Early infantile}
\begin{itemize}
\item pre-existing hepatosplenomegaly
\item delay in motor milestones 9 months - 2 years
\item survival < 6 years
\end{itemize}
\end{block}

\begin{block}{Late-infantile}
\begin{itemize}
\item classic NPC - 60-70\% of cases
\item language delay
\item Ataxia at 3-5 years
\item Cognitive dysfunction follows 6-12 years
\end{itemize}
\end{block}
\end{frame}
\end{document}

% Created 2018-10-18 Thu 13:00
\documentclass[presentation, smaller]{beamer}
\usepackage[utf8]{inputenc}
\usepackage[T1]{fontenc}
\usepackage{fixltx2e}
\usepackage{graphicx}
\usepackage{grffile}
\usepackage{longtable}
\usepackage{wrapfig}
\usepackage{rotating}
\usepackage[normalem]{ulem}
\usepackage{amsmath}
\usepackage{textcomp}
\usepackage{amssymb}
\usepackage{capt-of}
\usepackage{hyperref}
\hypersetup{colorlinks,linkcolor=white,urlcolor=blue}
\usepackage{textpos}
\usepackage{textgreek}
\usepackage[version=4]{mhchem}
\usepackage{chemfig}
\usepackage{siunitx}
\usepackage{gensymb}
\usepackage[usenames,dvipsnames]{xcolor}
\usepackage[T1]{fontenc}
\usepackage{lmodern}
\usepackage{verbatim}
\usepackage{tikz}
\usetikzlibrary{shapes.geometric,arrows,decorations.pathmorphing,backgrounds,positioning,fit,petri}
\usetheme{Ilmenau}
\usecolortheme{whale}
\author{Matthew Henderson, PhD, FCACB}
\date{\today}
\title{Disorders of Sphingolipid Synthesis}
\institute[NSO]{Newborn Screening Ontario | The University of Ottawa}
\titlegraphic{\includegraphics[height=1cm,keepaspectratio]{../logos/NSO_logo.pdf}\includegraphics[height=1cm,keepaspectratio]{../logos/cheo-logo.png} \includegraphics[height=1cm,keepaspectratio]{../logos/UOlogoBW.eps}}
\hypersetup{
 pdfauthor={Matthew Henderson, PhD, FCACB},
 pdftitle={Disorders of Sphingolipid Synthesis},
 pdfkeywords={},
 pdfsubject={},
 pdfcreator={Emacs 25.2.1 (Org mode 8.3.4)}, 
 pdflang={English}}
\begin{document}

\maketitle
%\logo{\includegraphics[width=1cm,height=1cm,keepaspectratio]{../logos/NSO_logo_small.pdf}~%
%    \includegraphics[width=1cm,height=1cm,keepaspectratio]{../logos/UOlogoBW.eps}%
%}

\vspace{220pt}
\beamertemplatenavigationsymbolsempty
\setbeamertemplate{caption}[numbered]
\setbeamerfont{caption}{size=\tiny}
% \addtobeamertemplate{frametitle}{}{%
% \begin{textblock*}{100mm}(.85\textwidth,-1cm)
% \includegraphics[height=1cm,width=2cm]{cat}
% \end{textblock*}}

\tikzstyle{chemical} = [rectangle, rounded corners, text width=5em, minimum height=1em,text centered, draw=black, fill=none]
\tikzstyle{hardware} = [rectangle, rounded corners, text width=5em, minimum height=1em,text centered, draw=black, fill=gray!30]
\tikzstyle{ms} = [rectangle, rounded corners, text width=5em, minimum height=1em,text centered, draw=orange, fill=none]
\tikzstyle{msw} = [rectangle, rounded corners, text width=7em, minimum height=1em,text centered, draw=orange, fill=none]
\tikzstyle{label} = [rectangle,text width=8em, minimum height=1em, text centered, draw=none, fill=none]
\tikzstyle{hl} = [rectangle, rounded corners, text width=5em, minimum height=1em,text centered, draw=black, fill=red!30]
\tikzstyle{box} = [rectangle, rounded corners, text width=5em, minimum height=5em,text centered, draw=black, fill=none]
\tikzstyle{arrow} = [thick,->,>=stealth]
\tikzstyle{hl-arrow} = [ultra thick,->,>=stealth,draw=red]


\section{Introduction}
\label{sec:orgheadline6}
\begin{frame}[label={sec:orgheadline1}]{Sphingolipids}
\begin{itemize}
\item Found in all mammalian cell membranes
\item Plasma lipoproteins
\item Structural role
\item Modulate numerous biological functions
\begin{itemize}
\item apoptosis
\end{itemize}
\item named after the Sphinx because of their enigmatic nature\ldots{}
\end{itemize}
\end{frame}

\begin{frame}[label={sec:orgheadline2}]{Sphingosine and Ceramide}
\begin{itemize}
\item long chain sphingolipid base
\item N-acylated by a variety of fatty acids
\end{itemize}
\centering
\definesubmol{x}{-[7,.3]-[1,.3]}
\definesubmol{y}{-[:+30,.3]=[:-30,.3]}
\definesubmol{a}{-[1,.3](=[2,.3]O)!x!x!x!x!x!x!x!x!x!x!x}
\chemfig{OH!x([2,.5]<HN)-[7,.3](-[6,.3]OH)-[1,.3]=[7,.3]-[1,.3]!x!x!x!x!x!x}
\chemfig{OH!x([2,.5]<HN!a)-[7,.3](-[6,.3]OH)-[1,.3]=[7,.3]-[1,.3]!x!x!x!x!x!x}
%%\chemfig{!b}
\end{frame}



\begin{frame}[label={sec:orgheadline3}]{Biosynthesis}
\begin{block}{ER}
\begin{itemize}
\item condensation of serine and palmitoyl-CoA \(\to\) sphinganine
\item N-acylation \(\to\) ceramide
\end{itemize}
\end{block}

\begin{block}{Golgi}
\begin{itemize}
\item stepwise addition of monosaccharides
\begin{itemize}
\item sphingomyelin
\item glucosylceramide
\item glycosphingolipids
\item gangliosides
\end{itemize}
\end{itemize}
\end{block}
\end{frame}


\begin{frame}[label={sec:orgheadline4}]{Sphingolipid Structure}
\includegraphics[width=\textwidth]{./figures/Sphingolipids_general_structures.png}
\end{frame}

\begin{frame}[label={sec:orgheadline5}]{Biosynthesis}
\centering
\includegraphics[width=0.6\textwidth]{./figures/synthesis.png}
\end{frame}


\section{Disorders of Sphingolipid Synthesis}
\label{sec:orgheadline16}

\begin{frame}[label={sec:orgheadline7}]{Serine Palmitoyltransferase (Subunit 1 or 2) Deficiency}
\begin{itemize}
\item Defect in first step of sphingolipid biosynthesis
\item Major cause of dominant Hereditary Sensory and Autonomic Neuropathies (HSAN1).
\begin{itemize}
\item Late onset (2-4th decade)
\item peripheral sensory neuropathy
\item distal sensory loss
\item ulcerative mutilations
\item hypohydrosis
\item there is a more severe early onset form
\end{itemize}
\item Accumulation of sphingoid bases \(\to\) pathology
\item mutations in serine palmitoyltransferase (SPCTLC1 or 2) alter
substrate specificity
\begin{itemize}
\item serine \(\to\) alanine and glycine
\end{itemize}
\item Elevated plasma 1-deoxy-sphingamine, 1-deoxy-methyl-sphingamine, 1-deoxy-ceramindes
\item Trial of serine supplementation
\end{itemize}
\end{frame}

\begin{frame}[label={sec:orgheadline8}]{Ceramide Synthases 1 and 2}
\begin{itemize}
\item Six human ceramide synthases
\begin{itemize}
\item tissue and acyl-CoA substrate specificity
\item Neurological CERS1 \&2
\begin{description}
\item[{CER1}] Myoclonic epilepsy, cognitive decline
\begin{itemize}
\item Decreased C18-ceramide in cultured fibroblasts
\end{itemize}
\item[{CER2}] Myoclonic epilepsy
\begin{itemize}
\item Decreased VLC-ceramide in cultured fibroblasts
\end{itemize}
\end{description}
\item Dermatologic CERS3
\begin{description}
\item[{CER3}] Ichthyosis
\begin{itemize}
\item Lack of VLC-ceramides in skin and fibroblasts
\end{itemize}
\end{description}
\end{itemize}
\end{itemize}
\end{frame}


\begin{frame}[label={sec:orgheadline9}]{Fatty Acid 2-Hydroxylase}
\begin{itemize}
\item spastic paraplegia
\begin{itemize}
\item fatty acid hydroxylase associated neurodegeneration (FAHN)
\end{itemize}
\item 38 patients, most in present in childhood
\item slowly progressing
\begin{itemize}
\item spastic paraplegia
\item dysarthria
\item mild cognitive decline
\item dystonia
\end{itemize}

\item Insufficiency production of 2-hydroxy-galactosphingolipids
\begin{itemize}
\item required in myelin
\item increase with brain development
\end{itemize}

\item Decreased hydroxylated sphingomyelin in cultured cells
\end{itemize}
\end{frame}

\begin{frame}[label={sec:orgheadline10}]{GM3 Synthase Deficiency}
\begin{itemize}
\item Autosomal recessive infantile-onset epilepsy
\begin{itemize}
\item Amish epilepsy syndrome
\end{itemize}
\item In first year \(\to\) generalized tonic-clonic seizures
\begin{itemize}
\item profound developmental stagnation and regression
\item salt and pepper syndrome
\begin{itemize}
\item hyper and hypo-pigmented skin maculae
\item facial dysmorphism scoliosis
\item intellectual disability
\item seizures
\item choreoathetosis
\item spasticity
\end{itemize}
\end{itemize}

\item lack of GM3, GD3 and higher gangliosides, and increased
lactosylceramide and Gb4 levels in plasma and cultured cells
\end{itemize}
\end{frame}


\begin{frame}[label={sec:orgheadline11}]{GM2/GD2 Synthase Deficiency}
\begin{itemize}
\item Mutations of B4GALNT1
\item SPG26, a slowly progressive complex hereditary spastic paraplegia
with mild to moderate cognitive impairment.

\item Cultured fibroblasts of patients have shown decreased GM2 levels
with an increase of its precursor, GM3.
\end{itemize}
\end{frame}

\begin{frame}[label={sec:orgheadline12}]{Non-lysosomal β-Glucosidase Deficiency}
\begin{itemize}
\item GBA2 is a membrane-associated protein localised at the ER and Golgi
\begin{itemize}
\item hydrolyse glucosylceramide to ceramide and glucose.
\end{itemize}
\item GBA2 is distinct from the lysosomal acid \(\beta\)-glucosidase GBA1 deficient in Gaucher disease
\item hereditary (complex) spastic paraplegia locus SPG46.
\item Starting in childhood marked spasticity in lower extremities with
progressive gait disturbances
\begin{itemize}
\item later, ataxia and other cerebellar signs
\end{itemize}
\end{itemize}
\end{frame}

\begin{frame}[label={sec:orgheadline13}]{Ceramide Synthase 3 and ULFA \(\omega\)-Hydroxylase}
\begin{itemize}
\item ceramides in skin maintain skin barrier homeostasis, prevent water
loss and protect against microbial infections
\item Autosomal recessive congenital ichthyosis (ARCI) is a heterogeneous
group of disorders of epidermal cornification
\item 9 causative genes have been identified including CERS3 and CYP4F22

\item[{CERS3}] ichthyosis
\begin{itemize}
\item lack of ceramides with VLCFA in cultured fibroblasts
\end{itemize}
\item[{CYP4F22}] ichthyosis
\begin{itemize}
\item lack of ceramides with ULCFA in cultured fibroblasts
\end{itemize}
\end{itemize}
\end{frame}

\begin{frame}[label={sec:orgheadline14}]{Classification}
\begin{block}{Primarily nervous system involvement}
\begin{itemize}
\item Serine palmitoyltransferase - peripheral sensory neuropathy
\item Ceramide synthase 1 - myoclonic epilepsy
\item Ceramide synthase 2 - myoclonic epilepsy
\item Fatty Acid 2-hydroxylase - SPG35
\item Nonlysosomal β-Glucosidase - SPG46
\item GM3 Synthase Deficiency - Amish infantile epilepsy
\item GM2/GD2 Synthase Deficiency - SPG26
\end{itemize}
\end{block}


\begin{block}{Primarily skin involvement}
\begin{itemize}
\item Ceramide synthase 3 - Ichthyosis
\item ULCFA \(\omega\)-hydrolase - Ichthyosis
\end{itemize}
\end{block}
\end{frame}

\begin{frame}[label={sec:orgheadline15}]{Next time}
\begin{itemize}
\item Back to LSDs with Sphingolipidoses
\end{itemize}
\end{frame}
\end{document}

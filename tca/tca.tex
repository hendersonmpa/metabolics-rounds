% Created 2019-03-21 Thu 12:44
% Intended LaTeX compiler: pdflatex
\documentclass[presentation, smaller]{beamer}
\usepackage[utf8]{inputenc}
\usepackage[T1]{fontenc}
\usepackage{graphicx}
\usepackage{grffile}
\usepackage{longtable}
\usepackage{wrapfig}
\usepackage{rotating}
\usepackage[normalem]{ulem}
\usepackage{amsmath}
\usepackage{textcomp}
\usepackage{amssymb}
\usepackage{capt-of}
\usepackage{hyperref}
\hypersetup{colorlinks,linkcolor=white,urlcolor=blue}
\usepackage{textpos}
\usepackage{textgreek}
\usepackage[version=4]{mhchem}
\usepackage{chemfig}
\usepackage{siunitx}
\usepackage{gensymb}
\usepackage[usenames,dvipsnames]{xcolor}
\usepackage[T1]{fontenc}
\usepackage{lmodern}
\usepackage{verbatim}
\usepackage{tikz}
\usetikzlibrary{shapes.geometric,arrows,decorations.pathmorphing,backgrounds,positioning,fit,petri}
\usetheme{Hannover}
\usecolortheme{whale}
\author{Matthew Henderson, PhD, FCACB}
\date{\today}
\title{Disorders of the Tricarboxylic Acid Cycle}
\institute[NSO]{Newborn Screening Ontario | The University of Ottawa}
\titlegraphic{\includegraphics[height=1cm,keepaspectratio]{../logos/NSO_logo.pdf}\includegraphics[height=1cm,keepaspectratio]{../logos/cheo-logo.png} \includegraphics[height=1cm,keepaspectratio]{../logos/UOlogoBW.eps}}
\hypersetup{
 pdfauthor={Matthew Henderson, PhD, FCACB},
 pdftitle={Disorders of the Tricarboxylic Acid Cycle},
 pdfkeywords={},
 pdfsubject={},
 pdfcreator={Emacs 26.1 (Org mode 9.1.9)}, 
 pdflang={English}}
\begin{document}

\maketitle

%\logo{\includegraphics[width=1cm,height=1cm,keepaspectratio]{../logos/NSO_logo_small.pdf}~%
%    \includegraphics[width=1cm,height=1cm,keepaspectratio]{../logos/UOlogoBW.eps}%
%}

\vspace{220pt}
\beamertemplatenavigationsymbolsempty
\setbeamertemplate{caption}[numbered]
\setbeamerfont{caption}{size=\tiny}
% \addtobeamertemplate{frametitle}{}{%
% \begin{textblock*}{100mm}(.85\textwidth,-1cm)
% \includegraphics[height=1cm,width=2cm]{cat}
% \end{textblock*}}

\section{Introduction}
\label{sec:orgafc44ed}

\begin{frame}[label={sec:orgdaa0a47}]{The Tricarboxylic Acid Cycle}
\begin{itemize}
\item Pathways for oxidation of fatty acids, glucose, amino acids and ketones produce acetyl-CoA
\end{itemize}
%%\setchemfig{lewis style=red}
\centering
\chemfig{\lewis{0.,H}-\lewis{0.2.4.6.,{\color{red}C}}(-[6]\lewis{2.,H})(-[2]\lewis{6.,H})-\lewis{4.,{\color{red}C}}(=[2]O)-[,,,,decorate, decoration=snake]SCoA}
\begin{itemize}
\item Part of aerobic respiration - where is the \ce{O2}?
\begin{itemize}
\item ETC regenerates \ce{NAD+} from NADH
\end{itemize}
\item Cofactors:
\begin{itemize}
\item niacin (\ce{NAD+})
\item riboflavin (FAD and FMN)
\item panthothenic acid (CoA)
\item thiamine
\item \ce{Mg^2+}, \ce{Ca^2+}, \ce{Fe+} and phosphate
\end{itemize}
\end{itemize}
\end{frame}


\begin{frame}[label={sec:org8dd6560}]{The Tricarboxylic Acid Cycle}
\begin{center}
\includegraphics[width=.9\textwidth]{./figures/TCACycle.png}
\end{center}

\centering
\tiny
\ce{AcetylCoA + 3NAD+ + FAD + GDP + Pi + 2H2O -> 2CO2 + CoA + 3NADH + FADH2 + GTP + 2H+}
\end{frame}

\begin{frame}[label={sec:orgdf7a804}]{Disorders of the TCA cycle}
\begin{itemize}
\item \(\alpha\)-Ketoglutarate Dehydrogenase Complex Deficiency
\item Succinate Dehydrogenase Deficiency
\item Fumarase Deficiency
\end{itemize}
\end{frame}

\begin{frame}[label={sec:org1478622}]{Disorders of the TCA cycle}
\begin{center}
\includegraphics[width=\textwidth]{./figures/TCA_disorders.png}
\end{center}
\end{frame}


\begin{frame}[label={sec:org9564000}]{Disorders of the TCA cycle}
\begin{figure}[htbp]
\centering
\includegraphics[width=0.7\textwidth]{./figures/gr2.png}
\caption{Model for a functional splitting of the Krebs cycle reactions into complementary mini-cycles.}
\end{figure}

\begin{itemize}
\item Rustin, P., Bourgeron, T., Parfait, B., Chretien, D., Munnich, A., \&
Rotig, A. (1997). Inborn errors of the Krebs cycle : a group of
unusual mitochondrial diseases in human, 185-197.
\end{itemize}
\end{frame}

\section{\(\alpha\)-ketoglutarate Dehydrogenase Complex Deficiency}
\label{sec:orgb116d04}
\begin{frame}[label={sec:org855b4c6}]{\(\alpha\)-ketoglutarate Dehydrogenase Complex Deficiency}
\begin{center}
\includegraphics[width=0.9\textwidth]{./figures/kgdh.png}
\end{center}
\end{frame}

\begin{frame}[label={sec:orge6c1b13}]{\(\alpha\)-ketoglutarate Dehydrogenase Complex}
\begin{itemize}
\item KDHC is a \(\alpha\)-ketoacid dehydrogenase analogous to PDHC and BCKD.
\end{itemize}

\ce{\alpha-ketoglutarate + NAD+ + CoA ->[KDHC] Succinyl CoA + CO2 + NADH}


\begin{center}
\begin{tabular}{llll}
Unit & Name & Gene & Cofactor\\
\hline
E1 & \(\alpha\)-ketoglutarate dehydrogenase & OGDH & thiamine pyrophosphate(TPP)\\
E2 & dihydrolipoyl succinyltransferase & DLST & lipoic acid, Coenzyme A\\
E3 & dihydrolipoyl dehydrogenase & DLD & FAD, NAD\\
\end{tabular}
\end{center}
\end{frame}

\begin{frame}[label={sec:org357bfd7}]{Clinical Presentation}
\begin{itemize}
\item Similar to PDHC
\item Developmental delay, hypotonia, opisthotonos and ataxia
\begin{itemize}
\item seizures less common
\end{itemize}
\item Present as neonate and early childhood
\end{itemize}
\end{frame}

\begin{frame}[label={sec:org8fd7069}]{Genetics}
\begin{itemize}
\item AR inheritence, encoded by nuclear DNA
\item E1 gene mapped to 7p13
\item E2 gene mapped to 14q24.3
\item Molecular basis of KDHC deficiencies is not resolved.

\item \(\alpha\)-ketoglutarate dehydrogenase deficiency is sometimes a feature of DLD deficiency
\end{itemize}
\end{frame}

\begin{frame}[label={sec:org297294f}]{Diagnostic Tests}
\begin{itemize}
\item Urine organic acids
\begin{itemize}
\item \(\uparrow\) \(\alpha\)-KGA, \textpm{} other TCA intermediates
\item \(\alpha\)-KGA is a common finding, not specific for KDHC deficiency
\end{itemize}
\item Blood lactate
\begin{itemize}
\item Normal or increased L/P
\end{itemize}
\item KDHC activity
\begin{itemize}
\item \ce{^14CO2} release from \ce{[1-^14C]} \(\alpha\)-ketoglutarate (or \ce{[1-^14C]} leucine)
\item cultured skin fibroblasts
\item muscle
\end{itemize}
\end{itemize}
\end{frame}

\begin{frame}[label={sec:orgf206480}]{Treatment}
\begin{itemize}
\item None to date
\end{itemize}
\end{frame}


\section{Succinate Dehydrogenase Deficiency}
\label{sec:orge269c62}

\begin{frame}[label={sec:org998913b}]{Succinate Dehydrogenase Deficiency}
\begin{center}
\includegraphics[width=0.9\textwidth]{./figures/sdh.png}
\end{center}
\end{frame}

\begin{frame}[label={sec:org292cbfb}]{Succinate Dehydrogenase | Complex II}
\begin{itemize}
\item Four subunits compose Complex II of the mitochondrial respiratory chain
\end{itemize}

\begin{center}
\begin{tabular}{ll}
Subunit name & Protein description\\
\hline
SdhA & Succinate dehydrogenase flavoprotein subunit\\
SdhB & Succinate dehydrogenase iron-sulfur subunit\\
SdhC & Succinate dehydrogenase cytochrome b560 subunit\\
SdhD & Succinate dehydrogenase cytochrome b small subunit\\
\end{tabular}
\end{center}

\begin{itemize}
\item The SdhA subunit contains an FAD binding site where succinate
is deprotonated and converted to fumarate.
\end{itemize}

succinate + ubiquinone \(\to\) fumarate + ubiquinol

\begin{itemize}
\item Electrons removed from succinate transfer to SdhA
\item transfer across SdhB through iron sulphur clusters to the SdhC/SdhD subunits
\begin{itemize}
\item SdhC/SdhD are anchored in the mitochondrial membrane.
\end{itemize}
\end{itemize}
\end{frame}

\begin{frame}[label={sec:org902c853}]{Succinate Dehydrogenase | Complex II}
\begin{center}
\includegraphics[width=0.9\textwidth]{./figures/SuccDeh.png}
\end{center}
\end{frame}


\begin{frame}[label={sec:org6f9c5e8}]{Clinical Presentation}
\begin{itemize}
\item Very rare disorder with highly variable phenotype
\item Complex II is part of the TCA cycle and ETC
\begin{itemize}
\item phenotype resembles defects in respiratory chain
\end{itemize}
\item Clinical picture can include:
\begin{itemize}
\item Kearns-Sayre syndrome
\item isolated hypertrophic cardiomyopathy
\item combined cardiac and skeletal myopathy
\item generalized muscle weakness, \(\uparrow\) fatiguability
\item early onset Leigh encephalopathy
\end{itemize}
\item Also:
\begin{itemize}
\item cerebral ataxia
\item optic atropy
\item tumour formation in adults
\end{itemize}
\end{itemize}
\end{frame}

\begin{frame}[label={sec:org907ba88}]{Genetics}
\begin{itemize}
\item All components of Complex II are encoded by nuclear DNA.
\end{itemize}

\begin{center}
\begin{tabular}{ll}
Gene & Location\\
\hline
SDHA & 5p15.33\\
SDHB & 1p36.13\\
SDHC & 1q23.3\\
SDHD & 11q23.1\\
\end{tabular}
\end{center}


\begin{itemize}
\item AR with highly variable phenotype
\item Case of affected sisters with one identified SDHA mutation suggested
dominant transmission
\item Mutations in SDHB, SDHC and SDHD cause susceptibility to familial
phaeochromocytoma and familial paraganglioma.
\end{itemize}
\end{frame}

\begin{frame}[label={sec:org5e0f7b5}]{Diagnostic Tests}
\begin{itemize}
\item Unlike other TCA cycle disorders Complex II deficiency does not always
result in characteristic organic aciduria
\begin{itemize}
\item succinic aciduria.
\end{itemize}
\item Organic acids can show variable amounts of lactate, pyruvate, succinate, fumarate and malate
\end{itemize}
\end{frame}

\begin{frame}[label={sec:orgcad0c27}]{Diagnostic Tests}
\begin{itemize}
\item Measurement of complex II activity in muscle is the most reliable
means of diagnosis
\begin{itemize}
\item there is no clear correlation between residual complex II activity
and severity or clinical outcome.
\end{itemize}
\end{itemize}

\begin{figure}[htbp]
\centering
\includegraphics[width=0.9\textwidth]{./figures/gr4.jpg}
\caption{\label{fig:org8cf1825}
Coupled spectrophotometric assay}
\end{figure}
\end{frame}

\begin{frame}[label={sec:orga920634}]{Treatment}
\begin{itemize}
\item In some cases, treatment with riboflavin may have clinical benefit
\end{itemize}
\end{frame}

\section{Fumarase Deficiency}
\label{sec:org2a7bdde}
\begin{frame}[label={sec:orgcd5b881}]{Fumarase Deficiency}
\begin{center}
\includegraphics[width=0.9\textwidth]{./figures/fumarase.png}
\end{center}
\end{frame}

\begin{frame}[label={sec:org448f351}]{Fumarase}
\begin{itemize}
\item Fumarase catalyses reversible hydration/dehydration of fumarate to malate
\item Two forms: mitochondrial and cytosolic.
\begin{itemize}
\item The mitochondrial isoenzyme is involved in the TCA Cycle
\item The cytosolic isoenzyme is involved in the metabolism of amino acids and fumarate.
\end{itemize}
\item Subcellular localization is established by the presence/absence of an N-terminal mitochondrial signal
sequence
\item Deficiency causes impaired energy production
\end{itemize}
\end{frame}

\begin{frame}[label={sec:orgfe84ce4}]{Clinical Presentation}
\begin{itemize}
\item Characterized by polyhydramnios and fetal brain abnormalities.
\item In the newborn period, findings include:
\begin{itemize}
\item severe neurologic abnormalities,
\item poor feeding,
\item failure to thrive
\item hypotonia.
\end{itemize}

\item Fumarase deficiency is suspected in infants with multiple severe
neurologic abnormalities in the absence of an acute metabolic
crisis.

\item Inactivity of both cytosolic and mitochondrial forms of
fumarase are potential causes.
\end{itemize}
\end{frame}

\begin{frame}[label={sec:org4fd43d2}]{Genetics}
\begin{itemize}
\item AR inheritance, encoded by nuclear DNA
\item Single gene and mRNA encode mito and cyto isoforms
\end{itemize}
\end{frame}

\begin{frame}[label={sec:orgb8c361a}]{Diagnostic Tests}
\begin{itemize}
\item Isolated, increased concentration of fumaric acid on urine organic
acid analysis is highly suggestive of fumarase deficiency.
\begin{itemize}
\item Succinate, \(\alpha\)-KGA can also be elevated
\end{itemize}
\item Molecular genetic testing for fumarase deficiency is currently available
\end{itemize}
\end{frame}
\end{document}
% Created 2017-11-16 Thu 12:54
\documentclass[presentation, smaller]{beamer}
\usepackage[utf8]{inputenc}
\usepackage[T1]{fontenc}
\usepackage{fixltx2e}
\usepackage{graphicx}
\usepackage{grffile}
\usepackage{longtable}
\usepackage{wrapfig}
\usepackage{rotating}
\usepackage[normalem]{ulem}
\usepackage{amsmath}
\usepackage{textcomp}
\usepackage{amssymb}
\usepackage{capt-of}
\usepackage{hyperref}
\hypersetup{colorlinks,linkcolor=white,urlcolor=blue}
\usepackage{textpos}
\usepackage{textgreek}
\usepackage[version=4]{mhchem}
\usepackage{chemfig}
\usepackage{siunitx}
\usepackage[usenames,dvipsnames]{xcolor}
\usepackage[T1]{fontenc}
\usepackage{lmodern}
\usepackage{verbatim}
\usepackage{tikz}
\usetikzlibrary{shapes.geometric,arrows,decorations.pathmorphing,backgrounds,positioning,fit,petri}
\usetheme{Frankfurt}
\usecolortheme{rose}
\useinnertheme{circles}
\author{Matthew Henderson, PhD, FCACB}
\date{\today}
\title{Fatty Acid Oxidation}
\institute[NSO]{Newborn Screening Ontario | The University of Ottawa}
\titlegraphic{\includegraphics[height=1cm,keepaspectratio]{../logos/NSO_logo.pdf}\includegraphics[height=1cm,keepaspectratio]{../logos/cheo-logo.png} \includegraphics[height=1cm,keepaspectratio]{../logos/UOlogoBW.eps}}
\hypersetup{
 pdfauthor={Matthew Henderson, PhD, FCACB},
 pdftitle={Fatty Acid Oxidation},
 pdfkeywords={},
 pdfsubject={},
 pdfcreator={Emacs 25.2.1 (Org mode 8.3.4)}, 
 pdflang={English}}
\begin{document}

\maketitle
%\logo{\includegraphics[width=1cm,height=1cm,keepaspectratio]{../logos/NSO_logo_small.pdf}~%
%    \includegraphics[width=1cm,height=1cm,keepaspectratio]{../logos/UOlogoBW.eps}%
%}

\vspace{220pt}
\beamertemplatenavigationsymbolsempty
\setbeamertemplate{caption}[numbered]
\setbeamerfont{caption}{size=\tiny}
% \addtobeamertemplate{frametitle}{}{%
% \begin{textblock*}{100mm}(.85\textwidth,-1cm)
% \includegraphics[height=1cm,width=2cm]{cat}
% \end{textblock*}}


\tikzstyle{chemical} = [rectangle, rounded corners, text width=5em, minimum height=1em,text centered, draw=black, fill=none]
\tikzstyle{hardware} = [rectangle, rounded corners, text width=5em, minimum height=1em,text centered, draw=black, fill=gray!30]
\tikzstyle{ms} = [rectangle, rounded corners, text width=5em, minimum height=1em,text centered, draw=orange, fill=none]
\tikzstyle{msw} = [rectangle, rounded corners, text width=7em, minimum height=1em,text centered, draw=orange, fill=none]
\tikzstyle{label} = [rectangle,text width=5em, minimum height=1em, text centered, draw=none, fill=none]
\tikzstyle{hl} = [rectangle, rounded corners, text width=5em, minimum height=1em,text centered, draw=black, fill=red!30]
\tikzstyle{arrow} = [thick,->,>=stealth]
\tikzstyle{hl-arrow} = [ultra thick,->,>=stealth,draw=red]


\section{Introduction}
\label{sec:orgheadline5}
\begin{frame}[label={sec:orgheadline1}]{Fatty Acids}
\begin{itemize}
\item usually straight aliphatic chains with a methyl group at one end
(\(\omega\)-carbon) and a carboxyl group and the other end.
\end{itemize}

\definesubmol{x}{-[1,.6]-[7,.6]}
\definesubmol{a}{-[1,.6]\beta{}-[7,.6]\alpha{}}
\definesubmol{y}{!x!x!x!x!x!x!x!x}
\definesubmol{b}{!x!x!x!x!x!x!x!a}
%\chemfig{H{_3}C!y-[1]C(=[1]O)-[7]O{^-}}
\chemname{\chemfig{\omega{}!b-[1]C(=[1]O)-[7]O{^-}}}{\small stearic acid 18:0}
\end{frame}

\begin{frame}[label={sec:orgheadline2}]{Fatty Acid Nomenclature}
\begin{itemize}
\item Non-systematic historical names most commonly used.
\begin{description}
\item[{Palmitic acid}] discovered in palm oil
\item[{Stearic acid}] from the Greek word "stear", which means tallow.
\item[{Oleic acid}] oleic means related to, or derived from, olive oil
\end{description}
\item The position of a double bond is designated by the number of the carbon in the double bond that is closest to the carboxyl group
\end{itemize}


\definesubmol{x}{-[1,.6]-[7,.6]}
\definesubmol{y}{-[7,.6]-[1,.6]}
\definesubmol{d}{=[0,.6](-[7,0.25,,,draw=none]\scriptstyle\color{red}9)-[1,.6]}
\definesubmol{e}{!x!x!x!x!d!y!y!y}
\chemname{\chemfig{\omega{}(-[3,0.25,,,draw=none]\scriptstyle\color{red}18)!e(-[2,0.25,,,draw=none]\scriptstyle\color{red}2)-[7,.6]COOH}}{\small Oleic acid 18:1,\Delta{}$^9$}

\begin{itemize}
\item 18 carbons
\item 1 double bond
\item \(\Delta^{\text{9}}\), double bond between 9th and 10th carbon.
\item Also 18:1(9)
\item Distance from \(\omega\) methyl group, \(\omega\)-9
\end{itemize}
\end{frame}

\begin{frame}[label={sec:orgheadline3}]{Fatty Acid Chain Length}
\begin{description}
\item[{Very long-chain}] > C20
\item[{Long-chain}] C12-C20
\item[{Medium-chain}] C6-C12
\item[{Short-chain}] C4
\end{description}
\end{frame}

\begin{frame}[label={sec:orgheadline4}]{Fatty Acids as an Energy Source}
\begin{itemize}
\item Long chain fatty acids released from adipose tissue triacylglycerol
stores during periods of increased fuel demand or fasting.
\item \(\downarrow\) insulin, \(\uparrow\) glucagon \(\to\) \(\uparrow\) lipolysis
\begin{itemize}
\item dietary lipids
\item triacylglycerols synthesis in the liver
\item palmitate (C16:0), oleate (C18:1, \(\Delta ^9\)) and stearate (C18:0)
\end{itemize}
\item FA transported to tissue bound to albumin
\item Energy derived from oxidation of FA to acetyl-CoA in B-oxidation.
\item The acetyl-CoA is oxidized in the TCA cycle or converted to ketone bodies in the liver.
\end{itemize}
\end{frame}

\section{B-oxidation}
\label{sec:orgheadline15}
\begin{frame}[label={sec:orgheadline6}]{Overview of Mitochondrial Long-Chain Fatty Acid Metabolism}
\begin{columns}
\begin{column}{0.5\columnwidth}
\centering
\includegraphics[height=0.8\textheight]{./figures/23_1.png}
\end{column}

\begin{column}{0.5\columnwidth}
\begin{enumerate}
\item Membrane transport
\begin{itemize}
\item FaBP
\end{itemize}
\item Activation
\begin{itemize}
\item Coenzyme A
\end{itemize}
\item Carnitine
\item B-oxidation
\item Ketone bodies
\end{enumerate}
\end{column}
\end{columns}
\end{frame}

\begin{frame}[label={sec:orgheadline7}]{Activation of Fatty Acids}
\includegraphics[width=.9\linewidth]{./figures/23_2.png}
\end{frame}

\begin{frame}[label={sec:orgheadline8}]{Transport of Long-Chain Fatty Acids into Mitochondria}
\begin{columns}
\begin{column}{0.6\columnwidth}
\centering
\includegraphics[height=0.6\textheight]{./figures/23_5.png}
\end{column}

\begin{column}{0.4\columnwidth}
\begin{itemize}
\item chain length specificity
\item Organ and organelle specificity
\begin{itemize}
\item Acyl-CoA synthetases
\item Acyltransferases
\item Acyl-CoA dehydrogenases
\end{itemize}
\end{itemize}
\end{column}
\end{columns}
\end{frame}


\begin{frame}[label={sec:orgheadline9}]{Chain Length Specificity}
\begin{columns}
\begin{column}{.5\columnwidth}
\begin{block}{Acyl-CoA Synthetases}
\scriptsize
\begin{center}
\begin{tabular}{lrl}
Enzyme & Length & location\\
\hline
V.L. chain & 14-26 & pex\\
L. chain & 12-20 & ER, mito, pex\\
M. chain & 6-12 & mito - kidney, liver\\
acetyl & 2-4 & cyto, ?mito?\\
\end{tabular}
\end{center}
\end{block}
\end{column}


\begin{column}{.5\columnwidth}
\begin{block}{Acyl-CoA Dehydrogenases}
\scriptsize
\begin{center}
\begin{tabular}{lrl}
Enzyme & Length & location\\
\hline
VLCAD & 14-20 & IMM\\
LCAD & 12-18 & MM\\
MCAD & 4-12 & MM\\
SCAD & 2-4 & MM\\
\end{tabular}
\end{center}
\end{block}
\end{column}
\end{columns}

\begin{block}{Other}
\scriptsize
\begin{center}
\begin{tabular}{lrl}
Enzyme & Length & comment\\
\hline
Enoyl-CoA hydratase,SC & >4 & \(\downarrow\) activity w \(\uparrow\) length\\
Hydroxyacyl-CoA dehydrogenase, SC & 4-16 & \(\downarrow\) activity w \(\uparrow\) length\\
Acetoacetyl-CoA thiolase & 4 & Acetoacetyl-CoA specific\\
Trifunctional protein & 12-16 & \(\uparrow\) activity w \(\uparrow\) length\\
\end{tabular}
\end{center}
\end{block}
\end{frame}


\begin{frame}[label={sec:orgheadline10}]{B-oxidation of Long-Chain Fatty Acids}
\centering
\includegraphics[height=0.85\textheight]{./figures/23_7.png}
\end{frame}

\begin{frame}[label={sec:orgheadline11}]{Oxidation of Unsaturated Fatty Acids}
\begin{columns}
\begin{column}{0.5\columnwidth}
\centering
\includegraphics[height=0.85\textheight]{./figures/23_9.png}
\end{column}

\begin{column}{0.5\columnwidth}
\begin{itemize}
\item isomerase and reductase change location of the double bonds
\begin{itemize}
\item correct configuration for B-oxidation
\end{itemize}
\end{itemize}
\end{column}
\end{columns}
\end{frame}
\begin{frame}[label={sec:orgheadline12}]{Odd-Chain Length Fatty Acids}
\centering
\includegraphics[height=0.5\textheight]{./figures/23_10.png}
\end{frame}

\begin{frame}[label={sec:orgheadline13}]{Oxidation of Medium-Chain Length Fatty Acids}
\begin{itemize}
\item \(\uparrow\) solubility
\item not stored in adipose triacylglycerol
\item gut \(\to\) portal vein \(\to\) liver
\item \(\to\) mito matrix via the monocarboxylate transporter
\item activated in the mito matrix
\item B-oxidation
\end{itemize}
\end{frame}

\begin{frame}[label={sec:orgheadline14}]{Regulation of B-oxidation}
\centering
\includegraphics[height=0.6\textheight]{./figures/23_12.png}

\begin{enumerate}
\item Lipolysis or gut
\item Regulation of CPT1 activity
\item Re-oxidation of NAD\(^{\text{+}}\) and FAD\(^{\text{2+}}\)
\end{enumerate}
\end{frame}

\section{Alternative Routes of Fatty Acid Oxidation}
\label{sec:orgheadline20}

\begin{frame}[label={sec:orgheadline16}]{Peroxisomal Oxidation of Fatty Acids}
\includegraphics[width=.9\linewidth]{./figures/23_14.png}

\begin{itemize}
\item very long chain FA C24-26 mandatory
\item long chain optional
\item carnitine not required for entry into peroxisomes
\end{itemize}
\end{frame}

\begin{frame}[label={sec:orgheadline17}]{Long-Chain Branched-Chain Fatty Acids}
\centering
\includegraphics[height=0.7\textheight]{./figures/ff22.png}

\begin{itemize}
\item \(\alpha\)-oxidation of phytanic acid takes place in peroxisomes.
\item Pristanic acid can then undergo \(\beta\)-oxidation.
\begin{itemize}
\item Propionyl-CoA is released when the \(\alpha\) carbon is substituted
\end{itemize}
\end{itemize}
\end{frame}

\begin{frame}[label={sec:orgheadline18}]{\(\omega\)-Oxidation of Fatty Acids}
\begin{columns}
\begin{column}{0.5\columnwidth}
\centering
\includegraphics[height=0.5\textheight]{./figures/23_16.png}
\end{column}


\begin{column}{0.5\columnwidth}
\begin{itemize}
\item occurs in the ER
\item the \(\omega\)-carbon is oxidized to an alcohol
\item dehydrogenated to a carboxylic acid \(\to\) dicarboxylic acid
\begin{itemize}
\item excreted in urine as medium chain dicarboxylic acids
\end{itemize}
\item xenobiotic compounds w FA like structure
\end{itemize}
\end{column}
\end{columns}
\end{frame}


\begin{frame}[label={sec:orgheadline19}]{Thanks}
\begin{itemize}
\item Next up:
\begin{itemize}
\item Fatty Acid Oxidation Defects
\item Carnitine and Aceylcarnitines methods
\end{itemize}
\end{itemize}
\end{frame}
\end{document}

% Created 2018-02-06 Tue 14:03
\documentclass[presentation, smaller]{beamer}
\usepackage[utf8]{inputenc}
\usepackage[T1]{fontenc}
\usepackage{fixltx2e}
\usepackage{graphicx}
\usepackage{grffile}
\usepackage{longtable}
\usepackage{wrapfig}
\usepackage{rotating}
\usepackage[normalem]{ulem}
\usepackage{amsmath}
\usepackage{textcomp}
\usepackage{amssymb}
\usepackage{capt-of}
\usepackage{hyperref}
\hypersetup{colorlinks,linkcolor=white,urlcolor=blue}
\usepackage{textpos}
\usepackage{textgreek}
\usepackage[version=4]{mhchem}
\usepackage{chemfig}
\usepackage{siunitx}
\usepackage{gensymb}
\usepackage[usenames,dvipsnames]{xcolor}
\usepackage[T1]{fontenc}
\usepackage{lmodern}
\usepackage{verbatim}
\usepackage{tikz}
\usetikzlibrary{shapes.geometric,arrows,decorations.pathmorphing,backgrounds,positioning,fit,petri}
\usetheme{Hannover}
\usecolortheme{whale}
\author{Matthew Henderson, PhD, FCACB}
\date{\today}
\title{Propionic Acidemia}
\institute[NSO]{Newborn Screening Ontario | The University of Ottawa}
\titlegraphic{\includegraphics[height=1cm,keepaspectratio]{../logos/NSO_logo.pdf}\includegraphics[height=1cm,keepaspectratio]{../logos/cheo-logo.png} \includegraphics[height=1cm,keepaspectratio]{../logos/UOlogoBW.eps}}
\hypersetup{
 pdfauthor={Matthew Henderson, PhD, FCACB},
 pdftitle={Propionic Acidemia},
 pdfkeywords={},
 pdfsubject={},
 pdfcreator={Emacs 25.2.1 (Org mode 8.3.4)}, 
 pdflang={English}}
\begin{document}

\maketitle
%\logo{\includegraphics[width=1cm,height=1cm,keepaspectratio]{../logos/NSO_logo_small.pdf}~%
%    \includegraphics[width=1cm,height=1cm,keepaspectratio]{../logos/UOlogoBW.eps}%
%}

\vspace{220pt}
\beamertemplatenavigationsymbolsempty
\setbeamertemplate{caption}[numbered]
\setbeamerfont{caption}{size=\tiny}
% \addtobeamertemplate{frametitle}{}{%
% \begin{textblock*}{100mm}(.85\textwidth,-1cm)
% \includegraphics[height=1cm,width=2cm]{cat}
% \end{textblock*}}

\tikzstyle{chemical} = [rectangle, rounded corners, text width=5em, minimum height=1em,text centered, draw=black, fill=none]
\tikzstyle{hardware} = [rectangle, rounded corners, text width=5em, minimum height=1em,text centered, draw=black, fill=gray!30]
\tikzstyle{ms} = [rectangle, rounded corners, text width=5em, minimum height=1em,text centered, draw=orange, fill=none]
\tikzstyle{msw} = [rectangle, rounded corners, text width=7em, minimum height=1em,text centered, draw=orange, fill=none]
\tikzstyle{label} = [rectangle,text width=8em, minimum height=1em, text centered, draw=none, fill=none]
\tikzstyle{hl} = [rectangle, rounded corners, text width=5em, minimum height=1em,text centered, draw=black, fill=red!30]
\tikzstyle{box} = [rectangle, rounded corners, text width=5em, minimum height=5em,text centered, draw=black, fill=none]
\tikzstyle{arrow} = [thick,->,>=stealth]
\tikzstyle{hl-arrow} = [ultra thick,->,>=stealth,draw=red]

\section{Propionic Acidemia}
\label{sec:orgheadline13}
\begin{frame}[label={sec:orgheadline1}]{PA Phenotype}
\begin{itemize}
\item Recurrent episodes of ketosis, acidosis and dehydration progressive to coma;
\item neutropenia, thrombocytopenia; osteoporosis;
\item hyperglycinemia;
\item propionic acidemia;
\item methylcitraturia;
\item deficiency of propionyl CoA carboxylase.
\end{itemize}
\end{frame}

\begin{frame}[label={sec:orgheadline2}]{PA History}
\begin{block}{Ketotic Hyperglycinemia}
\begin{itemize}
\item A patient with PA was reported in 1961 as hyperglycinemia
\begin{itemize}
\item recurrent ketoacidosis
\item \(\uparrow\) \(\uparrow\) glycine in blood and urine
\item attacks related to the intake of protein
\begin{itemize}
\item administration of BCAA, Thr, Met \(\to\) ketonuria
\end{itemize}
\end{itemize}
\end{itemize}
\end{block}
\begin{block}{Non-ketotic Hyperglycinemia}
\begin{itemize}
\item Methylmalonic acidemia
\end{itemize}
\end{block}
\end{frame}

\begin{frame}[label={sec:orgheadline3}]{BCAA catabolism}
\centering
\includegraphics[height=0.85\textheight]{./figures/bcaa.png}
\end{frame}
\begin{frame}[label={sec:orgheadline4}]{Propionic Acidemia Pathway}
\includegraphics[width=.9\linewidth]{./figures/pa_path.png}
\end{frame}

\begin{frame}[label={sec:orgheadline5}]{Propionyl Carboxylase}
\begin{itemize}
\item Composed of \(\alpha\) and \(\beta\) subunits
\begin{itemize}
\item \(\alpha_{\text{4}} \beta_{\text{4}}\) heteropolymer
\end{itemize}
\item Apoenzyme activated by covalent binding of biotin
\end{itemize}
\end{frame}

\begin{frame}[label={sec:orgheadline6}]{Acute presentation}
\begin{itemize}
\item Life-threatening illness early in life
\begin{itemize}
\item ketonuria
\begin{itemize}
\item acidosis
\item dehydration
\end{itemize}
\item vomiting
\item lethargy \(\to\) coma
\end{itemize}
\end{itemize}
\end{frame}

\begin{frame}[label={sec:orgheadline7}]{Recurrent Symptoms}
\begin{itemize}
\item ketotic episodes
\item infection
\item protein intolerance
\end{itemize}
\end{frame}


\begin{frame}[label={sec:orgheadline8}]{Clinical Chemistry}
\begin{itemize}
\item Acidosis in acute episodes
\begin{itemize}
\item accumulation of \(\beta\)-hydroxybutyrate and acetoacetate
\item Arterial pH as low as 6.9
\item Bicarb as low as 5 mEq/L
\end{itemize}
\item \(\uparrow\) lactic acid
\item Hypoglycemia
\item Hyperammonemia
\end{itemize}
\end{frame}

\begin{frame}[label={sec:orgheadline9}]{Amino Acids}
\begin{itemize}
\item \(\Uparrow\) glycine
\item \(\uparrow\) glutamine when hyperammonemia
\end{itemize}
\end{frame}


\begin{frame}[label={sec:orgheadline10}]{Long term}
\begin{itemize}
\item Variable developmental/cognitive outcome
\begin{itemize}
\item appears linked to incidence of illness
\end{itemize}
\item hypotonic
\begin{itemize}
\item developmental delay
\end{itemize}
\item A subset with exclusively neurological presentation
\begin{itemize}
\item \textpm{} ketoacidosis
\item hypotonia \(\to\) hypertonia
\end{itemize}
\item Propionyl-CoA toxic to bone marrow
\begin{itemize}
\item neutropenia
\item transient thrombocytopenia in infancy
\end{itemize}
\item Osteoporosis
\item Pancreatitis
\item Cardiomyopathy
\end{itemize}
\end{frame}


\begin{frame}[label={sec:orgheadline11}]{Neurological Findings}
\begin{columns}
\begin{column}{.5\columnwidth}
\begin{itemize}
\item Neonatal death
\begin{itemize}
\item spongy degeneration of white matter
\end{itemize}
\item Later death
\begin{itemize}
\item shrinkage and marbling in basal ganglia
\item neuronal loss
\item gliosis
\end{itemize}
\end{itemize}
\end{column}
\begin{column}{.5\columnwidth}
\includegraphics[width=.9\linewidth]{./figures/pa_mri.png}
\end{column}
\end{columns}
\end{frame}





\begin{frame}[label={sec:orgheadline12}]{Genetics}
\begin{itemize}
\item autosomal recessive trait
\item Propionyl CoA Carboxylase
\end{itemize}
\end{frame}
\end{document}

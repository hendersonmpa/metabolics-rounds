% Created 2018-02-14 Wed 16:55
\documentclass[presentation, smaller]{beamer}
\usepackage[utf8]{inputenc}
\usepackage[T1]{fontenc}
\usepackage{fixltx2e}
\usepackage{graphicx}
\usepackage{grffile}
\usepackage{longtable}
\usepackage{wrapfig}
\usepackage{rotating}
\usepackage[normalem]{ulem}
\usepackage{amsmath}
\usepackage{textcomp}
\usepackage{amssymb}
\usepackage{capt-of}
\usepackage{hyperref}
\hypersetup{colorlinks,linkcolor=white,urlcolor=blue}
\usepackage{textpos}
\usepackage{textgreek}
\usepackage[version=4]{mhchem}
\usepackage{chemfig}
\usepackage{siunitx}
\usepackage{gensymb}
\usepackage[usenames,dvipsnames]{xcolor}
\usepackage[T1]{fontenc}
\usepackage{lmodern}
\usepackage{verbatim}
\usepackage{tikz}
\usetikzlibrary{shapes.geometric,arrows,decorations.pathmorphing,backgrounds,positioning,fit,petri}
\usetheme{Hannover}
\usecolortheme{whale}
\author{Matthew Henderson, PhD, FCACB}
\date{\today}
\title{Propionic Acidemia}
\institute[NSO]{Newborn Screening Ontario | The University of Ottawa}
\titlegraphic{\includegraphics[height=1cm,keepaspectratio]{../logos/NSO_logo.pdf}\includegraphics[height=1cm,keepaspectratio]{../logos/cheo-logo.png} \includegraphics[height=1cm,keepaspectratio]{../logos/UOlogoBW.eps}}
\hypersetup{
 pdfauthor={Matthew Henderson, PhD, FCACB},
 pdftitle={Propionic Acidemia},
 pdfkeywords={},
 pdfsubject={},
 pdfcreator={Emacs 25.2.1 (Org mode 8.3.4)}, 
 pdflang={English}}
\begin{document}

\maketitle
%\logo{\includegraphics[width=1cm,height=1cm,keepaspectratio]{../logos/NSO_logo_small.pdf}~%
%    \includegraphics[width=1cm,height=1cm,keepaspectratio]{../logos/UOlogoBW.eps}%
%}

\vspace{220pt}
\beamertemplatenavigationsymbolsempty
\setbeamertemplate{caption}[numbered]
\setbeamerfont{caption}{size=\tiny}
% \addtobeamertemplate{frametitle}{}{%
% \begin{textblock*}{100mm}(.85\textwidth,-1cm)
% \includegraphics[height=1cm,width=2cm]{cat}
% \end{textblock*}}

\tikzstyle{chemical} = [rectangle, rounded corners, text width=5em, minimum height=1em,text centered, draw=black, fill=none]
\tikzstyle{hardware} = [rectangle, rounded corners, text width=5em, minimum height=1em,text centered, draw=black, fill=gray!30]
\tikzstyle{ms} = [rectangle, rounded corners, text width=5em, minimum height=1em,text centered, draw=orange, fill=none]
\tikzstyle{msw} = [rectangle, rounded corners, text width=7em, minimum height=1em,text centered, draw=orange, fill=none]
\tikzstyle{label} = [rectangle,text width=8em, minimum height=1em, text centered, draw=none, fill=none]
\tikzstyle{hl} = [rectangle, rounded corners, text width=5em, minimum height=1em,text centered, draw=black, fill=red!30]
\tikzstyle{box} = [rectangle, rounded corners, text width=5em, minimum height=5em,text centered, draw=black, fill=none]
\tikzstyle{arrow} = [thick,->,>=stealth]
\tikzstyle{hl-arrow} = [ultra thick,->,>=stealth,draw=red]

\section{Introduction}
\label{sec:orgheadline10}
\begin{frame}[label={sec:orgheadline1}]{History}
\begin{itemize}
\item Was called Ketotic Hyperglycinemia
\item A patient with PA was reported in 1961 with hyperglycinemia
\begin{itemize}
\item recurrent ketoacidosis
\item \(\uparrow\) \(\uparrow\) glycine in blood and urine
\item attacks related to the intake of protein
\begin{itemize}
\item administration of BCAA, Thr, Met \(\to\) ketonuria
\end{itemize}
\end{itemize}
\item Index patient and sister had defective propionyl carboxylase.
\item \(\uparrow\) clinical penetrance
\begin{itemize}
\item incidence of PA has not \(\uparrow\) with NBS
\end{itemize}
\end{itemize}
\end{frame}

\begin{frame}[label={sec:orgheadline2}]{Propionic Acid and Derivatives}
\centering

\vspace{2em}
\chemname{\chemfig[][scale=.5]{-[7]-[1]([2]=O)-[7]OH}}{\tiny propionic acid}
\hspace{4em}
\chemname{\chemfig[][scale=.5]{-[7]-[1]([2]=O)-[7]CoA}}{\tiny propionyl CoA}

\vspace{2em}
\chemname{\chemfig[][scale=.5]{-N^{+}([2]-)([6]-)-[1]-[7]([6]-O-([5]=O)-[7,.6]-[1,.6])-[1]-[7]([7]=O)([1]-O^{-})}}{\tiny propionyl-carnitine}
\hspace{4em}
\chemname{\chemfig[][scale=.5]{OH-[1]-[7]-[1]([2]=O)-[7]OH}}{\tiny 3-hydroxypropionic acid}
\end{frame}

\begin{frame}[label={sec:orgheadline3}]{Propionic Acidemia Pathway}
\includegraphics[width=.9\linewidth]{./figures/pa_path.png}
\end{frame}

\begin{frame}[label={sec:orgheadline4}]{BCAA catabolism}
\centering
\includegraphics[height=0.85\textheight]{./figures/bcaa.png}
\end{frame}

\begin{frame}[label={sec:orgheadline5}]{AA catabolism}
\includegraphics[width=.9\linewidth]{./figures/aa_met.png}
\end{frame}
\begin{frame}[label={sec:orgheadline6}]{Odd-Chain Length Fatty Acids}
\centering
\includegraphics[height=0.5\textheight]{./figures/23_10.png}
\end{frame}

\begin{frame}[label={sec:orgheadline7}]{Long-Chain Branched-Chain Fatty Acids}
\centering
\includegraphics[height=0.7\textheight]{./figures/ff22.png}

\begin{itemize}
\item \(\alpha\)-oxidation of phytanic acid takes place in peroxisomes.
\item Pristanic acid can then undergo \(\beta\)-oxidation.
\begin{itemize}
\item Propionyl-CoA is released when the \(\alpha\) carbon is substituted
\end{itemize}
\end{itemize}
\end{frame}

\begin{frame}[label={sec:orgheadline8}]{Sterol Catabolism}
\includegraphics[width=.9\linewidth]{./figures/gr3.jpg}
\end{frame}

\begin{frame}[label={sec:orgheadline9}]{Propionyl Carboxylase}
\centering
\includegraphics[height=0.75\textheight]{./figures/pc.jpg}

\begin{itemize}
\item Composed of \(\alpha\) and \(\beta\) subunits
\begin{itemize}
\item \(\alpha_{\text{4}} \beta_{\text{4}}\) heteropolymer
\end{itemize}
\item Apoenzyme activated by covalent binding to biotin
\end{itemize}
\end{frame}

\section{Genetics and Pathogenisis}
\label{sec:orgheadline14}

\begin{frame}[label={sec:orgheadline11}]{Genetics}
\begin{itemize}
\item autosomal recessive trait
\item Two complementation groups
\begin{itemize}
\item PccA \(\to\) \(\alpha\)
\item PccBC \(\to\) \(\beta\)
\end{itemize}
\item \(\alpha\) 13q32
\item \(\beta\) 3q13.3-22
\end{itemize}
\end{frame}


\begin{frame}[label={sec:orgheadline12}]{Propionyl CoA}
\begin{block}{Hyperglycinemia}
\begin{itemize}
\item PA inhibits synthesis of glycine cleaving enzyme.
\end{itemize}
\end{block}
\begin{block}{CBC}
\begin{itemize}
\item Propionyl-CoA toxic to bone marrow
\begin{itemize}
\item neutropenia
\item transient thrombocytopenia in infancy
\end{itemize}
\end{itemize}
\end{block}
\begin{block}{Hyperammonemia}
\begin{itemize}
\item PA of of carbamylphosphate synthetase
\end{itemize}
\end{block}
\begin{block}{Ketosis}
\begin{itemize}
\item PA inhibits mitochondrial oxidation of succinic acid and 2-ketoglutaric acid
\end{itemize}
\end{block}
\end{frame}

\begin{frame}[label={sec:orgheadline13}]{Biochemical Markers}
\begin{itemize}
\item 3-hydroxypropionic acid
\item tiglic acid / tiglyglycine
\item propionylglycine
\item methylcitrate
\begin{itemize}
\item condensation of propionyl-CoA \& oxaloacetic acid
\item metabolic end product and very stable
\end{itemize}
\end{itemize}
\end{frame}

\section{Laboratory Investigations}
\label{sec:orgheadline20}
\begin{frame}[label={sec:orgheadline15}]{NSO PA/MMA Screening Logic}
\begin{block}{Inital positive \textless{} 7 days}
(C3/C2 \(\ge\) 0.21 AND C3 \(\ge\) 4.0)
OR
(C3/C2 \(\ge\) 0.23 AND C3 \(\ge\) 3.5)
\end{block}
\begin{block}{Inital positive \textgreater{} 7 days}
(C3/C2 \(\ge\) 0.21 AND C3 \(\ge\) 2.6)
OR
(C3/C2 \(\ge\) 0.23 AND C3 \(\ge\) 2.4)
\begin{itemize}
\item Repeat overnight
\item No weekend reporting
\end{itemize}
\end{block}
\begin{block}{Alert}
C3/C2 \(\ge\) 0.3 AND C3 \(\ge\) 9.0
\begin{itemize}
\item Repeat same day
\item Weekend reporting
\end{itemize}
\end{block}
\begin{block}{Confirmation}
C3/C2 \(\ge\) 0.23 AND MCA \(\ge\) 0.5
\end{block}
\end{frame}
\begin{frame}[label={sec:orgheadline16}]{Elevated C3 ACT algorithm}
\includegraphics[width=.9\linewidth]{./figures/pa_act.png}
\end{frame}
\begin{frame}[label={sec:orgheadline17}]{Clinical Chemistry}
\begin{itemize}
\item Acidosis in acute episodes
\begin{itemize}
\item accumulation of \(\beta\)-hydroxybutyrate and acetoacetate
\item Arterial pH as low as 6.9
\item Bicarb as low as 5 mEq/L
\end{itemize}
\item \(\uparrow\) lactic acid
\item Hypoglycemia
\item Hyperammonemia
\end{itemize}
\end{frame}

\begin{frame}[label={sec:orgheadline18}]{Biochemical Genetics}
\begin{block}{Plasma Amino Acids}
\begin{itemize}
\item \(\Uparrow\) glycine
\item \(\uparrow\) glutamine when hyperammonemia
\end{itemize}
\end{block}
\begin{block}{Plasma Acylcarnitines}
\begin{itemize}
\item \(\uparrow\) propionyl carnitine (C3)
\end{itemize}
\end{block}
\begin{block}{Urine Organic Acids}
\begin{itemize}
\item 3-hydroxypropionic acid
\item methylcitric acid
\item lactic acid
\item BHB
\item acetoacetate
\item tiglic acid / tiglyglycine
\end{itemize}
\end{block}
\end{frame}

\begin{frame}[label={sec:orgheadline19}]{Urine Organic Acids}
\includegraphics[width=.9\linewidth]{./figures/pa_uoa.png}
\end{frame}

\section{Clinical Findings}
\label{sec:orgheadline26}
\begin{frame}[label={sec:orgheadline21}]{Acute presentation}
\begin{itemize}
\item Life-threatening illness early in life
\begin{itemize}
\item ketonuria
\begin{itemize}
\item acidosis
\item dehydration
\end{itemize}
\item vomiting
\item lethargy \(\to\) coma
\end{itemize}
\end{itemize}
\end{frame}

\begin{frame}[label={sec:orgheadline22}]{Recurrent Symptoms}
\begin{itemize}
\item ketotic episodes
\item infection
\item protein intolerance
\end{itemize}
\end{frame}

\begin{frame}[label={sec:orgheadline23}]{Long term}
\begin{itemize}
\item Variable developmental/cognitive outcomes
\begin{itemize}
\item appears linked to incidence of illness
\end{itemize}
\item hypotonic
\begin{itemize}
\item developmental delay
\end{itemize}
\item A subset with exclusively neurological presentation
\begin{itemize}
\item \textpm{} ketoacidosis
\item hypotonia \(\to\) hypertonia
\end{itemize}
\item Propionyl-CoA toxic to bone marrow
\begin{itemize}
\item neutropenia
\item transient thrombocytopenia in infancy
\end{itemize}
\item Osteoporosis
\item Pancreatitis
\item Cardiomyopathy
\item Fatty infiltration of liver on PM
\end{itemize}
\end{frame}

\begin{frame}[label={sec:orgheadline24}]{Neurological Findings}
\begin{columns}
\begin{column}{.5\columnwidth}
\begin{itemize}
\item Neonatal death
\begin{itemize}
\item spongy degeneration of white matter
\end{itemize}
\item Later death
\begin{itemize}
\item shrinkage and marbling in basal ganglia
\item neuronal loss
\item gliosis
\end{itemize}
\end{itemize}
\end{column}
\begin{column}{.5\columnwidth}
\includegraphics[width=.9\linewidth]{./figures/pa_mri.png}
\end{column}
\end{columns}
\end{frame}

\begin{frame}[label={sec:orgheadline25}]{Long-term Treatment}
\begin{block}{Diet}
\begin{itemize}
\item Limit Val, Ile, Thr, Met
\begin{itemize}
\item Monitor urine metabolites, plasma amino acids
\item Urine ketones (daily in infancy)
\item Monitor weight, nitrogen balance
\end{itemize}
\item Avoid fasting
\begin{itemize}
\item Catabolism
\item Propionate release from lipids
\end{itemize}
\end{itemize}
\end{block}

\begin{block}{Supplementation}
\begin{itemize}
\item Carnitine
\begin{itemize}
\item excretion of carnitine esters \(\to\) detoxification
\item Daily dose 60 to 100 mg/kg
\end{itemize}
\item Biotin
\begin{itemize}
\item conflicting information
\end{itemize}
\end{itemize}
\end{block}
\end{frame}
\end{document}

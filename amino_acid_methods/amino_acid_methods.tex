% Created 2017-12-14 Thu 09:53
\documentclass[presentation, smaller]{beamer}
\usepackage[utf8]{inputenc}
\usepackage[T1]{fontenc}
\usepackage{fixltx2e}
\usepackage{graphicx}
\usepackage{grffile}
\usepackage{longtable}
\usepackage{wrapfig}
\usepackage{rotating}
\usepackage[normalem]{ulem}
\usepackage{amsmath}
\usepackage{textcomp}
\usepackage{amssymb}
\usepackage{capt-of}
\usepackage{hyperref}
\hypersetup{colorlinks,linkcolor=white,urlcolor=blue}
\usepackage{textpos}
\usepackage{textgreek}
\usepackage[version=4]{mhchem}
\usepackage{chemfig}
\usepackage{siunitx}
\usepackage[usenames,dvipsnames]{xcolor}
\usepackage[T1]{fontenc}
\usepackage{lmodern}
\usepackage{verbatim}
\usepackage{tikz}
\usetikzlibrary{shapes.geometric,arrows,decorations.pathmorphing,backgrounds,positioning,fit,petri}
\usetheme[height=20pt]{Frankfurt}
\usecolortheme{whale}
\author{Matthew Henderson, PhD, FCACB}
\date{\today}
\title{Laboratory Methods for Quantitative Amino Acid Analysis in Biological Matrices}
\institute[NSO]{Newborn Screening Ontario | The University of Ottawa}
\titlegraphic{\includegraphics[height=1cm,keepaspectratio]{../logos/NSO_logo.pdf}\includegraphics[height=1cm,keepaspectratio]{../logos/cheo-logo.png} \includegraphics[height=1cm,keepaspectratio]{../logos/UOlogoBW.eps}}
\hypersetup{
 pdfauthor={Matthew Henderson, PhD, FCACB},
 pdftitle={Laboratory Methods for Quantitative Amino Acid Analysis in Biological Matrices},
 pdfkeywords={},
 pdfsubject={},
 pdfcreator={Emacs 25.2.1 (Org mode 8.3.4)}, 
 pdflang={English}}
\begin{document}

\maketitle

%\logo{\includegraphics[width=1cm,height=1cm,keepaspectratio]{../logos/NSO_logo_small.pdf}~%
%    \includegraphics[width=1cm,height=1cm,keepaspectratio]{../logos/UOlogoBW.eps}%
%}

\vspace{220pt}}
\beamertemplatenavigationsymbolsempty
\setbeamertemplate{caption}[numbered]
\setbeamerfont{caption}{size=\tiny}
% \addtobeamertemplate{frametitle}{}{%
% \begin{textblock*}{100mm}(.85\textwidth,-1cm)
% \includegraphics[height=1cm,width=2cm]{cat}
% \end{textblock*}}

\tikzstyle{chemical} = [rectangle, rounded corners, text width=5em, minimum height=1em,text centered, draw=black, fill=none]
\tikzstyle{hardware} = [rectangle, rounded corners, text width=5em, minimum height=1em,text centered, draw=black, fill=gray!30]
\tikzstyle{ms} = [rectangle, rounded corners, text width=5em, minimum height=1em,text centered, draw=orange, fill=none]
\tikzstyle{msw} = [rectangle, rounded corners, text width=7em, minimum height=1em,text centered, draw=orange, fill=none]
\tikzstyle{label} = [rectangle,text width=5em, minimum height=1em, text centered, draw=none, fill=none]
\tikzstyle{hl} = [rectangle, rounded corners, text width=5em, minimum height=1em,text centered, draw=black, fill=red!30]
\tikzstyle{arrow} = [thick,->,>=stealth]
\tikzstyle{hl-arrow} = [ultra thick,->,>=stealth,draw=red]

\section{Introduction}
\label{sec:orgheadline11}
\begin{frame}[label={sec:orgheadline1}]{Amino Acids: A Very Short Introduction}
\begin{itemize}
\item Amino acids are mono or dicarboxylic acids with one or more amino groups.
\begin{itemize}
\item Zwitterion at ph 7.45
\end{itemize}

\item Proteinogenic amino acids (22)
\begin{itemize}
\item 21 amino acids naturally incorportated into polypeptides in humans
\item 20 genetically encoded
\item selenocysteine
\end{itemize}

\item Non-proteinogenic
\begin{itemize}
\item post-translational modification
\begin{itemize}
\item hydroxylation of proline \(\to\) hydroxyproline
\end{itemize}
\item Not found in proteins
\begin{itemize}
\item gamma-aminobutryic acid
\item ornithine, citrulline
\end{itemize}
\end{itemize}

\item 76 amino acids of biological interest in humans
\end{itemize}
\end{frame}

\begin{frame}[label={sec:orgheadline2}]{Indications for Measurement of Amino Acids}
\begin{itemize}
\item Diagnosis of inborn errors of amino acid metabolism and transport
\item Diet monitoring in patients with known IEM
\item Nutritional assessment of patients with non-metabolic conditions [e.g. short bowel syndrome]
\end{itemize}
\end{frame}

\begin{frame}[label={sec:orgheadline3}]{Quantitative Amino Acid Analysis}
\begin{columns}
\begin{column}{0.5\columnwidth}
\includegraphics[width=.9\linewidth]{./figures/Moore_Stein.png}
\end{column}

\begin{column}{0.5\columnwidth}
\begin{itemize}
\item Stanford Moore (left) and William Stein (right) about 1965 in front
of the original amino acid analyzer.
\item Ion-exchange chromatography with post-column ninhydrin detection
\end{itemize}
\end{column}
\end{columns}
\end{frame}

\begin{frame}[label={sec:orgheadline4}]{Metbionet Guidelines for Amino Acid Analysis.}
\begin{block}{Plasma Profile for Diagnosis of Amino Acid Disorders}
\begin{columns}
\begin{column}{0.3\columnwidth}
\begin{itemize}
\item Alanine
\item Alloisoleucine
\item Arginine
\item Argininosuccinic acid
\item Cystine
\item Citrulline
\item Glutamic acid
\end{itemize}
\end{column}


\begin{column}{0.3\columnwidth}
\begin{itemize}
\item Glutamine
\item Glycine
\item Histidine
\item Homocysteine*
\item Isoleucine
\item Leucine
\item Lysine
\item Methionine
\end{itemize}
\end{column}
\begin{column}{0.3\columnwidth}
\begin{itemize}
\item Ornithine
\item Phenylalanine
\item Proline
\item Serine
\item Sulphocysteine**
\item Taurine
\item Threonine
\item Tyrosine
\item Valine
\end{itemize}
\end{column}
\end{columns}
\end{block}

\begin{itemize}
\item *Plasma total homocysteine is not detected by routine methods, plasma free homocystine analysis has poor clinical sensitivity.
\item **Sulphocysteine may not be detectable in plasma using routine methods
\end{itemize}
\end{frame}
\begin{frame}[label={sec:orgheadline5}]{Plasma}
\begin{itemize}
\item Patient prep
\begin{itemize}
\item Fasting (overnight preferred, 4 hours minimum). Infants and children should be
drawn just before next feeding (2-3 hours without TPN if possible).
\end{itemize}
\item Sample
\begin{itemize}
\item LiHeparin venous plasma
\item Prompt separation and deproteinisation is essential for accurate measurement of (free) sulphur containing amino acids.
\item Protein binding: cystine and homocystine
\item release of arginase from RBCs
\item store at -20\degree{}C to limit glutamine decomposition
\end{itemize}
\end{itemize}
\end{frame}

\begin{frame}[label={sec:orgheadline6}]{Plasma - Pre-analytical}
\begin{itemize}
\item Serum should not be used because there may be deamination (asparagine to
aspartic acid and glutamine to glutamic acid), loss of sulphur
containing amino acids and release of oligopeptides.
\item EDTA plasma is recommended in some labs as the specimen of
choice. The older literature reports ninhydrin positive artefacts in
EDTA plasma but modern tubes do not seem to have this problem.
\item Haemolysis will cause increases in serine, glycine, taurine,
phosphoethanolamine, aspartic acid, glutamic acid, ornithine and
decreased arginine.
\item Delayed separation or leucocyte and platelet contamination will
cause increased serine, glycine, taurine, phosphoethanolamine,
ornithine, glutamic acid and decreased arginine, homocystine,
cystine.
\item Phenylalanine and tyrosine increase if specimen separation is
delayed
\item Amino acids are more stable in frozen deproteinised plasma than
in frozen native plasma.
\item Capillary blood may be used with careful cleaning of the skin prior
to specimen collection provided the blood is flowing freely.
\item Free tryptophan may be lost when using sulphosalicylic acid as
deproteinising agent.
\begin{itemize}
\item trichloroacetic acid is the deproteinising agent of choice for
this amino acid.
\end{itemize}
\item Sodium metabisulphite, found in some intravenous preparations as a
preservative, can cause the conversion of cystine to sulphocysteine.
\end{itemize}
\end{frame}

\begin{frame}[label={sec:orgheadline7}]{Urine}
\begin{itemize}
\item Urine
\begin{itemize}
\item 24 hour or random urine
\item preservative free bottle
\item Specimen quality is checked by testing for nitrite and pH.
\begin{itemize}
\item Specimen deterioration causes:
\begin{itemize}
\item \(\downarrow\) serine
\item \(\uparrow\) \(\downarrow\) alanine
\item increased glycine
\item decarboxylation of glutamic acid \(\to\) \(\gamma\)-aminobutyric acid
\item breakdown of phosphoethanolamine \(\to\) ethanolamine and phosphate
\item breakdown of cystathionine \(\to\) homocystine
\item hydrolysis of peptides causing \(\uparrow\) proline
\end{itemize}
\end{itemize}
\item Fecal contamination causes increased proline, glutamic acid, branched chain amino acids but not hydroxyproline.
\item Faecal bacteria can produce \(\gamma\)-aminobutyric acid from glutamic acid and b-alanine from aspartic acid
\item reported in \textmu{}mol/g creatinine
\item Aminoaciduria due to overflow and amino acid transport defects.
\end{itemize}
\end{itemize}
\end{frame}

\begin{frame}[label={sec:orgheadline8}]{Cerebral Spinal Fluid}
\begin{itemize}
\item free of blood contamination
\item which tube is used?
\item investigation of neurological disorders and essential for the
diagnosis of non-ketotic hyperglycinaemia.
\item CSF/Plasma ratio of amino acids is more informative than an isolated CSF sample.
\begin{itemize}
\item A paired plasma sample should be obtained within two hours.
\end{itemize}
\end{itemize}
\end{frame}
\begin{frame}[label={sec:orgheadline9}]{Dried Blood Spot}
\begin{itemize}
\item Collected from free flowing blood spotted onto filter paper
\item Newborn Screening
\item Monitoring Therapy/Diet
\item Each DBS is assume to contain 3.2 \textmu{}L of blood
\item The quantity of blood present in the paper varies by
\begin{itemize}
\item hematocrit
\item degree of saturation
\item the cotton fiber paper
\item the environment  when applying blood (humidity and temperature).
\end{itemize}
\item Because of these numerous factors, a dried blood spot is a highly
imprecise specimen compared with liquids such as urine, blood, and plasma.

\item Amniotic Fluid
\begin{itemize}
\item prenatal diagnosis of IMD

\item Still used?
\end{itemize}
\end{itemize}
\end{frame}

\begin{frame}[label={sec:orgheadline10}]{ERNDIM Plasma Amino Acids Survey}
\includegraphics[height=0.9\textheight]{./figures/erndim.png}
\end{frame}

\section{Amino Acid Analyser}
\label{sec:orgheadline16}

\begin{frame}[label={sec:orgheadline12}]{Amino Acid Analyser Schematic}
\begin{center}
\begin{tikzpicture}[node distance=9em]
% nodes
\node(column)[msw, right of=extraction]{Chromatography};
\node(derivatization)[msw, right of=column]{Ninhyrin Derivatization};
\node(detector)[ms, right of=derivatization]{UV Detector};
% arrows
\draw[arrow](column) -- (derivatization);
\draw[arrow](derivatization) -- (detector);
\end{tikzpicture}

\vspace{2em}

\schemedebug{false}
\schemestart
\chemfig[][scale=.33]{*6(=*5(-(=O)-(-[6]OH)(-[8]OH)-(=O)-)-=-=-)}
\+
\chemfig[][scale=.5]{{\color{red}R}-[::-60](<[::-60]NH_3^+)-[::60](=[::60]O)-[::-60]OH}
\arrow{-U>[][{\tiny \ce{2H2O}}]}
\chemfig[][scale=.33]{*6(=*5(-(=O)-(=N-[::-60](-[::-60]{\color{red}R})-[::60](=[::60]O)-[::-60]OH)-(=O)-)-=-=-)}
\arrow{->[][]}
\chemfig[][scale=.33]{*6(=*5(-(=O)-(-N=*5(-(=O)-(*6(-=-=-))--(=O)-))-(=O)-)-=-=-)}
\schemestop

\schemedebug{false}
\schemestart
\chemfig[][scale=.33]{*6(=*5(-(=O)-(-[6]OH)(-[8]OH)-(=O)-)-=-=-)}
\+
\chemfig[][scale=.33]{H-*5(N----(-COOH)-)}
\arrow{->[][]}
\chemfig[][scale=.33]{*6(=*5(-(=O)-(-*5(N-----))-(=O)-)-=-=-)}
\schemestop
\end{center}
\end{frame}


\begin{frame}[label={sec:orgheadline13}]{Amino Acid Analyser}
\begin{itemize}
\item Cation-exchange chromatography using a lithium buffer system
followed by post-column derivatization with ninhydrin
\end{itemize}




\begin{columns}
\begin{column}{0.5\columnwidth}
\begin{itemize}
\item Samples are de-proteinized with sulfosalicylic acid prior to
injection
\item Utilizes a lithium-based cation-exchange column
\item Eluting amino acids undergo post column reaction with ninhydrin
\item Optical detection in the visible spectrum
\begin{itemize}
\item amino acids: 570nm
\item imino acids 440 nm
\end{itemize}
\end{itemize}
\end{column}

\begin{column}{0.5\columnwidth}
\includegraphics[width=.9\linewidth]{./figures/212.jpg}
\end{column}
\end{columns}
\end{frame}

\begin{frame}[label={sec:orgheadline14}]{Amino Acid Analyser}
\includegraphics[width=.9\linewidth]{./figures/aaachrom.png}
\end{frame}

\begin{frame}[label={sec:orgheadline15}]{Pros and Cons of Amino Acid Analysers}
\begin{block}{Pros}
\begin{itemize}
\item Standardized and established technology
\item Interpretive experience
\end{itemize}
\end{block}
\begin{block}{Cons}
\begin{itemize}
\item Long run time (90 – 150 minutes)
\item Lack of analyte specificity (identification by retention time)
\begin{itemize}
\item antibiotics (ampicillin, amoxicillin, and gentamicin), acetaminophen
\end{itemize}
\item Co-eluting substances cannot be separated and distinguished on a standard IEC chromatogram
\begin{itemize}
\item Homocitrulline co-elutes with methionine
\item ASA co-elutes with leucine
\item Alloisoleucine co-elutes with cystathionine
\item Tryptophan co-elutes with histidine
\end{itemize}
\end{itemize}
\end{block}
\end{frame}

\section{FIA-MS/MS}
\label{sec:orgheadline23}

\begin{frame}[label={sec:orgheadline17}]{FIA-MS/MS sample}
\begin{itemize}
\item Amino acids in the DBS eluate are esterified as butyl esters with butanol-hydrogen chloride.
\end{itemize}

\centering
\schemedebug{false}
\schemestart
\chemname{\chemfig[][scale=.33]{{\color{red}R}-[::-60](<[::-60]NH_3^+)-[::60](=[::60]O)-[::-60]OH}}{\tiny \textalpha{}-amino acid}
\+
\chemname{\chemfig[][scale=.33]{HO-[::30]-[::-60]-[::60]-[::-60]}}{\tiny n-butanol}
\arrow{-U>[][{\tiny \ce{H2O}}]}
\chemname{\chemfig[][scale=.33]{{\color{red}R}-[::-60](<[::-60]NH_3^+)-[::60](=[::60]O)-[::-60]O-[::60]-[::-60]-[::60]-[::-60]}}{\tiny AA butyl ester}
\schemestop


\begin{itemize}
\item Solvent delivery is via HPLC with no chromatography, called flow injection analysis.
\item 10 \textmu{}L of sample extract is injected into a flowing stream operating at \textasciitilde{} 0.15 ml/min.

\item Typical injection rates between samples are 2 min, giving a potential 400-
to 600-sample capacity per instrument per day.
\begin{itemize}
\item volume is typically 200-400 specimens per instrument per day
\item maintenance issues, repeat specimen analysis.
\end{itemize}
\end{itemize}
\end{frame}

\begin{frame}[label={sec:orgheadline18}]{FIA-MS/MS schematic}
\begin{center}
\begin{tikzpicture}[node distance=7em]
% nodes
\node(ms1)[ms]{MS1: Mass Filter};
\node(cc)[ms, right of=ms1]{Collision cell};
\node(ms2)[ms, right of=cc]{MS2: Mass Filter};
\node(ion)[ms, below of=ms1,yshift=3em]{Ionization};
\node(lc)[msw, below of=ion,yshift=3em]{Injection};
\node(detector)[ms, below of=ms2, yshift=3em]{Detector};
% arrows
\draw[arrow](lc) -- (ion);
\draw[arrow](ion) -- (ms1);
\draw[arrow](ms1) -- (cc);
\draw[arrow](cc) -- (ms2);
\draw[arrow](ms2) -- (detector);
\end{tikzpicture}
\end{center}
\end{frame}


\begin{frame}[label={sec:orgheadline19}]{FIA-MS/MS transitions}
\begin{itemize}
\item Electrospray ionization in positive mode
\item \(\alpha\)-amino acids fragment to produce the neutral butyl formate molecule (102 Da).
\item A neutral loss scan is used to identify parent molecules with a NL of 102 Da.
\item MRM is used to detected amino acids with basic functional groups
\begin{itemize}
\item arginine, ornithine and citrulline
\end{itemize}
\end{itemize}

\centering
\schemedebug{false}
\schemestart
\chemname{\chemfig[][scale=.33]{{\color{red}R}-[::-60](<[::-60]NH_3^+)-[::60](=[::60]O)-[::-60]O-[::60]-[::-60]-[::60]-[::-60]}}{\tiny AA butyl ester}
\arrow{->[{\tiny fragmentation}]}
\chemname{\chemfig[][scale=.33]{{\color{red}R}-[::60]=NH_2^{+}}}{\tiny fragment}
\+
\chemname{\chemfig[][scale=.33]{H-[::60](=[::60]O)-[::-60]O-[::60]-[::-60]-[::60]-[::-60]}}{\tiny butyl formate (102 Da)}
\schemestop
\end{frame}

\begin{frame}[label={sec:orgheadline20}]{FIA-MS/MS  MRM}
\begin{itemize}
\item Citrulline contains a labile amino group that fragments together with butyl formate.
\item CID results in net neutral fragmentation of butyl formate (102 Da) plus \ce{NH3} (17 Da)
\item \href{https://en.wikipedia.org/wiki/Selected_reaction_monitoring}{SRM} Citrulline-Bu 232.15 Da \(\to\) 113 Da , loss of 119 Da
\end{itemize}

\centering
\schemedebug{false}
\schemestart
\chemname{\chemfig[][scale=.33]{H_2N-[::30,,2,](=[::60]O)-[::-60]NH-[::60]-[::-60]-[::60]-[::-60](<[::-60]NH_3^+)-[::60](=[::60]O)-[::-60]OH}}{\tiny citrulline 175 Da}
\+
\chemname{\chemfig[][scale=.33]{HO-[::30]-[::-60]-[::60]-[::-60]}}{\tiny n-butanol 74 Da}
\arrow{-U>[][{\tiny \ce{H2O}}]}
\chemname{\chemfig[][scale=.33]{H_2N-[::30,,2,](=[::60]O)-[::-60]NH-[::60]-[::-60]-[::60]-[::-60](<[::-60]NH_3^+)-[::60](=[::60]O)-[::-60]O-[::60]-[::-60]-[::60]-[::-60]}}{\tiny 232 Da}
\schemestop



\centering
\schemedebug{false}
\schemestart
\chemname{\chemfig[][scale=.33]{H_2N-[::60]-[::-60]-[::60]-[::-60]-[::60]N=O=C}}{\tiny 113 Da}
\+
\chemname{\chemfig[][scale=.33]{H-[::60](=[::60]O)-[::-60]O-[::60]-[::-60]-[::60]-[::-60]}}{\tiny 102 Da}
\+
\chemname{\chemfig[][scale=.43]{NH_3}}{\tiny 17 Da}
\schemestop

\begin{itemize}
\item Its name is derived from citrullus, the Latin word for watermelon, from which it was first isolated in 1914 by Koga and Odake.
\end{itemize}
\end{frame}


\begin{frame}[label={sec:orgheadline21}]{FIA-MS/MS Amino Acid Scan}
\begin{block}{Quantified Amino Acids}
\begin{columns}
\begin{column}{0.5\columnwidth}
\begin{itemize}
\item Glycine
\item Alanine
\item Valine
\item Leucine
\item Methionine
\item Phenylalanine
\end{itemize}
\end{column}
\begin{column}{0.5\columnwidth}
\begin{itemize}
\item Tyrosine
\item Ornithine
\item Citruline
\item Arginine
\item \color{blue}Succinylacetone
\end{itemize}
\end{column}
\end{columns}
\end{block}
\end{frame}


\begin{frame}[label={sec:orgheadline22}]{Pros and Cons of FIA-MS/MS using DBS}
\begin{itemize}
\item As compared to AAA and LC-MS/MS
\end{itemize}
\begin{block}{Pros}
\begin{itemize}
\item \textasciitilde{} 2 min analysis time
\item Analyte specificity
\item Acylcarnitines quantified simultaneously
\end{itemize}
\end{block}

\begin{block}{Cons}
\begin{itemize}
\item Variability in DBS sample as described above
\item Iso-baric compounds
\begin{itemize}
\item leucine, Isoleucine, Alloisoleucine
\end{itemize}
\item Fewer AA quantified
\begin{itemize}
\item homocystine (free)
\item glutamine
\end{itemize}
\end{itemize}
\end{block}
\end{frame}

\section{LC-MS/MS}
\label{sec:orgheadline30}
\begin{frame}[label={sec:orgheadline24}]{LC-MS/MS schematic}
\begin{center}
\begin{tikzpicture}[node distance=7em]
% nodes
\node(ms1)[ms]{MS1: Mass Filter};
\node(cc)[ms, right of=ms1]{Collision cell};
\node(ms2)[ms, right of=cc]{MS2: Mass Filter};
\node(ion)[ms, below of=ms1,yshift=3em]{Ionization};
\node(lc)[msw, below of=ion,yshift=3em]{Chromatography};
\node(detector)[ms, below of=ms2, yshift=3em]{Detector};
% arrows
\draw[arrow](lc) -- (ion);
\draw[arrow](ion) -- (ms1);
\draw[arrow](ms1) -- (cc);
\draw[arrow](cc) -- (ms2);
\draw[arrow](ms2) -- (detector);
\end{tikzpicture}
\end{center}
\end{frame}



\begin{frame}[label={sec:orgheadline25}]{LC-MS/MS sample prep}
\begin{itemize}
\item 10 \textmu{}L of sample is mixed with 990 \textmu{}L of IS in 0.5 mM Perfluoroheptanoic acid and centrifuge to deproteinize.
\item 200 \textmu{}L of supernatant is removed
\item 7.5 \textmu{}L is injected onto an octadecylsilyl (C18) stationary phase.
\end{itemize}


\includegraphics[width=0.7\textwidth]{./figures/outletmethod.png}
\end{frame}


\begin{frame}[label={sec:orgheadline26}]{Ion-Pairing Chromatography}
\centering
\chemfig[][scale=.70]{CF_3-{(CF_2)_4}-CF_2-[::30](=[::60]O)-[::-60]OH}

\vspace{3em}

\ce{AA+ + PFHA-  <=> AA+ PFHA-}
\end{frame}


\begin{frame}[label={sec:orgheadline27}]{LC- MS/MS transitions}
\begin{itemize}
\item ESI in positive mode
\begin{itemize}
\item MRM
\end{itemize}
\end{itemize}

\begin{block}{Quantified amino acids}
\scriptsize
\begin{columns}
\begin{column}{0.3\columnwidth}
\begin{itemize}
\item phosphoserine
\item taurine
\item phosphoethanolamine
\item aspartate
\item hydroxyproline
\item threonine
\item serine
\item asparagine
\item glutamate
\item glutamine
\item sarcosine
\item aminoadipic
\item proline
\item glycine
\end{itemize}
\end{column}

\begin{column}{0.3\columnwidth}
\begin{itemize}
\item alanine
\item citulline
\item 2-aminobutyric
\item valine
\item cystine
\item saccharopine
\item methionine
\item alloisoleucine
\item cystathionine
\item isoleucine
\item leucine
\item arginosuccinic acid
\item tyrosine
\item \(\beta\)-alanine
\end{itemize}
\end{column}

\begin{column}{0.3\columnwidth}
\begin{itemize}
\item phenylalanine
\item aminoisobutyric
\item \(\gamma\)-aminobutryic
\item ethanolamine
\item hydroxylysine
\item ornithine
\item lysine
\item 1-methylhistidine
\item histidine
\item tryptophan
\item 3-methylhistidine
\item anserine
\item carnosine
\item arginine
\item s-sulfocyteine*
\end{itemize}
\end{column}
\end{columns}
\end{block}
\end{frame}




\begin{frame}[label={sec:orgheadline28}]{Pros and Cons of LC-MS/MS}
\begin{itemize}
\item As compared to FIA-MS/MS
\end{itemize}
\begin{block}{Pros}
\begin{itemize}
\item 43 vs 11 amino acids quantified
\begin{itemize}
\item Leu/Ile/Allo
\end{itemize}
\item Iso-baric compounds resolved
\begin{itemize}
\item Leucine, Isoleucine, Alloisoleucine
\end{itemize}
\end{itemize}
\end{block}
\begin{block}{Cons}
\begin{itemize}
\item Too slow for NBS
\item Manual peak integration
\end{itemize}
\end{block}
\end{frame}


\begin{frame}[label={sec:orgheadline29}]{Pros and Cons of LC-MS/MS}
\begin{itemize}
\item As compared to AAA
\end{itemize}
\begin{block}{Pros}
\begin{itemize}
\item \textasciitilde{} 30 min shorter analysis time
\item Analyte specificity
\begin{itemize}
\item Based on MRM rather than RT and ninhydrin reactivity
\begin{itemize}
\item gentamycin, acetaminophen, dopamine analogs
\end{itemize}
\item Co-eluting substances cannot be separated and distinguished on a
standard IEC chromatogram
\begin{itemize}
\item Homocitrulline co-elutes with methionine
\item ASA co-elutes with leucine
\item Alloisoleucine co-elutes with cystathionine
\item Tryptophan co-elutes with histidine
\end{itemize}
\end{itemize}
\item Long term reagent expense
\end{itemize}
\end{block}

\begin{block}{Cons}
\begin{itemize}
\item Upfront hardware expense
\item Manual peak integration
\item Lab developed test - not standardized
\item Changing LOQ with equipment age
\end{itemize}
\end{block}
\end{frame}
\end{document}

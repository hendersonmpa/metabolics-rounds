% Created 2019-10-23 Wed 17:00
% Intended LaTeX compiler: pdflatex
\documentclass[presentation, smaller]{beamer}
\usepackage[utf8]{inputenc}
\usepackage[T1]{fontenc}
\usepackage{graphicx}
\usepackage{grffile}
\usepackage{longtable}
\usepackage{wrapfig}
\usepackage{rotating}
\usepackage[normalem]{ulem}
\usepackage{amsmath}
\usepackage{textcomp}
\usepackage{amssymb}
\usepackage{capt-of}
\usepackage{hyperref}
\hypersetup{colorlinks,linkcolor=white,urlcolor=blue}
\usepackage{textpos}
\usepackage{textgreek}
\usepackage[version=4]{mhchem}
\usepackage{chemfig}
\usepackage{siunitx}
\usepackage{gensymb}
\usepackage[usenames,dvipsnames]{xcolor}
\usepackage[T1]{fontenc}
\usepackage{lmodern}
\usepackage{verbatim}
\usepackage{tikz}
\usepackage{wasysym}
\usetikzlibrary{shapes.geometric,arrows,decorations.pathmorphing,backgrounds,positioning,fit,petri}
\usetheme{Hannover}
\usecolortheme{whale}
\author{Matthew Henderson, PhD, FCACB}
\date{\today}
\title{Congenital Hyperinsulinemia and Disorders of Glucose Transport}
\institute[NSO]{Newborn Screening Ontario | The University of Ottawa}
\titlegraphic{\includegraphics[height=1cm,keepaspectratio]{../logos/NSO_logo.pdf}\includegraphics[height=1cm,keepaspectratio]{../logos/cheo-logo.png} \includegraphics[height=1cm,keepaspectratio]{../logos/UOlogoBW.eps}}
\hypersetup{
 pdfauthor={Matthew Henderson, PhD, FCACB},
 pdftitle={Congenital Hyperinsulinemia and Disorders of Glucose Transport},
 pdfkeywords={},
 pdfsubject={},
 pdfcreator={Emacs 26.1 (Org mode 9.1.9)}, 
 pdflang={English}}
\begin{document}

\maketitle

%\logo{\includegraphics[width=1cm,height=1cm,keepaspectratio]{../logos/NSO_logo_small.pdf}~%
%    \includegraphics[width=1cm,height=1cm,keepaspectratio]{../logos/UOlogoBW.eps}%
%}

\vspace{220pt}
\beamertemplatenavigationsymbolsempty
\setbeamertemplate{caption}[numbered]
\setbeamerfont{caption}{size=\tiny}
% \addtobeamertemplate{frametitle}{}{%
% \begin{textblock*}{100mm}(.85\textwidth,-1cm)
% \includegraphics[height=1cm,width=2cm]{cat}
% \end{textblock*}}

\section{Congenital Hyperinsulinemia}
\label{sec:orgf637a50}
\begin{frame}[label={sec:org57dc44d}]{Insulin Secretion}
\begin{itemize}
\item glucose is transported into the pancreatic \(\beta\)-cell and phosphorylated to G-6-P by glucokinase
\begin{itemize}
\item GCK Km \(\sim\) [glucose] in  blood
\item functions as a glucose sensor
\end{itemize}
\item \(\uparrow\) glycolysis \(\to\) \(\uparrow\) ATP
\item \(\uparrow\) ATP/ADP ratio detected by ATP/ADP-sensitive potassium channels (K\(_{\text{ATP}}\))
\begin{itemize}
\item \(\to\) channel closure depolarization of the plasma membrane
\item \(\to\) voltage-sensitive Ca\(^{\text{2+}}\) channel opens
\item influx of extracellular Ca\(^{\text{2+}}\) stimulates insulin secretion by
exocytosis from storage granules
\end{itemize}
\end{itemize}
\end{frame}

\begin{frame}[label={sec:org2ea5dc5}]{Insulin Secretion}
\begin{itemize}
\item Other mechanisms regulate the release of insulin
\begin{enumerate}
\item transcription factors, such as HNF1A and HNF4A

\item metabolic factors which modulate the ATP production
\begin{itemize}
\item leucine (through the activation of the glutamate dehydrogenase (GDH,GLUD1)
\item short-chain L-3-hydroxyacyl-CoA dehydrogenase (SCHAD, HADH)
\item monocarboxylate transporter (MCT1,SLC16A1),
\item mitochondrial uncoupling protein 2 (UCP2)
\end{itemize}
\item receptors for various hormones and neuropeptides including:
\begin{itemize}
\item somatostatin, insulin, GLP1,GIP, etc.
\end{itemize}
\end{enumerate}
\end{itemize}
\end{frame}

\begin{frame}[label={sec:org7057a8e}]{Insulin Secretion}
\begin{figure}[htbp]
\centering
\includegraphics[width=0.9\textwidth]{./figures/insulin.png}
\caption[insulin]{\label{fig:org2720a25}
Insulin Secretion}
\end{figure}
\end{frame}

\begin{frame}[label={sec:org9ab7fa5}]{Insulin Effects}
\begin{itemize}
\item Activation of insulin receptors:
\begin{itemize}
\item \(\uparrow\) glucose utilization
\item \(\downarrow\) lipid utilization
\item cellular growth
\item translocates the glucose transporter GLUT4 to the PM
\end{itemize}
\item Cerebral cells are poorly insulin-sensitive
\begin{itemize}
\item highly dependent on circulating glucose
\begin{itemize}
\item in hyperinsulinism, there is a significant risk of brain damage
from neuroglucopenia.
\end{itemize}
\end{itemize}
\end{itemize}
\end{frame}

\begin{frame}[label={sec:org468fef2}]{Congenital Hyperinsulinism}
\begin{itemize}
\item CHI includes all genetic causes of hyperinsulinaemic
hypoglycaemia due to a primary defect of the pancreatic
\(\beta\)-cell
\item CHI can present throughout childhood, most common in infancy
\item Severe CHI is responsible for recurrent severe hypoglycaemia in neonates
\begin{itemize}
\item delayed diagnosis or inappropriate medical management is responsible for brain damage in about 1/3
\end{itemize}
\item Two main histopathological variants of CHI: diffuse and focal
\item Three forms: transient, syndromic and isolated congenital HI
\end{itemize}
\end{frame}

\begin{frame}[label={sec:orgc475543}]{Transient and Syndromic HI}
\begin{itemize}
\item Transient neonatal HI
\begin{itemize}
\item can occur in newborns from diabetic mothers
\item small for gestational age
\item due to perinatal stress such as fetal distress or following birth asphyxia
\item Hypoglycaemia can be severe
\begin{itemize}
\item usually resolves within a few days or months
\end{itemize}
\end{itemize}
\item Syndromic HI
\begin{itemize}
\item HI is part of a developmental syndrome.
\item Hypoglycaemia can be the initial manifestation of a number of
different syndromes during the neonatal period
\begin{itemize}
\item Beckwith Wiedemann Syndrome (BWS)
\item CDGs (PMM2-CDG and PMMI-CDG)
\item Kabuki syndrome
\item Sotos syndrome
\end{itemize}
\end{itemize}
\end{itemize}
\end{frame}

\begin{frame}[label={sec:orga38750b}]{Isolated Congenital HI}
\begin{itemize}
\item HI is inherited but occurs primarily as an isolated abnormality
\item Hypoglycaemia can reveal the disease in all ages
\item Hypoglycaemia occurs both in the fasting and the post-prandial states
\item Most neonates (86\%) are resistant to treatment with diazoxide
\end{itemize}
\end{frame}

\begin{frame}[label={sec:org8413106}]{Metabolic Derangement}
\begin{itemize}
\item functional defect of the pancreatic \(\beta\)-cells.
\item inappropriate secretion of insulin \(\to\) hypoglycaemia
\begin{itemize}
\item \(\downarrow\) hepatic glucose release from glycogen and gluconeogenesis
\item \(\uparrow\) glucose uptake in muscular and fatty tissues.
\end{itemize}
\item CHI is heterogeneous, caused by various defects in regulation of insulin secretion.
\begin{itemize}
\item channelopathies affecting the ATP channel (ABCC8 and KCNJ11 mutations)
\item metabolic defects:
\begin{itemize}
\item enzymes deficiencies: glucokinase, glutamate dehydrogenase, or SCHAD
\item transporter deficiencies: MCT1 (SLC16A1 mutations) and the mitochondrial uncoupling protein 2 (UCP2)
\end{itemize}
\item transcription factors impairment, such as HNF1A and HNF4A.
\item exceptional cases, defect in the signalling pathway of the insulin
receptor.
\end{itemize}
\end{itemize}
\end{frame}

\begin{frame}[label={sec:orge3261c2}]{Genetics}
\begin{itemize}
\item estimated incidence of severe CHI is 1 in 50,000 live births
\begin{itemize}
\item in countries with substantial consanguinity it may be as high as 1 in 2,500
\end{itemize}
\item The pattern of inheritance can be dominant or recessive,
\begin{itemize}
\item and the genetic abnormality sometimes occurs de novo.
\end{itemize}
\item In isolated CHI, the inheritance is:
\begin{itemize}
\item autosomal recessive for ABCC8, KCNJ11 and HADH gene mutations
\item autosomal dominant or de novo for GLUD1, GCK, UCP2,SLC16A1, HNF1A, HNF4A mutations
\begin{itemize}
\item some cases for ABCC8 and KCNJ11 mutations.
\end{itemize}
\end{itemize}
\end{itemize}
\end{frame}

\begin{frame}[label={sec:orgdc91d2d}]{Diagnosis}
\begin{itemize}
\item Diagnosis of HI relies on 5 non-essential criteria:
\begin{enumerate}
\item Fasting and/or post-prandial hypoglycaemia (<2.5–3 mmol/l).
\item Inappropriate plasma insulin levels and c-peptide at the time of
hypoglycaemia (potentially missed by a single sample because of
the pulsatile secretion of insulin).
\item Absent/low blood \& urine ketones bodies and non-esterified fatty
acids (NEFA). However, in some cases, ketones bodies and NEFA are
not totally suppressed.
\item An increase in blood glucose greater than 1.7 mmol/l (30 mg/dl)
within 30–40 minutes after SC/IM or IV administration of 1 mg
glucagon.
\item The need for a high glucose infusion rate (GIR) to keep blood
glucose above 3 mmol/l is characteristic of an insulin related
hypoglycaemia
\end{enumerate}

\item Once HI is established molecular can identify a gene
\end{itemize}
\end{frame}

\section{Disorders of Glucose Transport}
\label{sec:org6c74859}
\begin{frame}[label={sec:orgfc88ad4}]{Glucose Transporters}
\begin{itemize}
\item Glucose is hydrophilic \(\therefore\) cannot easily cross cell membrane
\item Transporters exist in almost all cell types
\item glucose transporter proteins can be divided into two groups:
\begin{enumerate}
\item Sodium-dependent glucose transporters (SGLTs)
\begin{itemize}
\item symporter systems, active transporters encoded by members of
the SLC5 gene family
\item couple sugar transport to the electrochemical gradient of sodium
\item transport glucose \(\uparrow\) [gradient].
\end{itemize}
\item Facilitative glucose transporters (GLUTs)
\begin{itemize}
\item uniporter systems, passive transporters encoded by members of the SLC2 gene family
\item transport glucose \(\downarrow\) [gradient].
\end{itemize}
\end{enumerate}
\end{itemize}
\end{frame}

\begin{frame}[label={sec:orgd820566}]{Glucose Transporters}
\begin{figure}[htbp]
\centering
\includegraphics[width=0.9\textwidth]{./figures/glut.png}
\caption[glucose transporters]{\label{fig:org17f1542}
Glucose Transporters}
\end{figure}
\end{frame}

\begin{frame}[label={sec:org578264e}]{Congenital Defects of Glucose Transporters}
\begin{itemize}
\item Five congenital defects of monosaccharide transporters
\item Their clinical picture depends on tissue-specific expression and
substrate specificity of the affected transporter.

\begin{enumerate}
\item SGLT1 : Congenital Glucose/Galactose Malabsorption
\item SGLT2 : Renal Glucosuria
\item GLUT1 : Glucose Transporter-1 Deficiency
\item GLUT2 : Fanconi-Bickel Syndrome
\item GLUT10 : Arterial Tortuosity Syndrome
\end{enumerate}
\end{itemize}
\end{frame}

\begin{frame}[label={sec:orgb48daaf}]{Congenital Glucose/Galactose Malabsorption (SGLT1 Deficiency )}
\begin{itemize}
\item SGLT1 is a high-affinity, low-capacity sodium-dependent transporter
of the two monosaccharides, at the brush border of enterocytes.

\item SGLT1 deficiency \(\to\) intestinal glucose-galactose malabsorption
\item GGM is a rare autosomal recessive disorder.

\item presents with severe osmotic diarrhoea and dehydration soon after a
normal birth and pregnancy
\begin{itemize}
\item patients develop severe hypertonic dehydration, often with fever
\item patients die from hypovolaemic shock.
\end{itemize}

\item Treatement is a glucose and galactose free diet
\end{itemize}
\end{frame}

\begin{frame}[label={sec:org6d89a6c}]{Renal Glucosuria (SGLT2 Deficiency)}
\begin{itemize}
\item SGLT2 is the major cotransporter involved in glucose reabsorption in
the kidney
\item SGLT2 mutations result in isolated renal glucosuria,
\begin{itemize}
\item a harmless renal transport defect characterised by:
\begin{itemize}
\item normal blood glucose concentrations
\item absence of any other signs of renal tubular dysfunction
\end{itemize}
\end{itemize}
\end{itemize}
\end{frame}

\begin{frame}[label={sec:org1860ef4}]{Glucose Transporter-1 Deficiency (GLUT1 Deficiency)}
\begin{itemize}
\item GLUT1 is a membrane-spanning, glycosylated protein that facilitates
glucose transport across the blood-brain barrier
\begin{itemize}
\item low CSF glucose concentration (hypoglycorrhachia)
\end{itemize}

\item clinical symptoms include: microcephaly, epileptic encephalopathies,
paroxysmal movement disorders, tremor
\item haemolytic anaemia has also been observed
\end{itemize}
\end{frame}

\begin{frame}[label={sec:orgcf1d03d}]{Glucose Transporter-1 Deficiency (GLUT1 Deficiency)}
\begin{itemize}
\item both AD and AR inheritance have been described

\item GLUT1D should be suspected in any child with a CSF glucose
concentration \textless{} 2.5 mmol/L (range range 0.9-2.9 mmol/l)
\begin{itemize}
\item normal \textgreater{} 3.3 mmol/L
\end{itemize}

\item CSF to blood glucose ratio \textless{} 0.5 (range 0.19-0.52)
\begin{itemize}
\item normal \textgreater{} 0.6
\item in the absence of hypoglycaemia or a CNS infection is diagnostic.
\end{itemize}
\item ketogenic diet is used in treatment
\end{itemize}
\end{frame}

\begin{frame}[label={sec:orgafccd33}]{Fanconi-Bickel Syndrome (GLUT2 Deficiency )}
\begin{itemize}
\item GLUT2 is a high-K\(_{\text{m}}\) monosaccharide carrier 
\begin{itemize}
\item located in hepatocytes
\item at the basolateral membrane of cells in the proximal tubule
\item pancreatic \(\beta\)-cells
\end{itemize}

\item Typically presents with a combination of increased hepatic
glycogen storage, generalised renal tubular dysfunction, severe glucosuria.

\item In FBS GLUT2 acts as a malfunctioning glucose sensor:
\begin{itemize}
\item in the fasted state, [glucose] and [G-6-P] are inappropriately \(\uparrow\) in hepatocytes
\item stimulates glycogen synthesis, inhibits gluconeogenesis and glycogenolysis
\item predisposes to hypoglycaemia and hepatic glycogen accumulation
\end{itemize}

\item very rare autosomal recessive condition caused by mutations in
SLC2A2.
\end{itemize}
\end{frame}

\begin{frame}[label={sec:orgbe855b3}]{Fanconi-Bickel Syndrome (GLUT2 Deficiency )}
\begin{itemize}
\item Diagnosis suggested by the characteristic combination of an altered
glucose homeostasis, hepatic glycogen accumulation, and the typical
features of a Fanconi-type tubulopathy.

\item Elevated biotinidase activity in serum has been found to be a useful
screening test for hepatic glycogen storage disorders including FBS.

\item Only symptomatic treatment is available.
\end{itemize}
\end{frame}

\begin{frame}[label={sec:orgf0dbe4d}]{Arterial Tortuosity Syndrome (GLUT10 Deficiency)}
\begin{itemize}
\item GLUT10 function not entirely clear:
\begin{itemize}
\item localizes to mitochondria of smooth muscle and insulin-stimulated adipocytes
\item facilitates transport of l-dehydroascorbic acid (DHA), the oxidized
form of vitamin C, into mitochondria,
\end{itemize}

\item GLUT10 deficiency is characterised by hyperelastic connective tissue
and generalised tortuosity and elongation of all major arteries
including the aorta
\end{itemize}
\end{frame}

\begin{frame}[label={sec:org3db0c7c}]{Arterial Tortuosity Syndrome (GLUT10 Deficiency)}
\begin{itemize}
\item presents with acute infarction owing to ischaemic stroke or an
increased risk of thromboses.
\item Aortic regurgitation and multiple pulmonary artery stenoses are
typical intrathoracic manifestations.
\item closely resembles a connective tissue disorder in presentation.

\item rare, AR GLUT10 (SLC2A10)
\item Echocardiography, angiography, and/or CT scan are important to demonstrate vascular changes.
\item Diagnosis is based on molecular genetic methods
\end{itemize}
\end{frame}
\end{document}
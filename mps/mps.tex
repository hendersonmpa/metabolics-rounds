% Created 2018-09-05 Wed 21:51
% Intended LaTeX compiler: pdflatex
\documentclass[presentation, smaller]{beamer}
\usepackage[utf8]{inputenc}
\usepackage[T1]{fontenc}
\usepackage{graphicx}
\usepackage{grffile}
\usepackage{longtable}
\usepackage{wrapfig}
\usepackage{rotating}
\usepackage[normalem]{ulem}
\usepackage{amsmath}
\usepackage{textcomp}
\usepackage{amssymb}
\usepackage{capt-of}
\usepackage{hyperref}
\hypersetup{colorlinks,linkcolor=white,urlcolor=blue}
\usepackage{textpos}
\usepackage{textgreek}
\usepackage[version=4]{mhchem}
\usepackage{chemfig}
\usepackage{siunitx}
\usepackage{gensymb}
\usepackage[usenames,dvipsnames]{xcolor}
\usepackage[T1]{fontenc}
\usepackage{lmodern}
\usepackage{verbatim}
\usepackage{tikz}
\usetikzlibrary{shapes.geometric,arrows,decorations.pathmorphing,backgrounds,positioning,fit,petri}
\usetheme{Hannover}
\usecolortheme{whale}
\author{Matthew Henderson, PhD, FCACB}
\date{\today}
\title{Mucopolysaccharidoses}
\institute[NSO]{Newborn Screening Ontario | The University of Ottawa}
\titlegraphic{\includegraphics[height=1cm,keepaspectratio]{../logos/NSO_logo.pdf}\includegraphics[height=1cm,keepaspectratio]{../logos/cheo-logo.png} \includegraphics[height=1cm,keepaspectratio]{../logos/UOlogoBW.eps}}
\hypersetup{
 pdfauthor={Matthew Henderson, PhD, FCACB},
 pdftitle={Mucopolysaccharidoses},
 pdfkeywords={},
 pdfsubject={},
 pdfcreator={Emacs 26.1 (Org mode 9.1.9)}, 
 pdflang={English}}
\begin{document}

\maketitle

\begin{LaTeX}


\vspace{220pt}
\beamertemplatenavigationsymbolsempty
\setbeamertemplate{caption}[numbered]
\setbeamerfont{caption}{size=\tiny}

\tikzstyle{chemical} = [rectangle, rounded corners, text width=5em, minimum height=1em,text centered, draw=black, fill=none]
\tikzstyle{hardware} = [rectangle, rounded corners, text width=5em, minimum height=1em,text centered, draw=black, fill=gray!30]
\tikzstyle{ms} = [rectangle, rounded corners, text width=5em, minimum height=1em,text centered, draw=orange, fill=none]
\tikzstyle{msw} = [rectangle, rounded corners, text width=7em, minimum height=1em,text centered, draw=orange, fill=none]
\tikzstyle{label} = [rectangle,text width=8em, minimum height=1em, text centered, draw=none, fill=none]
\tikzstyle{hl} = [rectangle, rounded corners, text width=5em, minimum height=1em,text centered, draw=black, fill=red!30]
\tikzstyle{box} = [rectangle, rounded corners, text width=5em, minimum height=5em,text centered, draw=black, fill=none]
\tikzstyle{arrow} = [thick,->,>=stealth]
\tikzstyle{hl-arrow} = [ultra thick,->,>=stealth,draw=red]
\end{LaTeX}


\section{Introduction}
\label{sec:org9c666fb}
\begin{frame}[label={sec:orgb7a4dbe}]{Proteoglycans function}
\begin{itemize}
\item Structural proteins of the ECM are embedded in gels formed from
proteoglycans.
\item Composed of glycosaminoglycans (GAGs) linked to a protien core.
\item Negatively charged GAGs bind Na\(^{\text{+}}\)
\begin{itemize}
\item draws water to create a gel
\end{itemize}
\item Found in interstitial connective tissues such as: 
\begin{itemize}
\item synovial fluid
\item vitreous humour, cornea
\item arterial walls
\item bone, cartilage
\end{itemize}
\end{itemize}
\end{frame}


\begin{frame}[label={sec:org43daa8b}]{Proteoglycans synthesis}
\begin{center}
\includegraphics[width=0.8\textwidth]{./figures/ch3f1.jpg}
\label{org7480fb9}
\end{center}
\end{frame}

\begin{frame}[label={sec:org605e97e}]{Glycosaminoglycans}
\begin{enumerate}
\item heparin
\item heparin sulfate
\item chondroitin sulfate
\item dermatan sulfate
\item keratan sulfate
\item hyaluronan (not typically protein bound)
\end{enumerate}


\begin{itemize}
\item GAGs are composed of repeating units of disaccharides.
\begin{itemize}
\item hexosamine and a hexose or hexuronic acid
\end{itemize}
\end{itemize}
\end{frame}

\begin{frame}[label={sec:orga369c3d}]{GAGs: Dermatan sulfate}
\begin{itemize}
\item O-xylose-linked to core proteins
\item found mostly in skin,
\begin{itemize}
\item also blood vessels, heart valves, tendons, and lungs.
\end{itemize}
\item may have roles in coagulation, cardiovascular disease, carcinogenesis, infection, wound repair, and fibrosis
\end{itemize}

\begin{center}
\includegraphics[width=0.5\textwidth]{./figures/dermatan_sulfate.png}
\label{org3f5b39b}
\end{center}
\end{frame}


\begin{frame}[label={sec:org6a74776}]{GAGs:Heparin Sulfate}
\begin{itemize}
\item O-xylose-linked to core proteins
\item occurs as a proteoglycan in which two or three HS chains are
attached in close proximity to cell surface or extracellular matrix
proteins.
\item developmental processes,
\item angiogenesis
\item blood coagulation
\item tumour metastasis.
\end{itemize}

\begin{center}
\includegraphics[width=0.5\textwidth]{./figures/heparin_sulfate.png}
\label{org2d0db3b}
\end{center}
\end{frame}

\begin{frame}[label={sec:org4fe44ad}]{GAGs:Keratan sulfate}
\begin{itemize}
\item KSI was isolated from corneal tissue and KSII from skeletal tissue
\begin{itemize}
\item KSI is N-linked to specific asparagine amino acids via
N-acetylglucosamine
\item KSII is O-linked to specific Serine or Threonine amino acids via
N-acetyl galactosamine.
\item cornea, cartilage, and bone
\item synthesized in the central nervous system where it participates both
in development
\end{itemize}
\end{itemize}

\begin{center}
\includegraphics[width=0.5\textwidth]{./figures/keratan_sulfate.png}
\label{orgc2cff85}
\end{center}
\end{frame}

\section{Mucopolysaccharidoses}
\label{sec:orge93f51f}

\begin{frame}[label={sec:orgea366f4}]{Mucopolysaccharidoses}
\begin{itemize}
\item type of lysosomal storage disease
\item a group of metabolic disorders caused by the absence or
malfunctioning of lysosomal enzymes needed to break down
glycosaminoglycans.
\item can be a result of decreased expression, stability, and activity of
one of the eleven enzymes required for glycosaminoglycans
degradation
\item GAGs collect in the cells, blood and connective tissues.
\begin{itemize}
\item The result is permanent, progressive cellular damage which affects:
\begin{itemize}
\item appearance
\item physical abilities
\item organ and system functioning,
\item in most cases, mental development.
\end{itemize}
\end{itemize}
\end{itemize}
\end{frame}

\begin{frame}[label={sec:orga4c39d6}]{Classification}
\begin{itemize}
\item Presenting as a dysmorphic syndrome
\begin{itemize}
\item MPS I (Hurler)
\item MPS II (Hunter)
\item MPS VI (Maroteaux-Lamy)
\end{itemize}
\item Presenting with learning difficulties, behavioral disturbances and dementia
\begin{itemize}
\item MPS III (Sanfilippo)
\end{itemize}
\item Presenting with severe bone dysplasia
\begin{itemize}
\item MPS IV (Morquio)
\end{itemize}
\item Others rare
\begin{itemize}
\item MPS VII (Sly)
\item MPS IX (Natowicz)
\end{itemize}
\end{itemize}
\end{frame}



\begin{frame}[label={sec:org611a63f}]{MPS GAGs}
\begin{center}
\begin{tabular}{lll}
Name & Enzyme & GAG\\
\hline
MPS I (Hurler) & \(\alpha\)-iduronidase & HS,DS\\
MPS II (Hunter) & Iduronate-2-sulfatase & HS,DS\\
\hline
MPS IIIA (Sanfilippo A) & Heparin-N-Sulfatase & HS\\
MPS IIIB (Sanfilippo B) & N-acetyl glucosaminidase & HS\\
MPS IIIC (Sanfilippo C) & Acetyl CoA glucosamine & HS\\
 & N-acetyltransferase & \\
MPS IIID (Sanfilippo D) & N-acetyl-glucosamine & HS\\
 & 6-sulfatase & \\
\hline
MPS IVA (Morquio A) & N-acetylgalactosamine & KS,CS\\
 & 6-sulfatase & \\
MPS IVB (Morquio B) & \(\beta\)-galactosidase & KS\\
\hline
MPS VI (Maroteaux-Lamy) &  & DS\\
MPS VII (Sly) & \(\beta\)-glucuronidase & DS,HS,CS\\
MPS IX & hyaluronidase & HA\\
MSD (Austin) & formylglycine-generating & HS,DS\\
\end{tabular}
\end{center}
\end{frame}


\begin{frame}[label={sec:org2e3f9a5}]{Glycosaminoglycan degradation}
\begin{center}
\includegraphics[width=0.8\textwidth]{./figures/ch16f9.jpg}
\label{org1713e00}
\end{center}
\end{frame}


\begin{frame}[label={sec:orgadc67de}]{Dermatan sulfate degradation}
\begin{center}
\includegraphics[width=0.6\textwidth]{./figures/ds_degradation.jpg}
\label{org848dc6f}
\end{center}
\end{frame}


\begin{frame}[label={sec:org8328f4a}]{Keratan sulfate degradation}
\begin{center}
\includegraphics[width=0.6\textwidth]{./figures/ks_degradation.jpg}
\label{org15c9c5f}
\end{center}
\end{frame}


\begin{frame}[label={sec:org9d27f56}]{Heparin degradation}
\begin{center}
\includegraphics[width=0.5\textwidth]{./figures/hs_degradation.jpg}
\label{org399b88c}
\end{center}
\end{frame}
\end{document}

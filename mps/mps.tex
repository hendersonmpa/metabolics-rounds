% Created 2018-09-06 Thu 12:47
\documentclass[presentation, smaller]{beamer}
\usepackage[utf8]{inputenc}
\usepackage[T1]{fontenc}
\usepackage{fixltx2e}
\usepackage{graphicx}
\usepackage{grffile}
\usepackage{longtable}
\usepackage{wrapfig}
\usepackage{rotating}
\usepackage[normalem]{ulem}
\usepackage{amsmath}
\usepackage{textcomp}
\usepackage{amssymb}
\usepackage{capt-of}
\usepackage{hyperref}
\hypersetup{colorlinks,linkcolor=white,urlcolor=blue}
\usepackage{textpos}
\usepackage{textgreek}
\usepackage[version=4]{mhchem}
\usepackage{chemfig}
\usepackage{siunitx}
\usepackage{gensymb}
\usepackage[usenames,dvipsnames]{xcolor}
\usepackage[T1]{fontenc}
\usepackage{lmodern}
\usepackage{verbatim}
\usepackage{tikz}
\usetikzlibrary{shapes.geometric,arrows,decorations.pathmorphing,backgrounds,positioning,fit,petri}
\usetheme{Hannover}
\usecolortheme{whale}
\author{Matthew Henderson, PhD, FCACB}
\date{\today}
\title{Mucopolysaccharidoses Overview}
\institute[NSO]{Newborn Screening Ontario | The University of Ottawa}
\titlegraphic{\includegraphics[height=1cm,keepaspectratio]{../logos/NSO_logo.pdf}\includegraphics[height=1cm,keepaspectratio]{../logos/cheo-logo.png} \includegraphics[height=1cm,keepaspectratio]{../logos/UOlogoBW.eps}}
\hypersetup{
 pdfauthor={Matthew Henderson, PhD, FCACB},
 pdftitle={Mucopolysaccharidoses Overview},
 pdfkeywords={},
 pdfsubject={},
 pdfcreator={Emacs 25.2.1 (Org mode 8.3.4)}, 
 pdflang={English}}
\begin{document}

\maketitle


# %\logo{\includegraphics[width=1cm,height=1cm,keepaspectratio]{../logos/NSO_logo_small.pdf}~%
# %    \includegraphics[width=1cm,height=1cm,keepaspectratio]{../logos/UOlogoBW.eps}%
# }

\vspace{220pt}
\beamertemplatenavigationsymbolsempty
\setbeamertemplate{caption}[numbered]
\setbeamerfont{caption}{size=\tiny}
#  \addtobeamertemplate{frametitle}{}{%
#  \begin{textblock*}{100mm}(.85\textwidth,-1cm)
#  \includegraphics[height=1cm,width=2cm]{cat}
#  \end{textblock*}}

\tikzstyle{chemical} = [rectangle, rounded corners, text width=5em, minimum height=1em,text centered, draw=black, fill=none]
\tikzstyle{hardware} = [rectangle, rounded corners, text width=5em, minimum height=1em,text centered, draw=black, fill=gray!30]
\tikzstyle{ms} = [rectangle, rounded corners, text width=5em, minimum height=1em,text centered, draw=orange, fill=none]
\tikzstyle{msw} = [rectangle, rounded corners, text width=7em, minimum height=1em,text centered, draw=orange, fill=none]
\tikzstyle{label} = [rectangle,text width=8em, minimum height=1em, text centered, draw=none, fill=none]
\tikzstyle{hl} = [rectangle, rounded corners, text width=5em, minimum height=1em,text centered, draw=black, fill=red!30]
\tikzstyle{box} = [rectangle, rounded corners, text width=5em, minimum height=5em,text centered, draw=black, fill=none]
\tikzstyle{arrow} = [thick,->,>=stealth]
\tikzstyle{hl-arrow} = [ultra thick,->,>=stealth,draw=red]

\section{Introduction}
\label{sec:orgheadline9}
\begin{frame}[label={sec:orgheadline1}]{Proteoglycans}
\includegraphics[width=0.8\textwidth]{./figures/ch17f01.jpg}
\end{frame}

\begin{frame}[label={sec:orgheadline2}]{Proteoglycans function}
\begin{itemize}
\item Structural proteins of the ECM are embedded in gels formed from
proteoglycans.
\item Composed of glycosaminoglycans (GAGs) linked to a protien core.
\item Negatively charged GAGs bind Na\(^{\text{+}}\)
\begin{itemize}
\item draws water to create a gel
\end{itemize}
\item Found in interstitial connective tissues such as: 
\begin{itemize}
\item synovial fluid
\item vitreous humour, cornea
\item arterial walls
\item bone, cartilage
\end{itemize}
\end{itemize}
\end{frame}


\begin{frame}[label={sec:orgheadline3}]{Proteoglycans synthesis}
\includegraphics[width=0.8\textwidth]{./figures/ch3f1.jpg}
\end{frame}


\begin{frame}[label={sec:orgheadline4}]{Glycosaminoglycans}
\begin{enumerate}
\item heparin
\item heparin sulfate
\item chondroitin sulfate
\item dermatan sulfate
\item keratan sulfate
\item hyaluronan (not typically protein bound)
\end{enumerate}


\begin{itemize}
\item GAGs are composed of repeating units of disaccharides.
\begin{itemize}
\item hexosamine and a hexose or hexuronic acid
\end{itemize}
\end{itemize}
\end{frame}

\begin{frame}[label={sec:orgheadline5}]{Symbol Nomenclature for Glycans (SNFG)}
\includegraphics[width=0.8\textwidth]{./figures/snfg.png}
\end{frame}


\begin{frame}[label={sec:orgheadline6}]{Glycosaminoglycans}
\includegraphics[width=0.8\textwidth]{./figures/ch17f02.jpg}
\end{frame}

\begin{frame}[label={sec:orgheadline7}]{GAGs}
\begin{block}{Hyaluronan}
\begin{itemize}
\item not sulfated, not protein bound
\item forms in the plasma membrane instead of the Golgi
\item connective, epithelial, and neural tissues.
\item cell proliferation and migration
\end{itemize}
\end{block}

\begin{block}{Chondroitin sulfate}
\begin{itemize}
\item O-xylose-linked to core proteins
\item structural component of cartilage
\end{itemize}
\end{block}

\begin{block}{Dermatan sulfate}
\begin{itemize}
\item O-xylose-linked to core proteins
\item skin, blood vessels, heart valves, tendons, and lungs.O
\item coagulation, cardiovascular disease, carcinogenesis, infection, wound repair, and fibrosis
\end{itemize}
\end{block}
\end{frame}

\begin{frame}[label={sec:orgheadline8}]{GAGs}
\begin{block}{Heparin Sulfate}
\begin{itemize}
\item O-xylose-linked to core proteins
\item developmental processes,angiogenesis, blood coagulation, tumour metastasis.
\end{itemize}
\end{block}

\begin{block}{Keratan sulfate}
\begin{itemize}
\item KSI was isolated from corneal tissue and KSII from skeletal tissue
\begin{itemize}
\item KSI is N-linked to specific asparagine amino acids via
N-acetylglucosamine
\item KSII is O-linked to specific Serine or Threonine amino acids via
N-acetyl galactosamine.
\item cornea, cartilage,bone, CNS
\end{itemize}
\end{itemize}
\end{block}
\end{frame}


\section{Mucopolysaccharidoses}
\label{sec:orgheadline18}

\begin{frame}[label={sec:orgheadline10}]{Mucopolysaccharidoses}
\begin{itemize}
\item type of lysosomal storage disease
\item a group of metabolic disorders caused by the absence or
malfunctioning of lysosomal enzymes needed to break down
glycosaminoglycans.
\item can be a result of decreased expression, stability, and activity of
one of the eleven enzymes required for glycosaminoglycans
degradation
\item GAGs collect in the cells, blood and connective tissues.
\begin{itemize}
\item The result is permanent, progressive cellular damage which affects:
\begin{itemize}
\item appearance
\item physical abilities
\item organ and system functioning,
\item in most cases, mental development.
\end{itemize}
\end{itemize}
\end{itemize}
\end{frame}

\begin{frame}[label={sec:orgheadline11}]{Glycosaminoglycan degradation}
\includegraphics[width=0.8\textwidth]{./figures/ch16f9.jpg}
\end{frame}


\begin{frame}[label={sec:orgheadline12}]{Mucopolysaccharidoses}
\small
\begin{center}
\begin{tabular}{lll}
Name & Enzyme & GAG\\
\hline
MPS I (Hurler) & \(\alpha\)-iduronidase & HS,DS\\
MPS II (Hunter) & Iduronate-2-sulfatase & HS,DS\\
\hline
MPS IIIA (Sanfilippo A) & Heparin-N-Sulfatase & HS\\
MPS IIIB (Sanfilippo B) & N-acetyl glucosaminidase & HS\\
MPS IIIC (Sanfilippo C) & Acetyl CoA glucosamine & HS\\
 & N-acetyltransferase & \\
MPS IIID (Sanfilippo D) & N-acetyl-glucosamine & HS\\
 & 6-sulfatase & \\
\hline
MPS IVA (Morquio A) & N-acetylgalactosamine & KS,CS\\
 & 6-sulfatase & \\
MPS IVB (Morquio B) & \(\beta\)-galactosidase & KS\\
\hline
MPS VI (Maroteaux-Lamy) & N-acetylgalactosamine & DS\\
 & 4-sulfatase & \\
MPS VII (Sly) & \(\beta\)-glucuronidase & DS,HS,CS\\
MPS IX & hyaluronidase & HA\\
MSD (Austin) & formylglycine-generating & HS,DS\\
 & enzyme & \\
\end{tabular}
\end{center}
\end{frame}



\begin{frame}[label={sec:orgheadline13}]{Classification}
\begin{itemize}
\item Presenting as a dysmorphic syndrome
\begin{itemize}
\item MPS I (Hurler)
\item MPS II (Hunter)
\item MPS VI (Maroteaux-Lamy)
\end{itemize}
\item Presenting with learning difficulties, behavioral disturbances and dementia
\begin{itemize}
\item MPS III (Sanfilippo)
\end{itemize}
\item Presenting with severe bone dysplasia
\begin{itemize}
\item MPS IV (Morquio)
\end{itemize}
\item Others rare
\begin{itemize}
\item MPS VII (Sly)
\item MPS IX (Natowicz)
\end{itemize}
\end{itemize}
\end{frame}



\begin{frame}[label={sec:orgheadline14}]{Dermatan sulfate degradation}
\includegraphics[width=0.6\textwidth]{./figures/ds_degradation_disorders.png}
\end{frame}


\begin{frame}[label={sec:orgheadline15}]{Keratan sulfate degradation}
\includegraphics[width=0.6\textwidth]{./figures/ks_degradation_disorders.png}
\end{frame}


\begin{frame}[label={sec:orgheadline16}]{Heparin degradation}
\includegraphics[width=0.5\textwidth]{./figures/hs_degradation_disorders.png}
\end{frame}


\begin{frame}[label={sec:orgheadline17}]{Next Up}
\begin{itemize}
\item NBS for MPS I
\end{itemize}
\end{frame}
\end{document}

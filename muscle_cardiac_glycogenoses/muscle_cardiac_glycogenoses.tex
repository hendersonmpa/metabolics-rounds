% Created 2019-09-10 Tue 14:41
% Intended LaTeX compiler: pdflatex
\documentclass[presentation, smaller]{beamer}
\usepackage[utf8]{inputenc}
\usepackage[T1]{fontenc}
\usepackage{graphicx}
\usepackage{grffile}
\usepackage{longtable}
\usepackage{wrapfig}
\usepackage{rotating}
\usepackage[normalem]{ulem}
\usepackage{amsmath}
\usepackage{textcomp}
\usepackage{amssymb}
\usepackage{capt-of}
\usepackage{hyperref}
\hypersetup{colorlinks,linkcolor=white,urlcolor=blue}
\usepackage{textpos}
\usepackage{textgreek}
\usepackage[version=4]{mhchem}
\usepackage{chemfig}
\usepackage{siunitx}
\usepackage{gensymb}
\usepackage[usenames,dvipsnames]{xcolor}
\usepackage[T1]{fontenc}
\usepackage{lmodern}
\usepackage{verbatim}
\usepackage{tikz}
\usepackage{wasysym}
\usetikzlibrary{shapes.geometric,arrows,decorations.pathmorphing,backgrounds,positioning,fit,petri}
\usetheme{Hannover}
\usecolortheme{whale}
\author{Matthew Henderson, PhD, FCACB}
\date{\today}
\title{Muscle \& Cardiac Glycogenoses}
\institute[NSO]{Newborn Screening Ontario | The University of Ottawa}
\titlegraphic{\includegraphics[height=1cm,keepaspectratio]{../logos/NSO_logo.pdf}\includegraphics[height=1cm,keepaspectratio]{../logos/cheo-logo.png} \includegraphics[height=1cm,keepaspectratio]{../logos/UOlogoBW.eps}}
\hypersetup{
 pdfauthor={Matthew Henderson, PhD, FCACB},
 pdftitle={Muscle \& Cardiac Glycogenoses},
 pdfkeywords={},
 pdfsubject={},
 pdfcreator={Emacs 26.1 (Org mode 9.1.9)}, 
 pdflang={English}}
\begin{document}

\maketitle


\section{Introduction}
\label{sec:orged8bd5c}
\begin{frame}[label={sec:orgb2364d4}]{Muscle and Cardiac Glycogenoses}
\begin{itemize}
\item at rest muscle predominately utilizes fatty acids.
\begin{itemize}
\item uses blood glucose during sub-maximal exercise.
\end{itemize}
\item Energy is supplied by glycogenolysis during intense exercise.
\begin{itemize}
\item glycogenolysis supplies glucose for anaerobic glycolysis
\end{itemize}
\end{itemize}
\end{frame}

\begin{frame}[label={sec:org5262fab}]{Muscle and Cardiac Glycogenoses}
\scriptsize
\begin{center}
\begin{tabular}{llll}
Type & Enzyme & Gene & Phenotype\\
\hline
0b & Muscle glycogen synthase & GYS1 & cardiomyopathy \& myopathy\\
II (Pompe) & Acid \(\alpha\)-glucosidase & GAA & hypotonia, muscle dysfunction\\
III & Glycogen debrancher & AGL & hypoglycema, hepatomegaly\\
V (McArdle) & Muscle Glycogen Phosphorylase & PYGM & Excercise intolerance\\
Danon & LAMP2 & LAMP2 & cardiomyopathy, \textpm{} skeletal myopathy\\
AMPK & AMP-activated Protein Kinase & PRKAG2 & cardiomyopathy, \textpm{} skeletal myopathy\\
\end{tabular}
\end{center}
\end{frame}

\begin{frame}[label={sec:org0978dac}]{Glycogenoses}
\begin{figure}[htbp]
\centering
\includegraphics[width=0.75\textwidth]{./figures/gggmetab.png}
\caption[Glycogenoses]{\label{fig:org6d5dba5}
Glycogenoses}
\end{figure}
\end{frame}

\begin{frame}[label={sec:orgadeceb1}]{Muscle and Cardiac Glycogenoses}
\begin{figure}[htbp]
\centering
\includegraphics[width=0.75\textwidth]{./figures/gggmetab_muscle_cardiac.png}
\caption[Muscle and Cardiac Glycogenoses]{\label{fig:orgf30e5f8}
Muscle and Cardiac Glycogenoses}
\end{figure}
\end{frame}

\section{GSD Type 0}
\label{sec:org06431e3}
\begin{frame}[label={sec:orgd65057d}]{GSD Type 0}
\begin{figure}[htbp]
\centering
\includegraphics[width=0.75\textwidth]{./figures/gggmetab_muscle_cardiac.png}
\caption[Muscle, Cardiac Glycogenoses]{\label{fig:orgf41776c}
Muscle, Cardiac Glycogenoses}
\end{figure}
\end{frame}


\begin{frame}[label={sec:org9946010}]{GSD Type 0}
\begin{block}{Metabolic Derangement}
\begin{itemize}
\item Deficiency in muscle glycogen synthase
\item Ubiquitously expressed
\end{itemize}
\end{block}

\begin{block}{Genetics}
\begin{itemize}
\item AR, GYS1 encodes muscle isoform of GS
\end{itemize}
\end{block}

\begin{block}{Clinical Presentation}
\begin{itemize}
\item muscle fatigue
\item hypertrophic cardiomyopathy
\end{itemize}
\end{block}

\begin{block}{Diagnostic Tests}
\begin{itemize}
\item mutation analysis
\end{itemize}
\end{block}
\begin{block}{Treatment}
\begin{itemize}
\item \(\beta_{\text{1}}\)-receptor blockage for cardiac protection
\end{itemize}
\end{block}
\end{frame}

\section{GSD Type II (Pompe)}
\label{sec:org467f51d}
\begin{frame}[label={sec:org15d4cc5}]{GSD Type II (Pompe)}
\begin{figure}[htbp]
\centering
\includegraphics[width=0.75\textwidth]{./figures/gggmetab_muscle_cardiac.png}
\caption[Muscle, Cardiac Glycogenoses]{\label{fig:org0371dfb}
Muscle, Cardiac Glycogenoses}
\end{figure}
\end{frame}

\begin{frame}[label={sec:orgbebdf83}]{Metabolic Derangement}
\begin{itemize}
\item Acid \(\alpha\)-glucosidase deficiency
\item accumulation of glycogen within the lysosomes
\item with different critical thresholds depending on the organ
\end{itemize}
\end{frame}

\begin{frame}[label={sec:orgdbbb809}]{Genetics}
\begin{itemize}
\item AR, AGL gene
\item 200 mutations have been reported in GAA
\begin{itemize}
\item \textasciitilde{} 75\% of these being pathogenic mutations (www.pompecenter.nl)
\end{itemize}
\item some genotype-phenotype correlation
\begin{itemize}
\item severe mutations (such as del exon18) associated with the infantile form
\item ›leaky‹ mutations associated with the adult variant
\end{itemize}
\item c.-32-13T>G is the most common mutation in adults and children with
a slowly progressive course
\begin{itemize}
\item approximately 80\% of Caucasian patients.
\end{itemize}
\end{itemize}
\end{frame}

\begin{frame}[label={sec:orgc45513c}]{Clinical Presentation}
\begin{block}{Infantile}
\begin{itemize}
\item first months of life with hypotonia and hypertrophic cardiomyopathy
\item also dysphagia, smooth muscle dysfunction, enlargement of the tongue
and liver
\item Most untreated infantile onset patients die from cardiopulmonary
failure or aspiration pneumonia prior to one year of age
\end{itemize}
\end{block}
\begin{block}{Juvenile}
\begin{itemize}
\item predominant skeletal muscle dysfunction
\begin{itemize}
\item with motor and respiratory problems, rarely cardiac involvement.
\item Calf hypertrophy can be present, mimicking Duchenne muscular dystrophy in boys.
\end{itemize}
\item Myopathy and respiratory insufficiency deteriorate gradually, and patients may become dependent on a ventilator or wheelchair.
\end{itemize}
\end{block}
\begin{block}{Adult}
\begin{itemize}
\item 3rd or 4th decade and affects the trunk and proximal limb muscles
\begin{itemize}
\item mimicks inherited limb-girdle muscle dystrophies.
\end{itemize}
\item Involvement of the diaphragm is frequent,
\begin{itemize}
\item acute respiratory failure may be the initial symptom in some patients.
\end{itemize}
\item the heart is generally not affected.
\end{itemize}
\end{block}
\end{frame}
\begin{frame}[label={sec:org2fb24c3}]{Diagnostic Tests}
\begin{itemize}
\item Acid \(\alpha\)-glucosidase enzyme assay
\begin{itemize}
\item classic infantile \textasciitilde{} 1\% residual activity
\item Children and Adults \(\le\) 30\% activity
\end{itemize}
\item Skin Fibroblasts are best tissue
\begin{itemize}
\item Lower biochemical interferences (neutral \(\alpha\)-glucosidases)
\end{itemize}
\item mutation analysis
\end{itemize}
\end{frame}
\begin{frame}[label={sec:org808c450}]{Treatment}
\begin{itemize}
\item Recombinant acid \(\alpha\)-glucosidase (rhGAA)
\begin{itemize}
\item CHO cells (alglucosidase alfa)
\end{itemize}
\item Anti rhGAA IgG antibodies form
\item 1/3 of ERT treated were ventilator free
\item Better outcome if identified by NBS
\end{itemize}
\end{frame}

\section{GSD Type III}
\label{sec:org0664e2e}
\begin{frame}[label={sec:org051d1e1}]{GSD Type III}
\begin{figure}[htbp]
\centering
\includegraphics[width=0.75\textwidth]{./figures/gggmetab_muscle_cardiac.png}
\caption[Muscle, Cardiac Glycogenoses]{\label{fig:org77ebf82}
Muscle, Cardiac Glycogenoses}
\end{figure}
\end{frame}

\begin{frame}[label={sec:orga90f773}]{Metabolic Derangement}
\begin{itemize}
\item Glycogen debrancher enzyme (GDE) deficiency
\item has both glucosidase and transferase activity
\begin{itemize}
\item cleaves \(\alpha\)-1,4 glucose linkages of the terminal glucose
\item then breaks \(\alpha\)-1,6 linkage to remove branch point
\end{itemize}
\item accumulation of abnormal glycogen
\item limited glucose release from glycogen
\item gluconeogenesis functions normally
\end{itemize}
\end{frame}
\begin{frame}[label={sec:orgea0d703}]{Genetics}
\begin{itemize}
\item AR, AGL gene
\item mutations occur throughout AGL (GSD IIIa)
\begin{itemize}
\item defect in liver and muscle
\end{itemize}
\item two specific mutations in exon 3 (GSD IIIb)
\begin{itemize}
\item liver only
\end{itemize}
\end{itemize}
\end{frame}
\begin{frame}[label={sec:orge4354ce}]{Clinical Presentation}
\begin{itemize}
\item Hepatic glycogenosis and (in most cases) also myopathic
\item First year with poor growth, delayed motor milestones and abdominal
distension
\item Fasting hypoglycaemia 
\begin{itemize}
\item Fasting tolerance is usually longer than in GSD I
\end{itemize}
\item Fasting ketosis is prominent.
\item Gluconeogenesis is normal \(\therefore\) no fasting hyperlactataemia
\item Moderate post-prandial \(\uparrow\) lactate
\item Hyperlipdaemia
\item \(\uparrow\) \(\uparrow\) \(\uparrow\) liver transaminases
\item \(\uparrow\) CK in myopathic form
\end{itemize}
\end{frame}
\begin{frame}[label={sec:org0d580d0}]{Diagnostic Tests}
\begin{itemize}
\item DBE activity in leucocytes
\item mutation analysis
\end{itemize}
\end{frame}
\begin{frame}[label={sec:org077111e}]{Treatment}
\begin{itemize}
\item Aim is to maintain normoglycaemia, reduce the hyperlipidaemia and ketosis and
ensure adequate growth.
\item Regular meals and uncooked cornstarch
\item Overnight continuous feeding is less commonly needed in GSD III than
in GSD I
\item Long term outcome for individuals with GSD III is generally good
with survival into adulthood.
\end{itemize}
\end{frame}

\section{GSD Type V}
\label{sec:org0ff6bbe}
\begin{frame}[label={sec:orge6316ed}]{GSD Type V}
\begin{figure}[htbp]
\centering
\includegraphics[width=0.75\textwidth]{./figures/gggmetab_muscle_cardiac.png}
\caption[Muscle, Cardiac Glycogenoses]{\label{fig:org7074efc}
Muscle, Cardiac Glycogenoses}
\end{figure}
\end{frame}

\begin{frame}[label={sec:orgb1554ef}]{Metabolic Derangement}
\begin{itemize}
\item There are three isoforms of glycogen phosphorylase: brain/heart,
liver and muscle, all encoded by different genes.
\item GSD V is caused by deficient myophosphorylase activity.
\end{itemize}
\end{frame}

\begin{frame}[label={sec:org88d5288}]{Genetics}
\begin{itemize}
\item AR, PYGM
\item \textgreater{} 100 known pathogenic mutations
\item p.R50X mutation, most common in Caucasians
\begin{itemize}
\item 81\% of the alleles in British patients
\item 63\% of alleles in US patients
\end{itemize}
\item No genotype-phenotype correlations have been detected
\item ACE polymorphism may be a phenotype modulator
\end{itemize}
\end{frame}

\begin{frame}[label={sec:org1fdcaae}]{Clinical Presentation}
\begin{itemize}
\item exercise intolerance with myalgia and stiffness in exercising muscles
\begin{itemize}
\item relieved by rest.
\end{itemize}
\item Onset of the disease occurs during childhood
\begin{itemize}
\item diagnosis is frequently missed at an early age
\item affected children are often considered lazy.
\end{itemize}
\item Myoglobinuria is the major complication, and occurs in about half of
the patients.
\item Creatine kinase (CK) can increase to more than 100,000–1,000,000
UI/l during episodes of rhabdomyolysis
\item Risk of acute renal failure
\end{itemize}
\end{frame}

\begin{frame}[label={sec:org2e38acb}]{Diagnostic Tests}
\begin{itemize}
\item ischaemic forearm exercise test (IFET) was first used by McArdle to
describe the absence of elevation of lactate during exercise.
\begin{itemize}
\item \alert{Should not be used}
\end{itemize}

\item Non-ischemic FET has a sensitivity of 100\% in McArdle’s disease
\item Ammonia levels should be also assessed in parallel with lactate
\begin{itemize}
\item an abnormal increase in ammonia always observed in GSD V.
\end{itemize}
\item PYGM gene sequencing
\end{itemize}
\end{frame}

\begin{frame}[label={sec:org635e170}]{Treatment}
\begin{itemize}
\item no pharmacological treatment-
\item exercise intolerance may be alleviated by:
\begin{itemize}
\item aerobic conditioning programs
\item ingestion of oral sucrose
\end{itemize}
\end{itemize}
\end{frame}
\section{LAMP 2 Deficiency (Danon Disease)}
\label{sec:org393c284}
\begin{frame}[label={sec:orgf8278c6}]{LAMP2 Deficiency (Danon Disease)}
\begin{figure}[htbp]
\centering
\includegraphics[width=0.75\textwidth]{./figures/gggmetab_muscle_cardiac.png}
\caption[Muscle, Cardiac Glycogenoses]{\label{fig:org6b206ca}
Muscle, Cardiac Glycogenoses}
\end{figure}
\end{frame}

\begin{frame}[label={sec:orgd4e7571}]{LAMP2 Deficiency (Danon Disease)}
\begin{itemize}
\item Danon disease is a rare X-linked disorder
\item caused by a primary deficiency of lysosomal-associated membrane
protein 2 (LAMP2).
\item Presents after 1st decade
\begin{itemize}
\item cardiomyopathy all cases
\item mild skeletal myopathy and developmental delay 70\%
\end{itemize}
\item muscle biopsy shows glycogen filled vacuoles
\item consider cardiac transplantation
\end{itemize}
\end{frame}

\section{AMPK Deficiency}
\label{sec:org0d8eabf}
\begin{frame}[label={sec:orgfdc169f}]{AMPK Deficiency}
\begin{figure}[htbp]
\centering
\includegraphics[width=0.75\textwidth]{./figures/gggmetab_muscle_cardiac.png}
\caption[Muscle, Cardiac Glycogenoses]{\label{fig:org02a25af}
Muscle, Cardiac Glycogenoses}
\end{figure}
\end{frame}

\begin{frame}[label={sec:org7c31e08}]{AMPK Deficiency}
\begin{itemize}
\item AMPK controls whole-body glucose homeostasis by regulating metabolism in multiple peripheral tissues, such as
skeletal muscle, liver, adipose tissues, and pancreatic \(\beta\)-cells
\item activated \(\uparrow\) AMP/ATP ratio
\item stimulates glucose uptake and lipid oxidation to produce energy
\item inhibits energy-consuming processes including glucose and lipid production.
\end{itemize}
\end{frame}

\begin{frame}[label={sec:org1db1a6f}]{Metabolic Derangement}
\begin{itemize}
\item AMPK is a heterotrimeric complex comprising:
\begin{itemize}
\item a catalytic subunit (α)
\item two regulatory subunits (β and γ).
\end{itemize}
\item Three isoforms of the gamma subunits are known (γ1, γ2 and γ3) with different tissue
expression
\end{itemize}
\end{frame}

\begin{frame}[label={sec:orgb6c9a3a}]{Genetics}
\begin{itemize}
\item The PRKAG2 gene coding for the \(\gamma\)-subunit of AMPK is located on chromosome 7q36.
\item Mutations in the \(\gamma\)2-subunit of AMPK are transmitted as an
autosomal dominant trait with full penetrance.
\end{itemize}
\end{frame}

\begin{frame}[label={sec:orgd375c27}]{Diagnosis \& Treatment}
\begin{itemize}
\item The differential diagnosis includes Pompe, Danon (LAMP2) and Fabry diseases.

\item diagnosis, if clinically suspected, is based on ECG,
echocardiography and molecular genetics.

\item Treatment requires a pacemaker/defibrillator and heart transplant.
\end{itemize}
\end{frame}
\end{document}
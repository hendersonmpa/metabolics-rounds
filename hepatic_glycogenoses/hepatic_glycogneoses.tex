% Created 2019-06-27 Thu 12:48
% Intended LaTeX compiler: pdflatex
\documentclass[presentation, smaller]{beamer}
\usepackage[utf8]{inputenc}
\usepackage[T1]{fontenc}
\usepackage{graphicx}
\usepackage{grffile}
\usepackage{longtable}
\usepackage{wrapfig}
\usepackage{rotating}
\usepackage[normalem]{ulem}
\usepackage{amsmath}
\usepackage{textcomp}
\usepackage{amssymb}
\usepackage{capt-of}
\usepackage{hyperref}
\hypersetup{colorlinks,linkcolor=white,urlcolor=blue}
\usepackage{textpos}
\usepackage{textgreek}
\usepackage[version=4]{mhchem}
\usepackage{chemfig}
\usepackage{siunitx}
\usepackage{gensymb}
\usepackage[usenames,dvipsnames]{xcolor}
\usepackage[T1]{fontenc}
\usepackage{lmodern}
\usepackage{verbatim}
\usepackage{tikz}
\usepackage{wasysym}
\usetikzlibrary{shapes.geometric,arrows,decorations.pathmorphing,backgrounds,positioning,fit,petri}
\usetheme{Hannover}
\usecolortheme{whale}
\author{Matthew Henderson, PhD, FCACB}
\date{\today}
\title{Hepatic Glycogenoses}
\institute[NSO]{Newborn Screening Ontario | The University of Ottawa}
\titlegraphic{\includegraphics[height=1cm,keepaspectratio]{../logos/NSO_logo.pdf}\includegraphics[height=1cm,keepaspectratio]{../logos/cheo-logo.png} \includegraphics[height=1cm,keepaspectratio]{../logos/UOlogoBW.eps}}
\hypersetup{
 pdfauthor={Matthew Henderson, PhD, FCACB},
 pdftitle={Hepatic Glycogenoses},
 pdfkeywords={},
 pdfsubject={},
 pdfcreator={Emacs 26.1 (Org mode 9.1.9)}, 
 pdflang={English}}
\begin{document}

\maketitle

%\logo{\includegraphics[width=1cm,height=1cm,keepaspectratio]{../logos/NSO_logo_small.pdf}~%
%    \includegraphics[width=1cm,height=1cm,keepaspectratio]{../logos/UOlogoBW.eps}%
%}

\vspace{220pt}
\beamertemplatenavigationsymbolsempty
\setbeamertemplate{caption}[numbered]
\setbeamerfont{caption}{size=\tiny}
% \addtobeamertemplate{frametitle}{}{%
% \begin{textblock*}{100mm}(.85\textwidth,-1cm)
% \includegraphics[height=1cm,width=2cm]{cat}
% \end{textblock*}}

\section{Introduction}
\label{sec:orgff2581f}

\begin{frame}[label={sec:orgb801ec9}]{Hepatic Glycogenoses}
\scriptsize
\begin{center}
\begin{tabular}{llll}
Type & Enzyme & Gene & Phenotype\\
\hline
0a & liver glycogen synthase & GYS2 & ketotic hypoglycema\\
0b & Muscle glycogen synthase & GYS1 & cardiomyopathy \& myopathy\\
Ia & G6Pase \(\alpha\) & GSPC & hypoglycema, hepatomegaly, lactic acidosis\\
Ib & G6P transporter & SLC37A4 & Ia + neutrophil dysfunction, colitis\\
III & Glycogen debrancher & AGL & hypoglycema, hepatomegaly\\
IV & Glycogen brancher & GBE1 & cirrhosis, cardiomyopathy \& myopathy\\
 &  &  & adult polyglucosan body disease\\
VI & Liver glycogen phosphorylase & PYGL & hypoglycema, hepatomegaly, growth delay\\
IXa & phosphorylase kinase \(\alpha\)2 & PHKA2 & hypoglycema, hepatomegaly\\
IXb & phosphorylase kinase \(\beta\) & PHKB & \\
IXc & phosphorylase kinase \(\gamma\)TL & PHKG2 & \\
IXd & phosphorylase kinase \(\alpha\)1 & PHKA1 & myopathy\\
\end{tabular}
\end{center}
\end{frame}

\begin{frame}[label={sec:org13cfa7b}]{Hepatic Glycogenoses}
\begin{figure}[htbp]
\centering
\includegraphics[width=0.75\textwidth]{./figures/gggmetab.png}
\caption[Hepatic Glycogenoses]{\label{fig:orgb99161a}
Hepatic Glycogenoses}
\end{figure}
\end{frame}



\begin{frame}[label={sec:org2cc23ea}]{Hepatic Glycogenoses}
\begin{figure}[htbp]
\centering
\includegraphics[width=0.75\textwidth]{./figures/gggmetab_hepatic.png}
\caption[Hepatic Glycogenoses]{\label{fig:org5fbc06b}
Hepatic Glycogenoses}
\end{figure}
\end{frame}


\section{GSD Type 0}
\label{sec:orgb0f82b0}
\begin{frame}[label={sec:orga7abb6a}]{GSD Type 0}
\begin{figure}[htbp]
\centering
\includegraphics[width=0.75\textwidth]{./figures/gggmetab_hepatic.png}
\caption[Hepatic Glycogenoses]{\label{fig:org155ad6b}
Hepatic Glycogenoses}
\end{figure}
\end{frame}

\begin{frame}[label={sec:org8268871}]{Metabolic Derangement}
\begin{itemize}
\item Deficiency in hepatic glycogen synthase (GS2)
\item Very low liver glycogen
\item Fasting associated with ketosis
\item Post-prandial hyperglycemia with moderate hyperlactatemia
\begin{itemize}
\item reduced liver uptake of glucose
\end{itemize}
\end{itemize}
\end{frame}

\begin{frame}[label={sec:orgc1690f1}]{Genetics}
\begin{itemize}
\item AR, GYS2 encodes liver isoform of GS2
\end{itemize}
\end{frame}

\begin{frame}[label={sec:orgc462d8b}]{Clinical Presentation}
\begin{itemize}
\item ketotic hypoglycema with post-prandial hyperglycemia/uria
\item can have poor growth
\item no hepatomegaly
\end{itemize}
\end{frame}

\begin{frame}[label={sec:org151c960}]{Diagnostic Tests}
\begin{itemize}
\item mutation analysis
\end{itemize}
\end{frame}
\begin{frame}[label={sec:orge23def1}]{Treatment}
\begin{itemize}
\item avoid fasting
\item uncooked cornstarch prior to overnight fast and during infections
\end{itemize}
\end{frame}
\section{GSD Type I}
\label{sec:org88fc050}
\begin{frame}[label={sec:org47c7d1e}]{GSD Type I}
\begin{figure}[htbp]
\centering
\includegraphics[width=0.75\textwidth]{./figures/gggmetab_hepatic.png}
\caption[Hepatic Glycogenoses]{\label{fig:org4989af2}
Hepatic Glycogenoses}
\end{figure}
\end{frame}

\begin{frame}[label={sec:org7d26e31}]{Metabolic Derangement}
\begin{itemize}
\item GSD Ia, glucose 6 phosphatase-\(\alpha\)
\item GSD Ib, glucose 6 phosphate transporter
\item Disorder of glycogen metabolism and gluconeogenesis
\item Failure of glucose dephosphorylation inhibits hepatic glycogen breakdown
\item Hyperlactatemia occurs due to lack of gluconeogenesis
\begin{itemize}
\item protective
\end{itemize}
\item hyperlipidemia and hyperuricemia due to \(\uparrow\) G6P
\begin{itemize}
\item \(\uparrow\) G6P \(\to\) \emph{de novo} lipogenesis and flux through pentose phosphate pathway
\end{itemize}
\item G6P transporter required for normal neutrophil function
\end{itemize}
\end{frame}

\begin{frame}[label={sec:orgade0a49}]{Genetics}
\begin{itemize}
\item AR, 1:100,000, 80\% Ia
\item GSD Ia: G6PC
\item GSD Ib: SLC37A4
\item no genotype phenotype correlation established
\end{itemize}
\end{frame}

\begin{frame}[label={sec:orgc9ade21}]{Clinical Presentation}
\begin{block}{Ia and Ib}
\begin{itemize}
\item severe fasting hypoglycema, lactic acidosis
\item hepatomegaly
\item hyperlipidemia, hyperuricemia
\end{itemize}
\end{block}
\begin{block}{Ib}
\begin{itemize}
\item neutrophil dysfunction
\item increased infections
\end{itemize}
\end{block}
\end{frame}

\begin{frame}[label={sec:org37cc97d}]{Diagnostic Tests}
\begin{itemize}
\item mutation analysis
\end{itemize}
\end{frame}

\begin{frame}[label={sec:orgbf57580}]{Treatment}
\begin{itemize}
\item generally fatal if untreated
\item diet
\item liver transplant
\item treatment of sequelae
\begin{itemize}
\item hepatic tumors
\item GI disease - IBD in GSD Ib
\item renal disease - glycogen deposition
\item hematological disease
\begin{itemize}
\item anemia
\item coagulopathy
\item infections, GSD Ib
\end{itemize}
\item cardiovascular disease
\item bone disease
\end{itemize}
\end{itemize}
\end{frame}

\section{GSD Type III}
\label{sec:orgd7de33a}
\begin{frame}[label={sec:orga4dbe1e}]{GSD Type III}
\begin{figure}[htbp]
\centering
\includegraphics[width=0.75\textwidth]{./figures/gggmetab_hepatic.png}
\caption[Hepatic Glycogenoses]{\label{fig:orgbbd3049}
Hepatic Glycogenoses}
\end{figure}
\end{frame}

\begin{frame}[label={sec:org89918a5}]{Metabolic Derangement}
\begin{itemize}
\item Glycogen debrancher enzyme (GDE) deficiency
\item has both glucosidase and transferase activity
\begin{itemize}
\item cleaves \(\alpha\)-1,4 glucose linkages of the terminal glucose
\item then breaks \(\alpha\)-1,6 linkage to remove branch point
\end{itemize}
\item accumulation of abnormal glycogen
\item limited glucose release from glycogen
\item gluconeogenesis functions normally
\end{itemize}
\end{frame}
\begin{frame}[label={sec:orgd15e08f}]{Genetics}
\begin{itemize}
\item AR, AGL gene
\item mutations occur throughout AGL (GSD IIIa)
\begin{itemize}
\item defect in liver and muscle
\end{itemize}
\item two specific mutations in exon 3 (GSD IIIb)
\begin{itemize}
\item liver only
\end{itemize}
\end{itemize}
\end{frame}
\begin{frame}[label={sec:orgfe58275}]{Clinical Presentation}
\begin{itemize}
\item Hepatic glycogenosis and (in most cases) also myopathic
\item First year with poor growth, delayed motor milestones and abdominal
distension
\item Fasting hypoglycaemia 
\begin{itemize}
\item Fasting tolerance is usually longer than in GSD I
\end{itemize}
\item Fasting ketosis is prominent.
\item Gluconeogenesis is normal \(\therefore\) no fasting hyperlactataemia
\item Moderate post-prandial \(\uparrow\) lactate
\item Hyperlipdaemia
\item \(\uparrow\) \(\uparrow\) \(\uparrow\) liver transaminases
\item \(\uparrow\) CK in myopathic form
\end{itemize}
\end{frame}
\begin{frame}[label={sec:orgf6f41e5}]{Diagnostic Tests}
\begin{itemize}
\item DBE activity in leucocytes
\item mutation analysis
\end{itemize}
\end{frame}
\begin{frame}[label={sec:org338dd0d}]{Treatment}
\begin{itemize}
\item Aim is to maintain normoglycaemia, reduce the hyperlipidaemia and ketosis and
ensure adequate growth.
\item Regular meals and uncooked cornstarch
\item Overnight continuous feeding is less commonly needed in GSD III than
in GSD I
\item Long term outcome for individuals with GSD III is generally good
with survival into adulthood.
\end{itemize}
\end{frame}
\section{GSD Type IV}
\label{sec:org70cc1d2}
\begin{frame}[label={sec:org4f88420}]{GSD Type IV}
\begin{figure}[htbp]
\centering
\includegraphics[width=0.75\textwidth]{./figures/gggmetab_hepatic.png}
\caption[Hepatic Glycogenoses]{\label{fig:orgc9a4b11}
Hepatic Glycogenoses}
\end{figure}
\end{frame}

\begin{frame}[label={sec:org969b711}]{Metabolic Derangement}
\begin{itemize}
\item GSD IV is caused by deficiency in glycogen brancher enzyme (GBE).
\item GBE transfers short glucosyl chains to form branch points with an
\(\alpha\)-1,6 linkage.
\item Deficiency results in an abnormal poorly soluble glycogen with fewer branch points (polyglucosan)
\item This abnormal glycogen accumulates in liver, muscle, heart, nervous system and skin.
\begin{itemize}
\item leads to tissue damage.
\end{itemize}
\end{itemize}
\end{frame}

\begin{frame}[label={sec:org10a4dfa}]{Genetics}
\begin{itemize}
\item AR, GBE1
\item Common mutation in Ashkenazi Jewish pop
\begin{itemize}
\item adult polyglucosan body disease (APBD)
\end{itemize}
\end{itemize}
\end{frame}

\begin{frame}[label={sec:org6bcc8af}]{Clinical Presentation}
\begin{itemize}
\item Multiple phenotypes associated with GBE deficiency
\begin{itemize}
\item Ranges from death in utero to adult presentation
\end{itemize}
\end{itemize}

\begin{block}{Liver Disease}
\begin{itemize}
\item Progressive liver disease in infancy.
\begin{itemize}
\item Presents in first months of life with:
\begin{itemize}
\item Failure to thrive and hepatomegaly.
\end{itemize}
\item Cirrhosis develops with eventual end stage liver disease and
portal hypertension.
\item Death is usual by 5 years of age.
\end{itemize}
\item Non-progressive liver disease in childhood.
\begin{itemize}
\item Present with hepatomegaly, liver dysfunction, hypotonia and
myopathy.
\item Liver disease does not progress, survival into adulthood.
\end{itemize}
\end{itemize}
\end{block}
\end{frame}

\begin{frame}[label={sec:org0a32995}]{Clinical Presentation}
\begin{block}{Neuromuscular Disease}
\begin{itemize}
\item Congenital onset
\begin{itemize}
\item fetal loss in pregnancy,
\item fetal akinesia deformation sequence (FADS) with athrogryposis, hydrops and perinatal death
\item severe congenital myopathy similar to SMA with \textpm{} cardiomyopathy
\end{itemize}
\item Juvenile onset
\begin{itemize}
\item with a myopathy and/or cardiomyopathy
\end{itemize}
\item Adult onset
\begin{itemize}
\item adult polyglucosan body disease (APBD)
\item rarely myopathy
\end{itemize}
\end{itemize}
\end{block}
\end{frame}

\begin{frame}[label={sec:org5449319}]{Diagnostic Tests}
\begin{itemize}
\item \(\uparrow\) transaminases in those with hepatic involvement
\item Fasting hypoglycaemia is uncommon except in end stage liver failure
\item Liver and muscle histology show swollen hepatocytes that contain
periodic acid-Schiff (PAS)-positive and diastase resistance
inclusions and evidence of interstitial fibrosis.
\item Enzyme analysis can be undertaken in liver tissue, cultured skin
fibroblast, peripheral lymphocytes and muscle
\item Confirmed by GBE1 mutation analysis.
\end{itemize}
\end{frame}

\begin{frame}[label={sec:org7d36a4d}]{Treatment}
\begin{itemize}
\item Liver transplant is the only treatment for the progressive liver form
\item Heart transplant may be considered in those with heart failure caused by cardiomyopathy.
\item There is no specific treatment for the other forms of the disease.
\end{itemize}
\end{frame}
\section{GSD Type VI}
\label{sec:org1f560f6}
\begin{frame}[label={sec:org4542a39}]{GSD Type VI}
\begin{figure}[htbp]
\centering
\includegraphics[width=0.75\textwidth]{./figures/gggmetab_hepatic.png}
\caption[Hepatic Glycogenoses]{\label{fig:org5e54a88}
Hepatic Glycogenoses}
\end{figure}
\end{frame}

\begin{frame}[label={sec:org1d06990}]{Metabolic Derangement}
\begin{itemize}
\item GSD VI is caused by deficiency in hepatic glycogen phosphorylase
\item Catalyses the release and phosphorylation of terminal glucosyl units
from glycogen forming glucose-1-phosphate
\item Ketosis with or without hypoglycaemia may occur with fasting
\item Although plasma lipids may be raised
\item In severe variants recurrent hypoglycaemia and post-prandial lactic
acidosis can occur
\end{itemize}
\end{frame}
\begin{frame}[label={sec:orgb8b457a}]{Genetics}
\begin{itemize}
\item AR, PGYL gene
\end{itemize}
\end{frame}

\begin{frame}[label={sec:org36db393}]{Clinical Presentation}
\begin{itemize}
\item GSD VI is generally a mild disorder often diagnosed due to hepatomegaly.
\begin{itemize}
\item can present with symptomatic ketotic hypoglycaemia and growth retardation
\end{itemize}
\end{itemize}
\end{frame}
\begin{frame}[label={sec:org40ae42a}]{Diagnostic Tests}
\begin{itemize}
\item Diagnosis confirmed by mutation analysis or
\item Enzyme deficiency in hepatic tissue, erythrocytes, and leukocytes.
\item Enzyme activity may not always be reduced in blood and even in liver
tissue may be difficult to interpret due to residual activity and
the effect of other factors.
\item For example, deficiency of glycogen phosphorylase kinase will cause
low activity of glycogen phosphorylase.
\end{itemize}
\end{frame}
\begin{frame}[label={sec:org54f64de}]{Treatment}
\begin{itemize}
\item No treatment required for asymptomatic children
\item Those with growth failure or fasting ketosis benefit from regular
meals,snacks and uncooked cornstarch.
\item The outcome for individuals with GSD VI is generally excellent
\begin{itemize}
\item Catch up growth occurring for those with short stature in childhood.
\end{itemize}
\end{itemize}
\end{frame}
\section{GSD Type IX}
\label{sec:org82202d9}
\begin{frame}[label={sec:org3782a4f}]{GSD Type IX}
\begin{figure}[htbp]
\centering
\includegraphics[width=0.75\textwidth]{./figures/gggmetab_hepatic.png}
\caption[Hepatic Glycogenoses]{\label{fig:org64c4ce9}
Hepatic Glycogenoses}
\end{figure}
\end{frame}

\begin{frame}[label={sec:org3dac432}]{Metabolic Derangement}
\begin{itemize}
\item GSD IX is caused by deficiency in hepatic glycogen phosphorylase kinase (PHK)
\item PHK phosphorylates glycogen phosphatase \emph{b} \(\to\) \emph{a} form
\begin{itemize}
\item inactive \emph{b} \(\to\)  active \emph{a}
\end{itemize}
\item Decreased PHK activity \(\to\) \(\uparrow\)
\item PHK is homotetramer in which each subunit is itself a tetramer
\begin{itemize}
\item \(\alpha\), \(\beta\), \(\gamma\) and \(\delta\) subunits.
\end{itemize}
\item The \(\gamma\) subunit is catalytic and the other subunits regulatory
\item There are tissue specific isoforms of the \(\alpha\) and \(\gamma\) subunits.
\item The \(\delta\) subunit, calmodulin, is ubiquitous
\end{itemize}
\end{frame}

\begin{frame}[label={sec:orgf15e38b}]{Genetics}
\begin{center}
\begin{tabular}{lllll}
Type & Gene & Subunit & Inheritance & Tissue\\
\hline
IXa & PHKA2 & \(\alpha\)2 & XLR & liver \& blood\\
IXb & PHKB & \(\beta\) & AR & liver \& muscle\\
IXc & PHKG2 & \(\gamma\)TL & AR & live\\
IXd & PHKA1 & \(\alpha\)1 & AR & muscle\\
\end{tabular}
\end{center}
\end{frame}

\begin{frame}[label={sec:orgd034f25}]{Clinical Presentation}
\begin{itemize}
\item Usually a benign disorder, with hepatomegaly often detected
incidentally
\item possible short stature, fasting hypoglycaemia and ketosis, with
raised liver transaminases, cholesterol and triglycerides.
\item Blood lactate and uric acid are normal. There is usually resolution
of signs and symptoms by adulthood.
\item GSD IXc can be more severe with an increased risk of hepatic fibrosis and cirrhosis
\end{itemize}
\end{frame}

\begin{frame}[label={sec:org982db3e}]{Diagnostic Tests}
\begin{itemize}
\item Considered in children with unexplained hepatomegaly and in those with ketotic hypoglycaemia.
\item PHK can be measured in liver, erythrocytes and leukocytes.
\item However, in view of variable tissue expression enzyme assays may be
difficult to interpret.
\item Diagnosis is best achieved by mutation analysis using a DNA panel.
\end{itemize}
\end{frame}
\begin{frame}[label={sec:org630f5fe}]{Treatment}
\begin{itemize}
\item Asymptomatic patients may not need treatment.
\item growth failure or symptomatic hypoglycaemia frequent meals and
uncooked cornstarch may be used.
\item Protein can be increased to 15 to 20\% of calories to provide a gluconeogenesis substrate.
\item The outcome for most patients is good with resolution of
hepatomegaly and catch up growth by adulthood.
\end{itemize}
\end{frame}
\end{document}
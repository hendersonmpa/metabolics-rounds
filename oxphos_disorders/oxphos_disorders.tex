% Created 2019-04-18 Thu 12:52
% Intended LaTeX compiler: pdflatex
\documentclass[presentation, smaller]{beamer}
\usepackage[utf8]{inputenc}
\usepackage[T1]{fontenc}
\usepackage{graphicx}
\usepackage{grffile}
\usepackage{longtable}
\usepackage{wrapfig}
\usepackage{rotating}
\usepackage[normalem]{ulem}
\usepackage{amsmath}
\usepackage{textcomp}
\usepackage{amssymb}
\usepackage{capt-of}
\usepackage{hyperref}
\hypersetup{colorlinks,linkcolor=white,urlcolor=blue}
\usepackage{textpos}
\usepackage{textgreek}
\usepackage[version=4]{mhchem}
\usepackage{chemfig}
\usepackage{siunitx}
\usepackage{gensymb}
\usepackage[usenames,dvipsnames]{xcolor}
\usepackage[T1]{fontenc}
\usepackage{lmodern}
\usepackage{verbatim}
\usepackage{tikz}
\usetikzlibrary{shapes.geometric,arrows,decorations.pathmorphing,backgrounds,positioning,fit,petri}
\usetheme{Hannover}
\usecolortheme{whale}
\author{Matthew Henderson, PhD, FCACB}
\date{\today}
\title{Disorders of Oxidative Phosphorylation}
\institute[NSO]{Newborn Screening Ontario | The University of Ottawa}
\titlegraphic{\includegraphics[height=1cm,keepaspectratio]{../logos/NSO_logo.pdf}\includegraphics[height=1cm,keepaspectratio]{../logos/cheo-logo.png} \includegraphics[height=1cm,keepaspectratio]{../logos/UOlogoBW.eps}}
\hypersetup{
 pdfauthor={Matthew Henderson, PhD, FCACB},
 pdftitle={Disorders of Oxidative Phosphorylation},
 pdfkeywords={},
 pdfsubject={},
 pdfcreator={Emacs 26.1 (Org mode 9.1.9)}, 
 pdflang={English}}
\begin{document}

\maketitle

%\logo{\includegraphics[width=1cm,height=1cm,keepaspectratio]{../logos/NSO_logo_small.pdf}~%
%    \includegraphics[width=1cm,height=1cm,keepaspectratio]{../logos/UOlogoBW.eps}%
%}n

\vspace{220pt}
\beamertemplatenavigationsymbolsempty
\setbeamertemplate{caption}[numbered]
\setbeamerfont{caption}{size=\tiny}
% \addtobeamertemplate{frametitle}{}{%
% \begin{textblock*}{100mm}(.85\textwidth,-1cm)
% \includegraphics[height=1cm,width=2cm]{cat}
% \end{textblock*}}

\section{Metabolic Derangement}
\label{sec:orgfd75015}
\begin{frame}[label={sec:org7a43798}]{OxPhos Deficiency and Anaerobic Glycolysis}
\begin{itemize}
\item Insufficient ATP severely affects highly energy dependant tissues
\begin{itemize}
\item A complete loss of OxPhos is not observed in human disease.
\end{itemize}
\item In the absence of OxPhos cells survive using ATP from anaerobic glycolysis
\begin{itemize}
\item 20x less efficient, generates lactate
\item pyruvate \(\to\) alanine if glutamate is available
\end{itemize}
\item Lactate, pyruvate and alanine are the typical products of anaerobic glycolysis
\end{itemize}
\end{frame}

\begin{frame}[label={sec:org43481ca}]{Factors Affecting OxPhos System}
\begin{itemize}
\item \textasciitilde{} 90 subunits
\begin{itemize}
\item 13 subunits of Complexes I, III, IV and V encoded by mtDNA
\end{itemize}
\item mitochondrial replication, transcription and translation
\begin{itemize}
\item require \textgreater{} 200 proteins, rRNAs and tRNAs
\end{itemize}
\item Cofactors: coenzyme Q\(_{\text{10}}\), iron-sulfur clusters, heme, copper
\begin{itemize}
\item require synthesis and/or transport to OxPhos system
\end{itemize}
\item Cardiolipin required for cristae formation
\item Mitochondrial function
\begin{itemize}
\item protein import, turnover
\item fission, fusion
\end{itemize}
\item Toxic metabolites
\item \textgreater{} 1500 proteins in the human mitochondrial proteome
\begin{itemize}
\item other additional factors - lipids, cofactors
\item up to 10\% of human proteome may have role in maintaining mitochondrial function
\end{itemize}
\end{itemize}
\end{frame}

\begin{frame}[label={sec:org7c035c3}]{Types of genetic defects}
\begin{itemize}
\item OxPhos Subunit
\item Assembly Factor
\item mtDNA replication
\item mtDNA transcription
\item cofactor
\item mitochondrial homeostasis
\begin{itemize}
\item fission and fusion
\end{itemize}
\item the primary biochemical phenotype is impaired OxPhos
\end{itemize}
\end{frame}

\begin{frame}[label={sec:orgc3d91a2}]{Clinical Manifestations}
\begin{center}
\begin{tabular}{ll}
System & Manifestation\\
\hline
CNS & \textbf{Myoclonus}\\
 & \textbf{Seizures}\\
 & \textbf{Ataxia}\\
Skeletal Muscle & \textbf{Myopathy, hypotonia}\\
 & \textbf{CPEO}\\
 & myoglobinuria\\
Marrow & Sideroblastic anemia/pancytopenia\\
Kidney & Fanconi\\
Endocrine & \textbf{Diabetes}\\
 & Hypoparathyroidism,\\
 & growth/multiple hormone deficiency\\
Heart & Cardiomyopathy\\
 & Conduction defects\\
GI & pancreatic failure\\
 & villous atrophy\\
Ear & \textbf{Sensorineural deafness}\\
 & Aminoglycoside deafness\\
Systemic & \textbf{Lactic Acidosis}\\
\end{tabular}
\end{center}
\end{frame}


\section{OxPhos Clinical Manifestations}
\label{sec:org30c22c5}

\begin{frame}[label={sec:org0dc093a}]{Clinical Manifestations}
\begin{itemize}
\item Clinical diagnosis is extremely challenging
\begin{itemize}
\item can affect any organ system
\item antenatal (IUGR, birth defects) \(\to\) elderly (myopathy)
\end{itemize}
\end{itemize}

\begin{block}{Clinical Suspicion based on:}
\begin{enumerate}
\item Constellation of symptoms/signs consistent with a mitochondrial syndrome
\item Complex multi-system presentation involving two/more organ systems,
best described by an underlying disorder of energy generation.
\item Lactic acidosis, characteristic neuro-imaging, 3-methylglutaconic
aciduria, ragged red fiber mypopathy.
\item Pathogenic mutation in a known mitochondrial disease gene.
\end{enumerate}
\end{block}
\end{frame}

\begin{frame}[label={sec:org5c13562}]{Common Clinical Manifestations}
\begin{itemize}
\item Mitochondrial disease commonly presents with:
\begin{itemize}
\item Myopathy
\item Encephalopathy
\item Leber’s hereditary optic neuropathy
\item Pearson's Syndrome
\item Diabetes
\end{itemize}
\end{itemize}
\end{frame}

\begin{frame}[label={sec:org6600467}]{Myopathies}
\begin{itemize}
\item Chronic progressive external ophthalmoplegia (CPEO)
\begin{itemize}
\item w/wo retinitis pigmentosa
\item most common clinical manifestation
\item muscle biopsy is diagnostic
\end{itemize}
\item Kearns Sayre syndrome is a subtype of CPEO
\begin{itemize}
\item onset \textless{} 20
\item pigmentary retinopathy
\item cardiac conduction defect
\item ataxia, \(\uparrow\) CSF protein
\end{itemize}
\item Isolated limb myopathy
\end{itemize}
\end{frame}

\begin{frame}[label={sec:org32c2928}]{Encephalopathies}
\begin{itemize}
\item encephalopathic features:
\begin{itemize}
\item dementia/ID, ataxia, seizures, myoclonus, deafness, dystonia.
\end{itemize}
\item MELAS: myopathy, encephalopathy, lactic acidosis, stroke-like episodes
\begin{itemize}
\item most common mito encephalopathy
\end{itemize}
\item MERRF: myoclonic epilepsy w ragged red fibres
\begin{itemize}
\item ptosis, ataxia, deafness
\end{itemize}
\item Leigh Syndrome
\begin{itemize}
\item most frequent presentation of MD in childhood
\item subacute necrotising encephalomyelopathy
\item several biochemical defects including: PDH, OxPhos
\item MRI - lesions affecting basal ganglia and/or brain stem
\item \(\uparrow\) lactate blood and CSF
\item hypo/er-ventilation, spasticity, dystonia, ataxia, tremor, optic atrophy
\item cardiomyopathy, renal tubulopathy, GI disfunction
\item \textgreater{} 75 genes(mt and nuclear)
\item Saguenay-Lac-St-Jean type incidence 1/2000, gene prevelance 1/23
\end{itemize}
\end{itemize}
\end{frame}

\begin{frame}[label={sec:orgedde3d0}]{Leber’s Hereditary Optic Neuropathy}
\begin{itemize}
\item most common cause of blindness in otherwise healthy young men.
\item maternally inherited and manifests in late adolescence or early
adulthood as bilateral sequential visual failure.
\item 90\% of patients are affected by age 40
\end{itemize}
\end{frame}

\begin{frame}[label={sec:org54f5ef0}]{Pearson's Syndrome}
\begin{itemize}
\item transfusion dependant sideroblastic anemia/pancytopenia
\item exocrine pancreas failure
\item progressive liver disease
\item renal tubular disease
\end{itemize}
\end{frame}

\begin{frame}[label={sec:org322aeee}]{Neonatal and Infantile Presentation}
\begin{itemize}
\item Congenital Lactic Acidosis
\item Leigh Syndrome
\item MEGDEL: 3-methylglutaconic aciduria, deafness, encephalopathy and Leigh-like disease
\item Pearson's marrow-pancreas syndrome
\item MDDS: mitochondrial DNA depletion syndrome
\item Alper-Huttenlocher syndrome
\item Reversible infantile respiratory chain deficiency
\item Infantile onset Q\(_{\text{10}}\) biosynthetic defects
\end{itemize}
\end{frame}

\begin{frame}[label={sec:org85d8d3b}]{Childhood and Adolescent Presentation}
\begin{itemize}
\item Kearn-Sayre syndrome
\item MELAS: myopathy, encephalopathy, lactic acidosis, stroke-like episodes
\item MERRF: myoclonic epilepsy w ragged red fibres
\item NARP: neuropathy, ataxia, retinitis pigmentosa
\item LHON: Leber's Hereditary Optic Neuropathy
\item MEMSA: myoclonic epilepsy, myopathy, sensory ataxia
\item MNGIE: mitochondrial neurogastrointestinal encephalopathy
\end{itemize}
\end{frame}

\begin{frame}[label={sec:org088f710}]{Adult Presentation}
\begin{itemize}
\item MIDD: maternally inherited diabetes and deafness
\item PEO: Progressive External Opthalmoplegia
\item SANDO: Sensory Ataxic Neuropathy, dysarthria and opthalmoparesis
\end{itemize}
\end{frame}

\section{Investigations}
\label{sec:org2381ccc}

\begin{frame}[label={sec:org3e17cea}]{Biochemistry}
\begin{itemize}
\item blood lactate, CSF lactate
\item L/P \(\uparrow\) at rest, \(\Uparrow\) after excercise
\item renal tubular disfunction: urine anion gap, pH, serum K
\item Plasma amino acids:
\begin{itemize}
\item alanine \(\propto\) pyruvate
\item ala/lys normally \textless{} 3:1
\item \(\uparrow\) gly in lipoic acid biosynthesis defects
\item \(\downarrow\) cit and arg in Leigh, NARP, MELAS and Pearson
\end{itemize}
\item Urine organic acids
\begin{itemize}
\item lactate, pyruvate, TCA intermediates
\item 3-methylglutaconic acid in Barth, Sengers, MEGDEL, ATP synthase deficiency
\item ethylmalonic
\item mma in succinyl-CoA-ligase deficiency
\end{itemize}
\item Acylcarnitines
\begin{itemize}
\item flavin cofactor metabolism
\end{itemize}
\item Purine and pyrimidines (plasma or urine)
\begin{itemize}
\item MNGIE \(\uparrow\) thymidine and deoxyuridine
\end{itemize}
\item FGF-21, GDF15 and creatinine \(\propto\) mito disfunction in myopathy
\end{itemize}
\end{frame}

\begin{frame}[label={sec:org62aee3d}]{Imaging}
\begin{itemize}
\item Cranial CT shows cerebral and cerebellar atrophy in many encephalopathic patients
\begin{itemize}
\item basal ganglia calcification may be seen in MELAS.
\end{itemize}
\item MRI in MELAS-associated stroke reveals increased T2 weighted signals in the grey and white matter
\item Symmetrical changes in the basal ganglia and brainstem observed in Leigh syndrome.
\end{itemize}
\end{frame}

\begin{frame}[label={sec:orgc9e9e86}]{Histology}
\begin{columns}
\begin{column}{0.6\columnwidth}
\begin{itemize}
\item Muscle biopsy is diagnostic
\begin{itemize}
\item mitochondrial myopathy due to mtDNA mutations and LHON may have normal biopsies.
\end{itemize}
\item Ragged red fibres on Gomori trichrome staining, due to mitochondrial proliferation
\item fibres stain strongly for succinate dehydrogenase
\item fibres often negative for COX (complex IV) in CPEO, KSS, or MERRF but positive in MELAS.
\item Leigh syndrome patients may have no ragged red fibres and  COX-negative fibres only
\end{itemize}
\end{column}

\begin{column}{0.4\columnwidth}
\begin{figure}[htbp]
\centering
\includegraphics[width=0.9\textwidth]{./figures/Ragged_red_fibers_in_MELAS.jpg}
\caption[rrf]{\label{fig:org00a2077}
Ragged red fibers - Gomori stain}
\end{figure}
\end{column}
\end{columns}
\end{frame}

\begin{frame}[label={sec:orgaeaf1fb}]{Molecular}
\begin{itemize}
\item no strict relation between phenotype and genotype.
\item mtDNA tRNA mutations are most common of the single base change abnormalities.
\begin{itemize}
\item A3243G in the tRNA\(^{\text{Leu(UUR)}}\) gene is most frequently found in MELAS
\item G8344A in tRNA\(^{\text{Lys}}\) in MERRF.
\item Many other tRNA mutations have been associated with other clinical phenotypes.
\end{itemize}
\item The primary mutations associated with LHON (G11778A, G3460A,T14484C) are in complex I genes ND4, ND1, and ND6.
\begin{itemize}
\item G11778A is by far the commonest and is found in over 50\% of LHON families in the UK.
\end{itemize}
\end{itemize}
\end{frame}
\end{document}
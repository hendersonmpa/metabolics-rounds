% Created 2018-04-25 Wed 17:03
\documentclass[presentation, smaller]{beamer}
\usepackage[utf8]{inputenc}
\usepackage[T1]{fontenc}
\usepackage{fixltx2e}
\usepackage{graphicx}
\usepackage{grffile}
\usepackage{longtable}
\usepackage{wrapfig}
\usepackage{rotating}
\usepackage[normalem]{ulem}
\usepackage{amsmath}
\usepackage{textcomp}
\usepackage{amssymb}
\usepackage{capt-of}
\usepackage{hyperref}
\hypersetup{colorlinks,linkcolor=gray,urlcolor=blue}
\usepackage{textpos}
\usepackage{textgreek}
\usepackage[version=4]{mhchem}
\usepackage{chemfig}
\usepackage{siunitx}
\usepackage{gensymb}
\usepackage[usenames,dvipsnames]{xcolor}
\usepackage[T1]{fontenc}
\usepackage{lmodern}
\usepackage{verbatim}
\usepackage{tikz}
\usetikzlibrary{shapes.geometric,arrows,decorations.pathmorphing,backgrounds,positioning,fit,petri}
\AtBeginSection[]{\begin{frame}\tableofcontents[currentsection] \end{frame}}
\usetheme[height=20pt]{Boadilla}
\usecolortheme[RGB={170,160,80}]{{structure}}
\author{Matthew Henderson, PhD, FCACB}
\date{\today}
\title{Urine Organic Acids in Homo Sapiens}
\institute[NSO]{Newborn Screening Ontario}
\titlegraphic{\includegraphics[height=1cm,keepaspectratio]{../logos/NSO_logo.pdf}\includegraphics[height=1cm,keepaspectratio]{../logos/cheo-logo.png} \includegraphics[height=1cm,keepaspectratio]{../logos/UOlogoBW.eps}}
\hypersetup{
 pdfauthor={Matthew Henderson, PhD, FCACB},
 pdftitle={Urine Organic Acids in Homo Sapiens},
 pdfkeywords={},
 pdfsubject={},
 pdfcreator={Emacs 25.2.1 (Org mode 8.3.4)}, 
 pdflang={English}}
\begin{document}

\maketitle
\begin{frame}{Outline}
\tableofcontents
\end{frame}


\logo{\includegraphics[width=1cm,height=1cm,keepaspectratio]{../logos/NSO_logo_small.pdf}}

\vspace{220pt}
\beamertemplatenavigationsymbolsempty
\setbeamertemplate{caption}[numbered]
\setbeamerfont{caption}{size=\tiny}
% \addtobeamertemplate{frametitle}{}{%
% \begin{textblock*}{100mm}(.85\textwidth,-1cm)
% \includegraphics[height=1cm,width=2cm]{cat}
% \end{textblock*}}


\tikzstyle{chemical} = [rectangle, rounded corners, text width=5em, minimum height=1em,text centered, draw=black, fill=none]
\tikzstyle{hardware} = [rectangle, rounded corners, text width=5em, minimum height=1em,text centered, draw=black, fill=gray!30]
\tikzstyle{ms} = [rectangle, rounded corners, text width=5em, minimum height=1em,text centered, draw=orange, fill=none]
\tikzstyle{msw} = [rectangle, rounded corners, text width=7em, minimum height=1em,text centered, draw=orange, fill=none]
\tikzstyle{label} = [rectangle,text width=8em, minimum height=1em, text centered, draw=none, fill=none]
\tikzstyle{hl} = [rectangle, rounded corners, text width=5em, minimum height=1em,text centered, draw=black, fill=red!30]
\tikzstyle{box} = [rectangle, rounded corners, text width=5em, minimum height=5em,text centered, draw=black, fill=none]
\tikzstyle{arrow} = [thick,->,>=stealth]
\tikzstyle{hl-arrow} = [ultra thick,->,>=stealth,draw=red]

\section{Background}
\label{sec:orgheadline5}
\begin{frame}[label={sec:orgheadline1}]{What are urine organic acids?}
\begin{itemize}
\item Water soluble compounds containing \(\ge\) one carboxyl group(s) and
non-amino functional groups
\end{itemize}

\centering
\chemfig{X-C(-[2]X)(-[6]X)-C(-[2]X)(-[6]X)-C(-[7]OH)=[1]O}

\begin{block}{Acylglycines}
\begin{itemize}
\item Acylglycines are also detected in UOA analysis
\begin{itemize}
\item conjugation of acyl-CoA species to glycine
\item catalysed by glycine N-acylase
\end{itemize}
\end{itemize}
\end{block}
\end{frame}

\begin{frame}[label={sec:orgheadline2}]{Nomenclature}
\begin{center}
\begin{tabular}{lll}
Length & Monocarboxylic acid & Dicarboxylic acid\\
\hline
C2 & Acetic & Oxalic\\
C3 & Propionic & Malonic\\
C4 & Butyric & Succinic\\
 & Isobutyric & \\
C5 & Valeric & Glutaric\\
 & Isovaleric & \\
 & 2-Methylbutyric & \\
C6 & Hexanoic (caprioc) & Adipic\\
C7 & Heptanoic (enanthic) & Pimelic\\
C8 & Octanoic (caprylic) & Suberic\\
C9 & Nonanoic (pelargonic) & Azelaic\\
C10 & Decanoic (capric) & Sebacic\\
\end{tabular}
\end{center}
\end{frame}


\begin{frame}[label={sec:orgheadline3}]{Functional Groups}
\centering
\chemfig{X-C(-[2]X)(-[6]X)-C(-[2]X)(-[6]X)-C(-[7]OH)=[1]O}

\begin{center}
\begin{tabular}{ll}
Functional group & Formula\\
\hline
hydrogen & -H\\
keto & .= O\\
hydroxyl & -OH\\
carboxyl & -COOH\\
side chain & -(CH\(_2\))\(_n\)\\
\end{tabular}
\end{center}
\end{frame}

\begin{frame}[label={sec:orgheadline4}]{Side Chains}
\centering
\chemfig{X-C(-[2]X)(-[6]X)-C(-[2]X)(-[6]X)-C(-[7]OH)=[1]O}

\begin{center}
\begin{tabular}{ll}
Side chain & Structure\\
\hline
Methyl & \chemfig{CH_3-}\\
Ethyl & \chemfig{CH_3-CH_2-}\\
Propyl & \chemfig{CH_3-CH_2-CH_2-}\\
Butyl & \chemfig{CH_3-CH_2-CH_2-CH_2-}\\
\end{tabular}
\end{center}
\end{frame}


\section{Urine Organic Acids}
\label{sec:orgheadline12}
\begin{frame}[label={sec:orgheadline6}]{Where do they come from?}
\begin{block}{Endogenous Sources}
\begin{itemize}
\item Originate from the intermediate metabolism of all major groups of
organic cellular components
\begin{itemize}
\item amino acids
\item lipids
\item nucleotides
\item carbohydrates
\item nucleic acids
\item steroids
\end{itemize}
\end{itemize}
\end{block}

\begin{block}{Exogenous}
\begin{itemize}
\item food
\item environment
\item medications
\end{itemize}
\end{block}
\end{frame}

\begin{frame}[label={sec:orgheadline7}]{Why are they measured in urine?}
\begin{itemize}
\item not extensively reabsorbed in the kidney tubules after glomerular
filtration
\begin{itemize}
\item can be present at 100x concentration in blood
\end{itemize}
\item readily available sample type
\item less invasive than serum
\item Over 500 organic acids have been identified in urine.
\begin{itemize}
\item clinical metabolics
\end{itemize}
\end{itemize}
\end{frame}

\begin{frame}[label={sec:orgheadline8}]{Classes of organic acids detected in the urine of  healthly subjects}
\begin{itemize}
\item Tricarboxylica acid cycle acids
\begin{itemize}
\item citric
\end{itemize}
\item hydroxyaliphatic acids
\begin{itemize}
\item 3-hydroxybutyric
\end{itemize}
\item aliphatic keto acids
\begin{itemize}
\item pyrvic
\end{itemize}
\item aliphatic acids
\begin{itemize}
\item oxalic
\end{itemize}
\item aldonic and deoxyaldonic acids (sugar acids)
\begin{itemize}
\item 
\end{itemize}
\item aromatic acids
\begin{itemize}
\item hippuric
\end{itemize}
\end{itemize}
\end{frame}

\begin{frame}[label={sec:orgheadline9}]{Abnormal Urine Organic acids profiles}
\begin{itemize}
\item Elevated concentration of normal metabolites
\begin{itemize}
\item fumaric acid in fumarase deficiency
\item adipic, suberic, and sebacic acids in MCADD
\item ketones in fasting
\begin{itemize}
\item 3-hydroxybutyric
\item acetoacetic
\end{itemize}
\end{itemize}

\item Pathological metabolites
\begin{itemize}
\item succinylacetone, methylcitric acid
\end{itemize}

\item Food, medications, environment
\begin{itemize}
\item ethosuximide
\item adipic acid
\item cresol
\item 2-furaldehyde
\end{itemize}
\end{itemize}
\end{frame}

\begin{frame}[label={sec:orgheadline10}]{Disorders that can be idenfified by Urine Organic Acid analysis}
\begin{itemize}
\item classic organic acidemias due to inborn errors of amino acid metabolism 
\begin{itemize}
\item isovaleric acidemia [MIM 243500]
\item methylmalonic acidemia[s] propionic acidemia [MIM 606054]
\item glutaric acidemia type I [MIM 231670]
\end{itemize}
\item as well as other amino acid disorders
\begin{itemize}
\item phenylketonuria [MIM 261600]
\item tyrosinemia type I [MIM 276700]
\item alkaptonuria [MIM 203500]
\item 3-methylglutaconic aciduria type I [MIM 250950]
\item maple syrup urine disease [MIM 248600]
\end{itemize}
\item disorders of the urea cycle - orotic acid
\item fatty acid oxidation
\begin{itemize}
\item short-chain acyl-CoA dehydrogenase deficiency [MIM 201470]
\item medium-chain acyl-CoA dehydrogenase deficiency [MIM 201450]
\item multiple acyl-CoA dehydrogenase deficiency [MIM 231680]
\item energy metabolism (pyruvate dehydrogenase deficiency
\item fumarase deficiency [MIM 606812]
\item SUCLA2 deficiency [MIM 603921]
\end{itemize}
\item purine and pyrimidine metabolism
\begin{itemize}
\item uridine monophosphate synthetase deficiency [MIM 613891]
\item dihydropyrimidine dehydrogenase deficiency [MIM 274270]
\end{itemize}
\item neurotransmission
\begin{itemize}
\item aromatic L-amino acid decarboxylase deficiency [MIM 608643]
\item ethylmalonicencephalopathy pMIM 602473)
\item Canavan disease [MIM 271900]
\item glutathione synthetase deficiency [MIM 266130]
\item glycerol kinase deficiency [MIM 307030]
\item primary hyperoxaluria type I [MIM 259900]
\item primary hyperoxaluria type II [MIM 260000]
\end{itemize}
\end{itemize}
\end{frame}

\begin{frame}[label={sec:orgheadline11}]{Next time}
\begin{itemize}
\item Urine organic acid methods
\end{itemize}
\end{frame}
\end{document}

% Created 2018-04-23 Mon 14:58
\documentclass[presentation, smaller]{beamer}
\usepackage[utf8]{inputenc}
\usepackage[T1]{fontenc}
\usepackage{fixltx2e}
\usepackage{graphicx}
\usepackage{longtable}
\usepackage{float}
\usepackage{wrapfig}
\usepackage{rotating}
\usepackage[normalem]{ulem}
\usepackage{amsmath}
\usepackage{textcomp}
\usepackage{marvosym}
\usepackage{wasysym}
\usepackage{amssymb}
\usepackage{hyperref}
\tolerance=1000
\hypersetup{colorlinks,linkcolor=white,urlcolor=blue}
\usepackage{textpos}
\usepackage[version=4]{mhchem}
\usepackage{chemfig}
\usepackage{siunitx}
\usepackage[usenames,dvipsnames]{xcolor}
\usepackage[T1]{fontenc}
\usepackage{lmodern}
\usepackage{verbatim}
\institute[NSO]{Newborn Screening Ontario | The University of Ottawa}
\titlegraphic{\includegraphics[height=1cm,keepaspectratio]{../logos/NSO_logo.pdf} \includegraphics[height=1cm,keepaspectratio]{../logos/UOlogoBW.eps}}
\usetheme[height=20pt]{Boadilla}
\usecolortheme[RGB={170,160,80}]{{structure}}
\author{Matthew Henderson, PhD, FCACB}
\date{\today}
\title{Organic Acids in Homo Sapiens}
\hypersetup{
  pdfkeywords={},
  pdfsubject={},
  pdfcreator={Emacs 25.2.1 (Org mode 8.2.10)}}
\begin{document}

\maketitle
\logo{\includegraphics[width=1cm,height=1cm,keepaspectratio]{../logos/NSO_logo_small.pdf}~%
    \includegraphics[width=1cm,height=1cm,keepaspectratio]{../logos/UOlogoBW.eps}%
}

\vspace{220pt}}
\beamertemplatenavigationsymbolsempty
\setbeamertemplate{caption}[numbered]
\setbeamerfont{caption}{size=\tiny}

% \addtobeamertemplate{frametitle}{}{%
% \begin{textblock*}{100mm}(.85\textwidth,-1cm)
% \includegraphics[height=1cm,width=2cm]{cat}
% \end{textblock*}}

\section{Background}
\label{sec-1}
\begin{frame}[label=sec-1-1]{Where do they come from?}
\end{frame}

\begin{frame}[label=sec-1-2]{Why are they here?}
\end{frame}
\begin{frame}[label=sec-1-3]{Nomenclature}
\begin{center}
\begin{tabular}{lll}
Length & Monocarboxylic acid & Dicarboxylic acid\\
\hline
C2 & Acetic & Oxalic\\
C3 & Propionic & Malonic\\
C4 & Butyric & Succinic\\
 & Isobutyric & \\
C5 & Valeric & Glutaric\\
 & Isovaleric & \\
 & 2-Methylbutyric & \\
C6 & Hexanoic (caprioc) & Adipic\\
C7 & Heptanoic (enanthic) & Pimelic\\
C8 & Octanoic (caprylic) & Suberic\\
C9 & Nonanoic (pelargonic) & Azelaic\\
C10 & Decanoic (capric) & Sebacic\\
\end{tabular}
\end{center}
\end{frame}

\begin{frame}[label=sec-1-4]{Functional Groups}
\centering
\chemfig{X-C(-[2]X)(-[6]X)-C(-[2]X)(-[6]X)-C(-[7]OH)=[1]O}

\begin{center}
\begin{tabular}{ll}
Functional group & Formula\\
\hline
hydrogen & -H\\
keto & .= O\\
hydroxyl & -OH\\
carboxyl & -COOH\\
side chain & -(CH$_2$)$_n$\\
\end{tabular}
\end{center}
\end{frame}

\begin{frame}[label=sec-1-5]{Side Chains}
\centering
\chemfig{X-C(-[2]X)(-[6]X)-C(-[2]X)(-[6]X)-C(-[7]OH)=[1]O}

\begin{center}
\begin{tabular}{ll}
Side chain & Structure\\
\hline
Methyl & \chemfig{CH_3-}\\
Ethyl & \chemfig{CH_3-CH_2-}\\
Propyl & \chemfig{CH_3-CH_2-CH_2-}\\
Butyl & \chemfig{CH_3-CH_2-CH_2-CH_2-}\\
\end{tabular}
\end{center}
\end{frame}


\section{Urine Organic Acids by GC-MS}
\label{sec-2}

\begin{frame}[label=sec-2-1]{Oximation}
\begin{itemize}
\item Oximated with 10\% hydroxylamine-HCL
\begin{itemize}
\item avoids multiple TMS species due to keto-enol tautomerism
\end{itemize}
\end{itemize}

\centering
\schemedebug{false}
\schemestart
\chemname{\chemfig[][scale=.5]{R=[1](-[2]OH)-[7]R}}{\tiny enol}
\arrow{<=>}
\chemname{\chemfig[][scale=.5]{R-[1](=[2]O)-[7]R}}{\tiny ketone}
\+
\chemname{\chemfig[][scale=.5]{N(<:[::-160]H)(<[::-120]H)-O-[1]H}}{\tiny hydroxylamine}
\arrow{->}
\chemname{\chemfig[][scale=.5]{R-[1](=[2]N-[1]OH)-[7]R}}{\tiny ketoxime}
\schemestop
\end{frame}


\begin{frame}[label=sec-2-2]{BSTFA derivatization}
\begin{itemize}
\item Acidified and extracted twice with ethyl ether
\item Derivatised with BSTFA (N,O-bis(trimethylsilyl)trifluoroacetamide) \footnote{Stalling DL, Gehrke CW, Zumwalt RW. A new silylation
reagent for amino acids bis(trimethylsilyl)trifluoroacetamide
(BSTFA). Biochemical and Biophysical Research Communications. 1968 May
23;31(4):616-22.}
\begin{itemize}
\item forms organic acid TMS esters
\end{itemize}
\end{itemize}

\centering
\schemedebug{false}
\schemestart
\chemname{\chemfig[][scale=.5]{F{_3}C-C(-[1]OTMS)=[7]NTMS}}{\tiny BSTFA}
\+
\chemname{\chemfig[][scale=.5]{R-C(=[1]O)-[7]OH}}{\tiny carboxylic acid}
\arrow{->}
\chemname{\chemfig[][scale=.5]{R-C(=[1]O)-[7]OTMS}}{\tiny TMS ester}
\+
\chemname{\chemfig[][scale=.5]{F{_3}C-C(=[1]O)-[7]NTMS}}{\tiny TMS amide}
\schemestop
\end{frame}


\begin{frame}[label=sec-2-3]{Gas Chromatography}
\end{frame}

\begin{frame}[label=sec-2-4]{Mass-spectroscopy}
\end{frame}

\begin{frame}[label=sec-2-5]{Reporting}
\end{frame}
% Emacs 25.2.1 (Org mode 8.2.10)
\end{document}

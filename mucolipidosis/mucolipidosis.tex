% Created 2018-10-04 Thu 12:44
\documentclass[presentation, smaller]{beamer}
\usepackage[utf8]{inputenc}
\usepackage[T1]{fontenc}
\usepackage{fixltx2e}
\usepackage{graphicx}
\usepackage{grffile}
\usepackage{longtable}
\usepackage{wrapfig}
\usepackage{rotating}
\usepackage[normalem]{ulem}
\usepackage{amsmath}
\usepackage{textcomp}
\usepackage{amssymb}
\usepackage{capt-of}
\usepackage{hyperref}
\hypersetup{colorlinks,linkcolor=white,urlcolor=blue}
\usepackage{textpos}
\usepackage{textgreek}
\usepackage[version=4]{mhchem}
\usepackage{chemfig}
\usepackage{siunitx}
\usepackage{gensymb}
\usepackage[usenames,dvipsnames]{xcolor}
\usepackage[T1]{fontenc}
\usepackage{lmodern}
\usepackage{verbatim}
\usepackage{tikz}
\usetikzlibrary{shapes.geometric,arrows,decorations.pathmorphing,backgrounds,positioning,fit,petri}
\usetheme{Hannover}
\usecolortheme{whale}
\author{Matthew Henderson, PhD, FCACB}
\date{\today}
\title{Mucolipidosis}
\institute[NSO]{Newborn Screening Ontario | The University of Ottawa}
\titlegraphic{\includegraphics[height=1cm,keepaspectratio]{../logos/NSO_logo.pdf}\includegraphics[height=1cm,keepaspectratio]{../logos/cheo-logo.png} \includegraphics[height=1cm,keepaspectratio]{../logos/UOlogoBW.eps}}
\hypersetup{
 pdfauthor={Matthew Henderson, PhD, FCACB},
 pdftitle={Mucolipidosis},
 pdfkeywords={},
 pdfsubject={},
 pdfcreator={Emacs 25.2.1 (Org mode 8.3.4)}, 
 pdflang={English}}
\begin{document}

\maketitle
%\logo{\includegraphics[width=1cm,height=1cm,keepaspectratio]{../logos/NSO_logo_small.pdf}~%
%    \includegraphics[width=1cm,height=1cm,keepaspectratio]{../logos/UOlogoBW.eps}%
%}

\vspace{220pt}
\beamertemplatenavigationsymbolsempty
\setbeamertemplate{caption}[numbered]
\setbeamerfont{caption}{size=\tiny}
% \addtobeamertemplate{frametitle}{}{%
% \begin{textblock*}{100mm}(.85\textwidth,-1cm)
% \includegraphics[height=1cm,width=2cm]{cat}
% \end{textblock*}}

\tikzstyle{chemical} = [rectangle, rounded corners, text width=5em, minimum height=1em,text centered, draw=black, fill=none]
\tikzstyle{hardware} = [rectangle, rounded corners, text width=5em, minimum height=1em,text centered, draw=black, fill=gray!30]
\tikzstyle{ms} = [rectangle, rounded corners, text width=5em, minimum height=1em,text centered, draw=orange, fill=none]
\tikzstyle{msw} = [rectangle, rounded corners, text width=7em, minimum height=1em,text centered, draw=orange, fill=none]
\tikzstyle{label} = [rectangle,text width=8em, minimum height=1em, text centered, draw=none, fill=none]
\tikzstyle{hl} = [rectangle, rounded corners, text width=5em, minimum height=1em,text centered, draw=black, fill=red!30]
\tikzstyle{box} = [rectangle, rounded corners, text width=5em, minimum height=5em,text centered, draw=black, fill=none]
\tikzstyle{arrow} = [thick,->,>=stealth]
\tikzstyle{hl-arrow} = [ultra thick,->,>=stealth,draw=red]


\section{Introduction}
\label{sec:orgheadline3}
\begin{frame}[label={sec:orgheadline1}]{Mucolipidosis}
\begin{itemize}
\item The mucolipidoses derived their name from the similarity in
presentation to both mucopolysaccharidoses and sphingolipidoses.

\item A biochemical understanding of these conditions has changed how they
are classified.
\item Four conditions (I, II, III, and IV) have been labeled as
mucolipidoses:
\begin{itemize}
\item ML I: sialidosis - a glycoproteinosis
\item ML II: I-cell - a glycoproteinosis
\item ML III: pseudo-Hurler polydystropy - a glycoproteinosis
\item ML IV: ganglioside sialidase deficiency - gangliosidosis
\end{itemize}
\end{itemize}
\end{frame}

\begin{frame}[label={sec:orgheadline2}]{ML II \& III : Nomenclature}
\begin{center}
\begin{tabular}{lll}
 & Current & Proposed\\
\hline
I-cell & ML II & ML II alpha/beta\\
Pseudo-Hurler polydystropy & ML IIIA & ML III alpha/beta\\
ML III variant & ML IIIC & ML III gamma\\
\end{tabular}
\end{center}
\end{frame}

\section{ML II \& III}
\label{sec:orgheadline11}
\begin{frame}[label={sec:orgheadline4}]{Protein trafficking to lysosomes}
\begin{figure}[htb]
\centering
\includegraphics[width=0.8\textwidth]{./figures/lysosome_traffic.jpg}
\caption{\label{fig:traffic}
Protein trafficking to lysosomes}
\end{figure}

\begin{itemize}
\item ML II and III are due to incorrect protein trafficking to lysosomes
\end{itemize}
\end{frame}

\begin{frame}[label={sec:orgheadline5}]{ML II \& III : Biochemical Defect}
\begin{figure}[htb]
\centering
\includegraphics[width=0.8\textwidth]{./figures/ml_defect.png}
\caption{\label{fig:biochem}
N-acetylglucosamine (GlcNAc) phosphotransferase}
\end{figure}
\end{frame}

\begin{frame}[label={sec:orgheadline6}]{ML II \& III Biochemical Defect}
\begin{itemize}
\item Characterized by deficient activity of a large number of lysosomal enzymes:
\begin{itemize}
\item \(\beta\)-glucuronidase
\item \(\beta\)-galactosidase
\item \(\alpha\)-mannosidase
\item \(\alpha\)-fucosidase
\item N-acetyl-\(\beta\)-d-galactosiaminidase
\item arylsulfatase-A
\item glycosylasparaginase
\end{itemize}
\item The activities of the same lysosomal enzymes are high in the medium
surrounding cultured I-cell fibroblasts
\end{itemize}
\end{frame}


\begin{frame}[label={sec:orgheadline7}]{ML II \& III : I-cell}
\begin{figure}[htb]
\centering
\includegraphics[height=0.65\textheight]{./figures/icell.png}
\caption{\label{fig:icell}
I cell in fibroblast culture}
\end{figure}

\begin{itemize}
\item oligosaccharides, lipids, and glycosaminoglycan inclusions in lysosomes
\end{itemize}
\end{frame}

\begin{frame}[label={sec:orgheadline8}]{ML II \& III : Clinical}
\begin{block}{ML II}
\begin{itemize}
\item Low IQ
\item Short, max 75 cm by 2 y
\item early coarse features
\item gingival hypertrophy
\item limited joint movement
\item dysostosis multiplex
\item minimal to no hepatomegaly and splenomegaly
\end{itemize}
\end{block}

\begin{block}{ML III}
\begin{itemize}
\item The pathology of ML-III is not as well documented as that of ML-II
\item Available data indicate the presence of similar but less severe
findings.

\item ML-II and ML-III can be distinguished on clinical criteria and on progression of the disease
\end{itemize}
\end{block}
\end{frame}

\begin{frame}[label={sec:orgheadline9}]{ML II \& III :Genetics}
\begin{itemize}
\item Autosomal recessive
\item The GlcNAc-PT has been purified and characterized as a hexameric
(\(\alpha\)2\(\beta\)2\(\gamma\)2) protein,
\begin{itemize}
\item a 540-KDa complex of disulfide linked homodimers
\end{itemize}
\item the \(\alpha\) and \(\beta\) subunits are encoded as a single \(\alpha \beta\) polypeptide by the GNPTAB gene
\begin{itemize}
\item subunits acquire molecular maturity following post-translational proteolysis of the initial gene product
\item \(\alpha \beta\) is the catalytic center in the GlcNAc-PT enzyme complex.
\end{itemize}
\item Mutations in the GNPTAB gene cause ML II and ML IIIA
\item Sequencing of the GNPTAB and GNPTG coding regions detects
disease-causing mutations in over 95\% of patients.
\item Mutations in the GNPTG gene that encodes the \(\gamma\) subunit were
first identified in a large Druze family in the Middle-East with a
variant form of ML III, termed ML IIIC.
\end{itemize}
\end{frame}


\begin{frame}[label={sec:orgheadline10}]{ML II \& III : Labs}
\begin{itemize}
\item Diagnosis is generally made by assay of lysosomal enzymes
\begin{itemize}
\item in cultured fibroblasts there is a distinct deficiency
\item in the plasma or serum where there is as much as a 10- to 20-fold increase in enzyme activity
\end{itemize}
\item Assay of fibroblasts or plasma for glycosylasparaginase has been
reported as useful for the diagnosis of I-cell disease.
\item The diagnosis can also be made by assay of the GlcNAc
phosphotransferase in leukocytes or cultured fibroblasts
\end{itemize}


\begin{itemize}
\item Treatment is supportive
\end{itemize}
\end{frame}

\section{Sialidosis and ML IV}
\label{sec:orgheadline17}

\begin{frame}[label={sec:orgheadline12}]{Sialidosis (ML I)}
\begin{itemize}
\item Sialidosis is an autosomal recessive lysosomal storage disorder.

\item \textbf{Type I sialidosis}, the milder form of this disorder, is
characterized by the development of ocular cherry-red spots and
generalized myoclonus in the second or third decade of life.
\item Additional findings, reported in more than 50 percent of patients,
include seizures, hyperreflexia, and ataxia.

\item \textbf{Type II sialidosis} is distinguished from this milder form by the
early onset of a progressive, rather severe,
mucopolysaccharidosis-like phenotype with visceromegaly, dysostosis
multiplex, and mental retardation.
\end{itemize}
\end{frame}

\begin{frame}[label={sec:orgheadline13}]{Sialidosis (ML I)}
\begin{itemize}
\item Both forms of the disease result from deficiency of the
neuraminidase (NEU1) that normally cleaves terminal \(\alpha\)2 \(\to\) 3 and
\(\alpha\)2 \(\to\) 6 sialyl linkages of several oligosaccharides and glycopeptides

\item found in increased amounts in tissues and fluids of affected patients.

\item Test urine samples for both oligosaccharides and glycopeptides

\item definitive diagnosis - measurement of sialidase activity in fresh tissue
samples, i.e., fibroblasts, cultured amniotic fluid cells, or white
blood cells.

\item supportive treatment
\end{itemize}
\end{frame}

\begin{frame}[label={sec:orgheadline14}]{ML IV}
\begin{itemize}
\item autosomal recessive inborn error of intracellular membrane trafficking
\begin{itemize}
\item associated with lysosomal inclusions in a variety of cell types.
\item mucolipin-1, a transmembrane protein of the transient receptor
potential channel family, causes MLIV.
\item it is unclear why a deficiency or malfunction of mucolipin-1 causes MLIV.
\end{itemize}

\item Clinical presentation includes:
\begin{itemize}
\item severe motor developmental delay
\item iron deficiency anemia
\item corneal clouding
\item progressive retinal degeneration
\item achlorhydria.
\end{itemize}

\item Notably absent are dysplastic bone abnormalities and enlargement of
organs such as the liver and the spleen.

\item blood gastrin levels should be measured, and elevated levels in
this setting are virtually diagnostic of MLIV
\end{itemize}
\end{frame}

\begin{frame}[label={sec:orgheadline15}]{ML IV}
\begin{itemize}
\item MLIV is pan-ethnic, but most patients are of Ashkenazi-Jewish
ancestry, in which the most prevalent mutation occurs at a frequency
of approximately 1/100.

\item g.5534A \(\to\) G and g.511-6944del, are present in 95\% of all
Ashkenazi-Jewish patients. Population-based screening for these
mutations is useful for the identification and counseling of MLIV
carriers. Identification of mutations in MCOLN1 should be used for
prenatal diagnosis.
\end{itemize}
\end{frame}


\begin{frame}[label={sec:orgheadline16}]{Next up}
\begin{itemize}
\item Lipid Storage Disorders
\end{itemize}
\end{frame}
\end{document}

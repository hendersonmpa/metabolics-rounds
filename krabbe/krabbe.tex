% Created 2019-01-24 Thu 12:47
% Intended LaTeX compiler: pdflatex
\documentclass[presentation, smaller]{beamer}
\usepackage[utf8]{inputenc}
\usepackage[T1]{fontenc}
\usepackage{graphicx}
\usepackage{grffile}
\usepackage{longtable}
\usepackage{wrapfig}
\usepackage{rotating}
\usepackage[normalem]{ulem}
\usepackage{amsmath}
\usepackage{textcomp}
\usepackage{amssymb}
\usepackage{capt-of}
\usepackage{hyperref}
\hypersetup{colorlinks,linkcolor=white,urlcolor=blue}
\usepackage{textpos}
\usepackage{textgreek}
\usepackage[version=4]{mhchem}
\usepackage{chemfig}
\usepackage{siunitx}
\usepackage{gensymb}
\usepackage[usenames,dvipsnames]{xcolor}
\usepackage[T1]{fontenc}
\usepackage{lmodern}
\usepackage{verbatim}
\usepackage{tikz}
\usetikzlibrary{shapes.geometric,arrows,decorations.pathmorphing,backgrounds,positioning,fit,petri}
\usetheme{Hannover}
\usecolortheme{whale}
\author{Matthew Henderson, PhD, FCACB}
\date{\today}
\title{Krabbe Disease}
\subtitle{Disorders of Sphingolipid Degradation}
\institute[NSO]{Newborn Screening Ontario | The University of Ottawa}
\titlegraphic{\includegraphics[height=1cm,keepaspectratio]{../logos/NSO_logo.pdf}\includegraphics[height=1cm,keepaspectratio]{../logos/cheo-logo.png} \includegraphics[height=1cm,keepaspectratio]{../logos/UOlogoBW.eps}}
\hypersetup{
 pdfauthor={Matthew Henderson, PhD, FCACB},
 pdftitle={Krabbe Disease},
 pdfkeywords={},
 pdfsubject={},
 pdfcreator={Emacs 26.1 (Org mode 9.1.9)}, 
 pdflang={English}}
\begin{document}

\maketitle

%\logo{\includegraphics[width=1cm,height=1cm,keepaspectratio]{../logos/NSO_logo_small.pdf}~%
%    \includegraphics[width=1cm,height=1cm,keepaspectratio]{../logos/UOlogoBW.eps}%
%}

\vspace{220pt}
\beamertemplatenavigationsymbolsempty
\setbeamertemplate{caption}[numbered]
\setbeamerfont{caption}{size=\tiny}
% \addtobeamertemplate{frametitle}{}{%
% \begin{textblock*}{100mm}(.85\textwidth,-1cm)
% \includegraphics[height=1cm,width=2cm]{cat}
% \end{textblock*}}

\tikzstyle{chemical} = [rectangle, rounded corners, text width=5em, minimum height=1em,text centered, draw=black, fill=none]
\tikzstyle{hardware} = [rectangle, rounded corners, text width=5em, minimum height=1em,text centered, draw=black, fill=gray!30]
\tikzstyle{ms} = [rectangle, rounded corners, text width=5em, minimum height=1em,text centered, draw=orange, fill=none]
\tikzstyle{msw} = [rectangle, rounded corners, text width=7em, minimum height=1em,text centered, draw=orange, fill=none]
\tikzstyle{label} = [rectangle,text width=8em, minimum height=1em, text centered, draw=none, fill=none]
\tikzstyle{hl} = [rectangle, rounded corners, text width=5em, minimum height=1em,text centered, draw=black, fill=red!30]
\tikzstyle{box} = [rectangle, rounded corners, text width=5em, minimum height=5em,text centered, draw=black, fill=none]
\tikzstyle{arrow} = [thick,->,>=stealth]
\tikzstyle{hl-arrow} = [ultra thick,->,>=stealth,draw=red]

\section{Introduction}
\label{sec:org596897a}
\begin{frame}[label={sec:org84a3c88}]{Krabbe Disease}
\begin{itemize}
\item A rapidly progressive CNS degenerative disease
\item Krabbe is both a leukodystrophy, affecting white matter of the central
and peripheral nervous systems, and an LSD

\item Incidence of 1:100,000 births
\item Cause by deficiency in galactocerebrosidase activity
\begin{itemize}
\item catabolism of galactosylceramide, a major lipid in myelin, kidney, and epithelial cells of the small intestine and colon.
\item results in accumulation of galactosylceramide in pathognomonic globoid cells
\begin{itemize}
\item Multinucleated microglia/macrophages seen in the white matter
\end{itemize}
\end{itemize}
\item accumulation of galactosylspingosine (psychosine) in oligodendrocytes and Schwann cells
\end{itemize}
\end{frame}

\begin{frame}[label={sec:org8d2b867}]{Galactocerebrosidase}
\begin{figure}[htbp]
\centering
\includegraphics[width=0.8\textwidth]{./figures/beta-galactosidase.png}
\caption{\label{fig:orgdc0415e}
Galactocerebrosidase}
\end{figure}

\begin{itemize}
\item Galactocerebrosidase is a lysosomal enzyme
\item Hydrolyzes the galactose ester bonds of galactocerebroside, galactosylsphingosine, lactosylceramide, and monogalactosyldiglyceride.
\item Requires saposin A cofactor
\end{itemize}
\end{frame}
\begin{frame}[label={sec:org70b1e45}]{Saposin A cofactor deficiency}
\begin{itemize}
\item atypical Krabbe disease due to saposin A deficiency is caused by mutation in the prosaposin gene (PSAP; 176801).
\item Sphingolipid activator proteins (saposins A, B, C and D) are small
homologous glycoproteins derived from a common precursor protein
(prosaposin) encoded by a single gene.
\item They are required for in vivo degradation of sphingolipids with short carbohydrate chains.
\item probably act by isolating the lipid substrate from the membrane
surroundings, thus making it more accessible to the soluble
degradative enzyme
\end{itemize}
\end{frame}

\begin{frame}[label={sec:orgec48afd}]{Sphingolipid degradation}
\begin{figure}[htbp]
\centering
\includegraphics[width=0.6\textwidth]{./figures/sl_degradation.png}
\caption[deg]{\label{fig:orge95ae29}
Sphingolipid degradation}
\end{figure}
\end{frame}

\begin{frame}[label={sec:orgbd90334}]{Lysosomal Protein Trafficking}
\begin{figure}[htbp]
\centering
\includegraphics[width=0.65\textwidth]{./figures/lysosome_trafficking.jpeg}
\caption[traf]{\label{fig:org598a03c}
Lysosomal protein trafficking receptors}
\end{figure}

\footnotesize
\begin{itemize}
\item lysosomal trafficking of galactocerebrosidase by the mannose 6-phosphate receptor.
\item enzyme is secreted in ML II
\end{itemize}
\end{frame}

\begin{frame}[label={sec:org694ab0e}]{Genetics}
\begin{itemize}
\item Autosomal recessive
\item The GALC gene is situated at 14q31 and consists of 17 exons.
\item A recurrent 30 kb deletion has been described which extends from
intron 10 to intron 17 of the GALC gene and in the homozygous state
is associated with infantile onset disease.
\item The allele frequency of this deletion in Krabbe patients is reported
to be approximately 50\% in Dutch patients and 35\% in non-Dutch
European patients
\end{itemize}
\end{frame}

\section{Clinical Findings}
\label{sec:org59b7aaf}
\begin{frame}[label={sec:orgefb75f1}]{Disease Spectrum}
\begin{itemize}
\item Krabbe disease is a spectrum from infantile to late-onset.

\item Infantile-onset characterized by normal development in the first few
months followed by rapid severe neurologic deterioration
\begin{itemize}
\item the average age of death is 24 months (range 8 months to 9 years).
\end{itemize}

\item later-onset disease manifests after 12 months and as late as the
seventh decade.

\item Historically 85\%-90\% of symptomatic individuals with Krabbe disease
diagnosed by enzyme activity alone have infantile-onset Krabbe
disease and 10\%-15\% have later-onset Krabbe disease,

\item NBS suggests that the proportion of individuals with later-onset
Krabbe disease is higher than previously thought.
\end{itemize}
\end{frame}

\begin{frame}[label={sec:org0a86f3e}]{Clinical Findings}
\begin{block}{Age <12 months (infantile-onset Krabbe disease)}
\small
\begin{itemize}
\item Excessive crying to extreme irritability
\item Feeding difficulties, gastroesophageal reflux disease
\item Spasticity of lower extremities and fist clenching, with axial hypotonia
\item Loss of acquired milestones (smiling, cooing, and head control)
\item Staring episodes
\item Peripheral neuropathy
\item the average age of death is 24 months (range 8 months to 9 years).
\end{itemize}
\end{block}

\begin{block}{Age >12 months (later-onset Krabbe disease)}
\small
\begin{itemize}
\item Slow development of motor milestones or loss of milestones (e.g.,
sitting without support, walking), slurred speech
\item Spasticity of extremities with truncal hypotonia
\item Vision loss, esotropia
\item Seizures
\item Peripheral neuropathy
\end{itemize}
\end{block}
\end{frame}

\section{Diagnosis}
\label{sec:orgc84e4e2}
\begin{frame}[label={sec:org46c7836}]{Symptomatic presentation}
\begin{itemize}
\item The diagnosis of Krabbe disease, suspected in a symptomatic proband
based on clinical findings and other supportive laboratory,
neuroimaging, and electrophysiologic findings, is established by:
\begin{itemize}
\item detection of deficient GALC enzyme activity in leukocytes.
\item Abnormal results require follow-up molecular genetic testing of GALC
\item elevated psychosine levels can also help establish the diagnosis.
\end{itemize}
\end{itemize}
\end{frame}

\begin{frame}[label={sec:org67162ff}]{Screen positive}
\begin{itemize}
\item In an asymptomatic newborn with low GALC enzyme activity
on dried blood spot specimens on NBS
\item urgent time-critical measurement of:
\begin{itemize}
\item blood psychosine levels
\item GALC molecular genetic testing
\end{itemize}
\item is necessary to identify, before age 14 days, those newborns with
evidence of infantile-onset Krabbe disease who are candidates for
early HSCT
\end{itemize}
\end{frame}

\begin{frame}[label={sec:orge2d093c}]{NBS follow-up}
\begin{figure}[htbp]
\centering
\includegraphics[width=0.8\textwidth]{./figures/NBS_follow_up.png}
\caption{\label{fig:org4d9e71a}
NBS follow-up at Mayo}
\end{figure}
\end{frame}


\section{Laboratory Investigations}
\label{sec:orgbcd3a4b}

\begin{frame}[label={sec:org0c48f72}]{CSF protein}
\begin{itemize}
\item protein in cerebrospinal fluid is elevated at the time of first symptoms
\item with increased albumin and decrease in \(\beta\)-globulins
\item Increase permeability of the blood-brain barrier?
\end{itemize}
\end{frame}

\begin{frame}[label={sec:org745acfd}]{galactocerebrosidase assay}
\begin{itemize}
\item HSC
\item Leukocytes preferred
\item Draw 5-6 mL of heparinized peripheral blood
\item Fresh heparinized blood should be drawn early enough in the day to arrive in the laboratory by 3:00 p.m. that day
\item Several of the assays available can be performed on a single leukocyte pellet or plasma sample

\item cleavage of 6-hexadecanoylamino-4-methylumbelliferyl-\(\beta\)-d-galactopyranoside
\end{itemize}
\end{frame}


\begin{frame}[label={sec:org1c9b09c}]{Newborn Screening}
\begin{block}{New York State - retrospective analysis}
\begin{itemize}
\item Almost 2 million infants screened.
\item Five infants diagnosed with early infantile Krabbe disease.
\item Three died, two from HSCT-related complications and one from untreated disease.
\item Two children who received HSCT have moderate to severe developmental delays.
\item Forty-six currently asymptomatic children are considered to be at
moderate or high risk for development of later-onset Krabbe disease.
\end{itemize}
\end{block}
\end{frame}


\begin{frame}[label={sec:orgea5222a}]{Multiplex DBS  Enzyme Assay}
\begin{itemize}
\item The DBS screening assay tests for:
\begin{itemize}
\item Gaucher
\item Krabbe
\item Niemann-Pick-A/B
\item Pompe
\item Fabry
\item MPS-I
\end{itemize}
\item a single 3-mm DBS punch, which is incubated in a single-assay
cocktail with all substrates and internal standards.
\item After incubation and liquid-liquid extraction, samples are analyzed by flow injection MS/MS.
\item All deuterated internal standards correspond to enzymatically generated products.
\end{itemize}
\end{frame}


\begin{frame}[label={sec:org1458870}]{DBS Psychosine}
\begin{itemize}
\item As an amphipathic molecule, psychosine partitions largely into
cellular membranes.
\item This test is used as a second-tier assay for infants who have
abnormal newborn screens with reduced GALC (galactocerebrosidase)
activity and to diagnose and monitor patients with Krabbe disease
and Saposin A cofactor deficiency.

\item psychosine is elevated in DBS samples of newborns with Krabbe.

\item The original DBS specimens from the first four infantile
KD cases identified through NBS had very elevated psychosine
concentrations, whereas the psychosine levels of all of the
asymptomatic high- and moderate-risk infants were only slightly
elevated compared with DBS from infants with normal GALC activities.
\end{itemize}
\end{frame}


\begin{frame}[label={sec:org1e83bcd}]{Treatment}
\begin{itemize}
\item Treatment of manifestations:
\begin{itemize}
\item Treatment of a child who is symptomatic before age six months is
supportive and focused on increasing the quality of life and
avoiding complications. For older individuals, treatment with HSCT
is individualized based on disease burden and manifestations.
\end{itemize}

\item Prevention of primary manifestations:
\begin{itemize}
\item Consensus guidelines recommend that asymptomatic newborns
identified by either prenatal/neonatal evaluation because of a
positive family history of Krabbe disease or an abnormal NBS
result undergo additional testing to identify those with
infantile-onset Krabbe disease. Those with laboratory findings
consistent with infantile-onset Krabbe disease are candidates for
HSCT before age 30 days.
\end{itemize}

\item Surveillance:
\begin{itemize}
\item Monitor symptomatic individuals with Krabbe disease for
development of: hydrocephalus, swallowing difficulties and chronic
microaspiration, scoliosis, hip subluxation, and osteopenia,
decreased vision, and corneal ulcerations.
\end{itemize}
\end{itemize}
\end{frame}
\end{document}

% Created 2019-07-21 Sun 13:02
% Intended LaTeX compiler: pdflatex
\documentclass[presentation, smaller]{beamer}
\usepackage[utf8]{inputenc}
\usepackage[T1]{fontenc}
\usepackage{graphicx}
\usepackage{grffile}
\usepackage{longtable}
\usepackage{wrapfig}
\usepackage{rotating}
\usepackage[normalem]{ulem}
\usepackage{amsmath}
\usepackage{textcomp}
\usepackage{amssymb}
\usepackage{capt-of}
\usepackage{hyperref}
\hypersetup{colorlinks,linkcolor=white,urlcolor=blue}
\usepackage{textpos}
\usepackage{textgreek}
\usepackage[version=4]{mhchem}
\usepackage{chemfig}
\usepackage{siunitx}
\usepackage{gensymb}
\usepackage[usenames,dvipsnames]{xcolor}
\usepackage[T1]{fontenc}
\usepackage{lmodern}
\usepackage{verbatim}
\usepackage{tikz}
\usepackage{wasysym}
\usetikzlibrary{shapes.geometric,arrows,decorations.pathmorphing,backgrounds,positioning,fit,petri}
\usetheme{Hannover}
\usecolortheme{whale}
\author{Matthew Henderson, PhD, FCACB}
\date{\today}
\title{Carbohydrate Metabolism}
\institute[NSO]{Newborn Screening Ontario | The University of Ottawa}
\titlegraphic{\includegraphics[height=1cm,keepaspectratio]{../logos/NSO_logo.pdf}\includegraphics[height=1cm,keepaspectratio]{../logos/cheo-logo.png} \includegraphics[height=1cm,keepaspectratio]{../logos/UOlogoBW.eps}}
\hypersetup{
 pdfauthor={Matthew Henderson, PhD, FCACB},
 pdftitle={Carbohydrate Metabolism},
 pdfkeywords={},
 pdfsubject={},
 pdfcreator={Emacs 26.1 (Org mode 9.1.9)}, 
 pdflang={English}}
\begin{document}

\maketitle

%\logo{\includegraphics[width=1cm,height=1cm,keepaspectratio]{../logos/NSO_logo_small.pdf}~%
%    \includegraphics[width=1cm,height=1cm,keepaspectratio]{../logos/UOlogoBW.eps}%
%}

\vspace{220pt}
\beamertemplatenavigationsymbolsempty
\setbeamertemplate{caption}[numbered]
\setbeamerfont{caption}{size=\tiny}
% \addtobeamertemplate{frametitle}{}{%
% \begin{textblock*}{100mm}(.85\textwidth,-1cm)
% \includegraphics[height=1cm,width=2cm]{cat}
% \end{textblock*}}

\section{Introduction}
\label{sec:orgdfe585e}
\begin{frame}[label={sec:org030bfd9}]{Carbohydrate Digestion}
\begin{figure}[htbp]
\centering
\includegraphics[width=0.7\textwidth]{./figures/carb_digest.PNG}
\caption{\label{fig:org8eab136}
Carbohydrate Digestions}
\end{figure}
\end{frame}

\begin{frame}[label={sec:orgc257b4a}]{Major Pathways of Glucose Metabolism}
\begin{figure}[htbp]
\centering
\includegraphics[width=0.9\textwidth]{./figures/glucose_pathways.PNG}
\caption{\label{fig:orgc60db26}
Major Pathways of Glucose Metabolism}
\end{figure}
\end{frame}

\begin{frame}[label={sec:org644f287}]{Conversion to AAs and FAs}
\begin{figure}[htbp]
\centering
\includegraphics[width=0.65\textwidth]{./figures/glucose_conversion.PNG}
\caption{\label{fig:org53da104}
Conversion of Glucose}
\end{figure}
\end{frame}

\section{Regulation}
\label{sec:org15d113e}
\begin{frame}[label={sec:org2d9742c}]{Major Hormones of Metabolic Homeostasis}
\begin{itemize}
\item \textbf{Insulin} is the main anabolic hormone
\begin{itemize}
\item promotes use of glucose as fuel
\begin{itemize}
\item \(\uparrow\) transport into cells
\end{itemize}
\item storage of glucose as glycogen
\item conversion of glucose \(\to\) TAGs
\item TAG storage in adipose tissue
\item AA uptake and protein synthesis in muscle
\end{itemize}
\item \textbf{Glucagon} is catabolic
\begin{itemize}
\item maintain fuel availability in the absence of dietary glucose
\item stimulates glycogenolysis
\item stimulates gluconeogenesis from lactate, glycerol and AAs
\item mobilizing FAs from adipose TAGs
\item acts on liver and adipose, muscle has no receptor
\end{itemize}
\end{itemize}
\end{frame}

\begin{frame}[label={sec:org4357624}]{Major Hormones of Metabolic Homeostasis}
\begin{figure}[htbp]
\centering
\includegraphics[width=0.9\textwidth]{./figures/regulation.PNG}
\caption{\label{fig:orgc23fb59}
Glucose Homeostasis}
\end{figure}
\end{frame}


\begin{frame}[label={sec:org5f8d400}]{Insulin and Counterregulatory Hormones}
\begin{center}
\begin{tabular}{lll}
Hormone & Function & Pathway\\
\hline
Insulin & \(\uparrow\) storage & glucose \(\to\) glycogen\\
 & \(\uparrow\) growth & FA synthesis and storage\\
 &  & AA uptake, protein synthesis\\
\hline
Glucagon & mobilizes stores & \(\uparrow\) gluconeogenesis\\
 & maintain blood glucose & \(\uparrow\) glycogenolysis\\
 & during a fast & FA release\\
\hline
Epinephrine & mobilize fuel during & \(\uparrow\) glycogenolysis\\
 & acute stress & FA release\\
\hline
Cortisol & long-term fuel requirements & \(\uparrow\) AA mobilization\\
 &  & from muscle\\
 &  & \(\uparrow\) gluconeogenesis for\\
 &  & glycogen synthesis\\
 &  & \(\uparrow\) FA release\\
\end{tabular}
\end{center}
\end{frame}

\begin{frame}[label={sec:org8bc9844}]{Post Prandial Production}
\begin{figure}[htbp]
\centering
\includegraphics[width=0.5\textwidth]{./figures/meal.PNG}
\caption{\label{fig:org28dbcc9}
Carbohydrate rich meal}
\end{figure}
\end{frame}

\begin{frame}[label={sec:org5d24e97}]{Fasting}
\begin{figure}[htbp]
\centering
\includegraphics[width=0.9\textwidth]{./figures/counter_hormones.PNG}
\caption{\label{fig:orga0ecc9d}
Low Blood Glucose}
\end{figure}
\end{frame}

\section{Transport}
\label{sec:orgc4df01f}
\begin{frame}[label={sec:org045401f}]{Digestion}
\begin{figure}[htbp]
\centering
\includegraphics[width=0.9\textwidth]{./figures/digestion.PNG}
\caption{\label{fig:orgaaeed64}
Digestion of Carbohydrates}
\end{figure}
\end{frame}



\begin{frame}[label={sec:org1642635}]{Absorption from the Intestine}
\begin{itemize}
\item NA-dependent transporter
\begin{itemize}
\item transport glucose \(\uparrow\) the concentration gradient
\item transport NA \(\downarrow\) the concentration gradient
\end{itemize}
\item Facilitative Glucose Transport
\begin{itemize}
\item transport glucose \(\downarrow\) the concentration gradient
\item GLUT1 to GLUT5
\end{itemize}
\item Galactose and Fructose Transport
\begin{itemize}
\item Gal uses same mechanism as glucose
\item Fructose relies on facilitated diffusion via GLUT5
\end{itemize}
\end{itemize}
\end{frame}

\begin{frame}[label={sec:orgfa49fb9}]{Absorption from the Intestine}
\begin{figure}[htbp]
\centering
\includegraphics[width=0.9\textwidth]{./figures/intestine.PNG}
\caption{\label{fig:orgbac8fea}
Absorption from the intestine}
\end{figure}
\end{frame}



\begin{frame}[label={sec:orge7bf13f}]{GLUTs}
\begin{center}
\begin{tabular}{lll}
Transporter & Distribution & Comments\\
\hline
GLUT1 & erythrocyte & barrier cells\\
 & brain barrier & \(\uparrow\) affinity transporter\\
 & retina barrier & \\
 & placenta barrier & \\
 & testis barrier & \\
\hline
GLUT2 & Liver & \(\uparrow\) capacity, \(\downarrow\) affinity\\
 & Kidney & may be glucose sensor\\
 & Pancratic \(\beta\)-cell & in pancreas\\
 & intestine & \\
\hline
GLUT3 & Neurons & \(\uparrow\) affinity  transporter in CNS\\
\hline
GLUT4 & Adipose & insulin sensitive transport\\
 & Skeletal muscle & \(\uparrow\) insuline \(\to\) \(\uparrow\) number\\
 & Heart muscle & \(\uparrow\) affinity\\
\hline
GLUT5 & Intestinal epithelium & fructose transport\\
 & spermatozoa & \\
\end{tabular}
\end{center}
\end{frame}

\section{Glycogen}
\label{sec:orge374b8c}

\begin{frame}[label={sec:orga7e78c9}]{Glycogen}
\begin{itemize}
\item glycogen is the storage form of glucose found in glycogen particles
\item large polymer of branched glucose polysaccharide
\item composed of glucosyl chains linked by \(\alpha\)-1-4-glycosidic bonds
\item \(\alpha\)-1-6-branches every 8 to 10 residues
\begin{itemize}
\item allows parallel processing
\item \(\uparrow\) solubility
\end{itemize}
\end{itemize}
\end{frame}

\begin{frame}[label={sec:orgfb10a72}]{Synthesis}
\begin{figure}[htbp]
\centering
\includegraphics[width=0.9\textwidth]{./figures/glycogen_synth.PNG}
\caption{\label{fig:org2198a09}
Glycogen Synthesis}
\end{figure}
\end{frame}

\begin{frame}[label={sec:org135d9d1}]{Degradation}
\begin{figure}[htbp]
\centering
\includegraphics[width=0.9\textwidth]{./figures/glycogen_degradation.PNG}
\caption{\label{fig:org0c90329}
Glycogen Synthesis and Degradation}
\end{figure}
\end{frame}

\begin{frame}[label={sec:orgedf829b}]{Synthesis and Degradation}
\begin{figure}[htbp]
\centering
\includegraphics[width=0.9\textwidth]{./figures/glycogen_synth_deg.PNG}
\caption{\label{fig:org6803cf2}
Glycogen Synthesis and Degradation}
\end{figure}
\end{frame}

\begin{frame}[label={sec:org7596d61}]{Regulation}
\begin{figure}[htbp]
\centering
\includegraphics[width=0.9\textwidth]{./figures/glycogen_enzyme_reg.PNG}
\caption{\label{fig:orgd0ce86a}
Regulations of Glycogen Synthesis and Degradation}
\end{figure}
\end{frame}

\begin{frame}[label={sec:orgaf11009}]{Regulation}
\begin{block}{Liver}
\begin{center}
\begin{tabular}{lll}
state & regulators & response\\
\hline
Fasting & \(\uparrow\) glucagon & \(\uparrow\) degradation\\
 & \(\downarrow\) insulin & \\
 & \(\uparrow\) cAMP & \\
CHO meal & \(\downarrow\) glucagon & \(\uparrow\) synthesis\\
 & \(\uparrow\) insulin & \\
 & \(\downarrow\) cAMP & \\
exercise \& & \(\uparrow\) epinephrine & \(\uparrow\) degradation\\
stress & \(\uparrow\) cAMP & \\
\end{tabular}
\end{center}
\end{block}


\begin{block}{Muscle}
\begin{center}
\begin{tabular}{lll}
state & regulators & response\\
\hline
Fasting & \(\downarrow\) insulin & \(\uparrow\) degradation\\
(rest) &  & \(\downarrow\) gluc transport\\
 & \(\uparrow\) cAMP & \\
CHO meal & \(\uparrow\) insulin & \(\uparrow\) synthesis\\
(rest) &  & \(\uparrow\) gluc transport\\
 &  & \\
exercise & \(\uparrow\) epinephrine & glycolysis\\
 & \(\uparrow\) cAMP & \(\downarrow\) synthesis\\
 & \(\downarrow\) AMP & \(\downarrow\) degradation\\
\end{tabular}
\end{center}
\end{block}
\end{frame}

\section{Sugar Metabolism Pathways}
\label{sec:org0e5527d}
\begin{frame}[label={sec:orgedbc8d4}]{Pathways}
\begin{itemize}
\item Fructose
\item Galactose
\item Pentose Phosphate Pathway
\end{itemize}
\end{frame}
\begin{frame}[label={sec:org11b1c15}]{Fructose Metabolism}
\begin{figure}[htbp]
\centering
\includegraphics[width=0.9\textwidth]{./figures/fructose_met.PNG}
\caption{\label{fig:orgdb48116}
Fructose Metabolism}
\end{figure}
\end{frame}

\begin{frame}[label={sec:org16f347f}]{Fructose Synthesis}
\begin{columns}
\begin{column}{0.5\columnwidth}
\begin{figure}[htbp]
\centering
\includegraphics[width=0.9\textwidth]{./figures/fructose_syn.PNG}
\caption{\label{fig:org088c3c3}
Fructose Synthesis}
\end{figure}
\end{column}

\begin{column}{0.5\columnwidth}
\begin{itemize}
\item polyol pathway
\item present in most tissues
\end{itemize}
\end{column}
\end{columns}
\end{frame}

\begin{frame}[label={sec:orgf1a3968}]{Galactose Metabolism}
\begin{figure}[htbp]
\centering
\includegraphics[width=0.9\textwidth]{./figures/gal_met.PNG}
\caption{\label{fig:org2cbd24b}
Galactose Metabolism}
\end{figure}
\end{frame}

\begin{frame}[label={sec:orgeb96930}]{Pentose Phosphate Pathway}
\begin{block}{Oxidative Phase}
\begin{itemize}
\item glucose 6-P \(\to\) NADPH + ribose 5-P
\item Glucose 6-P dehydrogenase catalyses first step
\item NADPH is for reducing reactions
\begin{itemize}
\item NADPH/NADP\^{}+ \textgreater{}\textgreater{}\textgreater{} NADH/NAD\^{}+
\item NADH is rapidly converted to NAD\^{}+ in the ETC
\end{itemize}
\end{itemize}
\end{block}
\begin{block}{Non-oxidative Phase}
\begin{itemize}
\item reversible rxns
\item convert glycolytic intermediates to 5 carbon sugars
\end{itemize}
\end{block}
\end{frame}
\begin{frame}[label={sec:org1b1a5ac}]{Pentose Phosphate Pathway}
\begin{itemize}
\item ribose-5-P required for purine and pyrimidine synthesis
\item NADPH required for detoxification and synthetic reaction
\item Detoxification
\begin{itemize}
\item Reduction of oxidized glutathione
\item Cytochrome p450 monoxygenases
\end{itemize}
\item Synthetic reactions
\begin{itemize}
\item FA synthesis
\item Cholesterol
\item neurotransmitters
\item deoxynucleotide
\item superoxide
\end{itemize}
\end{itemize}
\end{frame}

\section{Synthesis}
\label{sec:orgd397ad3}
\section{Gluconeogenesis}
\label{sec:orgeeae2a2}
\end{document}
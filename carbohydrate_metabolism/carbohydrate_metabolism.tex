% Created 2019-07-17 Wed 15:29
% Intended LaTeX compiler: pdflatex
\documentclass[presentation, smaller]{beamer}
\usepackage[utf8]{inputenc}
\usepackage[T1]{fontenc}
\usepackage{graphicx}
\usepackage{grffile}
\usepackage{longtable}
\usepackage{wrapfig}
\usepackage{rotating}
\usepackage[normalem]{ulem}
\usepackage{amsmath}
\usepackage{textcomp}
\usepackage{amssymb}
\usepackage{capt-of}
\usepackage{hyperref}
\hypersetup{colorlinks,linkcolor=white,urlcolor=blue}
\usepackage{textpos}
\usepackage{textgreek}
\usepackage[version=4]{mhchem}
\usepackage{chemfig}
\usepackage{siunitx}
\usepackage{gensymb}
\usepackage[usenames,dvipsnames]{xcolor}
\usepackage[T1]{fontenc}
\usepackage{lmodern}
\usepackage{verbatim}
\usepackage{tikz}
\usepackage{wasysym}
\usetikzlibrary{shapes.geometric,arrows,decorations.pathmorphing,backgrounds,positioning,fit,petri}
\usetheme{Hannover}
\usecolortheme{whale}
\author{Matthew Henderson, PhD, FCACB}
\date{\today}
\title{Carbohydrate Metabolism}
\institute[NSO]{Newborn Screening Ontario | The University of Ottawa}
\titlegraphic{\includegraphics[height=1cm,keepaspectratio]{../logos/NSO_logo.pdf}\includegraphics[height=1cm,keepaspectratio]{../logos/cheo-logo.png} \includegraphics[height=1cm,keepaspectratio]{../logos/UOlogoBW.eps}}
\hypersetup{
 pdfauthor={Matthew Henderson, PhD, FCACB},
 pdftitle={Carbohydrate Metabolism},
 pdfkeywords={},
 pdfsubject={},
 pdfcreator={Emacs 26.1 (Org mode 9.1.9)}, 
 pdflang={English}}
\begin{document}

\maketitle

%\logo{\includegraphics[width=1cm,height=1cm,keepaspectratio]{../logos/NSO_logo_small.pdf}~%
%    \includegraphics[width=1cm,height=1cm,keepaspectratio]{../logos/UOlogoBW.eps}%
%}

\vspace{220pt}
\beamertemplatenavigationsymbolsempty
\setbeamertemplate{caption}[numbered]
\setbeamerfont{caption}{size=\tiny}
% \addtobeamertemplate{frametitle}{}{%
% \begin{textblock*}{100mm}(.85\textwidth,-1cm)
% \includegraphics[height=1cm,width=2cm]{cat}
% \end{textblock*}}

\section{Introduction}
\label{sec:orgfc30a68}
\begin{frame}[label={sec:orgbee72d6}]{Carbohydrate Digestion}
\begin{figure}[htbp]
\centering
\includegraphics[width=0.7\textwidth]{./figures/carb_digest.PNG}
\caption{\label{fig:orgc694711}
Carbohydrate Digestions}
\end{figure}
\end{frame}

\begin{frame}[label={sec:org608f721}]{Major Pathways of Glucose Metabolism}
\begin{figure}[htbp]
\centering
\includegraphics[width=0.9\textwidth]{./figures/glucose_pathways.PNG}
\caption{\label{fig:org5283a92}
Major Pathways of Glucose Metabolism}
\end{figure}
\end{frame}

\begin{frame}[label={sec:orgd4f606e}]{Conversion to AAs and FAs}
\begin{figure}[htbp]
\centering
\includegraphics[width=0.65\textwidth]{./figures/glucose_conversion.PNG}
\caption{\label{fig:org6b7d870}
Conversion of Glucose}
\end{figure}
\end{frame}

\section{Regulation}
\label{sec:org57ec49d}
\begin{frame}[label={sec:org72a5e1c}]{Major Hormones of Metabolic Homeostasis}
\begin{itemize}
\item \alert{Insulin} is the main anabolic hormone
\begin{itemize}
\item promotes use of glucose as fuel
\begin{itemize}
\item \(\uparrow\) transport into cells
\end{itemize}
\item storage of glucose as glycogen
\item conversion of glucose \(\to\) TAGs
\item TAG storage in adipose tissue
\item AA uptake and protein synthesis in muscle
\end{itemize}
\item \alert{Glucagon} is catabolic
\begin{itemize}
\item maintain fuel availability in the absence of dietary glucose
\item stimulates glycogenolysis
\item stimulates gluconeogenesis from lactate, glycerol and AAs
\item mobilizing FAs from adipose TAGs
\item acts on liver and adipose, muscle has no receptor
\end{itemize}
\end{itemize}
\end{frame}

\begin{frame}[label={sec:orgbcfa7b3}]{Major Hormones of Metabolic Homeostasis}
\begin{figure}[htbp]
\centering
\includegraphics[width=0.9\textwidth]{./figures/regulation.PNG}
\caption{\label{fig:org4d60338}
Glucose Homeostasis}
\end{figure}
\end{frame}


\begin{frame}[label={sec:org7b513fe}]{Insulin and Counterregulatory Hormones}
\begin{center}
\begin{tabular}{lll}
Hormone & Function & Pathway\\
\hline
Insulin & \(\uparrow\) storage & glucose \(\to\) glycogen\\
 & \(\uparrow\) growth & FA synthesis and storage\\
 &  & AA uptake, protein synthesis\\
\hline
Glucagon & mobilizes stores & \(\uparrow\) gluconeogenesis\\
 & maintain blood glucose & \(\uparrow\) glycogenolysis\\
 & during a fast & FA release\\
\hline
Epinephrine & mobilize fuel during & \(\uparrow\) glycogenolysis\\
 & acute stress & FA release\\
\hline
Cortisol & long-term fuel requirements & \(\uparrow\) AA mobilization\\
 &  & from muscle\\
 &  & \(\uparrow\) gluconeogenesis for\\
 &  & glycogen synthesis\\
 &  & \(\uparrow\) FA release\\
\end{tabular}
\end{center}
\end{frame}

\begin{frame}[label={sec:org0e5d7cd}]{Post Prandial Production}
\begin{figure}[htbp]
\centering
\includegraphics[width=0.5\textwidth]{./figures/meal.PNG}
\caption{\label{fig:org9903ab2}
Carbohydrate rich meal}
\end{figure}
\end{frame}

\begin{frame}[label={sec:org583420e}]{Fasting}
\begin{figure}[htbp]
\centering
\includegraphics[width=0.9\textwidth]{./figures/counter_hormones.PNG}
\caption{\label{fig:org7964309}
Low Blood Glucose}
\end{figure}
\end{frame}

\section{Transport}
\label{sec:org9642a58}
\begin{frame}[label={sec:orgfc551ca}]{Digestion}
\begin{figure}[htbp]
\centering
\includegraphics[width=0.9\textwidth]{./figures/digestion.PNG}
\caption{\label{fig:org906a726}
Digestion of Carbohydrates}
\end{figure}
\end{frame}



\begin{frame}[label={sec:org93c7fc5}]{Absorption from the Intestine}
\begin{itemize}
\item NA-dependent transporter
\begin{itemize}
\item transport glucose \(\uparrow\) the concentration gradient
\item transport NA \(\downarrow\) the concentration gradient
\end{itemize}
\item Facilitative Glucose Transport
\begin{itemize}
\item transport glucose \(\downarrow\) the concentration gradient
\item GLUT1 to GLUT5
\end{itemize}
\item Galactose and Fructose Transport
\begin{itemize}
\item Gal uses same mechanism as glucose
\item Fructose relies on facilitated diffusion via GLUT5
\end{itemize}
\end{itemize}
\end{frame}

\begin{frame}[label={sec:org62d00f9}]{Absorption from the Intestine}
\begin{figure}[htbp]
\centering
\includegraphics[width=0.9\textwidth]{./figures/intestine.PNG}
\caption{\label{fig:org111b321}
Absorption from the intestine}
\end{figure}
\end{frame}



\begin{frame}[label={sec:orgdf508cc}]{GLUTs}
\begin{center}
\begin{tabular}{lll}
Transporter & Distribution & Comments\\
\hline
GLUT1 & erythrocyte & barrier cells\\
 & brain barrier & \(\uparrow\) affinity transporter\\
 & retina barrier & \\
 & placenta barrier & \\
 & testis barrier & \\
\hline
GLUT2 & Liver & \(\uparrow\) capacity, \(\downarrow\) affinity\\
 & Kidney & may be glucose sensor\\
 & Pancratic \(\beta\)-cell & in pancreas\\
 & intestine & \\
\hline
GLUT3 & Neurons & \(\uparrow\) affinity  transporter in CNS\\
\hline
GLUT4 & Adipose & insulin sensitive transport\\
 & Skeletal muscle & \(\uparrow\) insuline \(\to\) \(\uparrow\) number\\
 & Heart muscle & \(\uparrow\) affinity\\
\hline
GLUT5 & Intestinal epithelium & fructose transport\\
 & spermatozoa & \\
\end{tabular}
\end{center}
\end{frame}

\section{Glycogen}
\label{sec:orgd6dcd1f}
\section{Sugar Metabolism}
\label{sec:org48dbe2d}
\section{Synthesis}
\label{sec:org9add830}
\section{Gluconeogenesis}
\label{sec:org82ebff2}
\end{document}
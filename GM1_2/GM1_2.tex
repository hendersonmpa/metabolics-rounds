% Created 2018-12-07 Fri 16:37
\documentclass[presentation, smaller]{beamer}
\usepackage[utf8]{inputenc}
\usepackage[T1]{fontenc}
\usepackage{fixltx2e}
\usepackage{graphicx}
\usepackage{grffile}
\usepackage{longtable}
\usepackage{wrapfig}
\usepackage{rotating}
\usepackage[normalem]{ulem}
\usepackage{amsmath}
\usepackage{textcomp}
\usepackage{amssymb}
\usepackage{capt-of}
\usepackage{hyperref}
\hypersetup{colorlinks,linkcolor=white,urlcolor=blue}
\usepackage{textpos}
\usepackage{textgreek}
\usepackage[version=4]{mhchem}
\usepackage{chemfig}
\usepackage{siunitx}
\usepackage{gensymb}
\usepackage[usenames,dvipsnames]{xcolor}
\usepackage[T1]{fontenc}
\usepackage{lmodern}
\usepackage{verbatim}
\usepackage{tikz}
\usetikzlibrary{shapes.geometric,arrows,decorations.pathmorphing,backgrounds,positioning,fit,petri}
\usetheme{Ilmenau}
\usecolortheme{whale}
\author{Matthew Henderson, PhD, FCACB}
\date{\today}
\title{Sphingolipid Degradation: GM1/2 Gangliosidosis}
\institute[NSO]{Newborn Screening Ontario | The University of Ottawa}
\titlegraphic{\includegraphics[height=1cm,keepaspectratio]{../logos/NSO_logo.pdf}\includegraphics[height=1cm,keepaspectratio]{../logos/cheo-logo.png} \includegraphics[height=1cm,keepaspectratio]{../logos/UOlogoBW.eps}}
\hypersetup{
 pdfauthor={Matthew Henderson, PhD, FCACB},
 pdftitle={Sphingolipid Degradation: GM1/2 Gangliosidosis},
 pdfkeywords={},
 pdfsubject={},
 pdfcreator={Emacs 25.2.1 (Org mode 8.3.4)}, 
 pdflang={English}}
\begin{document}

\maketitle
%\logo{\includegraphics[width=1cm,height=1cm,keepaspectratio]{../logos/NSO_logo_small.pdf}~%
%    \includegraphics[width=1cm,height=1cm,keepaspectratio]{../logos/UOlogoBW.eps}%
%}

\vspace{220pt}
\beamertemplatenavigationsymbolsempty
\setbeamertemplate{caption}[numbered]
\setbeamerfont{caption}{size=\tiny}
% \addtobeamertemplate{frametitle}{}{%
% \begin{textblock*}{100mm}(.85\textwidth,-1cm)
% \includegraphics[height=1cm,width=2cm]{cat}
% \end{textblock*}}

\tikzstyle{chemical} = [rectangle, rounded corners, text width=5em, minimum height=1em,text centered, draw=black, fill=none]
\tikzstyle{hardware} = [rectangle, rounded corners, text width=5em, minimum height=1em,text centered, draw=black, fill=gray!30]
\tikzstyle{ms} = [rectangle, rounded corners, text width=5em, minimum height=1em,text centered, draw=orange, fill=none]
\tikzstyle{msw} = [rectangle, rounded corners, text width=7em, minimum height=1em,text centered, draw=orange, fill=none]
\tikzstyle{label} = [rectangle,text width=8em, minimum height=1em, text centered, draw=none, fill=none]
\tikzstyle{hl} = [rectangle, rounded corners, text width=5em, minimum height=1em,text centered, draw=black, fill=red!30]
\tikzstyle{box} = [rectangle, rounded corners, text width=5em, minimum height=5em,text centered, draw=black, fill=none]
\tikzstyle{arrow} = [thick,->,>=stealth]
\tikzstyle{hl-arrow} = [ultra thick,->,>=stealth,draw=red]

\section{Introduction}
\label{sec:orgheadline12}

\begin{frame}[label={sec:orgheadline1}]{Sphingolipid degradation}
\begin{figure}[htb]
\centering
\includegraphics[width=0.6\textwidth]{./figures/sl_degradation.png}
\caption[deg]{\label{fig:sld}
Sphingolipid degradation}
\end{figure}
\end{frame}


\begin{frame}[label={sec:orgheadline2}]{GM1 Gangliosidosis}
\begin{itemize}
\item Defect in \(\beta\)-gangliosidase
\item GM1 ganglioside accumulates in the brain and visera
\item Infantile, juvenile and adult forms
\begin{itemize}
\item Residual enzyme function
\end{itemize}
\item Devastating degenerative disease
\item GM1 gangliosidosis of all types is estimated to occur in 1:100,000 - 300,000
\end{itemize}

\begin{block}{MPS IVB - Morquio B}
\begin{itemize}
\item Clinically indistinguishable from MPS IVA
\begin{itemize}
\item skeletal changes, including short stature and skeletal dysplasia.
\item normal intelligence
\end{itemize}
\item The prevalence of MPS IVB has been reported as 1:250,000 - 1,000,000
\end{itemize}
\end{block}
\end{frame}


\begin{frame}[label={sec:orgheadline3}]{\(\beta\)-galactosidase}
\begin{figure}[htb]
\centering
\includegraphics[width=0.7\textwidth]{./figures/bgalatosidase.png}
\caption[bgal]{\label{fig:bgal}
\(\beta\)-galactosidase}
\end{figure}
\end{frame}


\begin{frame}[label={sec:orgheadline4}]{Lysosomal multi-enzyme complex}
\begin{columns}
\begin{column}{0.6\columnwidth}
\begin{itemize}
\item \(\beta\)-galactosidase forms a heterotrimeric complex with:
\begin{itemize}
\item cathepsin A/PPCA : CTSA
\item neuraminidase: NEU1
\end{itemize}

\item \(\downarrow\) cathepsin A \(\to\) 2\degree  deficiency of NEU1
\begin{itemize}
\item ML-1 (sialidosis)
\end{itemize}
\end{itemize}
\end{column}

\begin{column}{0.5\columnwidth}
\begin{figure}[htb]
\centering
\includegraphics[width=\textwidth]{./figures/lmc.jpg}
\caption[lmc]{\label{fig:lmc}
lysosomal multi-enzyme complex}
\end{figure}
\end{column}
\end{columns}
\end{frame}


\begin{frame}[label={sec:orgheadline5}]{GM2 Gangliosidosis}
\begin{itemize}
\item Three genetic and biochemical subtypes
\begin{itemize}
\item Tay-Sachs disease
\item Sandhoff disease
\item GM2 activator deficiency
\end{itemize}
\item Impaired lysosomal catabolism of GM2 ganglioside.
\item GM2 storage in neurons in Tay-Sachs and Sandhoff
\begin{itemize}
\item Sandhoff \(\uparrow\) asialo-GM2 in brain, globoside and oligosacarides in viseral organs
\end{itemize}
\item Progressive cerebral degeneration
\item Prior to population-based carrier screening the incidence of TSD was \textasciitilde{}1:3600 Ashkenazi Jewish births.
\begin{itemize}
\item Incidence of TSD in the Ashkenazi Jewish population of North America has reduced > 90\%
\end{itemize}
\item French-Canadian founder effect
\end{itemize}
\end{frame}


\begin{frame}[label={sec:orgheadline6}]{GM2 ganglioside storage diseases}
\begin{center}
\begin{tabular}{llrl}
Disorder & Onset & Death (y) & Enzyme\\
\hline
Tay-Sachs disease & 3-6 months & 2-4 & Hex A\\
Sandhoff disease & 3-6 months & 2-4 & Hex A\&B\\
AB variant & 3-6 months &  & Activator\\
Adult GM 2 gangliosidosis & 2-6 years & 5-15 & Hex A\\
Juvenile GM 2 gangliosidosis & 2 yrs-adult & Variable & Hex A\\
\end{tabular}
\end{center}
\end{frame}

\begin{frame}[label={sec:orgheadline7}]{Lysosomal \(\beta\)-Hexosaminidase enzymes}
\begin{itemize}
\item Functional lysosomal \(\beta\)-hexosaminidase enzymes are dimeric.
\item Three isozymes are produced through the combination of \(\alpha\)
  and \(\beta\) subunits
\end{itemize}

\begin{center}
\begin{tabular}{lll}
Isozyme & Dimer composition & Function\\
\hline
A & \(\alpha\)/\(\beta\) & hydrolyzes GM2 ganglioside\\
B & \(\beta\)/\(\beta\) & non-GM2 gangliosides w terminal hexosamine\\
S & \(\alpha\)/\(\alpha\) & no known physiological function\\
\end{tabular}
\end{center}
\end{frame}


\begin{frame}[label={sec:orgheadline8}]{Hexosaminidase A: Tay-Sachs}
\begin{figure}[htb]
\centering
\includegraphics[width=0.8\textwidth]{./figures/hexosaminidasea.png}
\caption[hexa]{\label{fig:hexa}
Hexosaminidase A}
\end{figure}
\end{frame}


\begin{frame}[label={sec:orgheadline9}]{Hexosaminidase A \& B: Sandhoff disease}
\begin{figure}[htb]
\centering
\includegraphics[width=0.8\textwidth]{./figures/hexosaminidaseab.png}
\caption[hexb]{\label{fig:hexb}
Hexosaminidase A \& B}
\end{figure}
\end{frame}


\begin{frame}[label={sec:orgheadline10}]{Lysosomal Trafficking}
\begin{figure}[htb]
\centering
\includegraphics[width=0.8\textwidth]{./figures/lysosome_trafficking.jpeg}
\caption[traf]{\label{fig:traf}
Lysosomal protein trafficking receptors}
\end{figure}

\footnotesize
\begin{itemize}
\item \(\beta\)-galactosidase, hexoaminidase A and B require the M6P-receptor
\item GM2 activator protein - sortilin
\end{itemize}
\end{frame}


\begin{frame}[label={sec:orgheadline11}]{Genetics}
\begin{block}{GM1}
\begin{itemize}
\item GLB1: autosomal recessive
\item \textasciitilde{} 150 mutations in GLB1 have been described
\item Neither the type or location correlate with phenotype
\end{itemize}
\end{block}

\begin{block}{GM2}
\begin{itemize}
\item HEXA, HEXB and GM2A: autosomal recessive
\item > 130 mutations in HEXA
\begin{itemize}
\item > 3 alleles comprise \textasciitilde{}95\% of Askenazi Jewish disease alleles
\item Good correlation with phenotype
\end{itemize}
\item > 40 mutations in HEXB
\item 6 in GM2A
\end{itemize}
\end{block}
\end{frame}

\section{Clinical Findings}
\label{sec:orgheadline15}

\begin{frame}[label={sec:orgheadline13}]{GM1 Signs and Symptoms}
\footnotesize

\begin{center}
\begin{tabular}{lllll}
Finding & Infantile & Juvenile & Adult & MPS IVB\\
\hline
Onset of symptoms & <1 year & 1-10 years & 10+ years & 3-5 years\\
Eye findings & CRS & CC & +/– CC & CC\\
Motor abnormalities & + & + & Extrapyramidal & \footnotemark\\
Hepatosplenomegaly & + & +/– & – & –\\
Cardiac involvement & +/– & +/– & +/– & +\\
Coarse facial features & +/– & – & – & \footnotemark[1]{}\\
Skeletal findings & + & +/– & – & +\\
Neuroimaging & PA & PA & +/– mild atrophy & \footnotemark[1]{}\\
Urine (GAG) & \footnotemark & \footnotemark[2]{} & \footnotemark[2]{} & Keratan sulfate \footnotemark\\
\end{tabular}
\end{center}\footnotetext[1]{Secondary to bony changes}\footnotetext[2]{Oligosacaride with terminal galactose}\footnotetext[3]{FN have been observed}
\end{frame}


\begin{frame}[label={sec:orgheadline14}]{GM2 Signs and Symptoms}
\begin{center}
\begin{tabular}{llll}
Finding & Infantile & Juvenile & Adult\\
\hline
Onset of symptoms & <1 year & 2-10 years & 10+ years\\
Eye findings & CRS, blindness & +/- CRS & \\
movement & weakness & ataxia, dysarthria & dystonia, ataxia\\
Neurological & startle response, & seizures & psychosis\\
 & seizures &  & \\
\end{tabular}
\end{center}
\end{frame}


\section{Laboratory Investigations}
\label{sec:orgheadline18}

\begin{frame}[label={sec:orgheadline16}]{Biochemistry}
\begin{block}{GM1}
\begin{itemize}
\item Urine oligosacarides
\item Mucopolysacarides: \(\uparrow\) keratin sulfate
\item \emph{in vitro} \(\beta\)-galactosidase activity: leukocytes and DBS
\begin{itemize}
\item 4-MU-\(\beta\)-d-galactopyranoside
\end{itemize}
\end{itemize}
\end{block}

\begin{block}{GM2}
\begin{itemize}
\item \emph{in vitro} Hexoaminidase activity: leukocytes, fibroblasts
\begin{itemize}
\item 4-MU-6-sulfo-\(\beta\)-glucosaminide
\item specific for the \(\alpha\) subunit
\item \(\uparrow\) in Sandoff
\item normal in GM2 activator deficiency
\end{itemize}
\end{itemize}
\end{block}
\end{frame}

\begin{frame}[label={sec:orgheadline17}]{Pathology}
\begin{columns}
\begin{column}{0.5\columnwidth}
\begin{itemize}
\item Widespread deposition of Gb3
\item Vacuoles seen in variety of cells, \(\uparrow\) endothelium of blood vessels
\end{itemize}
\end{column}

\begin{column}{0.5\columnwidth}
\begin{figure}[htb]
\centering
\includegraphics[width=0.7\textwidth]{./figures/Fabrys-disease.jpg}
\caption[em]{\label{fig:biopsy}
EM showing concentric or lamellar structure of lysosomal inclusions in Fabry disease renal biopsy}
\end{figure}
\end{column}
\end{columns}
\end{frame}


\section{Treatment}
\label{sec:orgheadline22}
\begin{frame}[label={sec:orgheadline19}]{Prenatal Screening for Tay-Sachs}
\begin{center}
\begin{tabular}{ll}
Group & number\\
\hline
Total screened & 9.53 x 10\(^{\text{6}}\) (seven countries)\\
Carriers identified & 36 418\\
Couples at risk & 1056\\
Pregnancies monitored & 2415 \footnotemark[2]{}\\
Affected fetuses & 469\\
Aborted & 451\\
Normal offspring born & 1881\\
Birth/year w Tay-Sachs & \\
Prior to 1969 & 100 (US \& Canada) 80\% Jewish\\
1980 & 13 80\% non-Jewish\\
1985–1992 & 3-10 80\% non-Jewish\\
\end{tabular}
\end{center}

\begin{itemize}
\item > 90\% reduction in the disease in Jewish population
\end{itemize}
\end{frame}

\begin{frame}[label={sec:orgheadline20}]{Treatment}
\begin{block}{GM1}
\begin{itemize}
\item no curative treatment to date
\end{itemize}
\end{block}
\begin{block}{GM2}
\begin{itemize}
\item treat seizures
\item no curative treatment to date
\end{itemize}
\end{block}
\end{frame}
\begin{frame}[label={sec:orgheadline21}]{Next time}
\begin{itemize}
\item Disorders of Sphingolipid Degradation continued\ldots{}
\begin{itemize}
\item Krabbe and Metachromatic Leukodystrophy
\end{itemize}
\end{itemize}
\end{frame}
\end{document}

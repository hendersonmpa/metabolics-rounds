% Created 2019-10-16 Wed 20:55
% Intended LaTeX compiler: pdflatex
\documentclass[presentation, smaller]{beamer}
\usepackage[utf8]{inputenc}
\usepackage[T1]{fontenc}
\usepackage{graphicx}
\usepackage{grffile}
\usepackage{longtable}
\usepackage{wrapfig}
\usepackage{rotating}
\usepackage[normalem]{ulem}
\usepackage{amsmath}
\usepackage{textcomp}
\usepackage{amssymb}
\usepackage{capt-of}
\usepackage{hyperref}
\hypersetup{colorlinks,linkcolor=white,urlcolor=blue}
\usepackage{textpos}
\usepackage{textgreek}
\usepackage[version=4]{mhchem}
\usepackage{chemfig}
\usepackage{siunitx}
\usepackage{gensymb}
\usepackage[usenames,dvipsnames]{xcolor}
\usepackage[T1]{fontenc}
\usepackage{lmodern}
\usepackage{verbatim}
\usepackage{tikz}
\usepackage{wasysym}
\usetikzlibrary{shapes.geometric,arrows,decorations.pathmorphing,backgrounds,positioning,fit,petri}
\usetheme{Hannover}
\usecolortheme{whale}
\author{Matthew Henderson, PhD, FCACB}
\date{\today}
\title{Disorders of Fructose Metabolism}
\institute[NSO]{Newborn Screening Ontario | The University of Ottawa}
\titlegraphic{\includegraphics[height=1cm,keepaspectratio]{../logos/NSO_logo.pdf}\includegraphics[height=1cm,keepaspectratio]{../logos/cheo-logo.png} \includegraphics[height=1cm,keepaspectratio]{../logos/UOlogoBW.eps}}
\hypersetup{
 pdfauthor={Matthew Henderson, PhD, FCACB},
 pdftitle={Disorders of Fructose Metabolism},
 pdfkeywords={},
 pdfsubject={},
 pdfcreator={Emacs 26.1 (Org mode 9.1.9)}, 
 pdflang={English}}
\begin{document}

\maketitle

%\logo{\includegraphics[width=1cm,height=1cm,keepaspectratio]{../logos/NSO_logo_small.pdf}~%
%    \includegraphics[width=1cm,height=1cm,keepaspectratio]{../logos/UOlogoBW.eps}%
%}

\vspace{220pt}
\beamertemplatenavigationsymbolsempty
\setbeamertemplate{caption}[numbered]
\setbeamerfont{caption}{size=\tiny}
% \addtobeamertemplate{frametitle}{}{%
% \begin{textblock*}{100mm}(.85\textwidth,-1cm)
% \includegraphics[height=1cm,width=2cm]{cat}
% \end{textblock*}}

\section{Introduction}
\label{sec:org5750aa4}
\begin{frame}[label={sec:orgda1e948}]{Fructose Metabolism}
\begin{itemize}
\item Fructose is mainly metabolised in the liver, renal cortex and small intestinal mucosa
\item Pathway composed of fructokinase , aldolase B and triokinase .
\item Aldolase B is also involved in the glycolytic-gluconeogenic pathway
\end{itemize}
\end{frame}

\begin{frame}[label={sec:org28a8028}]{Disorders of Fructose Metabolism}
\begin{itemize}
\item Three inborn errors are known in the pathway of fructose metabolism.
\item Essential fructosuria is a harmless anomaly
\begin{itemize}
\item characterised by the appearance of fructose in the urine after the intake of fructose-containing food.
\end{itemize}
\item Hereditary fructose intolerance (HFI)
\begin{itemize}
\item fructose may provoke prompt gastrointestinal discomfort and hypoglycaemia upon ingestion
\item fructose may cause liver and kidney failure when taken persistently,
\item IV fructose can be lethal
\end{itemize}
\item Fructose-1,6-bisphosphatase (FBPase) deficiency
\begin{itemize}
\item considered an inborn error of fructose metabolism although, it is a defect of gluconeogenesis.
\item hypoglycaemia and lactic acidosis (neonatally, or later during fasting or induced by fructose)
\item may be life-threatening
\end{itemize}
\end{itemize}
\end{frame}

\begin{frame}[label={sec:orgb434875}]{Fructose Metabolism}
pathway
\end{frame}

\section{Essential Fructosuria}
\label{sec:orgb7bde87}
\begin{frame}[label={sec:org0d80129}]{Clinical Presentation}
\begin{itemize}
\item Rare non-disease
\item Detected a when test urine for reducing substances
\item Deficiency in Fructokinase
\end{itemize}
\end{frame}
\begin{frame}[label={sec:org5244a60}]{Metabolic Derangement}
\begin{itemize}
\item 10-20\% of ingested fructose is excreted in urine
\item remainder is slowly metabolised in adipose tissue and muscle
\end{itemize}
\end{frame}
\begin{frame}[label={sec:orgc3774ac}]{Genetics}
\begin{itemize}
\item AR, 1:130,000 (underestimate?)
\item KHK undergoes tissue specific alternative splicing
\begin{itemize}
\item Two isoforms
\item ketohexokinase A, widely expressed, no clear role
\item ketohexokinase C, adult liver, kidney and small intestine
\begin{itemize}
\item affected in essential fructosuria
\end{itemize}
\item Only 2 mutations found
\end{itemize}
\end{itemize}
\end{frame}
\begin{frame}[label={sec:orgc7fa4e6}]{Diagnosis and Treatment}
\begin{itemize}
\item \(\uparrow\) reducing substances in urine
\item glucose oxidase negative
\item fructosuria is dependant on fructose intake
\item no treatent required
\item prognosis is excellent
\end{itemize}
\end{frame}

\section{Hereditary Fructose Intolerance}
\label{sec:org2c0a9bf}
\begin{frame}[label={sec:orga0b0966}]{Clinical Presentation}
\begin{itemize}
\item perfectly healthy as long as they do not ingest food containing fructose, sucrose and/or sorbitol
\item no metabolic derangement occurs during breast-feeding
\item younger the child and the higher the dietary fructose load, the more severe the reaction
\item first symptoms appear with the intake of fruits and vegetables or fructose containing formula
\begin{itemize}
\item gastrointestinal discomfort, nausea, vomiting, restlessness,
pallor, sweating, trembling, lethargy and, eventually, apathy,
coma, jerks and convulsion
\end{itemize}
\item laboratory tests indicate acute liver failure and generalised dysfunction of the renal proximal tubules.
\item hypoglycaemia after fructose ingestion is short-lived and can be easily missed or masked by concomitant glucose intake
\item food aversions form
\item approximately half of all adults with HFI are free of caries
\item affected subjects may remain undiagnosed and still have a normal life span.
\end{itemize}
\end{frame}

\begin{frame}[label={sec:org3010db3}]{Metabolic Derangement}
\begin{itemize}
\item FI is caused by deficiency of the second enzyme of the fructose pathway, aldolase B
\begin{itemize}
\item splits fructose-1-phosphate (F-1-P) into dihydroxyacetone phosphate and glyceraldehyde
\end{itemize}
\item The high activity of fructokinase after intake of fructose results in accumulation of F-1-P and trapping of phosphate.
\item This has two major effects:
\begin{enumerate}
\item inhibition of glucose production by blockage of gluconeogenesis
(inhibition of aldolase A) and glycogenolysis (inhibition of glycogen phosphorylase A)
\begin{itemize}
\item induces a rapid drop in blood glucose
\end{itemize}
\item overutilization and diminished regeneration of ATP.
\begin{itemize}
\item depletion of ATP results in an increased production of uric acid
\item a release of magnesium,
\item and a series of other disturbances,including impaired protein
synthesis and ultrastructural lesions which are responsible for
hepatic and renal dysfunction
\end{itemize}
\end{enumerate}
\item glycolysis and gluconeogenesis are not impaired in the fasted state in HFI patients due to activity of aldolase A

\item same process happens in IV fructose to normal patients
\item the use of fructose, sorbitol and invert sugar has been strongly discouraged for parenteral nutrition in general
\end{itemize}
\end{frame}

\begin{frame}[label={sec:org69d550e}]{Genetics}
\begin{itemize}
\item AR
\item Three aldolase genes
\item B is the major fructaldolase of liver, renal cortex, and small intestine
\item A muscle
\item C brain
\end{itemize}
\end{frame}

\begin{frame}[label={sec:org6909ef7}]{Diagnosis and Treatment}
\begin{itemize}
\item nutritional history
\item response to fructose withdrawl
\item First tier molecular diagnosis
\item Second tier (no mutations) Enzymatic
\item Liver biopsy Aldo B activity
\begin{itemize}
\item False low Aldo B secondary to liver damage
\end{itemize}

\item acute intoxication:
\begin{itemize}
\item fresh frozen plasma
\end{itemize}
\item Remove fructose (sucrose and sorbitol) from diet
\item Prognosis on diet is excellent with normal growth,
intelligence and life span
\end{itemize}
\end{frame}

\section{Fructose-1,6-Bisphosphatase Deficiency}
\label{sec:org1502c8f}
\begin{frame}[label={sec:orgf138f00}]{Clinical Presentation}
\begin{itemize}
\item 1/2 present in the first 1-4 days of life
\begin{itemize}
\item severe hyperventilation
\begin{itemize}
\item lactic acidosis
\item hypoglycaemia
\end{itemize}
\end{itemize}
\item later irritability, apnoeic spells, tachycardia, muscle hypotonia
\item chronic ingestion of fructose does not lead to gastrointestinal symptoms
\begin{itemize}
\item no aversion to sweet foods or failure to thrive, and only rarely \(\downarrow\) liver function.
\end{itemize}
\end{itemize}
\end{frame}

\begin{frame}[label={sec:org1256596}]{Metabolic Derangement}
\begin{itemize}
\item Deficiency of hepatic FBPase, key enzyme in gluconeogenesis, impairs
the formation of glucose from all gluconeogenic precursors, including dietary fructose
\item normoglycaemia in patients is dependent on glucose (and galactose)
intake and degradation of hepatic glycogen
\item hypoglycaemia occurs when glycogen reserves are limited (newborns, fasting)
\item accumulation of the gluconeogenic substrates lactate, pyruvate, alanine, and glycerol.
\end{itemize}
\end{frame}
\begin{frame}[label={sec:orgd7e5339}]{Genetics}
\begin{itemize}
\item AR
\item Liver isoform, FBP1 gene
\item 35 mutations in all regions of the gene have been published
\end{itemize}
\end{frame}

\begin{frame}[label={sec:org2924a99}]{Diagnosis}
\begin{itemize}
\item plasma during acute episodes
\begin{itemize}
\item \(\uparrow\) lactate (up to 15–25 mM)
\item \(\downarrow\) pH
\item \(\uparrow\) lactate/pyruvate ratio (up to 40)
\item hyperalaninaemia,
\item \(\uparrow\) glycerol which may mimic hypertriglyceridaemia
\item glucagon-resistant hypoglycaemia
\item \(\uparrow\) free fatty acids and uric acid.
\end{itemize}
\item Urinary analysis reveals
\begin{itemize}
\item \(\uparrow\) lactate, alanine, glycerol,
\item in most cases, ketones and glycerol-3-phosphate.
\end{itemize}

\item molecular analysis on DNA from peripheral leukocytes
\item if no mutations found
\begin{itemize}
\item enzymatic activity in a liver biopsy
\item the residual activity may vary from zero to 30\% of normal
\end{itemize}
\end{itemize}
\end{frame}

\begin{frame}[label={sec:org7e6fae7}]{Differential Diagnosis}
\begin{itemize}
\item other disturbances in gluconeogenesis and pyruvate oxidation should be considered, including:
\begin{enumerate}
\item pyruvate dehydrogenase deficiency characterised by a low
lactate/pyruvate ratio, absence of hypoglycaemia and aggravation
of lactic acidosis by glucose infusion
\item pyruvate carboxylase deficiency
\item respiratory chain disorders
\item glycogenosis type Ia and Ib presenting with the same metabolic profile
\begin{itemize}
\item fasting hypoglycaemia and lactic acidosis and hepato nephromegaly, hyperlipidaemia, and hyperuricaemia
\end{itemize}
\item fatty acid oxidation defects presenting with fasting hypoketotic hypoglycaemia and hyperlactataemia
\end{enumerate}
\end{itemize}
\end{frame}

\begin{frame}[label={sec:orgff8a8d1}]{Treatment}
\begin{itemize}
\item acute life-threatening episodes should be treated with an IV bolus
of 20\% glucose
\item followed by a continuous infusion of glucose and bicarbonate to
control hypoglycaemia and acidosis.
\item Maintenance therapy should be aimed at avoiding fasting,
particularly during febrile episodes
\begin{itemize}
\item slowly absorbed carbohydrates (uncooked starch), and a gastric
drip, if necessary.
\end{itemize}
\item absence of any triggering effects leading to metabolic
decompensation, individuals with FBPase deficiency are healthy and
no carbohydrate supplements are needed.
\end{itemize}
\end{frame}
\end{document}
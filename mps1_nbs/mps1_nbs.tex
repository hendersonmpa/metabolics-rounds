% Created 2018-09-17 Mon 16:27
\documentclass[presentation, smaller]{beamer}
\usepackage[utf8]{inputenc}
\usepackage[T1]{fontenc}
\usepackage{fixltx2e}
\usepackage{graphicx}
\usepackage{grffile}
\usepackage{longtable}
\usepackage{wrapfig}
\usepackage{rotating}
\usepackage[normalem]{ulem}
\usepackage{amsmath}
\usepackage{textcomp}
\usepackage{amssymb}
\usepackage{capt-of}
\usepackage{hyperref}
\hypersetup{colorlinks,linkcolor=white,urlcolor=blue}
\usepackage{textpos}
\usepackage{textgreek}
\usepackage[version=4]{mhchem}
\usepackage{chemfig}
\usepackage{siunitx}
\usepackage{gensymb}
\usepackage[usenames,dvipsnames]{xcolor}
\usepackage[T1]{fontenc}
\usepackage{lmodern}
\usepackage{verbatim}
\usepackage{tikz}
\usetikzlibrary{shapes.geometric,arrows,decorations.pathmorphing,backgrounds,positioning,fit,petri}
\usetheme{Hannover}
\usecolortheme{whale}
\author{Matthew Henderson, PhD, FCACB}
\date{\today}
\title{Newborn Screening for MPS I}
\institute[NSO]{Newborn Screening Ontario | The University of Ottawa}
\titlegraphic{\includegraphics[height=1cm,keepaspectratio]{../logos/NSO_logo.pdf}\includegraphics[height=1cm,keepaspectratio]{../logos/cheo-logo.png} \includegraphics[height=1cm,keepaspectratio]{../logos/UOlogoBW.eps}}
\hypersetup{
 pdfauthor={Matthew Henderson, PhD, FCACB},
 pdftitle={Newborn Screening for MPS I},
 pdfkeywords={},
 pdfsubject={},
 pdfcreator={Emacs 25.2.1 (Org mode 8.3.4)}, 
 pdflang={English}}
\begin{document}

\maketitle


# %\logo{\includegraphics[width=1cm,height=1cm,keepaspectratio]{../logos/NSO_logo_small.pdf}~%
# %    \includegraphics[width=1cm,height=1cm,keepaspectratio]{../logos/UOlogoBW.eps}%
# }

\vspace{220pt}
\beamertemplatenavigationsymbolsempty
\setbeamertemplate{caption}[numbered]
\setbeamerfont{caption}{size=\tiny}
#  \addtobeamertemplate{frametitle}{}{%
#  \begin{textblock*}{100mm}(.85\textwidth,-1cm)
#  \includegraphics[height=1cm,width=2cm]{cat}
#  \end{textblock*}}

\tikzstyle{chemical} = [rectangle, rounded corners, text width=5em, minimum height=1em,text centered, draw=black, fill=none]
\tikzstyle{hardware} = [rectangle, rounded corners, text width=5em, minimum height=1em,text centered, draw=black, fill=gray!30]
\tikzstyle{ms} = [rectangle, rounded corners, text width=5em, minimum height=1em,text centered, draw=orange, fill=none]
\tikzstyle{msw} = [rectangle, rounded corners, text width=7em, minimum height=1em,text centered, draw=orange, fill=none]
\tikzstyle{label} = [rectangle,text width=8em, minimum height=1em, text centered, draw=none, fill=none]
\tikzstyle{hl} = [rectangle, rounded corners, text width=5em, minimum height=1em,text centered, draw=black, fill=red!30]
\tikzstyle{box} = [rectangle, rounded corners, text width=5em, minimum height=5em,text centered, draw=black, fill=none]
\tikzstyle{arrow} = [thick,->,>=stealth]
\tikzstyle{hl-arrow} = [ultra thick,->,>=stealth,draw=red]

\section{Introduction}
\label{sec:orgheadline4}
\begin{frame}[label={sec:orgheadline1}]{MPS I}
\begin{itemize}
\item Defective \(\alpha\)-L-iduronidase enzyme
\begin{itemize}
\item degradation of dermatan sulfate and heparin sulfate
\end{itemize}
\item Three clinical subtypes IH, IS and IHS

\item The estimated prevalence for MPS1 (all forms) from NBS pilot studies is: 
\begin{itemize}
\item 1/35,500 Washington ​(Scott et al. 2013)
\item 1/14,567 Missouri ​(Hopkins et al. 2015)
\item 1/17,643 Taiwan ​(Lin et al. 2013)
\end{itemize}
\end{itemize}


\begin{figure}[htb]
\centering
\includegraphics[width=0.8\textwidth]{./figures/idua.png}
\label{fig:idua}
\end{figure}
\end{frame}


\begin{frame}[label={sec:orgheadline2}]{MPS I disease spectrum}
\begin{figure}[htb]
\centering
\includegraphics[width=0.8\textwidth]{./figures/mps1clinical.png}
\label{fig:mps1}
\end{figure}
\end{frame}


\begin{frame}[label={sec:orgheadline3}]{Diagnosis and Treatment}
\begin{figure}[htb]
\centering
\includegraphics[width=0.8\textwidth]{./figures/registry.png}
\label{fig:reg}
\end{figure}


\begin{block}{Diagnosis}
\begin{itemize}
\item \(\alpha\)-L-iduronidase enzyme activity levels
\begin{itemize}
\item measured in WBCs, plasma or fibroblasts \(\to\) low to undetectable
\item unable to predict severity of MPS I
\end{itemize}
\item Two ​IDUA mutations in trans (identified by sequencing)
\end{itemize}
\end{block}

\begin{block}{Treatment}
\begin{itemize}
\item hematopoietic stem cell transplantation
\item enzyme replacement therapy with Laronidase
\item ERT + HSCT in children <2 years of age
\end{itemize}
\end{block}
\end{frame}


\section{Biochemistry}
\label{sec:orgheadline8}
\begin{frame}[label={sec:orgheadline5}]{Hydrolysis reaction catalyzed by Alpha-L-Iduronidase:}
\begin{figure}[htb]
\centering
\includegraphics[width=0.8\textwidth]{./figures/nihms3970f3.jpg}
\caption[mech]{\label{fig:mech}
Hydrolysis reaction catalyzed by Alpha-L-Iduronidase:}
\end{figure}
\end{frame}

\begin{frame}[label={sec:orgheadline6}]{IDUA exhibits Michaelis-Menten kinetics}
\begin{figure}[htb]
\centering
\includegraphics[width=0.8\textwidth]{./figures/kinetics.pdf}
\caption{\label{fig:mm}
IDUA Kinetics}
\end{figure}
\end{frame}



\begin{frame}[label={sec:orgheadline7}]{Substrate concentration}
\begin{itemize}
\item substrate concentrations 10-fold the Km recommended
\item interference from contaminating 4MU-\(\beta\)-D-glucuronide
\item the very high cost of 4MU-\(\alpha\)-L-iduronide are limiting factors.

\item \(\alpha\)-L-iduronidase enzyme assays should be conducted either under
substrate saturating conditions or, when substrate concentrations
are significantly below enzyme saturation, the values are adjusted
by arithmetic conversion using the Michaelis–Menten model
\end{itemize}

\[ 
V  = \frac{Vmax[S]}{Km + [S]}
\]
\end{frame}
\section{Screening Methods}
\label{sec:orgheadline17}


\begin{frame}[label={sec:orgheadline9}]{IDUA: Colorimetric}
\begin{itemize}
\item phenyl \(\alpha\)-L-iduronide (Phi)
\begin{itemize}
\item Hydrolysis of phenyl \(\alpha\)-L-iduronide and subsequent
colorimetric assay of the liberated phenol
\item Digests contained 5 mM phenyl \(\alpha\)-L-iduronide, 0.1 M NaCI,
and 0.08 M sodium formate buffer of pH 3.5;
\item they were incubated for 18 hr at 25\degree C under a layer of \textbf{toluene}.
\item Recovery of the phenol liberated was completed by additional extractions with \textbf{toluene}.
\item Phenol was returned to an aqueous phase for analysis by extraction of the pooled \textbf{toluene} layers with 0.05 M NaOH.
\item Turbidity was then removed by a treatment with \textbf{chloroform-amyl alcohol}.
\end{itemize}
\end{itemize}
\end{frame}


\begin{frame}[label={sec:orgheadline10}]{Spectrofluorometric}
\begin{itemize}
\item 4-Methylumbelliferyl-\(\alpha\)-L-iduronide is a fluorogenic substrate of \(\alpha\)-L-iduronidase
\item emission maximum at 445-454 nm.
\item excitation maximum for 4-MU is pH-dependent: 330, 370, and 385 nm at pH 4.6, 7.4, and 10.4
\end{itemize}

\begin{figure}[htb]
\centering
\includegraphics[width=0.4\textwidth]{./figures/9001600.png}
\caption[4MUI]{\label{fig:4mui}
4-Methylumbelliferyl-\(\alpha\)-L-Iduronide 2-sulfate}
\end{figure}
\end{frame}


\begin{frame}[label={sec:orgheadline11}]{Spectrofluorometric}
\begin{enumerate}
\item Elute one 3.1 mm DBS punch
\begin{itemize}
\item D-saccharic acid-1,4-lactone
\item 2 mM 4MU-\(\alpha\)-L-iduronide
\end{itemize}
\item Incubate for 20hrs at 37\degree C
\item Add glycine-carbonate and vortex to stop reaction
\item 30 minutes at RT
\item Measure fluorescence
\begin{itemize}
\item 4MU calibrator
\end{itemize}
\item Results uM/L blood/20 hr
\end{enumerate}
\end{frame}


\begin{frame}[label={sec:orgheadline12}]{Psuedodeficiency}
\begin{itemize}
\item Low IDUA activity \emph{in vitro}
\item A300T, steric hindrance at active site E299
\item homozygosity for p.A79T,
\item heterozygosity for p.A79T and p.V322E
\item heterozygosity for p.A79T and p. D223N
\item heterozygosity for p.D223N and p. V322E
\end{itemize}

\begin{figure}[htb]
\centering
\includegraphics[width=0.8\textwidth]{./figures/pd.png}
\label{fig:pd}
\end{figure}


\begin{itemize}
\item Missouri program
\end{itemize}
\end{frame}


\begin{frame}[label={sec:orgheadline13}]{IDUA: FIA-MS/MS}
\begin{figure}[htb]
\centering
\includegraphics[width=0.8\textwidth]{./figures/F2large.jpg}
\caption{\label{fig:msmswf}
MS/MS workflow}
\end{figure}
\end{frame}


\begin{frame}[label={sec:orgheadline14}]{IDUA: FIA-MS/MS}
\begin{itemize}
\item Positive mode ESI
\item Ten µL of the 150 µL sample via flow injection
\begin{itemize}
\item 80/20 acetonitrile/water with 0.2\% formic acid
\item flow-rate of 0.1 mL/min for 1 min then 1 mL/min for 0.5 min.
\item Data was collected during 1.5 minute of infusion,
\end{itemize}
\end{itemize}

\begin{table}[htb]
\caption{\label{tab:mrm}
IDUA transitions}
\centering
\begin{tabular}{ll}
Analyte & transition\\
\hline
IDUA-IS & 377.2 -> 277.1\\
IDUA-P & 391.2 -> 291.2\\
\end{tabular}
\end{table}
\end{frame}

\begin{frame}[label={sec:orgheadline15}]{GAGs: LC-MS/MS}
\begin{itemize}
\item Dried blood spot punches (1/8 in. diameter) were eluted for 10 min at RT and sonicated for 15 min.
\item Heparan sulfate and dermatan sulfate in the DBS punches were
digested to disaccharides by with 5 mIU of each heparinase I, II,
III and 50 mIU chondroitinase B.
\item 2 h of incubation at 30\degree C, 15 μL 150 mM EDTA (pH7.0) and
125 ng internal standard, 4UA-2S-GlcNCOEt-6S, was added and the
reaction was stopped and proteins denatured by boiling for 5 min.
\item The reaction mixture was centrifuged at 16,000 g for 5 min at room
temperature. The supernatant was subsequently applied to an Amicon
Ultra 30 K centrifugal filter and centrifuged at 14,000 g
\end{itemize}
\end{frame}

\begin{frame}[label={sec:orgheadline16}]{GAGs: LC-MS/MS}
\includegraphics[width=.9\linewidth]{./figures/outletmethod.png}


\[
\frac{7.5 min/sample \cdot 1000 samples/day}{60 min/hour \cdot 7 instruments}
= 17.86 hours/instrument/day
\]
\end{frame}
\end{document}

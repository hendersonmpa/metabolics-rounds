% Created 2018-09-14 Fri 17:11
\documentclass[presentation, smaller]{beamer}
\usepackage[utf8]{inputenc}
\usepackage[T1]{fontenc}
\usepackage{fixltx2e}
\usepackage{graphicx}
\usepackage{grffile}
\usepackage{longtable}
\usepackage{wrapfig}
\usepackage{rotating}
\usepackage[normalem]{ulem}
\usepackage{amsmath}
\usepackage{textcomp}
\usepackage{amssymb}
\usepackage{capt-of}
\usepackage{hyperref}
\hypersetup{colorlinks,linkcolor=white,urlcolor=blue}
\usepackage{textpos}
\usepackage{textgreek}
\usepackage[version=4]{mhchem}
\usepackage{chemfig}
\usepackage{siunitx}
\usepackage{gensymb}
\usepackage[usenames,dvipsnames]{xcolor}
\usepackage[T1]{fontenc}
\usepackage{lmodern}
\usepackage{verbatim}
\usepackage{tikz}
\usetikzlibrary{shapes.geometric,arrows,decorations.pathmorphing,backgrounds,positioning,fit,petri}
\usetheme{Hannover}
\usecolortheme{whale}
\author{Matthew Henderson, PhD, FCACB}
\date{\today}
\title{Newborn Screening for MPS I}
\institute[NSO]{Newborn Screening Ontario | The University of Ottawa}
\titlegraphic{\includegraphics[height=1cm,keepaspectratio]{../logos/NSO_logo.pdf}\includegraphics[height=1cm,keepaspectratio]{../logos/cheo-logo.png} \includegraphics[height=1cm,keepaspectratio]{../logos/UOlogoBW.eps}}
\hypersetup{
 pdfauthor={Matthew Henderson, PhD, FCACB},
 pdftitle={Newborn Screening for MPS I},
 pdfkeywords={},
 pdfsubject={},
 pdfcreator={Emacs 25.2.1 (Org mode 8.3.4)}, 
 pdflang={English}}
\begin{document}

\maketitle


# %\logo{\includegraphics[width=1cm,height=1cm,keepaspectratio]{../logos/NSO_logo_small.pdf}~%
# %    \includegraphics[width=1cm,height=1cm,keepaspectratio]{../logos/UOlogoBW.eps}%
# }

\vspace{220pt}
\beamertemplatenavigationsymbolsempty
\setbeamertemplate{caption}[numbered]
\setbeamerfont{caption}{size=\tiny}
#  \addtobeamertemplate{frametitle}{}{%
#  \begin{textblock*}{100mm}(.85\textwidth,-1cm)
#  \includegraphics[height=1cm,width=2cm]{cat}
#  \end{textblock*}}

\tikzstyle{chemical} = [rectangle, rounded corners, text width=5em, minimum height=1em,text centered, draw=black, fill=none]
\tikzstyle{hardware} = [rectangle, rounded corners, text width=5em, minimum height=1em,text centered, draw=black, fill=gray!30]
\tikzstyle{ms} = [rectangle, rounded corners, text width=5em, minimum height=1em,text centered, draw=orange, fill=none]
\tikzstyle{msw} = [rectangle, rounded corners, text width=7em, minimum height=1em,text centered, draw=orange, fill=none]
\tikzstyle{label} = [rectangle,text width=8em, minimum height=1em, text centered, draw=none, fill=none]
\tikzstyle{hl} = [rectangle, rounded corners, text width=5em, minimum height=1em,text centered, draw=black, fill=red!30]
\tikzstyle{box} = [rectangle, rounded corners, text width=5em, minimum height=5em,text centered, draw=black, fill=none]
\tikzstyle{arrow} = [thick,->,>=stealth]
\tikzstyle{hl-arrow} = [ultra thick,->,>=stealth,draw=red]

\section{Introduction}
\label{sec:orgheadline4}
\begin{frame}[label={sec:orgheadline1}]{Hurler}
\begin{itemize}
\item Hurler
\item Sheie
\item Hurler-Sheie
\item Psuedodeficiency
\begin{itemize}
\item A300T
\end{itemize}
\end{itemize}
\end{frame}

\begin{frame}[label={sec:orgheadline2}]{Hydrolysis reaction catalyzed by Alpha-L-Iduronidase:}
\begin{figure}[htb]
\centering
\includegraphics[width=0.8\textwidth]{./figures/nihms3970f3.jpg}
\caption[mech]{\label{fig:mech}
Hydrolysis reaction catalyzed by Alpha-L-Iduronidase:}
\end{figure}
\end{frame}

\begin{frame}[label={sec:orgheadline3}]{Treatment}
\begin{itemize}
\item BMT
\item ERT
\item 
\end{itemize}
\end{frame}

\section{Screening Methods}
\label{sec:orgheadline12}

\begin{frame}[label={sec:orgheadline5}]{Michaelis-Menten Kinetics}
\begin{figure}[htb]
\centering
\includegraphics[width=0.8\textwidth]{./figures/kinetics.png}
\caption{\label{fig:mm}
IDUA Kinetics}
\end{figure}
\end{frame}



\begin{frame}[label={sec:orgheadline6}]{IDUA: Colorimetric}
\begin{itemize}
\item phenyl \(\alpha\)-Liduronide (Phi)
\begin{itemize}
\item Hydrolysis of phenyl \(\alpha\)-L-iduronide and subsequent
colorimetric assay of the liberated phenol
\item Digests contained 5 mM phenyl \(\alpha\)-L-iduronide, 0.1 M NaCI,
and 0.08 M sodium formate buffer of pH 3.5;
\item they were incubated for 18 hr at 25 ° under a layer of toluene.
\item Recovery of the phenol liberated was completed by additional extractions with toluene.
\item Phenol was returned to an aqueous phase for analysis by extraction of the pooled toluene layers with 0.05 M NaOH.
\item Turbidity was then removed by a treatment with chloroform-amyl alcohol.
\end{itemize}
\end{itemize}
\end{frame}

\begin{frame}[label={sec:orgheadline7}]{Spectrofluorometric}
\begin{itemize}
\item 4-Methylumbelliferyl-\(\alpha\)-L-iduronide is a fluorogenic substrate of \(\alpha\)-L-iduronidase
\item emission maximum at 445-454 nm.
\item excitation maximum for 4-MU is pH-dependent: 330, 370, and 385 nm at pH 4.6, 7.4, and 10.4
\end{itemize}

\begin{figure}[htb]
\centering
\includegraphics[width=0.8\textwidth]{./figures/9001600.png}
\caption[4MUI]{\label{fig:4mui}
4-Methylumbelliferyl-α-L-Iduronide 2-sulfate}
\end{figure}
\end{frame}

\begin{frame}[label={sec:orgheadline8}]{IDUA: FIA-MS/MS}
\begin{figure}[htb]
\centering
\includegraphics[width=0.8\textwidth]{./figures/F2large.jpg}
\caption{\label{fig:msmswf}
MS/MS workflow}
\end{figure}
\end{frame}


\begin{frame}[label={sec:orgheadline9}]{IDUA: FIA-MS/MS}
\begin{itemize}
\item Positive mode ESI
\item Ten µL of the 150 µL sample via flow injection
\begin{itemize}
\item 80/20 acetonitrile/water with 0.2\% formic acid
\item flow-rate of 0.1 mL/min for 1 min then 1 mL/min for 0.5 min.
\item Data was collected during 1.5 minute of infusion,
\end{itemize}
\end{itemize}

\begin{table}[htb]
\caption{\label{tab:mrm}
IDUA transitions}
\centering
\begin{tabular}{ll}
Analyte & transition\\
\hline
IDUA-IS & 377.2 -> 277.1\\
IDUA-P & 391.2 -> 291.2\\
\end{tabular}
\end{table}
\end{frame}



\begin{frame}[label={sec:orgheadline10}]{GAGs: LC-MS/MS}
\begin{itemize}
\item Dried blood spot punches (1/8 in. diameter) were eluted for 10 min at RT and sonicated for 15 min.
\item Heparan sulfate and dermatan sulfate in the DBS punches were
digested to disaccharides by with 5 mIU of each heparinase I, II,
III and 50 mIU chondroitinase B.
\item 2 h of incubation at 30 \degree{} C, 15 μL 150 mM EDTA (pH7.0) and
125 ng internal standard, 4UA-2S-GlcNCOEt-6S, was added and the
reaction was stopped and proteins denatured by boiling for 5 min.
\item The reaction mixture was centrifuged at 16,000 g for 5 min at room
temperature. The supernatant was subsequently applied to an Amicon
Ultra 30 K centrifugal filter and centrifuged at 14,000 g
\end{itemize}
\end{frame}

\begin{frame}[label={sec:orgheadline11}]{GAGs: LC-MS/MS}
\includegraphics[width=.9\linewidth]{./figures/outletmethod.png}


\[
\frac{7.5 min/sample \cdot 1000 samples/day}{60 min/hour \cdot 7 instruments}
= 17.86 hours/instrument/day
\]
\end{frame}
\end{document}

% Created 2019-01-24 Thu 11:02
% Intended LaTeX compiler: pdflatex
\documentclass[presentation, smaller]{beamer}
\usepackage[utf8]{inputenc}
\usepackage[T1]{fontenc}
\usepackage{graphicx}
\usepackage{grffile}
\usepackage{longtable}
\usepackage{wrapfig}
\usepackage{rotating}
\usepackage[normalem]{ulem}
\usepackage{amsmath}
\usepackage{textcomp}
\usepackage{amssymb}
\usepackage{capt-of}
\usepackage{hyperref}
\hypersetup{colorlinks,linkcolor=white,urlcolor=blue}
\usepackage{textpos}
\usepackage{textgreek}
\usepackage[version=4]{mhchem}
\usepackage{chemfig}
\usepackage{siunitx}
\usepackage{gensymb}
\usepackage[usenames,dvipsnames]{xcolor}
\usepackage[T1]{fontenc}
\usepackage{lmodern}
\usepackage{verbatim}
\usepackage{tikz}
\usetikzlibrary{shapes.geometric,arrows,decorations.pathmorphing,backgrounds,positioning,fit,petri}
\usetheme{Hannover}
\usecolortheme{whale}
\author{Matthew Henderson, PhD, FCACB}
\date{\today}
\title{Newborn Screening for MPS I}
\institute[NSO]{Newborn Screening Ontario | The University of Ottawa}
\titlegraphic{\includegraphics[height=1cm,keepaspectratio]{../logos/NSO_logo.pdf}\includegraphics[height=1cm,keepaspectratio]{../logos/cheo-logo.png} \includegraphics[height=1cm,keepaspectratio]{../logos/UOlogoBW.eps}}
\hypersetup{
 pdfauthor={Matthew Henderson, PhD, FCACB},
 pdftitle={Newborn Screening for MPS I},
 pdfkeywords={},
 pdfsubject={},
 pdfcreator={Emacs 26.1 (Org mode 9.1.9)}, 
 pdflang={English}}
\begin{document}

\maketitle


# %\logo{\includegraphics[width=1cm,height=1cm,keepaspectratio]{../logos/NSO_logo_small.pdf}~%
# %    \includegraphics[width=1cm,height=1cm,keepaspectratio]{../logos/UOlogoBW.eps}%
# }

\vspace{220pt}
\beamertemplatenavigationsymbolsempty
\setbeamertemplate{caption}[numbered]
\setbeamerfont{caption}{size=\tiny}
#  \addtobeamertemplate{frametitle}{}{%
#  \begin{textblock*}{100mm}(.85\textwidth,-1cm)
#  \includegraphics[height=1cm,width=2cm]{cat}
#  \end{textblock*}}

\tikzstyle{chemical} = [rectangle, rounded corners, text width=5em, minimum height=1em,text centered, draw=black, fill=none]
\tikzstyle{hardware} = [rectangle, rounded corners, text width=5em, minimum height=1em,text centered, draw=black, fill=gray!30]
\tikzstyle{ms} = [rectangle, rounded corners, text width=5em, minimum height=1em,text centered, draw=orange, fill=none]
\tikzstyle{msw} = [rectangle, rounded corners, text width=7em, minimum height=1em,text centered, draw=orange, fill=none]
\tikzstyle{label} = [rectangle,text width=8em, minimum height=1em, text centered, draw=none, fill=none]
\tikzstyle{hl} = [rectangle, rounded corners, text width=5em, minimum height=1em,text centered, draw=black, fill=red!30]
\tikzstyle{box} = [rectangle, rounded corners, text width=5em, minimum height=5em,text centered, draw=black, fill=none]
\tikzstyle{arrow} = [thick,->,>=stealth]
\tikzstyle{hl-arrow} = [ultra thick,->,>=stealth,draw=red]

\section{Introduction}
\label{sec:org4dbc379}
\begin{frame}[label={sec:org6b2c76e}]{MPS I}
\begin{itemize}
\item Defective \(\alpha\)-L-iduronidase enzyme
\begin{itemize}
\item degradation of dermatan sulfate and heparin sulfate
\end{itemize}
\item Three clinical subtypes IH, IS and IHS

\item The estimated prevalence for MPS1 (all forms) from NBS pilot studies is: 
\begin{itemize}
\item 1/35,500 Washington ​(n\textasciitilde{}110,000, Scott et al. 2013)
\item 1/14,567 Missouri ​(n=43,701, Hopkins et al. 2015)
\item 1/17,643 Taiwan ​(n=35,285, Lin et al. 2013)
\end{itemize}
\end{itemize}


\begin{figure}[htbp]
\centering
\includegraphics[width=0.8\textwidth]{./figures/idua.png}
\label{fig:orgc1074be}
\end{figure}
\end{frame}


\begin{frame}[label={sec:orgffc4b7c}]{MPS I disease spectrum}
\begin{center}
\includegraphics[width=0.8\textwidth]{./figures/mps1clinical.png}
\label{orgc5ff427}
\end{center}
\end{frame}


\begin{frame}[label={sec:orgd164dba}]{Diagnosis and Treatment}
\begin{block}{Diagnosis}
\begin{itemize}
\item \(\alpha\)-L-iduronidase enzyme activity levels
\begin{itemize}
\item measured in WBCs, plasma or fibroblasts \(\to\) low to undetectable
\item unable to distinguish between H/HS/S, there is overlap in the enzymatic levels.
\end{itemize}
\item Two ​IDUA mutations in trans (identified by sequencing)
\begin{itemize}
\item two ‘severe’ alleles \(\to\) severe MPS IH
\item one ‘severe’ and one allele that allows some residual enzyme
activity \(\to\) attenuated MPS I
\end{itemize}
\end{itemize}
\end{block}

\begin{block}{Treatment}
\begin{itemize}
\item hematopoietic stem cell transplantation
\item enzyme replacement therapy with Laronidase
\item ERT + HSCT in children <2 years of age
\end{itemize}
\end{block}
\end{frame}


\section{Biochemistry}
\label{sec:org09305ad}
\begin{frame}[label={sec:org9482acb}]{IDUA exhibits Michaelis-Menten kinetics}
\begin{figure}[htbp]
\centering
\includegraphics[width=0.8\textwidth]{./figures/kinetics.pdf}
\caption{\label{fig:orgc4dc581}
IDUA Kinetics, Km = 191 , Vmax = 4.42}
\end{figure}
\end{frame}


\begin{frame}[label={sec:org92b3c7c}]{Substrate concentration}
\begin{itemize}
\item substrate concentrations 10-fold the Km recommended
\item interference from contaminating 4MU-\(\beta\)-D-glucuronide
\item high cost of 4MU-\(\alpha\)-L-iduronide
\item \(\alpha\)-L-iduronidase enzyme assays should be conducted with either:
\begin{itemize}
\item substrate saturating conditions
\item substrate concentrations significantly below enzyme saturation,
the values are adjusted using Michaelis-Menten model.
\end{itemize}
\end{itemize}

\[ 
V  = \frac{Vmax[S]}{Km + [S]}
\]
\end{frame}


\section{Screening Methods}
\label{sec:org326719c}

\begin{frame}[label={sec:org384cfe0}]{IDUA: Colorimetric}
\begin{itemize}
\item phenyl \(\alpha\)-l-iduronide
\begin{itemize}
\item Hydrolysis of phenyl \(\alpha\)-L-iduronide and subsequent
colorimetric assay of the liberated phenol
\item Digests contained 5 mM phenyl \(\alpha\)-l-iduronide, 0.1 M NaCI,
and 0.08 M sodium formate buffer of pH 3.5;
\item they were incubated for 18 hr at 25\degree C under a layer of \alert{toluene}.
\item Recovery of the phenol liberated was completed by additional extractions with \alert{toluene}.
\item Phenol was returned to an aqueous phase for analysis by extraction of the pooled \alert{toluene} layers with 0.05 M NaOH.
\item Turbidity was then removed by a treatment with \alert{chloroform-amyl alcohol}.
\end{itemize}
\end{itemize}
\end{frame}

\begin{frame}[label={sec:orgc2a1f70}]{Spectrofluorometric}
\begin{itemize}
\item 4-Methylumbelliferyl-\(\alpha\)-L-iduronide is a fluorogenic substrate of \(\alpha\)-L-iduronidase
\item emission maximum at 445-454 nm.
\item excitation maximum for 4-MU is pH-dependent: 330, 370, and 385 nm at pH 4.6, 7.4, and 10.4
\end{itemize}

\begin{figure}[htbp]
\centering
\includegraphics[width=0.4\textwidth]{./figures/9001600.png}
\caption[4MUI]{\label{fig:org00bcda6}
4-Methylumbelliferyl-\(\alpha\)-L-Iduronide 2-sulfate}
\end{figure}
\end{frame}


\begin{frame}[label={sec:orgd6e3b05}]{Spectrofluorometric}
\begin{enumerate}
\item Elute one 3.1 mm DBS punch
\begin{itemize}
\item D-saccharic acid-1,4-lactone: \(\beta\)-glucuronidase inhibitor
\item 2 mM 4MU-\(\alpha\)-L-iduronide: substrate
\end{itemize}
\item Incubate for 20hrs at 37\degree C
\item Add glycine-carbonate and vortex to stop reaction
\item 30 minutes at RT
\item Measure fluorescence
\begin{itemize}
\item 4MU calibrator
\end{itemize}
\item Results uM/L blood/20 hr
\end{enumerate}
\end{frame}


\begin{frame}[label={sec:orge9e0219}]{Spectrofluorometric}
\begin{figure}[htbp]
\centering
\includegraphics[width=0.8\textwidth]{./figures/image001.png}
\caption{\label{fig:org84a86da}
SpotCheck Pro}
\end{figure}

\begin{itemize}
\item modified SpotCheck Pro
\end{itemize}
\end{frame}

\begin{frame}[label={sec:orgf42741b}]{Psuedodeficiency}
\begin{itemize}
\item Low IDUA \emph{in vitro} activity with 4-MU substrate
\item p.A300T, steric hindrance at active site E299
\item p.A79T, p.H82Q, p.V322E, p.D223N, p.V322E
\end{itemize}
\end{frame}


\begin{frame}[label={sec:orgba49197}]{Missouri Program}
\begin{itemize}
\item 43,701 samples screened by Missouri program using the Baebies spectrofluorometric assay
\end{itemize}

\begin{figure}[htbp]
\centering
\includegraphics[width=0.8\textwidth]{./figures/pd.png}
\label{fig:org65621ab}
\end{figure}


\[
 PPV = \frac{TP}{TP + FP}  = \frac{1}{1 + (2 + 7 + 2 + 16)} = 0.037
\]

\begin{block}{Screen Positive and Pseudodeficiency rates}
\begin{itemize}
\item SPR = 32/43701 =  0.0007322487
\begin{itemize}
\item 150000 samples/year * SPR \textasciitilde{}  110 SP/year
\end{itemize}
\item PD rate = 7/43701 = 0.0001601794
\begin{itemize}
\item 150000 samples/year * PD rate \textasciitilde{} 24 PD/year
\end{itemize}
\end{itemize}
\end{block}
\end{frame}

\begin{frame}[label={sec:org75de139}]{IDUA: FIA-MS/MS}
\begin{itemize}
\item Positive mode ESI
\item Ten uL of the 150 uL sample via flow injection
\begin{itemize}
\item 80/20 acetonitrile/water with 0.2\% formic acid
\item flow-rate of 0.1 mL/min for 1 min then 1 mL/min for 0.5 min.
\item Data was collected during 1.5 minute of infusion,
\end{itemize}
\end{itemize}

\begin{table}[htbp]
\caption{\label{tab:orgbe481db}
IDUA transitions}
\centering
\begin{tabular}{ll}
Analyte & transition\\
\hline
IDUA-IS & 377.2 -> 277.1\\
IDUA-P & 391.2 -> 291.2\\
\end{tabular}
\end{table}

\begin{itemize}
\item 25/26 patients with reduced leukocyte activity also had reduced
activity in DBS using MS/MS substrate (Pollard, presentation)
\end{itemize}
\end{frame}

\begin{frame}[label={sec:org1c85205}]{IDUA: FIA-MS/MS}
\begin{figure}[htbp]
\centering
\includegraphics[width=0.8\textwidth]{./figures/F2large.jpg}
\caption{\label{fig:org39b3027}
MS/MS workflow}
\end{figure}


\[
\frac{1.7 \text{min/sample} \cdot 1000 \text{samples/day}}{60 \text{min/hour} \cdot 2 \text{instruments}}
= 14.17 \text{hours/instrument/day}
\]
\end{frame}

\begin{frame}[label={sec:org47b6e40}]{GAGs: LC-MS/MS}
\begin{itemize}
\item Dried blood spot punch eluted for 10 min at RT and sonicated for 15 min.
\item Heparan sulfate and dermatan sulfate in the DBS punches were
digested to disaccharides with 5 mIU of each heparinase I, II, III
and 50 mIU chondroitinase B.
\item 2 h of incubation at 30\degree C,
\item 15 μL 150 mM EDTA (pH7.0), 125 ng internal standard, 4UA-2S-GlcNCOEt-6S
\item reaction was stopped and proteins denatured by boiling for 5 min.
\item centrifuged at 16,000 g for 5 min at room temp.
\item supernatant applied to an Amicon Ultra 30 K filter and centrifuged at 14,000 g
\end{itemize}
\end{frame}

\begin{frame}[label={sec:orge07798a}]{GAGs: LC-MS/MS}
\begin{center}
\includegraphics[width=.9\linewidth]{./figures/outletmethod.png}
\end{center}


\[
\frac{7.5 \text{min/sample} \cdot 1000 \text{samples/day}}{60 \text{min/hour} \cdot 7 \text{instruments}}
= 17.86 \text{hours/instrument/day}
\]
\end{frame}

\begin{frame}[label={sec:orge067d56}]{Proposed Hurler workflow}
\begin{figure}[htbp]
\centering
\includegraphics[height=0.8\textheight]{./figures/wf.pdf}
\caption{\label{fig:org3f8c0f3}
Proposed Hurler workflow}
\end{figure}
\end{frame}

\begin{frame}[label={sec:org0672c8f}]{Hurler screening time-lines}
\begin{block}{First, second and third tier assays validated}
\begin{itemize}
\item 2019-12
\end{itemize}
\end{block}
\begin{block}{Start retrospective population study}
\begin{itemize}
\item 2020-04
\end{itemize}
\end{block}
\begin{block}{Launch of NBS for Hurler}
\begin{itemize}
\item 2021-04
\end{itemize}
\end{block}
\end{frame}
\end{document}

% Created 2018-05-07 Mon 14:44
\documentclass[presentation, smaller]{beamer}
\usepackage[utf8]{inputenc}
\usepackage[T1]{fontenc}
\usepackage{fixltx2e}
\usepackage{graphicx}
\usepackage{grffile}
\usepackage{longtable}
\usepackage{wrapfig}
\usepackage{rotating}
\usepackage[normalem]{ulem}
\usepackage{amsmath}
\usepackage{textcomp}
\usepackage{amssymb}
\usepackage{capt-of}
\usepackage{hyperref}
\hypersetup{colorlinks,linkcolor=white,urlcolor=blue}
\usepackage{textpos}
\usepackage{textgreek}
\usepackage[version=4]{mhchem}
\usepackage{chemfig}
\usepackage{siunitx}
\usepackage{gensymb}
\usepackage[usenames,dvipsnames]{xcolor}
\usepackage[T1]{fontenc}
\usepackage{lmodern}
\usepackage{verbatim}
\usepackage{tikz}
\usetikzlibrary{shapes.geometric,arrows,decorations.pathmorphing,backgrounds,positioning,fit,petri}
\usetheme{Hannover}
\usecolortheme{whale}
\author{Matthew Henderson, PhD, FCACB}
\date{\today}
\title{Maple Syrup Urine Disease}
\institute[NSO]{Newborn Screening Ontario | The University of Ottawa}
\titlegraphic{\includegraphics[height=1cm,keepaspectratio]{../logos/NSO_logo.pdf}\includegraphics[height=1cm,keepaspectratio]{../logos/cheo-logo.png} \includegraphics[height=1cm,keepaspectratio]{../logos/UOlogoBW.eps}}
\hypersetup{
 pdfauthor={Matthew Henderson, PhD, FCACB},
 pdftitle={Maple Syrup Urine Disease},
 pdfkeywords={},
 pdfsubject={},
 pdfcreator={Emacs 25.2.1 (Org mode 8.3.4)}, 
 pdflang={English}}
\begin{document}

\maketitle
%\logo{\includegraphics[width=1cm,height=1cm,keepaspectratio]{../logos/NSO_logo_small.pdf}~%
%    \includegraphics[width=1cm,height=1cm,keepaspectratio]{../logos/UOlogoBW.eps}%
%}

\vspace{220pt}
\beamertemplatenavigationsymbolsempty
\setbeamertemplate{caption}[numbered]
\setbeamerfont{caption}{size=\tiny}
% \addtobeamertemplate{frametitle}{}{%
% \begin{textblock*}{100mm}(.85\textwidth,-1cm)
% \includegraphics[height=1cm,width=2cm]{cat}
% \end{textblock*}}

\tikzstyle{chemical} = [rectangle, rounded corners, text width=5em, minimum height=1em,text centered, draw=black, fill=none]
\tikzstyle{hardware} = [rectangle, rounded corners, text width=5em, minimum height=1em,text centered, draw=black, fill=gray!30]
\tikzstyle{ms} = [rectangle, rounded corners, text width=5em, minimum height=1em,text centered, draw=orange, fill=none]
\tikzstyle{msw} = [rectangle, rounded corners, text width=7em, minimum height=1em,text centered, draw=orange, fill=none]
\tikzstyle{label} = [rectangle,text width=8em, minimum height=1em, text centered, draw=none, fill=none]
\tikzstyle{hl} = [rectangle, rounded corners, text width=5em, minimum height=1em,text centered, draw=black, fill=red!30]
\tikzstyle{box} = [rectangle, rounded corners, text width=5em, minimum height=5em,text centered, draw=black, fill=none]
\tikzstyle{arrow} = [thick,->,>=stealth]
\tikzstyle{hl-arrow} = [ultra thick,->,>=stealth,draw=red]

\section{Introduction}
\label{sec:orgheadline3}
\begin{frame}[label={sec:orgheadline1}]{BCAAs}
\centering
\chemname{\chemfig[][scale=.75]{^{+}H_3N-C(-[2]COO^{-})(-[6]CH(-[7]CH_3)(-[5]CH_3))-H}}{\small valine}
\chemname{\chemfig[][scale=.75]{^{+}H_3N-C(-[2]COO^{-})(-[6]CH_2-[6]CH(-[7]CH_3)(-[5]CH_3))-H}}{\small leucine}
\chemname{\chemfig[][scale=.75]{^{+}H_3N-C(-[2]COO^{-})(-[6]CH(-CH_3)-[6]CH_2-[6]CH_3)-H}}{\small isoleucine}
\end{frame}

\begin{frame}[label={sec:orgheadline2}]{BCAA Catabolism}
\centering
\includegraphics[height=0.85\textheight]{./figures/bcaa.png}
\end{frame}

\section{Biochemistry}
\label{sec:orgheadline13}
\begin{frame}[label={sec:orgheadline4}]{Maple Syrup Urine Disease}
\begin{itemize}
\item Only BCAA disorder detected by plasma amino acids.
\begin{itemize}
\item \(\Uparrow\) \(\Uparrow\) leucine, also isoleucine and valine
\begin{itemize}
\item Presence of alloisoleucine is pathognomonic
\end{itemize}
\end{itemize}
\item Urine organic acids
\begin{itemize}
\item \(\Uparrow\) \(\alpha\)-ketoisocaproate,
\(\alpha\)-keto-\(\beta\)-methylvaleric and \(\alpha\)-ketoisovaleric
\begin{itemize}
\item Also elevated in fasting state
\end{itemize}
\end{itemize}
\end{itemize}
\end{frame}
\begin{frame}[label={sec:orgheadline5}]{Rules}
\includegraphics[width=.9\linewidth]{./figures/aa_rules.png}


VICKERY, H. B. (1947). Rules for the nomenclature of the natural amino
acids and related substances. The Journal of Biological Chemistry,
169(2), 237–245.
\end{frame}

\begin{frame}[label={sec:orgheadline6}]{L-Alloisoleucine}
\begin{itemize}
\item L-alloisoleucine is a branched chain amino acid and is a
stereo-isomer of L-isoleucine.
\item It is a common low [ ] constituent of human plasma.
\item Produced as a byproduct of isoleucine transamination.
\item L-alloisoleucine differs from L-isoleucine by having
different stereochemistry around its beta carbon.
\end{itemize}

\centering
\chemname{\chemfig[][scale=.75]{H_{3}C-[1]-[7](<[6]CH_3)-[1](<:[2]NH_2)-[7](=[6]0)-[1]OH}}{\small L-isoleucine}
\chemname{\chemfig[][scale=.75]{H_{3}C-[1]-[7](<:[6]CH_3)-[1](<:[2]NH_2)-[7](=[6]0)-[1]OH}}{\small L-alloisoleucine}
\end{frame}

\begin{frame}[label={sec:orgheadline7}]{BCAA Catabolism}
\begin{itemize}
\item First two steps are shared
\item \textbf{BCAT}: reversible transammination
\begin{itemize}
\item BCAAs escape first pass hepatic metabolism
\begin{itemize}
\item \(\downarrow\) expression of BCAT in the liver
\end{itemize}
\item Metabolized in skeletal muscle and adipose tissue
\end{itemize}
\item \textbf{BCKDC}: irreversible oxidative decarboxylation and thioesterification
\begin{itemize}
\item CoA derivatives are committed to oxidative phosphorylation
\item BCKDC is the regulated step in BCAA metabolism
\begin{itemize}
\item product inhibition
\item phosporylation
\item transcription
\end{itemize}
\end{itemize}
\end{itemize}
\end{frame}

\begin{frame}[label={sec:orgheadline8}]{BCAT and BCKDC Reaction}
\centering
\includegraphics[height=0.90\textheight]{./figures/BCKD_Reaction.png}
\end{frame}

\begin{frame}[label={sec:orgheadline9}]{Branched Chain \(\alpha\)-Ketoacid-Dehydrogenase Metabolic Machine}
\begin{itemize}
\item 4-megadalton metabolic machine on the IMM
\item organized around a 24-meric transacylase (E2b) cubic core
\item encoded by six genetic loci
\begin{itemize}
\item E1\(\alpha\)
\item E1\(\beta\)
\item E2
\item E3
\item BCKD kinase
\item BCKD phosphatase
\end{itemize}
\item multiple copies of the above are attached to E2b via ionic interactions
\item MSUD is classified into four molecular groups
\begin{itemize}
\item IA, IB, II, and III
\item E1\(\alpha\), E1\(\beta\), E2, and E3
\end{itemize}
\end{itemize}
\end{frame}

\begin{frame}[label={sec:orgheadline10}]{Regulation of the BCKD Metabolic Machine}
\begin{block}{Feedback Inhibition}
\begin{itemize}
\item product inhibition by branched chain acyl-CoAs
\item \(\uparrow\) NADH:NAD\(^{\text{+}}\) ratio
\end{itemize}
\end{block}
\begin{block}{Phosphorylation}
\begin{itemize}
\item Inhibited by phosphorylation of E1\(\alpha\) Ser 293
\item \(\uparrow\) protein diet, adrenaline and glucogon dephosphorylate BCKD
\item Phenylbutyrate prevents dephosphorylation
\end{itemize}
\end{block}
\begin{block}{Gene Expression}
\begin{itemize}
\item subunit expression differentially regulated
\end{itemize}
\end{block}
\end{frame}

\begin{frame}[label={sec:orgheadline11}]{Structure of the BCKD Metabolic Machine}
\centering
\includegraphics[height=0.90\textheight]{./figures/bckdmm.jpg}
\end{frame}

\begin{frame}[label={sec:orgheadline12}]{Leucine}
\centering
\includegraphics[width=.9\linewidth]{./figures/leu.png}
\end{frame}

\section{Diagnosis}
\label{sec:orgheadline19}
\begin{frame}[label={sec:orgheadline14}]{NSO Screening Logic}
\begin{block}{Screening}
\begin{description}
\item[{Inital positive}] LEU/ALA \(\ge\) 0.85 AND LEU \(\ge\) 250
\begin{itemize}
\item Repeat overnight
\item No weekend reporting
\end{itemize}
\item[{Alert}] LEU/ALA \(\ge\) 1.25 AND LEU \(\ge\) 300
\begin{itemize}
\item Repeat same day
\item Weekend reporting
\end{itemize}
\end{description}
\end{block}

\begin{block}{Confirmation}
\begin{description}
\item[{Screen Positive}] LEU/ALA \(\ge\) 1.0 AND LEU \(\ge\) 300
\end{description}
\end{block}
\end{frame}

\begin{frame}[label={sec:orgheadline15}]{Newborn Screening ACT Sheet}
\begin{block}{YOU SHOULD TAKE THE FOLLOWING ACTIONS IMMEDIATELY:}
\begin{itemize}
\item Contact family to inform them of the newborn screening result and
ascertain clinical status (poor feeding,vomiting, lethargy,
tachypnea).
\item Consult with pediatric metabolic specialist.
\item Evaluate the newborn (poor feeding, lethargy, tachypnea, alternating
hypertonia/hypotonia, seizures).
\item If any sign is present or infant is ill, transport to hospital for
further treatment in consultation with metabolic specialist.
\item Initiate timely confirmatory/diagnostic testing and management, as
recommended by specialist.
\item Provide the family with basic information about MSUD and dietary
management.
\item Report findings to newborn screening program.
\end{itemize}
\end{block}
\end{frame}

\begin{frame}[label={sec:orgheadline16}]{Diagnostic Evaluation}
\includegraphics[width=.9\linewidth]{./figures/leu_elevated.png}
\end{frame}

\begin{frame}[label={sec:orgheadline17}]{Hydroxyprolinemia}
\begin{itemize}
\item hydroxyproline is primarily derived from collagen turn-over
\item hydroxyprolinemia is due to a defect in hydroxyproline dehydrogenase (HYDH)
\item Hydroxyprolinemia is probably benign.
\end{itemize}

\centering
\schemedebug{false}
\schemestart
\chemname{\chemfig[][scale=.55]{COO^{-}>[6]*5(-^{+}H_2N--(<:[7]OH)--)}}{\small hydroyproline}
\arrow{->[{\tiny HYDH}]}
\chemname{\chemfig[][scale=.55]{COO^{-}>[6]*5(-H_N^{+}=-(<:[7]OH)--)}}{\small hydroyproline}
\arrow{->}
\schemestop

\begin{itemize}
\item hydroxyproline, leucine, isoleucine and alloisoleucine are isobaric
\begin{itemize}
\item commonly termed as "Xle"
\end{itemize}
\item share the same parent and product in FIA-MS/MS with butanol derivatization
\begin{itemize}
\item 188 m/z \(\to\) 86 m/z , NL of 102
\end{itemize}
\end{itemize}
\end{frame}

\begin{frame}[label={sec:orgheadline18}]{Clinical Considerations}
\begin{itemize}
\item MSUD presents in the neonate with:
\begin{itemize}
\item feeding intolerance
\item failure to thrive
\item vomiting
\item lethargy
\item maple syrup odor to urine and cerumen.
\end{itemize}
\item If untreated, it will progress to:
\begin{itemize}
\item irreversible mental retardation
\item hyperactivity
\item failure to thrive
\item seizures
\item coma
\item cerebral edema
\end{itemize}
\end{itemize}
\end{frame}

\section{Treatment}
\label{sec:orgheadline24}
\begin{frame}[label={sec:orgheadline20}]{Acute Management in the Newborn}
\begin{itemize}
\item Metabolic decompensation
\begin{itemize}
\item hemodialysis and hemofiltration
\item high energy dietary treatment
\item leu reduced to \(\le\) 1 mmol/l in hours
\end{itemize}
\item Recovery
\begin{itemize}
\item BCAA-free formula
\item Monitoring plasma [aa] daily
\begin{itemize}
\item Val, Ile
\end{itemize}
\end{itemize}
\end{itemize}
\end{frame}

\begin{frame}[label={sec:orgheadline21}]{Long-term management}
\begin{itemize}
\item Similar principle to PKU
\item BCAA free amino acid supplementation
\item Prevent catabolism
\item Serial monitoring of blood [BCAA]

\item type II missense mutations that appear correlate with the milder
thiamine-responsive form of MSUD
\end{itemize}


\begin{block}{Liver Transplantation}
\begin{itemize}
\item transplanted patient no longer require protein restriction
\item \(\downarrow\) risk of metabolic decompensation
\item candidate for domino transplantation
\end{itemize}
\end{block}
\end{frame}

\begin{frame}[label={sec:orgheadline22}]{Prognosis}
\begin{itemize}
\item The age at diagnosis and the subsequent course are the most
important determinants.
\item Treatment initiated before 10 days of age gives the best results
\item Patients identified by the Massachusetts Newborn Screening Program
\begin{itemize}
\item Carefully monitored and achieved good compliance
\begin{itemize}
\item only a few mild episodes of metabolic decompensation.
\end{itemize}
\item Some are now college graduates in their 20s and 30s.
\end{itemize}
\end{itemize}
\end{frame}
\begin{frame}[label={sec:orgheadline23}]{Next time}
\begin{itemize}
\item Disorders of Branched Chain Amino Acid Catabolism
\begin{itemize}
\item Part 2: Branch Chain Organic Acidurias
\end{itemize}
\end{itemize}
\end{frame}
\end{document}

% Created 2018-01-24 Wed 09:18
\documentclass[presentation, smaller]{beamer}
\usepackage[utf8]{inputenc}
\usepackage[T1]{fontenc}
\usepackage{fixltx2e}
\usepackage{graphicx}
\usepackage{grffile}
\usepackage{longtable}
\usepackage{wrapfig}
\usepackage{rotating}
\usepackage[normalem]{ulem}
\usepackage{amsmath}
\usepackage{textcomp}
\usepackage{amssymb}
\usepackage{capt-of}
\usepackage{hyperref}
\hypersetup{colorlinks,linkcolor=white,urlcolor=blue}
\usepackage{textpos}
\usepackage{textgreek}
\usepackage[version=4]{mhchem}
\usepackage{chemfig}
\usepackage{siunitx}
\usepackage{gensymb}
\usepackage[usenames,dvipsnames]{xcolor}
\usepackage[T1]{fontenc}
\usepackage{lmodern}
\usepackage{verbatim}
\usepackage{tikz}
\usetikzlibrary{shapes.geometric,arrows,decorations.pathmorphing,backgrounds,positioning,fit,petri}
\usetheme{Hannover}
\usecolortheme{whale}
\author{Matthew Henderson, PhD, FCACB}
\date{\today}
\title{Disorders of Branched Chain Amino Acid Catabolism}
\subtitle{Part 1: Maple Syrup Urine Disease}
\institute[NSO]{Newborn Screening Ontario | The University of Ottawa}
\titlegraphic{\includegraphics[height=1cm,keepaspectratio]{../logos/NSO_logo.pdf}\includegraphics[height=1cm,keepaspectratio]{../logos/cheo-logo.png} \includegraphics[height=1cm,keepaspectratio]{../logos/UOlogoBW.eps}}
\hypersetup{
 pdfauthor={Matthew Henderson, PhD, FCACB},
 pdftitle={Disorders of Branched Chain Amino Acid Catabolism},
 pdfkeywords={},
 pdfsubject={},
 pdfcreator={Emacs 25.2.1 (Org mode 8.3.4)}, 
 pdflang={English}}
\begin{document}

\maketitle
%\logo{\includegraphics[width=1cm,height=1cm,keepaspectratio]{../logos/NSO_logo_small.pdf}~%
%    \includegraphics[width=1cm,height=1cm,keepaspectratio]{../logos/UOlogoBW.eps}%
%}

\vspace{220pt}
\beamertemplatenavigationsymbolsempty
\setbeamertemplate{caption}[numbered]
\setbeamerfont{caption}{size=\tiny}
% \addtobeamertemplate{frametitle}{}{%
% \begin{textblock*}{100mm}(.85\textwidth,-1cm)
% \includegraphics[height=1cm,width=2cm]{cat}
% \end{textblock*}}

\tikzstyle{chemical} = [rectangle, rounded corners, text width=5em, minimum height=1em,text centered, draw=black, fill=none]
\tikzstyle{hardware} = [rectangle, rounded corners, text width=5em, minimum height=1em,text centered, draw=black, fill=gray!30]
\tikzstyle{ms} = [rectangle, rounded corners, text width=5em, minimum height=1em,text centered, draw=orange, fill=none]
\tikzstyle{msw} = [rectangle, rounded corners, text width=7em, minimum height=1em,text centered, draw=orange, fill=none]
\tikzstyle{label} = [rectangle,text width=8em, minimum height=1em, text centered, draw=none, fill=none]
\tikzstyle{hl} = [rectangle, rounded corners, text width=5em, minimum height=1em,text centered, draw=black, fill=red!30]
\tikzstyle{box} = [rectangle, rounded corners, text width=5em, minimum height=5em,text centered, draw=black, fill=none]
\tikzstyle{arrow} = [thick,->,>=stealth]
\tikzstyle{hl-arrow} = [ultra thick,->,>=stealth,draw=red]

\section{Introduction}
\label{sec:orgheadline3}
\begin{frame}[label={sec:orgheadline1}]{BCAAs}
\centering
\chemname{\chemfig[][scale=.75]{^{+}H_3N-C(-[2]COO^{-})(-[6]CH(-[7]CH_3)(-[5]CH_3))-H}}{\small valine}
\chemname{\chemfig[][scale=.75]{^{+}H_3N-C(-[2]COO^{-})(-[6]CH_2-[6]CH(-[7]CH_3)(-[5]CH_3))-H}}{\small leucine}
\chemname{\chemfig[][scale=.75]{^{+}H_3N-C(-[2]COO^{-})(-[6]CH(-CH_3)-[6]CH_2-[6]CH_3)-H}}{\small isoleucine}
\end{frame}

\begin{frame}[label={sec:orgheadline2}]{BCAA Catabolism}
\centering
\includegraphics[height=0.85\textheight]{./figures/bcaa.png}
\end{frame}


\section{Biochemistry}
\label{sec:orgheadline13}
\begin{frame}[label={sec:orgheadline4}]{Maple Syrup Urine Disease}
\begin{itemize}
\item Only BCAA disorder detected by plasma amino acids.
\begin{itemize}
\item \(\Uparrow\) \(\Uparrow\) Leucine, also isoleucine and valine
\begin{itemize}
\item Presence of Allo-isoleucine is pathognomonic
\end{itemize}
\end{itemize}
\item - \(\Uparrow\) \(\alpha\)-ketoisocaproate, \(\alpha\)-keto-\(\beta\)-methylvaleric and \(\alpha\)-ketoisovaleric
\end{itemize}
\end{frame}
\begin{frame}[label={sec:orgheadline5}]{L-Alloisoleucine}
\begin{itemize}
\item L-alloisoleucine is a branched chain amino acid and is a
stereo-isomer of L-isoleucine.
\item It is a common constituent of human plasma at low levels.
\item Produced as a byproduct of isoleucine transamination.
\item L-alloisoleucine differs from L-isoleucine by having having
different stereochemistry around its beta carbon.
\end{itemize}

\centering
\chemname{\chemfig[][scale=.75]{H_{3}C-[1]-[7](<[6]CH_3)-[1](<:[2]NH_2)-[7](=[6]0)-[1]OH}}{\small L-isoleucine}
\chemname{\chemfig[][scale=.75]{H_{3}C-[1]-[7](<:[6]CH_3)-[1](<:[2]NH_2)-[7](=[6]0)-[1]OH}}{\small L-alloisoleucine}
\end{frame}

\begin{frame}[label={sec:orgheadline6}]{Rules}
\includegraphics[width=.9\linewidth]{./figures/aa_rules.png}


VICKERY, H. B. (1947). Rules for the nomenclature of the natural amino
acids and related substances. The Journal of Biological Chemistry,
169(2), 237–245.
\end{frame}
\begin{frame}[label={sec:orgheadline7}]{BCAA Catabolism}
\begin{itemize}
\item First two steps are shared
\item \textbf{BCAT}: reversible transammination
\begin{itemize}
\item BCAAs escape first pass hepatic metabolism
\begin{itemize}
\item \(\downarrow\) expression of BCAT in the liver
\end{itemize}
\item Metabolized in skeletal muscle and adipose tissue
\end{itemize}
\item \textbf{BCKDC}: irreversible oxidative decarboxylation and thioesterification
\begin{itemize}
\item CoA derivatives are committed to oxidative phosphorylation
\item BCKDC is the regulated step in BCAA metabolism
\begin{itemize}
\item product inhibition
\item phosporylation
\item transcription
\end{itemize}
\end{itemize}
\end{itemize}
\end{frame}
\begin{frame}[label={sec:orgheadline8}]{BCAT and BCKDC Reaction}
\centering
\includegraphics[height=0.90\textheight]{./figures/BCKD_Reaction.png}
\end{frame}

\begin{frame}[label={sec:orgheadline9}]{Branched Chain \(\alpha\)-Ketoacid-Dehdrogenase Metabolic Machine}
\begin{itemize}
\item 4-megadalton metabolic machine on the IMM organized around a 24-meric
transacylase (E2b) cubic core
\item E2b contains 24 identical subunits made up of three folded domains:
\begin{itemize}
\item lipoyl (LD),
\item E1b/E3-binding (BD)
\item E2b core domains
\end{itemize}
\item multiple copies of the following are attached to E2b via ionic interactions
\begin{itemize}
\item decarboxylase (E1b),
\item dehydrogenase (E3),
\item specific kinase
\item specific phosphatase
\end{itemize}
\end{itemize}
\end{frame}

\begin{frame}[label={sec:orgheadline10}]{Regulation of the BCKD Metabolic Machine}
\begin{block}{Feedback Inhibition}
\begin{itemize}
\item product inhibition by branched chain acyl-CoAs
\item \(\uparrow\) NADH:NAD\(^{\text{+}}\) ratio
\end{itemize}
\end{block}
\begin{block}{Phosphorylation}
\begin{itemize}
\item Inhibited by phosphorylation of E1\(\alpha\) Ser 293
\item \(\uparrow\) protein diet, adrenaline and glucogon dephosphorylate BCKD
\item Phenylbutyrate prevents dephosphorylation
\end{itemize}
\end{block}
\begin{block}{Gene Expression}
\begin{itemize}
\item subunit expression differentially regulated
\end{itemize}
\end{block}
\end{frame}

\begin{frame}[label={sec:orgheadline11}]{Structure of the BCKD Metabolic Machine}
\centering
\includegraphics[height=0.90\textheight]{./figures/bckdmm.jpg}
\end{frame}

\begin{frame}[label={sec:orgheadline12}]{Leucine}
\centering
\includegraphics[width=.9\linewidth]{./figures/leu.png}
\end{frame}

\section{Testing}
\label{sec:orgheadline15}
\begin{frame}[label={sec:orgheadline14}]{NBS}
\end{frame}
\end{document}

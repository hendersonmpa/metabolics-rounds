% Created 2018-11-28 Wed 14:42
\documentclass[presentation, smaller]{beamer}
\usepackage[utf8]{inputenc}
\usepackage[T1]{fontenc}
\usepackage{fixltx2e}
\usepackage{graphicx}
\usepackage{grffile}
\usepackage{longtable}
\usepackage{wrapfig}
\usepackage{rotating}
\usepackage[normalem]{ulem}
\usepackage{amsmath}
\usepackage{textcomp}
\usepackage{amssymb}
\usepackage{capt-of}
\usepackage{hyperref}
\hypersetup{colorlinks,linkcolor=white,urlcolor=blue}
\usepackage{textpos}
\usepackage{textgreek}
\usepackage[version=4]{mhchem}
\usepackage{chemfig}
\usepackage{siunitx}
\usepackage{gensymb}
\usepackage[usenames,dvipsnames]{xcolor}
\usepackage[T1]{fontenc}
\usepackage{lmodern}
\usepackage{verbatim}
\usepackage{tikz}
\usetikzlibrary{shapes.geometric,arrows,decorations.pathmorphing,backgrounds,positioning,fit,petri}
\usetheme{Ilmenau}
\usecolortheme{whale}
\author{Matthew Henderson, PhD, FCACB}
\date{\today}
\title{Sphingolipid Degradation: Fabry}
\institute[NSO]{Newborn Screening Ontario | The University of Ottawa}
\titlegraphic{\includegraphics[height=1cm,keepaspectratio]{../logos/NSO_logo.pdf}\includegraphics[height=1cm,keepaspectratio]{../logos/cheo-logo.png} \includegraphics[height=1cm,keepaspectratio]{../logos/UOlogoBW.eps}}
\hypersetup{
 pdfauthor={Matthew Henderson, PhD, FCACB},
 pdftitle={Sphingolipid Degradation: Fabry},
 pdfkeywords={},
 pdfsubject={},
 pdfcreator={Emacs 25.2.1 (Org mode 8.3.4)}, 
 pdflang={English}}
\begin{document}

\maketitle
%\logo{\includegraphics[width=1cm,height=1cm,keepaspectratio]{../logos/NSO_logo_small.pdf}~%
%    \includegraphics[width=1cm,height=1cm,keepaspectratio]{../logos/UOlogoBW.eps}%
%}

\vspace{220pt}
\beamertemplatenavigationsymbolsempty
\setbeamertemplate{caption}[numbered]
\setbeamerfont{caption}{size=\tiny}
% \addtobeamertemplate{frametitle}{}{%
% \begin{textblock*}{100mm}(.85\textwidth,-1cm)
% \includegraphics[height=1cm,width=2cm]{cat}
% \end{textblock*}}

\tikzstyle{chemical} = [rectangle, rounded corners, text width=5em, minimum height=1em,text centered, draw=black, fill=none]
\tikzstyle{hardware} = [rectangle, rounded corners, text width=5em, minimum height=1em,text centered, draw=black, fill=gray!30]
\tikzstyle{ms} = [rectangle, rounded corners, text width=5em, minimum height=1em,text centered, draw=orange, fill=none]
\tikzstyle{msw} = [rectangle, rounded corners, text width=7em, minimum height=1em,text centered, draw=orange, fill=none]
\tikzstyle{label} = [rectangle,text width=8em, minimum height=1em, text centered, draw=none, fill=none]
\tikzstyle{hl} = [rectangle, rounded corners, text width=5em, minimum height=1em,text centered, draw=black, fill=red!30]
\tikzstyle{box} = [rectangle, rounded corners, text width=5em, minimum height=5em,text centered, draw=black, fill=none]
\tikzstyle{arrow} = [thick,->,>=stealth]
\tikzstyle{hl-arrow} = [ultra thick,->,>=stealth,draw=red]

\section{Introduction}
\label{sec:orgheadline9}

\begin{frame}[label={sec:orgheadline1}]{Fabry}
\begin{itemize}
\item AKA: angiokeratoma corporis diffusum universale
\item Second most common LSD
\item 1:339,000 heterozygote females in the UK
\item First described in 1898 independently by Anderson and Fabry
\begin{itemize}
\item Dermatologists
\end{itemize}
\item Defect in \(\alpha\)-galactosidase A (ceramide trihexosidase)
\begin{itemize}
\item Inability to cleave terminal galactose from the sphingolipid globotriasylceramide Gb3 (galactosylgalactosylglucoceramide)
\item 3 to 20\% activity in hemizygote males
\end{itemize}
\item Lack of \(\alpha\)-GalA leads to accumulation of Gb3 in blood vessels and other tissues
\begin{itemize}
\item wide range of symptoms including kidney, heart, and skin symptoms
\item \(\uparrow\) [Gb3] in kidney and blood group B antigenic glycosphingolipid
\end{itemize}
\end{itemize}
\end{frame}

\begin{frame}[label={sec:orgheadline2}]{Sphingolipid degradation}
\begin{figure}[htb]
\centering
\includegraphics[width=0.6\textwidth]{./figures/sl_degradation.png}
\caption[deg]{\label{fig:sld}
Sphingolipid degradation}
\end{figure}
\end{frame}


\begin{frame}[label={sec:orgheadline3}]{Globotriasylceramide (Gb3): the Fabry lipid}
\begin{figure}[htb]
\centering
\includegraphics[width=0.5\textwidth]{./figures/globotriasylceramide.png}
\caption[gluc]{\label{fig:galac}
globotriasylceramide}
\end{figure}
\end{frame}

\begin{frame}[label={sec:orgheadline4}]{\(\alpha\)-galactosidase A}
\begin{figure}[htb]
\centering
\includegraphics[width=0.5\textwidth]{./figures/galactosidaseA.png}
\caption[block]{\label{fig:sidase}
\(\beta\)-glucocerebrosidase}
\end{figure}

\begin{itemize}
\item Located in the lumen of lysosomes
\end{itemize}
\end{frame}

\begin{frame}[label={sec:orgheadline5}]{Lysosomal Trafficking}
\begin{figure}[htb]
\centering
\includegraphics[width=0.8\textwidth]{./figures/lysosome_trafficking.jpeg}
\caption[traf]{\label{fig:traf}
Lysosomal Trafficking}
\end{figure}
\begin{itemize}
\item Sortilin dependant pathway
\begin{itemize}
\item Not affected in ML II \& III
\end{itemize}
\end{itemize}
\end{frame}

\begin{frame}[label={sec:orgheadline6}]{Sortilin}
\begin{itemize}
\item Sortilin is a type I transmembrane protein found in lysosomes
\begin{itemize}
\item can transport several lysosomal proteins from the TGN or PM to the endosomes,
\begin{itemize}
\item Several of these proteins exhibit normal levels in some cell types and tissues of I-cell disease patients
\end{itemize}
\end{itemize}
\item Tissues from sortilin knock-out mice exhibit normal morphology
\item Sortilin may transport selected acid hydrolases in a subset of cell types
\item under stress conditions (e.g. Man-6-P pathway is deficient)
\end{itemize}
\end{frame}

\begin{frame}[label={sec:orgheadline7}]{Megalin}
\begin{itemize}
\item a cell surface receptor involved in reabsorption of proteins at the kidney proximal tubule
\item megalin mediated endocytosis of \(\alpha\)-galactosidase kidney proximal tubule
\item megalin also mediates the endocytosis of \(\alpha\)-galactosidase in renal podocytes
\end{itemize}
\end{frame}

\begin{frame}[label={sec:orgheadline8}]{Genetics}
\begin{itemize}
\item The \(\alpha\)-galactosidase A gene is on the X chromosome
\begin{itemize}
\item Xq22.1
\end{itemize}
\item X-linked with penetrance in female heterozygotes
\begin{itemize}
\item may be considered X-linked dominant
\end{itemize}
\item More that 300 of mutations have been found
\item Single nucleotide missense mutations identified in the majority of families
\begin{itemize}
\item Most private mutations
\end{itemize}
\end{itemize}
\end{frame}

\section{Clinical Findings}
\label{sec:orgheadline11}

\begin{frame}[label={sec:orgheadline10}]{Signs and Symptoms}
\begin{itemize}
\item postprandial pain or diarrhea
\begin{itemize}
\item may be sole complaint
\end{itemize}
\item degradation of interphalangeal joints
\item cerebrovascular - stroke, seizures
\item ocular lesions
\end{itemize}


\begin{center}
\begin{tabular}{ll}
Age & Signs\\
\hline
Childhood & Pain in extremities, fever, Fabry crisis \footnotemark\\
Adolescence & Angiokeratomas\\
Adulthood & Central nervous system symptoms\\
 & Myocardial and pulmonary disease\\
Middle age & Renal failure, lymphedema\\
\end{tabular}
\end{center}\footnotetext[1]{May be induced by heat, cold, fatigue or emotional stress}
\end{frame}


\section{Laboratory Investigations}
\label{sec:orgheadline14}

\begin{frame}[label={sec:orgheadline12}]{Biochemistry}
\begin{itemize}
\item Deficient \(\alpha\)-galactosidase A activity in leukocytes
\item NBS via \(\alpha\)-galactosidase A activity in DBS
\begin{itemize}
\item Taiwan, MO, IL
\end{itemize}
\item Elevated urine Gb3 and Gb2 in hemizygote males and heterozygote females
\item Plasma lyso-Gb3 is a sensitive biomarker
\end{itemize}
\end{frame}

\begin{frame}[label={sec:orgheadline13}]{Pathology}
\begin{columns}
\begin{column}{0.5\columnwidth}
\begin{itemize}
\item Widespread deposition of Gb3
\item Vacuoles seen in variety of cells, \(\uparrow\) endothelium of blood vessels
\end{itemize}
\end{column}

\begin{column}{0.5\columnwidth}
\begin{figure}[htb]
\centering
\includegraphics[width=0.7\textwidth]{./figures/Fabrys-disease.jpg}
\caption[em]{\label{fig:biopsy}
EM showing concentric or lamellar structure of lysosomal inclusions in Fabry disease renal biopsy}
\end{figure}
\end{column}
\end{columns}
\end{frame}




\section{Treatment}
\label{sec:orgheadline17}

\begin{frame}[label={sec:orgheadline15}]{Treatment}
\begin{itemize}
\item Alleviate pain
\item Treat renal and cardiac disease
\begin{itemize}
\item Dialysis or renal transplantation
\end{itemize}
\item Long term experience with ERT
\begin{itemize}
\item agalsidase alpha or beta)
\item Reduces left ventricular hypertrophy
\item Less effect on renal function
\item Does not prevent progression
\end{itemize}
\item Chaperone therapy - migalastat
\begin{itemize}
\item Amenable mutations
\end{itemize}
\end{itemize}
\end{frame}

\begin{frame}[label={sec:orgheadline16}]{Next time}
\begin{itemize}
\item Disorders of Sphingolipid Degradation continued\ldots{}
\end{itemize}
\end{frame}
\end{document}

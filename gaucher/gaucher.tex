% Created 2018-10-31 Wed 14:15
\documentclass[presentation, smaller]{beamer}
\usepackage[utf8]{inputenc}
\usepackage[T1]{fontenc}
\usepackage{fixltx2e}
\usepackage{graphicx}
\usepackage{grffile}
\usepackage{longtable}
\usepackage{wrapfig}
\usepackage{rotating}
\usepackage[normalem]{ulem}
\usepackage{amsmath}
\usepackage{textcomp}
\usepackage{amssymb}
\usepackage{capt-of}
\usepackage{hyperref}
\hypersetup{colorlinks,linkcolor=white,urlcolor=blue}
\usepackage{textpos}
\usepackage{textgreek}
\usepackage[version=4]{mhchem}
\usepackage{chemfig}
\usepackage{siunitx}
\usepackage{gensymb}
\usepackage[usenames,dvipsnames]{xcolor}
\usepackage[T1]{fontenc}
\usepackage{lmodern}
\usepackage{verbatim}
\usepackage{tikz}
\usetikzlibrary{shapes.geometric,arrows,decorations.pathmorphing,backgrounds,positioning,fit,petri}
\usetheme{Ilmenau}
\usecolortheme{whale}
\author{Matthew Henderson, PhD, FCACB}
\date{\today}
\title{Sphingolipid Degradation: Gaucher}
\institute[NSO]{Newborn Screening Ontario | The University of Ottawa}
\titlegraphic{\includegraphics[height=1cm,keepaspectratio]{../logos/NSO_logo.pdf}\includegraphics[height=1cm,keepaspectratio]{../logos/cheo-logo.png} \includegraphics[height=1cm,keepaspectratio]{../logos/UOlogoBW.eps}}
\hypersetup{
 pdfauthor={Matthew Henderson, PhD, FCACB},
 pdftitle={Sphingolipid Degradation: Gaucher},
 pdfkeywords={},
 pdfsubject={},
 pdfcreator={Emacs 25.2.1 (Org mode 8.3.4)}, 
 pdflang={English}}
\begin{document}

\maketitle
%\logo{\includegraphics[width=1cm,height=1cm,keepaspectratio]{../logos/NSO_logo_small.pdf}~%
%    \includegraphics[width=1cm,height=1cm,keepaspectratio]{../logos/UOlogoBW.eps}%
%}

\vspace{220pt}
\beamertemplatenavigationsymbolsempty
\setbeamertemplate{caption}[numbered]
\setbeamerfont{caption}{size=\tiny}
% \addtobeamertemplate{frametitle}{}{%
% \begin{textblock*}{100mm}(.85\textwidth,-1cm)
% \includegraphics[height=1cm,width=2cm]{cat}
% \end{textblock*}}

\tikzstyle{chemical} = [rectangle, rounded corners, text width=5em, minimum height=1em,text centered, draw=black, fill=none]
\tikzstyle{hardware} = [rectangle, rounded corners, text width=5em, minimum height=1em,text centered, draw=black, fill=gray!30]
\tikzstyle{ms} = [rectangle, rounded corners, text width=5em, minimum height=1em,text centered, draw=orange, fill=none]
\tikzstyle{msw} = [rectangle, rounded corners, text width=7em, minimum height=1em,text centered, draw=orange, fill=none]
\tikzstyle{label} = [rectangle,text width=8em, minimum height=1em, text centered, draw=none, fill=none]
\tikzstyle{hl} = [rectangle, rounded corners, text width=5em, minimum height=1em,text centered, draw=black, fill=red!30]
\tikzstyle{box} = [rectangle, rounded corners, text width=5em, minimum height=5em,text centered, draw=black, fill=none]
\tikzstyle{arrow} = [thick,->,>=stealth]
\tikzstyle{hl-arrow} = [ultra thick,->,>=stealth,draw=red]

\section{Introduction}
\label{sec:orgheadline6}

\begin{frame}[label={sec:orgheadline1}]{Gaucher}
\begin{itemize}
\item Caused by accumulation of glucosylceramide (glucocerebroside).
\item Defect in \(\beta\)-glucocerebrosidase
\begin{itemize}
\item Extremely rare SAP-C deficiency
\end{itemize}
\item Most common LSD
\begin{itemize}
\item 1:40,000 to 1:50,000 live births
\end{itemize}
\item Three types:
\begin{description}
\item[{Type 1}] No neurological symptoms
\item[{Type 2}] Acute neuronopathic
\item[{Type 3}] Sub-acute or chronic neuronopathic
\end{description}
\item Type 1 disease is common in Western Europe, the Americas and Israel
\item In many other countries neuronopathic forms of Gaucher disease predominate
\end{itemize}
\end{frame}

\begin{frame}[label={sec:orgheadline2}]{Sphingolipid degradation}
\begin{figure}[htb]
\centering
\includegraphics[width=0.6\textwidth]{./figures/sl_degradation.png}
\caption[deg]{\label{fig:sld}
Sphingolipid degradation}
\end{figure}
\end{frame}

\begin{frame}[label={sec:orgheadline3}]{Glucocerebroside: the Gaucher lipid}
\begin{figure}[htb]
\centering
\includegraphics[width=0.5\textwidth]{./figures/glucocerebroside.png}
\caption[gluc]{\label{fig:gluc}
Glucocerebroside}
\end{figure}
\end{frame}

\begin{frame}[label={sec:orgheadline4}]{Glucocerebrosidase}
\begin{figure}[htb]
\centering
\includegraphics[width=0.5\textwidth]{./figures/glucocerebrosidase.png}
\caption[block]{\label{fig:sidase}
\(\beta\)-glucocerebrosidase}
\end{figure}

\begin{itemize}
\item Located in the lumen of lysosomes
\item LIMP-2 is responsible for mannose 6-phosphate receptor independent
lysosomal targeting of \(\beta\)-glucocerebrosidase
\end{itemize}
\end{frame}

\begin{frame}[label={sec:orgheadline5}]{Genetics}
\begin{itemize}
\item Autosomal recessive, GBA gene
\item Most of the >300 disease alleles in Gaucher disease are missense
mutations
\begin{itemize}
\item result in \(\beta\)-glucocerebrosidase with decreased catalytic
function and/or stability.
\end{itemize}
\item A variety of complex mutations/rearrangements also causes Gaucher
disease:
\begin{itemize}
\item missense mutations, frame shift mutations, splicing mutations,
deletions, gene fusions with the pseudogene, examples of gene
conversions, and total deletions.
\end{itemize}
\item genotype/phenotype correlations exist for:
\begin{itemize}
\item type 1 disease (N370S)
\item types 2 and 3 (L444P)
\end{itemize}
\item within these categories there is variable penetrance and
expressivity between individuals and ethnic groups.
\end{itemize}
\end{frame}


\section{Clinical Findings}
\label{sec:orgheadline13}
\begin{frame}[fragile,label={sec:orgheadline7}]{Clinical Presentation}
 \begin{table}[htb]
\caption[variants]{\label{tab:variants}
Gaucher Clinical Variants}
\centering
\begin{tabular}{llll}
 & Type 1 & Type 2 & Type 3\\
\hline
Onset & Infant/Child/Adult & 3-6 months & Childhood\\
Neurodegeneration & Absent & \texttt{++++} & \texttt{++} \(\to\) \texttt{++++}\\
Survival & 6 - 80+ years & < 2 years & 2nd - 4th decade\\
Splenomegaly & \texttt{++++} & \texttt{++} & \texttt{++}\\
Hepatomegaly & \texttt{++} & \texttt{++} & \texttt{+}\\
Fractures, bone crises & \texttt{+} & - & \texttt{+}\\
Enrichment & Ashkenazi & None & Norrbottnian\\
 &  &  & Swedish\\
\end{tabular}
\end{table}
\end{frame}

\begin{frame}[label={sec:orgheadline8}]{Clinical Presentation}
\begin{figure}[htb]
\centering
\includegraphics[width=\textwidth]{./figures/variants.png}
\caption[variants]{\label{tab:variants2}
Gaucher Clinical Variants}
\end{figure}


\tiny 
= * Parkinson Disease, ** Myoclonic Seizures =
\end{frame}
\begin{frame}[label={sec:orgheadline9}]{Gaucher type 1}
\begin{itemize}
\item Clinical manifestations of Gaucher type 1 are linked to macrophages
engorged with glucosylceramide.

\item An undefined mechanism results in:
\begin{itemize}
\item enlargement and dysfunction of the liver and spleen
\item displacement of normal bone marrow by storage cells
\item osteoclastic-osteoblastic imbalances
\begin{itemize}
\item subsequent damage leading to bone infarctions and fractures.
\end{itemize}
\item Occasionally, involvement of other organs (e.g. lung) contributes
to the overall clinical picture.
\item Hypermetabolism and cachexia can be present
\item Thrombocytopenia is the most common peripheral blood abnormality
\end{itemize}

\item Neurological manifestations include:
\begin{itemize}
\item a high incidence of Parkinsonism
\item spinal cord compression
\item nerve root compression
\item polyneuropathy.
\end{itemize}
\end{itemize}
\end{frame}

\begin{frame}[label={sec:orgheadline10}]{Gaucher type 2}
\begin{itemize}
\item Of the two classic neuronopathic variants, type 2 is rare
\item early infantile onset of acute neuronopathic disease
\item progressing rapidly to death before age 2 years.

\begin{itemize}
\item retroflexion of the neck
\item developmental delay, poor weight gain,
\item protuberant abdomen due to hepatosplenomegaly
\item Bulbar signs are prominent including:
\begin{itemize}
\item convergent squint,
\item ocular paresis,
\item trismus,
\item dysphagia
\end{itemize}
\end{itemize}

\item The perinatal-lethal subtype is the most severe form of Gaucher
disease. It leads to death in utero or within hours to days after
birth
\end{itemize}
\end{frame}

\begin{frame}[label={sec:orgheadline11}]{Gaucher type 3}
\begin{itemize}
\item type 3 disease has a later onset, with slower progression of
neurologic manifestations and variable degrees of systemic
involvement.
\item phenotype in type 3 Gaucher disease is considerably more
heterogeneous than that in type 2.

\item onset of symptoms occurs later, and neurologic involvement
progresses more slowly

\item includes abnormalities in:
\begin{itemize}
\item eye movements
\item seizures
\item intellectual deterioration.
\end{itemize}

\item The same systemic manifestations occur as in type 1 disease.
\begin{itemize}
\item many type 3 patients may be incorrectly classified as type 1 when
first seen
\end{itemize}
\end{itemize}
\end{frame}

\begin{frame}[label={sec:orgheadline12}]{Gaucher type 3}
\begin{block}{Gaucher type 3a}
\begin{itemize}
\item progressive myoclonus and dementia
\end{itemize}
\end{block}

\begin{block}{Gaucher type 3b}
\begin{itemize}
\item horizontal supranuclear gaze palsy without other major
neurologic signs
\item aggressive systemic disease
\end{itemize}
\end{block}

\begin{block}{Gaucher type 3c}
\begin{itemize}
\item present in late childhood or later
\item only mild visceral signs of classic Gaucher disease
\item distinguishing clinical signs include:
\begin{itemize}
\item impaired horizontal ocular saccades
\item corneal opacities
\item cardiac/aortic valvular calcification
\end{itemize}
\end{itemize}
\end{block}
\end{frame}

\section{Laboratory Investigations}
\label{sec:orgheadline17}
\begin{frame}[label={sec:orgheadline14}]{Gaucher Cells}
\begin{figure}[htb]
\centering
\includegraphics[width=0.8\textwidth]{./figures/Gaucher_Cells_with_Fibrillar_Appearing_Cytoplasm.jpg}
\caption[cells]{\label{fig:cells}
Gaucher Cells}
\end{figure}
\end{frame}

\begin{frame}[label={sec:orgheadline15}]{Biochemistry}
\begin{block}{Enzyme Assay}
\begin{itemize}
\item assay of the \(\beta\)-glucocerebrosidase activity in any nucleated cell
\begin{itemize}
\item the enzyme does not normally occur in plasma/serum or erythrocytes
\end{itemize}
\item Glucocerebrosidase activity in:
\begin{itemize}
\item peripheral blood lymphocytes/leukocytes
\item dried blood spots
\end{itemize}
\item 4MU-\(\beta\)-D--glucopyranoside substrate
\end{itemize}
\end{block}

\begin{block}{Monitoring}
\begin{itemize}
\item chitotriosidase, chemokine CLL18/PARK, glucosylsphingosine
\end{itemize}
\end{block}
\end{frame}

\begin{frame}[label={sec:orgheadline16}]{Molecular}
\begin{itemize}
\item GBA gene sequencing, >300 disease alleles
\item Patients homozygous for the L444P mutation have severe visceral
disease, highly predisposed to the development of CNS disease.
\item The N370S mutant enzyme appears to preclude the development of classical CNS disease of Gaucher disease.
\item The D409H mutation manifests a characteristic phenotype:
\begin{itemize}
\item including cardiac calcification, oculomotor apraxia, and corneal opacities.
\end{itemize}
\end{itemize}
\end{frame}


\section{Treatment}
\label{sec:orgheadline18}
\end{document}

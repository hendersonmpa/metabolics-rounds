% Created 2018-10-30 Tue 15:01
\documentclass[presentation, smaller]{beamer}
\usepackage[utf8]{inputenc}
\usepackage[T1]{fontenc}
\usepackage{fixltx2e}
\usepackage{graphicx}
\usepackage{grffile}
\usepackage{longtable}
\usepackage{wrapfig}
\usepackage{rotating}
\usepackage[normalem]{ulem}
\usepackage{amsmath}
\usepackage{textcomp}
\usepackage{amssymb}
\usepackage{capt-of}
\usepackage{hyperref}
\hypersetup{colorlinks,linkcolor=white,urlcolor=blue}
\usepackage{textpos}
\usepackage{textgreek}
\usepackage[version=4]{mhchem}
\usepackage{chemfig}
\usepackage{siunitx}
\usepackage{gensymb}
\usepackage[usenames,dvipsnames]{xcolor}
\usepackage[T1]{fontenc}
\usepackage{lmodern}
\usepackage{verbatim}
\usepackage{tikz}
\usetikzlibrary{shapes.geometric,arrows,decorations.pathmorphing,backgrounds,positioning,fit,petri}
\usetheme{Ilmenau}
\usecolortheme{whale}
\author{Matthew Henderson, PhD, FCACB}
\date{\today}
\title{Sphingolipid Degradation: Gaucher}
\institute[NSO]{Newborn Screening Ontario | The University of Ottawa}
\titlegraphic{\includegraphics[height=1cm,keepaspectratio]{../logos/NSO_logo.pdf}\includegraphics[height=1cm,keepaspectratio]{../logos/cheo-logo.png} \includegraphics[height=1cm,keepaspectratio]{../logos/UOlogoBW.eps}}
\hypersetup{
 pdfauthor={Matthew Henderson, PhD, FCACB},
 pdftitle={Sphingolipid Degradation: Gaucher},
 pdfkeywords={},
 pdfsubject={},
 pdfcreator={Emacs 25.2.1 (Org mode 8.3.4)}, 
 pdflang={English}}
\begin{document}

\maketitle
%\logo{\includegraphics[width=1cm,height=1cm,keepaspectratio]{../logos/NSO_logo_small.pdf}~%
%    \includegraphics[width=1cm,height=1cm,keepaspectratio]{../logos/UOlogoBW.eps}%
%}

\vspace{220pt}
\beamertemplatenavigationsymbolsempty
\setbeamertemplate{caption}[numbered]
\setbeamerfont{caption}{size=\tiny}
% \addtobeamertemplate{frametitle}{}{%
% \begin{textblock*}{100mm}(.85\textwidth,-1cm)
% \includegraphics[height=1cm,width=2cm]{cat}
% \end{textblock*}}

\tikzstyle{chemical} = [rectangle, rounded corners, text width=5em, minimum height=1em,text centered, draw=black, fill=none]
\tikzstyle{hardware} = [rectangle, rounded corners, text width=5em, minimum height=1em,text centered, draw=black, fill=gray!30]
\tikzstyle{ms} = [rectangle, rounded corners, text width=5em, minimum height=1em,text centered, draw=orange, fill=none]
\tikzstyle{msw} = [rectangle, rounded corners, text width=7em, minimum height=1em,text centered, draw=orange, fill=none]
\tikzstyle{label} = [rectangle,text width=8em, minimum height=1em, text centered, draw=none, fill=none]
\tikzstyle{hl} = [rectangle, rounded corners, text width=5em, minimum height=1em,text centered, draw=black, fill=red!30]
\tikzstyle{box} = [rectangle, rounded corners, text width=5em, minimum height=5em,text centered, draw=black, fill=none]
\tikzstyle{arrow} = [thick,->,>=stealth]
\tikzstyle{hl-arrow} = [ultra thick,->,>=stealth,draw=red]

\section{Introduction}
\label{sec:orgheadline7}

\begin{frame}[label={sec:orgheadline1}]{Gaucher}
\begin{itemize}
\item Accumulation of glucosylceramide (glucocerebroside).
\item Defect in \(\beta\)-glucocerebrosidase
\begin{itemize}
\item Extremely rare SAP-C deficiency
\end{itemize}
\item Most common LSD
\begin{itemize}
\item 1:40,000 to 1:50,000 live births
\end{itemize}
\item Three types
\begin{description}
\item[{Type 1}] No neurological symptoms
\item[{Type 2}] Acute neuronopathic
\item[{Type 3}] Subacute or chronic neuronopathic
\end{description}
\item Type 1 disease is common in Western Europe, the Americas and Israel,
but in many other countries neuronopathic forms of Gaucher disease
predominate
\end{itemize}
\end{frame}


\begin{frame}[label={sec:orgheadline2}]{Gaucher Cells}
\begin{figure}[htb]
\centering
\includegraphics[width=0.8\textwidth]{./figures/Gaucher_Cells_with_Fibrillar_Appearing_Cytoplasm.jpg}
\caption[cells]{\label{fig:cells}
Gaucher Cells}
\end{figure}
\end{frame}




\begin{frame}[label={sec:orgheadline3}]{Sphingolipid degradation}
\begin{figure}[htb]
\centering
\includegraphics[width=0.6\textwidth]{./figures/sl_degradation.png}
\caption[deg]{\label{fig:sld}
Sphingolipid degradation}
\end{figure}
\end{frame}


\begin{frame}[label={sec:orgheadline4}]{Glucocerebroside: the Gaucher lipid}
\begin{figure}[htb]
\centering
\includegraphics[width=0.5\textwidth]{./figures/glucocerebroside.png}
\caption[gluc]{\label{fig:gluc}
Glucocerebroside}
\end{figure}
\end{frame}



\begin{frame}[label={sec:orgheadline5}]{Glucocerebrosidase}
\begin{figure}[htb]
\centering
\includegraphics[width=0.5\textwidth]{./figures/glucocerebrosidase.png}
\caption[block]{\label{fig:sidase}
\(\beta\)-glucocerebrosidase}
\end{figure}

\begin{itemize}
\item LIMP-2 is responsible for mannose 6-phosphate receptor
independent lysosomal targeting of \(\beta\)-glucocerebrosidase
\end{itemize}
\end{frame}

\begin{frame}[label={sec:orgheadline6}]{Genetics}
\begin{itemize}
\item Autosomal recessive, GBA gene
\item Most of the >300 disease alleles in Gaucher disease are missense
mutations that lead to acid β-glucosidases with decreased catalytic
function and/or stability.
\item A variety of complex mutations/rearrangements also causes Gaucher
disease:
\begin{itemize}
\item missense mutations, frame shift mutations, splicing mutations, deletions, gene fusions with the pseudogene, examples of gene conversions, and total deletions.
\end{itemize}
\item genotype/phenotype correlations exist for type 1 disease (N370S) and
types 2 and 3 (L444P)
\begin{itemize}
\item within these categories there is variable penetrance and
expressivity between individuals and ethnic groups.
\end{itemize}
\end{itemize}
\end{frame}

\section{Clinical Findings}
\label{sec:orgheadline14}
\begin{frame}[fragile,label={sec:orgheadline8}]{Clinical Presentation}
 \begin{center}
\begin{tabular}{llll}
 & Type 1 & Type 2 & Type 3\\
\hline
Onset & Infant/Child/Adult & 3-6 months & Childhood\\
Neurodegeneration & Absent & \texttt{++++} & \texttt{++} \(\to\) \texttt{++++}\\
Survival & 6 - 80+ years & < 2 years & 2nd - 4th decade\\
Splenomegaly & \texttt{++++} & \texttt{++} & \texttt{++}\\
Hepatomegaly & \texttt{++} & \texttt{++} & \texttt{+}\\
Fractures, bone crises & \texttt{+} & - & \texttt{+}\\
Enrichment & Ashkenazi & Panethnic & Norrbottnian Swedish\\
\end{tabular}
\end{center}
\end{frame}



\begin{frame}[label={sec:orgheadline9}]{Clinical Presentation}
\begin{figure}[htb]
\centering
\includegraphics[width=0.9\textwidth]{./figures/variants.png}
\caption[variants]{\label{tab:variants}
Clinical Variants}
\end{figure}
\end{frame}


\begin{frame}[label={sec:orgheadline10}]{Gaucher type 1}
\begin{itemize}
\item The clinical manifestations of Gaucher disease type 1 are
pathogenically linked to macrophages engorged with glucosylceramide

\item leads (undefined mechanism) to enlargement and dysfunction of the
liver and spleen, displacement of normal bone marrow by storage
cells, osteoclastic-osteoblastic imbalances, and subsequent damage
leading to bone infarctions and fractures.

\item Occasionally, involvement of other organs, e.g., lung, contributes
to the overall clinical picture.

\item Hypermetabolism and cachexia can be present

\item Thrombocytopenia is the most common peripheral blood abnormality

\item Neurological manifestations include a high incidence of Parkinsonism,
spinal cord compression, nerve root compression, and polyneuropathy.
\end{itemize}
\end{frame}

\begin{frame}[label={sec:orgheadline11}]{Gaucher type 2}
\begin{itemize}
\item Of the two classic neuronopathic variants, type 2 is rare
\item early infantile onset of acute neuronopathic disease
\item progressing rapidly to death before age 2 years.

\item retroflexion of the neck, developmental delay, poor weight gain,
and a protuberant abdomen due to hepatosplenomegaly.
\item Bulbar signs are prominent, including a convergent squint, ocular paresis, trismus, and dysphagia.
\item The perinatal-lethal subtype is the most severe form of Gaucher
disease. It leads to death in utero or within hours to days after
birth
\end{itemize}
\end{frame}

\begin{frame}[label={sec:orgheadline12}]{Gaucher type 3}
\begin{itemize}
\item type 3 disease has a later onset, with slower progression of
neurologic manifestations and variable degrees of systemic
involvement.
\item phenotype in type 3 Gaucher disease is considerably more
heterogeneous than that in type 2.

\item onset of symptoms occurs later, and neurologic involvement
progresses more slowly and includes abnormalities in eye movements,
seizures, and intellectual deterioration.

\item The same systemic manifestations occur as in type 1 disease.

\begin{itemize}
\item many type 3 patients may be incorrectly classified as type 1 when
first seen
\end{itemize}
\end{itemize}
\end{frame}

\begin{frame}[label={sec:orgheadline13}]{Gaucher type 3}
\begin{block}{Gaucher type 3a}
\begin{itemize}
\item progressive myoclonus and dementia
\item 
\end{itemize}
\end{block}

\begin{block}{Gaucher type 3b}
\begin{itemize}
\item horizontal supranuclear gaze palsy (see below) without other major
neurologic signs and with aggressive systemic disease
\end{itemize}
\end{block}

\begin{block}{Gaucher type 3c}
\begin{itemize}
\item present in late childhood or later and have only mild visceral signs
of classic Gaucher disease
\item distinguishing clinical signs include impaired horizontal ocular
saccades, corneal opacities, and cardiac/aortic valvular
calcification
\end{itemize}
\end{block}
\end{frame}
\section{Laboratory Investigations}
\label{sec:orgheadline15}

\section{Treatment}
\label{sec:orgheadline16}
\end{document}

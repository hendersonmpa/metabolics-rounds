% Created 2017-08-24 Thu 16:28
\documentclass[presentation, smaller]{beamer}
\usepackage[utf8]{inputenc}
\usepackage[T1]{fontenc}
\usepackage{fixltx2e}
\usepackage{graphicx}
\usepackage{grffile}
\usepackage{longtable}
\usepackage{wrapfig}
\usepackage{rotating}
\usepackage[normalem]{ulem}
\usepackage{amsmath}
\usepackage{textcomp}
\usepackage{amssymb}
\usepackage{capt-of}
\usepackage{hyperref}
\hypersetup{colorlinks,linkcolor=white,urlcolor=blue}
\usepackage{textpos}
\usepackage[version=4]{mhchem}
\usepackage{chemfig}
\usepackage[usenames,dvipsnames]{xcolor}
\usepackage[T1]{fontenc}
\usepackage{lmodern}
\usepackage{verbatim}
\usetheme[height=20pt]{Boadilla}
\usecolortheme[RGB={170,160,80}]{{structure}}
\author{Matthew Henderson, PhD, FCACB}
\date{\today}
\title{Neonatal Hyperammonemia}
\institute[NSO]{Newborn Screening Ontario | The University of Ottawa}
\titlegraphic{\includegraphics[height=1cm,keepaspectratio]{../logos/NSO_logo.pdf} \includegraphics[height=1cm,keepaspectratio]{../logos/UOlogoBW.eps}}
\hypersetup{
 pdfauthor={Matthew Henderson, PhD, FCACB},
 pdftitle={Neonatal Hyperammonemia},
 pdfkeywords={},
 pdfsubject={},
 pdfcreator={Emacs 25.2.1 (Org mode 8.3.4)}, 
 pdflang={English}}
\begin{document}

\maketitle
\logo{\includegraphics[width=1cm,height=1cm,keepaspectratio]{../logos/NSO_logo_small.pdf}~%
    \includegraphics[width=1cm,height=1cm,keepaspectratio]{../logos/UOlogoBW.eps}%
}

\vspace{220pt}}
\beamertemplatenavigationsymbolsempty
\setbeamertemplate{caption}[numbered]
\setbeamerfont{caption}{size=\tiny}

% \addtobeamertemplate{frametitle}{}{%
% \begin{textblock*}{100mm}(.85\textwidth,-1cm)
% \includegraphics[height=1cm,width=2cm]{cat}
% \end{textblock*}}

\section{Background}
\label{sec:orgheadline6}
\begin{frame}[label={sec:orgheadline1}]{Production}
\begin{itemize}
\item Ammonia is produced via the metabolism of nitrogen containing compounds:
\begin{itemize}
\item amino acids
\item purines and pyrimidines
\end{itemize}
\item Intestinal bacteria produce ammonia by splitting urea in the gut.
\item Ammonia in the body of a healthy individual is typically present as ammonium ions.
\end{itemize}
\centering
  \ce{NH4+ <=>[pKa = 9.3] NH3 + H+}
\begin{itemize}
\item Ammonium ions cannot cross membranes
\end{itemize}
\end{frame}

\begin{frame}[label={sec:orgheadline2}]{Transport from Muscle: Glucose/Alanine Cycle}
\centering
   \includegraphics[width=.9\linewidth]{./figures/glucose_alanine_cycle.png}
\end{frame}

\begin{frame}[label={sec:orgheadline3}]{Transport from Muscle: Glutamine}
\centering
   \includegraphics[width=.9\linewidth]{./figures/nitrogen_glutamine.png}
\end{frame}

\begin{frame}[label={sec:orgheadline4}]{Urea Cycle: Next Session UCD}
\centering
   \includegraphics[width=.9\linewidth]{./figures/urea_cycle.png}
\end{frame}

\begin{frame}[label={sec:orgheadline5}]{Liver Lobule}
\centering
\includegraphics[width=.9\linewidth]{./figures/liver_lobule.png}

\begin{description}
\item[{periportal hepatocytes}] \(\uparrow\) capacity, \(\downarrow\) affinity
\item[{perivenous hepatocytes}] \(\downarrow\) capacity, \(\uparrow\) affinity
\end{description}
\end{frame}


\section{Laboratory}
\label{sec:orgheadline11}
\begin{frame}[label={sec:orgheadline7}]{Measurement at CHEO}
\begin{itemize}
\item The VITROS AMON Slide has a multilayered analytical element coated
on a polyester support.
\item A drop of patient sample is deposited on the slide.
\item An  alkaline buffer converts ammonium ions to gaseous ammonia.
\item A semipermeable membrane allows only ammonia to pass through
\item After incubation the reflection density of the dye is measured.
\end{itemize}

\begin{block}{}
\centering
\ce{NH3 + bromophenol blue -> blue dye (600 nm)}
\end{block}

\begin{itemize}
\item bromophenol blue changes from yellow at pH 3.0 to blue at pH 4.6
\end{itemize}
\end{frame}

\begin{frame}[label={sec:orgheadline8}]{Measurement at TOH}
\begin{itemize}
\item The Dimension Vista Ammonia (AMM) method uses glutamate dehydrogenase (GLDH) and a NADPH analog.
\end{itemize}

\begin{block}{}
\centering
\ce{\alpha-ketoglutarate + NH4+ + NADPH ->[GLDH] L-glutamate + NADP+ + H2O}
\end{block}

\begin{itemize}
\item The decrease in absorbance due to the oxidation of the reduced
cofactor is monitored at 340/700 nm and is proportional to the
ammonia concentration.
\end{itemize}
\end{frame}

\begin{frame}[label={sec:orgheadline9}]{Point of Care Ammonia Meters}
\begin{block}{PocketChem BA}
\begin{itemize}
\item blood ammonia analysis used in veterinary medicine.
\item \href{http://www.woodleyequipment.com/laboratory-diagnostics/clinical-chemistry/pocketchem-ba-blood-ammonia-analyser-474-140-.php}{link}
\item \href{http://onlinelibrary.wiley.com/doi/10.1111/j.1939-165X.2008.00024.x/abstract;jsessionid=365F3D4A44D8D11C6511CA0B223D065B.f03t03}{validation}
\end{itemize}
\end{block}

\begin{block}{Brannelly, N. (2017) PhD, University of the West of England.}
-The development of a point of care device for measuring blood ammonia.
\begin{itemize}
\item polyaniline nanoparticle ink
\end{itemize}
\end{block}
\end{frame}

\begin{frame}[label={sec:orgheadline10}]{Ammonia Interpretation}
\begin{itemize}
\item In a healthy person, ammonia is relatively tightly controlled;
\end{itemize}
\begin{block}{CHEO}
\begin{center}
\begin{tabular}{lrr}
Age & RI (umol/L) & Critical\\
\hline
0-<1 month & 10-55 & >55\\
1-<3 months & <30 & >55\\
3 months-<18 yrs & <30 & >100\\
>18 years & <30 & >200\\
\end{tabular}
\end{center}
\end{block}

\begin{block}{TOH}
\begin{center}
\begin{tabular}{lrl}
Age & RI (umol/L) & Critical\\
\hline
All & <35 & None\\
\end{tabular}
\end{center}
\end{block}
\end{frame}


\section{Clinical}
\label{sec:orgheadline21}
\begin{frame}[label={sec:orgheadline12}]{Neonatal Symptoms}
\begin{columns}
\begin{column}{0.5\columnwidth}
\begin{itemize}
\item poor feeding
\item vomiting
\item seizures
\item respiratory distress
\item poor peripheral blood circulation
\end{itemize}
\end{column}
\begin{column}{0.5\columnwidth}
\begin{itemize}
\item hypotonia
\item vomiting
\item "abnormal neurologic changes"
\begin{itemize}
\item stupor
\end{itemize}
\item inhibition of insulin secretion
\end{itemize}
\end{column}
\end{columns}
\begin{block}{Outcome}
\begin{itemize}
\item outcome \(\propto\) \(\frac{1}{duration + [\ce{NH4+}]}\)
\begin{itemize}
\item irreparable brain damage
\end{itemize}
\end{itemize}
\end{block}
\end{frame}
\begin{frame}[label={sec:orgheadline13}]{Causes of Hyperammonemia}
\begin{block}{Increased ammonia production}
\begin{itemize}
\item High protein diets
\item Massive hemolysis
\item Parenteral nutrition with high nitrogen content
\item Protein catabolism (kwashiorkor)
\item Infection
\item \textbf{Pre-analytical}
\end{itemize}
\end{block}

\begin{block}{Decreased ammonia elimination}
\begin{itemize}
\item liver disease
\item IEM
\begin{itemize}
\item urea cycle defects
\item fatty acid oxidation defects
\item organic acidemias
\end{itemize}
\end{itemize}
\end{block}
\end{frame}

\begin{frame}[label={sec:orgheadline14}]{Pre-analytical Considerations in Ammonia Testing}
\begin{itemize}
\item Capillary ammonia is significantly higher than arterial and venous
\begin{itemize}
\item Capillary samples - sweat contamination
\end{itemize}
\item Delayed analysis
\begin{itemize}
\item erythrocytes and platlets \(\to\) ammonia
\item GGT activity
\item Serum is unsuitable
\end{itemize}
\item Hemolysis -  \(\uparrow\) [ammonia] RBC
\item Detergent contamination
\end{itemize}
\end{frame}

\begin{frame}[label={sec:orgheadline15}]{Specimen Collection and Handling}
\begin{itemize}
\item Free flowing venous or arterial sample
\item Pre-chilled lithium heparin (green top)
\item Transport on ice
\item Separated w/in 15 min of collection
\item Analysed immediately
\item Once separated stable: 4 hr @ 4\textdegree C , 24hr @-20\textdegree C
\end{itemize}
\end{frame}

\begin{frame}[label={sec:orgheadline16}]{Biochemical Testing in Neonate with Hyperammonemia}
\begin{itemize}
\item First line
\begin{itemize}
\item Blood gas analysis
\begin{itemize}
\item UCD \(\to\) Respiratory alkalosis
\item UCD rarely acidotic
\item Acidosis suggests OAD or mitochondrial disorder.
\end{itemize}
\item Urea
\item Glucose
\begin{itemize}
\item hypoglycemia - FAOD, HI, liver failure
\end{itemize}
\item Liver Function tests
\item Lactate
\begin{itemize}
\item mitochondrial disorders,organic acidemias and FAODs
\end{itemize}
\end{itemize}

\item Specialist Investigations
\begin{itemize}
\item Urine and Plasma amino acids
\begin{itemize}
\item citruline
\item argininosuccinic acid
\end{itemize}
\item Urine Organic Acids
\begin{itemize}
\item orotic acid
\end{itemize}
\end{itemize}
\end{itemize}
\end{frame}

\begin{frame}[label={sec:orgheadline17}]{Differential Diagnosis in the Neonate}
\begin{columns}
\begin{column}{0.5\columnwidth}
\begin{itemize}
\item Sepsis
\item Liver dysfunction
\item Portocaval shunt
\item Perinatal asphyxia
\item Sampling artifact
\end{itemize}
\end{column}


\begin{column}{0.5\columnwidth}
\begin{itemize}
\item IEMs
\begin{itemize}
\item Urea cycle disorders
\item Organic acidemias
\item Fatty acid oxidation disorders
\item Mitochondrial disorders
\item Amino acid transporter deficiency
\end{itemize}
\end{itemize}
\end{column}
\end{columns}

\begin{block}{}
Inborn errors of metabolism are a very important part of the differential diagnosis in a neonate who has hyperammonemia -- 100 umol/L or higher
\end{block}
\end{frame}

\begin{frame}[label={sec:orgheadline18}]{Differential Diagnosis in the Neonate}
\centering
\includegraphics[width=.9\linewidth]{./figures/THANvIEM.png}
\end{frame}

\begin{frame}[label={sec:orgheadline19}]{Ammonia Interpretation}
\begin{center}
\begin{tabular}{ll}
Ammonia (umol/L) & Conditions\\
\hline
> 1500 & THAN\\
> 600 & UCD, PA, Valproate\\
200 - 600 & OA, FAOD,\\
< 200 & Acquired\\
\end{tabular}
\end{center}
\end{frame}


\begin{frame}[label={sec:orgheadline20}]{References}
\begin{itemize}
\item \url{http://www.metbio.net/metbioGuidelines.asp}
\item Hudak, M. L., Jones, M. D., \& Brusilow, S. W. (1985).
Differentiation of transient hyperammonemia of the newborn and urea
cycle enzyme defects by clinical presentation. The Journal of
Pediatrics, 107(5), 712–719.
\item Haberle, J. (2011). Clinical practice: The management of
hyperammonemia. European Journal of Pediatrics, 170(1), 21–34.
\item Auron, A., \& Brophy, P. D. (2012). Hyperammonemia in review:
Pathophysiology, diagnosis, and treatment. Pediatric Nephrology,
27(2), 207–222.
\end{itemize}
\end{frame}
\end{document}

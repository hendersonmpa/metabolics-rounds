% Created 2017-10-12 Thu 12:58
\documentclass[presentation, smaller]{beamer}
\usepackage[utf8]{inputenc}
\usepackage[T1]{fontenc}
\usepackage{fixltx2e}
\usepackage{graphicx}
\usepackage{grffile}
\usepackage{longtable}
\usepackage{wrapfig}
\usepackage{rotating}
\usepackage[normalem]{ulem}
\usepackage{amsmath}
\usepackage{textcomp}
\usepackage{amssymb}
\usepackage{capt-of}
\usepackage{hyperref}
\hypersetup{colorlinks,linkcolor=white,urlcolor=blue}
\usepackage{textpos}
\usepackage{textgreek}
\usepackage[version=4]{mhchem}
\usepackage{chemfig}
\usepackage{siunitx}
\usepackage[usenames,dvipsnames]{xcolor}
\usepackage[T1]{fontenc}
\usepackage{lmodern}
\usepackage{verbatim}
\usepackage{tikz}
\usetikzlibrary{shapes.geometric,arrows,decorations.pathmorphing,backgrounds,positioning,fit,petri}
\usetheme[height=20pt]{Boadilla}
\usecolortheme[RGB={170,160,80}]{{structure}}
\author{Matthew Henderson, PhD, FCACB}
\date{\today}
\title{Protein and Amino Acid Metabolism}
\institute[NSO]{Newborn Screening Ontario | The University of Ottawa}
\titlegraphic{\includegraphics[height=1cm,keepaspectratio]{../logos/NSO_logo.pdf}\includegraphics[height=1cm,keepaspectratio]{../logos/cheo-logo.png} \includegraphics[height=1cm,keepaspectratio]{../logos/UOlogoBW.eps}}
\hypersetup{
 pdfauthor={Matthew Henderson, PhD, FCACB},
 pdftitle={Protein and Amino Acid Metabolism},
 pdfkeywords={},
 pdfsubject={},
 pdfcreator={Emacs 25.2.1 (Org mode 8.3.4)}, 
 pdflang={English}}
\begin{document}

\maketitle
%\logo{\includegraphics[width=1cm,height=1cm,keepaspectratio]{../logos/NSO_logo_small.pdf}~%
%    \includegraphics[width=1cm,height=1cm,keepaspectratio]{../logos/UOlogoBW.eps}%
%}

\vspace{220pt}}
\beamertemplatenavigationsymbolsempty
\setbeamertemplate{caption}[numbered]
\setbeamerfont{caption}{size=\tiny}
% \addtobeamertemplate{frametitle}{}{%
% \begin{textblock*}{100mm}(.85\textwidth,-1cm)
% \includegraphics[height=1cm,width=2cm]{cat}
% \end{textblock*}}

\tikzstyle{core} = [rectangle, rounded corners, text width=2cm, minimum height=.5cm,text centered, draw=black, fill=blue!30]
\tikzstyle{io} = [rectangle, rounded corners, text width=2cm, minimum height=.5cm,text centered, draw=black, fill=gray!30]
\tikzstyle{hl} = [rectangle, rounded corners, text width=2cm, minimum height=.5cm,text centered, draw=black, fill=red!30]
\tikzstyle{arrow} = [thick,->,>=stealth]
\tikzstyle{hl-arrow} = [ultra thick,->,>=stealth,draw=red]

\section{Protein Digestion}
\label{sec:orgheadline6}
\begin{frame}[label={sec:orgheadline1}]{Protein Metabolism}
\begin{center}
\begin{tikzpicture}[node distance=2cm]
% nodes
\node(protein)[core]{Protein (\textasciitilde{}11 kg)};
\node(faa)[core, below of=protein, yshift=-1cm]{Free amino acids (\textasciitilde{}70 g/d)};
\node(proteolysis)[io, left of=faa, xshift=-1cm, yshift=1.5cm]{Proteolysis (\textasciitilde{}300 g/d)};
\node(food)[io, left of=faa, xshift=-1cm]{Food intake};
\node(synthesis)[io, left of=faa, xshift=-1cm, yshift=-1cm]{Amino acid synthesis};
\node(protsyn)[io, right of=faa, xshift=1cm, yshift=1.5cm]{Protein synthesis (\textasciitilde{}300 g/d)};
\node(degradation)[io, right of=faa, xshift=1cm]{Degradation};
\node(conversion)[io, right of=faa, xshift=1cm, yshift=-1cm]{Conversion};

% arrows
\draw[arrow](protein) -| (proteolysis);
\draw[arrow](protsyn) |- (protein);
\draw[arrow](proteolysis) -- (faa);
\draw[arrow](faa) -- (protsyn);
\draw[arrow](food) -- (faa);
\draw[arrow](synthesis) -- (faa);
\draw[arrow](faa) -- (degradation);
\draw[arrow](faa) -- (conversion);

\end{tikzpicture}
\end{center}
\\
\begin{flushright}
\tiny{Adapted from SIMD-NAMA}
\end{flushright}

\begin{itemize}
\item 70 kg adult
\end{itemize}
\end{frame}
\begin{frame}[label={sec:orgheadline2}]{Protein Metabolism: Food Intake}
\begin{center}
\begin{tikzpicture}[node distance=2cm]
% nodes
\node(protein)[core]{Protein (\textasciitilde{}11 kg)};
\node(faa)[core, below of=protein, yshift=-1cm]{Free amino acids (\textasciitilde{}70 g/d)};
\node(proteolysis)[io, left of=faa, xshift=-1cm, yshift=1.5cm]{Proteolysis (\textasciitilde{}300 g/d)};
\node(food)[hl, left of=faa, xshift=-1cm]{Food intake};
\node(synthesis)[io, left of=faa, xshift=-1cm, yshift=-1cm]{Amino acid synthesis};
\node(protsyn)[io, right of=faa, xshift=1cm, yshift=1.5cm]{Protein synthesis (\textasciitilde{}300 g/d)};
\node(degradation)[io, right of=faa, xshift=1cm]{Degradation};
\node(conversion)[io, right of=faa, xshift=1cm, yshift=-1cm]{Conversion};

% arrows
\draw[arrow](protein) -| (proteolysis);
\draw[arrow](protsyn) |- (protein);
\draw[arrow](proteolysis) -- (faa);
\draw[arrow](faa) -- (protsyn);
\draw[hl-arrow](food) -- (faa);
\draw[arrow](synthesis) -- (faa);
\draw[arrow](faa) -- (degradation);
\draw[arrow](faa) -- (conversion);

\end{tikzpicture}
\end{center}
\
\begin{flushright}
\tiny{Adapted from SIMD-NAMA}
\end{flushright}
\end{frame}

\begin{frame}[label={sec:orgheadline3}]{Pancreatic and Intestinal Proteases}
\centering
\includegraphics[height=0.9\textheight]{./figures/proteolysis.png}

\begin{itemize}
\item Cystic Fibrosis
\end{itemize}
\end{frame}
\begin{frame}[label={sec:orgheadline4}]{Absorption from Small Intestine}
\begin{itemize}
\item Most AA have \textgreater{} 1 transport system
\item Driven by low intracellular \ce{Na+}
\begin{itemize}
\item secondary active transport
\end{itemize}
\end{itemize}

\begin{columns}
\begin{column}{0.5\columnwidth}
\begin{itemize}
\item Hartnup disease
\begin{itemize}
\item non-polar amino acids
\end{itemize}
\item cystinuria
\end{itemize}
\end{column}


\begin{column}{0.5\columnwidth}
\includegraphics[width=.9\linewidth]{./figures/transport.png}
\end{column}
\end{columns}
\end{frame}

\begin{frame}[label={sec:orgheadline5}]{20 Essential \& Non-essential Amino Acids}
\begin{columns}
\begin{column}{0.45\columnwidth}
\begin{block}{Essential}
\begin{itemize}
\item Histidine
\item Isoleucine
\item Leucine
\item Lysine
\item Methionine
\item Phenylalanine
\item Threonine
\item Tryptophan
\item Valine
\end{itemize}
\end{block}
\end{column}

\begin{column}{0.45\columnwidth}
\begin{block}{Non-essential}
\begin{itemize}
\item Glycine
\item Alanine
\item Serine
\item \emph{Cysteine (Met)}
\item Glutamate
\item Glutamine
\item Proline
\item Arginine
\item Asparagine
\item Aspartate
\item \emph{Tyrosine (Phe)}
\end{itemize}
\end{block}
\end{column}
\end{columns}
\end{frame}

\section{Proteolysis}
\label{sec:orgheadline9}
\begin{frame}[label={sec:orgheadline7}]{Protein Metabolism: Proteolysis}
\begin{center}
\begin{tikzpicture}[node distance=2cm]
% nodes
\node(protein)[core]{Protein (\textasciitilde{}11 kg)};
\node(faa)[core, below of=protein, yshift=-1cm]{Free amino acids (\textasciitilde{}70 g/d)};
\node(proteolysis)[hl, left of=faa, xshift=-1cm, yshift=1.5cm]{Proteolysis (\textasciitilde{}300 g/d)};
\node(food)[io, left of=faa, xshift=-1cm]{Food intake};
\node(synthesis)[io, left of=faa, xshift=-1cm, yshift=-1cm]{Amino acid synthesis};
\node(protsyn)[io, right of=faa, xshift=1cm, yshift=1.5cm]{Protein synthesis (\textasciitilde{}300 g/d)};
\node(degradation)[io, right of=faa, xshift=1cm]{Degradation};
\node(conversion)[io, right of=faa, xshift=1cm, yshift=-1cm]{Conversion};

% arrows
\draw[hl-arrow](protein) -| (proteolysis);
\draw[arrow](protsyn) |- (protein);
\draw[hl-arrow](proteolysis) -- (faa);
\draw[arrow](faa) -- (protsyn);
\draw[arrow](food) -- (faa);
\draw[arrow](synthesis) -- (faa);
\draw[arrow](faa) -- (degradation);
\draw[arrow](faa) -- (conversion);

\end{tikzpicture}
\end{center}
\\
\begin{flushright}
\tiny{Adapted from SIMD-NAMA}
\end{flushright}
\end{frame}

\begin{frame}[label={sec:orgheadline8}]{Protein Turnover - Intracellular amino acid pool}
\begin{itemize}
\item Proteolysis happens in catabolic circumstances (energetic requirements > intake)
\begin{itemize}
\item Fasting
\item Physiologic stress (fever, infections, stress, surgery)
\end{itemize}
\item Skeletal muscle is the main source of amino acids
\item 30 g muscle \(\to\) 6.25 g protein  \(\to\) 1 g nitrogen
\end{itemize}
\begin{block}{Lysosomal protein turnover}
\begin{itemize}
\item autophagy: engulf cytoplasm into vesicles
\begin{itemize}
\item can be triggered by starvation
\end{itemize}
\item cathepsin digests engulfed protein \(\to\) amino acids
\end{itemize}
\end{block}
\begin{block}{Ubiquitin-proteasome pathway}
\begin{itemize}
\item covalent binding of ubiquitin targets proteins for degradation by proteasome.
\item PEST sequences increase ubiquitination and decrease half-life.
\end{itemize}
\end{block}
\end{frame}

\section{Protein Synthesis}
\label{sec:orgheadline12}
\begin{frame}[label={sec:orgheadline10}]{Protein Metabolism: Synthesis}
\begin{center}
\begin{tikzpicture}[node distance=2cm]
% nodes
\node(protein)[core]{Protein (\textasciitilde{}11 kg)};
\node(faa)[core, below of=protein, yshift=-1cm]{Free amino acids (\textasciitilde{}70 g/d)};
\node(proteolysis)[io, left of=faa, xshift=-1cm, yshift=1.5cm]{Proteolysis (\textasciitilde{}300 g/d)};
\node(food)[io, left of=faa, xshift=-1cm]{Food intake};
\node(synthesis)[io, left of=faa, xshift=-1cm, yshift=-1cm]{Amino acid synthesis};
\node(protsyn)[hl, right of=faa, xshift=1cm, yshift=1.5cm]{Protein synthesis (\textasciitilde{}300 g/d)};
\node(degradation)[io, right of=faa, xshift=1cm]{Degradation};
\node(conversion)[io, right of=faa, xshift=1cm, yshift=-1cm]{Conversion};

% arrows
\draw[arrow](protein) -| (proteolysis);
\draw[hl-arrow](protsyn) |- (protein);
\draw[arrow](proteolysis) -- (faa);
\draw[hl-arrow](faa) -- (protsyn);
\draw[arrow](food) -- (faa);
\draw[arrow](synthesis) -- (faa);
\draw[arrow](faa) -- (degradation);
\draw[arrow](faa) -- (conversion);

\end{tikzpicture}
\end{center}
\\
\begin{flushright}
\tiny{Adapted from SIMD-NAMA}
\end{flushright}
\end{frame}

\begin{frame}[label={sec:orgheadline11}]{Protein Synthesis}
\begin{itemize}
\item Amino acid composition of proteins is genetically determined
\item Essential amino acids are limiting
\end{itemize}
\begin{columns}
\begin{column}{0.45\columnwidth}
\begin{block}{Control}
\begin{itemize}
\item 20g natural protein intake
\item 9 essential AAs,Phe 4\% = 800mg
\item Synthesized protein 10-16 g
\item Muscle 100g
\end{itemize}
\end{block}
\end{column}
\begin{column}{0.45\columnwidth}
\begin{block}{Classical PKU}
\begin{itemize}
\item 5-8 g natural protein intake
\begin{itemize}
\item Phe = 200-320 mg
\end{itemize}
\item Phe free AA mixture 15g
\item 8 essential AAs w Phe (300mg)
\item Synthesized protein 5-8 g
\item Muscle 25-35 g
\end{itemize}
\end{block}
\end{column}
\end{columns}
\begin{flushright}
\tiny{Adapted from SIMD-NAMA}
\end{flushright}
\end{frame}
\section{Amino Acid Synthesis}
\label{sec:orgheadline21}
\begin{frame}[label={sec:orgheadline13}]{Protein Metabolism: Amino Acid Synthesis}
\begin{center}
\begin{tikzpicture}[node distance=2cm]
% nodes
\node(protein)[core]{Protein (\textasciitilde{}11 kg)};
\node(faa)[core, below of=protein, yshift=-1cm]{Free amino acids (\textasciitilde{}70 g/d)};
\node(proteolysis)[io, left of=faa, xshift=-1cm, yshift=1.5cm]{Proteolysis (\textasciitilde{}300 g/d)};
\node(food)[io, left of=faa, xshift=-1cm]{Food intake};
\node(synthesis)[hl, left of=faa, xshift=-1cm, yshift=-1cm]{Amino acid synthesis};
\node(protsyn)[io, right of=faa, xshift=1cm, yshift=1.5cm]{Protein synthesis (\textasciitilde{}300 g/d)};
\node(degradation)[io, right of=faa, xshift=1cm]{Degradation};
\node(conversion)[io, right of=faa, xshift=1cm, yshift=-1cm]{Conversion};

% arrows
\draw[arrow](protein) -| (proteolysis);
\draw[arrow](protsyn) |- (protein);
\draw[arrow](proteolysis) -- (faa);
\draw[arrow](faa) -- (protsyn);
\draw[arrow](food) -- (faa);
\draw[hl-arrow](synthesis) -- (faa);
\draw[arrow](faa) -- (degradation);
\draw[arrow](faa) -- (conversion);

\end{tikzpicture}
\end{center}
\\
\begin{flushright}
\tiny{Adapted from SIMD-NAMA}
\end{flushright}
\end{frame}


\begin{frame}[label={sec:orgheadline14}]{Synthesis of Non-essential Amino Acids}
\includegraphics[width=.9\linewidth]{./figures/AA_synthesis.png}
\end{frame}

\begin{frame}[label={sec:orgheadline15}]{The AA transaminases}
\begin{itemize}
\item One specific transaminase for each amino acid
\begin{itemize}
\item \textalpha{}-amino transamination
\item Transamination of other amines 
\begin{itemize}
\item \textdelta{}-ornithine
\item \textgamma{}-GABA
\end{itemize}
\end{itemize}
\item Common characteristics:
\begin{itemize}
\item The cofactor: pyridoxal-P (vitamin B6)
\item All use \textalpha{}-ketoglutaric acid as nitrogen acceptor
\item All produce glutamic acid, which is central to nitrogen metabolism
\item Allows carbon skeleton to be oxidized by specific pathways
\end{itemize}
\end{itemize}
\end{frame}
\begin{frame}[label={sec:orgheadline16}]{The Role of Cofactors in Amino Acid Metabolism}
\begin{itemize}
\item AA metabolism requires three important cofactors:
\begin{itemize}
\item Pyridoxal phosphate (PLP)
\begin{itemize}
\item transamination reactions
\end{itemize}
\item Tetrahydrofolate (\ce{FH4})
\begin{itemize}
\item one-carbon transfer
\end{itemize}
\item Tetrahydrobiopterin (\ce{BH4})
\begin{itemize}
\item hydroxylation reactions
\end{itemize}
\end{itemize}
\end{itemize}
\end{frame}
\begin{frame}[label={sec:orgheadline17}]{AA biosynthetic families}
\begin{itemize}
\item Diverse biosynthetic pathways
\item Carbon skeletons of non-essential AAs come from either:
\begin{itemize}
\item intermediates of glycolysis
\item citric acid cycle
\end{itemize}
\end{itemize}
\end{frame}

\begin{frame}[label={sec:orgheadline18}]{AAs Derived from Glycolysis Intermediates}
\centering
\chemname{\chemfig[][scale=.75]{^{+}H_3N-C(-[2]COO^{-})(-[6]H)-H}}{\small glycine}
\chemname{\chemfig[][scale=.75]{^{+}H_3N-C(-[2]COO^{-})(-[6]CH_3)-H}}{\small alanine}
\chemname{\chemfig[][scale=.75]{^{+}H_3N-C(-[2]COO^{-})(-[6]CH_2-[6]OH)-H}}{\small serine}
\chemname{\chemfig[][scale=.75]{^{+}H_3N-C(-[2]COO^{-})(-[6]CH_3-[6]{\color{red}S}H)-H}}{\small cysteine}
\end{frame}

\begin{frame}[label={sec:orgheadline19}]{AAs derived from TCA cycle intermediates: \(\alpha\)-ketoglutarate}
\centering
\chemname{\chemfig[][scale=.75]{^{-}OOC-[7](=[1]O)-[6]CH_2-[6]CH_2-[6]COO^{-}}}{\small \textalpha{}-ketoglutarate}
\chemname{\chemfig[][scale=.75]{^{+}H_3N-C(-[2]COO^{-})(-[6]CH_2-[6]CH_2-[6]COO^{-})-H}}{\small glutamate}
\chemname{\chemfig[][scale=.75]{^{+}H_3N-C(-[2]COO^{-})(-[6]CH_2-[6]CH_2-[6]C(=O)-[6]NH_2)-H}}{\small glutamine}
\chemname{\chemfig[][scale=.75]{^{-}OOC-[6]*5(-^{+}H_2N----)}}{\small proline}
\chemname{\chemfig[][scale=.75]{^{+}H_3N-C(-[2]COO^{-})(-[6]{(CH_2)_3}-[6]NH-[6]C(=NH_2^{+})-[6]NH_2)-H}}{\small arginine}
%\chemname{\chemfig[][scale=.75]{^{+}H_3N-C(-[2]COO^{-})(-[6]CH_2-[6]*5(=-N=-NH-))-H}}{\small histidine}
\end{frame}

\begin{frame}[label={sec:orgheadline20}]{AAs derived from TCA cycle intermediate: oxaloacetate}
\centering
\chemname{\chemfig[][scale=.75]{^{-}OOC-[7](=[1]O)-[6]CH_2-[6]COO^{-}}}{\small oxaloacetate}
\chemname{\chemfig[][scale=.75]{^{+}H_3N-C(-[2]COO^{-})(-[6]CH_2-[6]COO^{-})-H}}{\small aspartate}
\chemname{\chemfig[][scale=.75]{^{+}H_3N-C(-[2]COO^{-})(-[6]CH_2-[6]C(=O)-[6]NH_2)-H}}{\small asparagine}
\end{frame}

\section{Amino Acids Catabolism}
\label{sec:orgheadline26}
\begin{frame}[label={sec:orgheadline22}]{Protein Metabolism: Amino Acid Catabolism}
\begin{center}
\begin{tikzpicture}[node distance=2cm]
% nodes
\node(protein)[core]{Protein (\textasciitilde{}11 kg)};
\node(faa)[core, below of=protein, yshift=-1cm]{Free amino acids (\textasciitilde{}70 g/d)};
\node(proteolysis)[io, left of=faa, xshift=-1cm, yshift=1.5cm]{Proteolysis (\textasciitilde{}300 g/d)};
\node(food)[io, left of=faa, xshift=-1cm]{Food intake};
\node(synthesis)[io, left of=faa, xshift=-1cm, yshift=-1cm]{Amino acid synthesis};
\node(protsyn)[io, right of=faa, xshift=1cm, yshift=1.5cm]{Protein synthesis (\textasciitilde{}300 g/d)};
\node(degradation)[hl, right of=faa, xshift=1cm]{Degradation};
\node(conversion)[hl, right of=faa, xshift=1cm, yshift=-1cm]{Conversion};

% arrows
\draw[arrow](protein) -| (proteolysis);
\draw[arrow](protsyn) |- (protein);
\draw[arrow](proteolysis) -- (faa);
\draw[arrow](faa) -- (protsyn);
\draw[arrow](food) -- (faa);
\draw[arrow](synthesis) -- (faa);
\draw[hl-arrow](faa) -- (degradation);
\draw[hl-arrow](faa) -- (conversion);

\end{tikzpicture}
\end{center}
\\
\begin{flushright}
\tiny{Adapted from SIMD-NAMA}
\end{flushright}
\end{frame}

\begin{frame}[label={sec:orgheadline23}]{Amino Acid Catabolism}
\centering
\includegraphics[height=0.9\textheight]{./figures/kgaas.png}
\end{frame}

\begin{frame}[label={sec:orgheadline24}]{Amino Acid Nitrogen}
\begin{itemize}
\item Urea Cycle, see previous!
\end{itemize}

\centering
\includegraphics[width=0.7\textwidth]{./figures/urea_cycle_crop.png}
\end{frame}


\begin{frame}[label={sec:orgheadline25}]{Next up}
\begin{itemize}
\item Amino Acid Analysis
\begin{itemize}
\item HPLC
\item FIA-MS/MS
\item LC- MS/MS
\end{itemize}
\end{itemize}
\end{frame}
\end{document}

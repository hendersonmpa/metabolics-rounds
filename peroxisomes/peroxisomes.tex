% Created 2019-10-30 Wed 16:00
% Intended LaTeX compiler: pdflatex
\documentclass[presentation, smaller]{beamer}
\usepackage[utf8]{inputenc}
\usepackage[T1]{fontenc}
\usepackage{graphicx}
\usepackage{grffile}
\usepackage{longtable}
\usepackage{wrapfig}
\usepackage{rotating}
\usepackage[normalem]{ulem}
\usepackage{amsmath}
\usepackage{textcomp}
\usepackage{amssymb}
\usepackage{capt-of}
\usepackage{hyperref}
\hypersetup{colorlinks,linkcolor=white,urlcolor=blue}
\usepackage{textpos}
\usepackage{textgreek}
\usepackage[version=4]{mhchem}
\usepackage{chemfig}
\usepackage{siunitx}
\usepackage{gensymb}
\usepackage[usenames,dvipsnames]{xcolor}
\usepackage[T1]{fontenc}
\usepackage{lmodern}
\usepackage{verbatim}
\usepackage{tikz}
\usepackage{wasysym}
\usetikzlibrary{shapes.geometric,arrows,decorations.pathmorphing,backgrounds,positioning,fit,petri}
\usetheme{Hannover}
\usecolortheme{whale}
\author{Matthew Henderson, PhD, FCACB}
\date{\today}
\title{Peroxisomal Disorders}
\institute[NSO]{Newborn Screening Ontario | The University of Ottawa}
\titlegraphic{\includegraphics[height=1cm,keepaspectratio]{../logos/NSO_logo.pdf}\includegraphics[height=1cm,keepaspectratio]{../logos/cheo-logo.png} \includegraphics[height=1cm,keepaspectratio]{../logos/UOlogoBW.eps}}
\hypersetup{
 pdfauthor={Matthew Henderson, PhD, FCACB},
 pdftitle={Peroxisomal Disorders},
 pdfkeywords={},
 pdfsubject={},
 pdfcreator={Emacs 26.1 (Org mode 9.1.9)}, 
 pdflang={English}}
\begin{document}

\maketitle

%\logo{\includegraphics[width=1cm,height=1cm,keepaspectratio]{../logos/NSO_logo_small.pdf}~%
%    \includegraphics[width=1cm,height=1cm,keepaspectratio]{../logos/UOlogoBW.eps}%
%}

\vspace{220pt}
\beamertemplatenavigationsymbolsempty
\setbeamertemplate{caption}[numbered]
\setbeamerfont{caption}{size=\tiny}
% \addtobeamertemplate{frametitle}{}{%
% \begin{textblock*}{100mm}(.85\textwidth,-1cm)
% \includegraphics[height=1cm,width=2cm]{cat}
% \end{textblock*}}

\section{Introduction}
\label{sec:orgae95cfb}
\begin{frame}[label={sec:orgcd9547c}]{Peroxisomal FA oxidation}
\begin{itemize}
\item \(\beta\)-oxidation of:
\begin{itemize}
\item very-long-chain fatty acids
\begin{itemize}
\item C22:6-CoA, C22:5-CoA, C26:1-CoA, C26:0-CoA
\item do not degrade fatty acids to completion
\end{itemize}
\item bile acid intermediates di- and trihydroxycholestanoic acid
\item a range of eicosanoids
\end{itemize}
\item \(\alpha\)-oxidation of phytanoyl-CoA \(\to\) pristanoly-CoA
\begin{itemize}
\item \(\beta\)-oxidation of pristanoly-CoA
\end{itemize}

\item Two acyl-CoA oxidases (ACOX1 \& 2)
\item Two bifunctional proteins (LBP \& DBP)
\item Two thiolases (ACAA1 \& SCPx).
\item Single enzyme or transport protein defect result in abnormalities in some fatty acids.
\end{itemize}
\end{frame}

\begin{frame}[label={sec:org22d6ad9}]{Peroxisomal Disorders}
\begin{itemize}
\item Peroxisomal \(\beta\)-oxidation disorders can be subdivided into two groups:
\begin{enumerate}
\item single peroxisomal enzyme or transport protein deficiencies
\begin{itemize}
\item X-linked adrenoleukodystrophy (X-ALD)
\end{itemize}
\item generalized peroxisomal beta-oxidation deficiencies.
\begin{itemize}
\item Zellweger syndrome
\end{itemize}
\end{enumerate}

\item Peroxisomal \(\alpha\)-oxidation disorders
\begin{itemize}
\item Adult Refsum
\end{itemize}
\item Sjogren Larsson Syndrome
\end{itemize}
\end{frame}

\begin{frame}[label={sec:orgdd7a7d8}]{Peroxisomal Disorders (Exam?)}
\begin{itemize}
\item Etherphospholipid Biosynthesis
\begin{itemize}
\item Rhizomelic Chondrodysplasia Punctata (RCDP) 1-4
\end{itemize}
\item FA Elongation disorders
\begin{itemize}
\item EVOL4, EVOL5,
\item Trans-2,3-Enoyl-CoA Reductase (TER), 3-Hydroxyacyl-CoA Dehydratase 1 (HACD1)
\end{itemize}
\item Eicosanoid Metabolism
\begin{itemize}
\item Primary Hypertrophic Osteoarthropathy Type 1
\item LTC4-Synthase Deficiency
\end{itemize}
\end{itemize}
\end{frame}

\begin{frame}[label={sec:orge9fe90f}]{Non-mitochondrial FA metabolism}
\begin{figure}[htbp]
\centering
\includegraphics[width=\textwidth]{./figures/non_mito_FA_met.png}
\caption[Non-mitochondrial FA metabolism]{\label{fig:org09c655a}
Non-mitochondrial FA metabolism}
\end{figure}
\end{frame}

\section{X-linked Adrenoleukodystrophy}
\label{sec:org39a7e54}
\begin{frame}[label={sec:orgc51d832}]{Clinical Presentation}
\begin{itemize}
\item 1/20,000 males and females
\end{itemize}
\begin{block}{ALD}
\begin{itemize}
\item childhood cerebral form is the most severe
\begin{itemize}
\item neurological symptoms between 4-12 years
\item vegetative state and death
\end{itemize}
\item Males may present with ADD or behavioural changes
\begin{itemize}
\item due to visuospatial deficits \textpm{} central hearing loss
\end{itemize}
\item followed by severe visual and hearing impairment, quadriplegia and
cerebellar ataxia.
\end{itemize}
\end{block}
\begin{block}{AMN}
\begin{itemize}
\item Adrenomyeloneuropathy affects 65\% of adult X-ALD male patients (20-50 years)
\begin{itemize}
\item up to 88\% of heterozygous women (\textgreater{}40 years)
\end{itemize}
\item Presentation is progressive spastic paraparesis and sensory ataxia.
\end{itemize}
\end{block}
\end{frame}

\begin{frame}[label={sec:org0b20daf}]{Metabolic Derangement}
\begin{itemize}
\item Deficiency in ALDP a peroxisomal ABC transporter
\item \(\to\) \(\uparrow\) VLCFA-CoA in cytosol
\begin{itemize}
\item incorporation into a variety of lipids including:
\item cholesterol esters, phospholipids, and sphingolipids
\end{itemize}
\item VLCFAs accumulate in virtually all tissues, including erythrocytes,
white blood cells and plasma.
\end{itemize}
\end{frame}

\begin{frame}[label={sec:orgfa5d2e0}]{Genetics}
\begin{itemize}
\item X-ALD is caused by mutations in ABCD1
\item X-linked
\end{itemize}
\end{frame}

\begin{frame}[label={sec:org885e872}]{Diagnostic Tests}
\begin{itemize}
\item Quantitative VLCFAs including:C22:0, C24:0, and C26:0 in plasma
\begin{itemize}
\item after alkaline and acid hydrolysis
\item releases the VLCFAs from all lipid species
\item \(\uparrow\) concentration of C26:0
\item \(\uparrow\) C26:0/C22:0 and C24:0/C22:0 ratios
\end{itemize}

\item In females, normal plasma VLCFA does not exclude XALD
\begin{itemize}
\item mutation analysis is the most reliable method for diagnosis in females
\end{itemize}

\item False-positive results have been reported in:
\begin{itemize}
\item patients on a ketogenic diet
\item recent ingestion of peanut butter
\end{itemize}
\end{itemize}
\end{frame}

\begin{frame}[label={sec:org9884363}]{Treatment}
\begin{itemize}
\item A boy born with X-ALD has a 35\% risk of developing cerebral ALD
between the age of 4-12 years
\item 100\% risk of developing AMN between the age of 25-50 years.
\item Cerebral X-ALD can be treated in boys and adult males
\begin{itemize}
\item only at a very early stage of the disease,
\item when patients start to develop cerebral demyelination on brain MRI
but have no or minimal neurologic symptoms.
\end{itemize}
\item HCT can arrest the cerebral demyelination when the procedure is
performed at a very early stage
\end{itemize}
\end{frame}

\section{Zellweger Spectrum Disorders}
\label{sec:orge87d50c}
\begin{frame}[label={sec:orgf769a84}]{Clinical Presentation}
\begin{itemize}
\item Prototypical ZSD:
\begin{enumerate}
\item Typical cranial facial dysmorphia including:
\begin{itemize}
\item high forehead
\item large interior fontanelle
\item hypoplastic supraorbital ridges
\item epicanthal folds
\item flat nasal bridge
\item deformed ear lobes
\end{itemize}
\item Profound neurological abnormalities
\end{enumerate}
\end{itemize}
\end{frame}

\begin{frame}[label={sec:orgd22c658}]{Metabolic Derangement}
\begin{itemize}
\item Absence or marked deficiency of peroxisomes
\begin{itemize}
\item assessed by catalase-staining in fibroblasts
\item using immunofluorescence microscopy analysis
\end{itemize}
\item All peroxisomal functions are impaired.
\item In classical ZSD abnormalities include:
\begin{itemize}
\item \(\uparrow\) VLCFAs
\item \(\uparrow\) pristanic acid,
\item \(\uparrow\) di- and trihydroxycholestanoic acid,
\item \(\uparrow\) pipecolic acid
\item \(\downarrow\) plasmalogens in erythrocytes
\end{itemize}
\end{itemize}
\end{frame}

\begin{frame}[label={sec:orgf62a90c}]{Genetics}
\begin{itemize}
\item The genetic basis of the ZSD is heterogeneous
\item Biallelic mutations identified in:
\begin{itemize}
\item PEX1, PEX2, PEX3, PEX5, PEX6, PEX10, PEX12, PEX13, PEX14, PEX16, PEX19, and PEX26
\end{itemize}
\item All disorders are autosomal recessive
\end{itemize}
\end{frame}

\begin{frame}[label={sec:org2405c80}]{Diagnostic Tests}
\begin{itemize}
\item VLCFA analysis is a good initial biochemical test
\item erythrocyte plasmalogens
\item pipecolic acid upon amino acid analysis
\item DNA-panel containing all PEX genes or all genes coding for
peroxisomal protein
\end{itemize}
\end{frame}

\begin{frame}[label={sec:orgf13ce7d}]{Treatment}
\begin{itemize}
\item No treatment available
\item supplementation with docosahexaenoic acid (DHA) is not beneficial
\item Investigating cholic acid supplementation to reduce formation of the
toxic bile acid intermediates DHCA and THCA
\end{itemize}
\end{frame}

\section{Adult Refsum Disease}
\label{sec:orgb0f7e27}
\begin{frame}[label={sec:org225f415}]{Clinical Presentation}
\begin{itemize}
\item present in late childhood with:
\begin{itemize}
\item progressive loss of night vision
\item decline in visual capacity
\item anosmia
\end{itemize}
\item After \(\ge\) 10 years patients may develop:
\begin{itemize}
\item deafness, ataxia, polyneuropathy, ichthyosis, fatigue, and cardiac
conduction disturbances
\end{itemize}
\item full constellation of features defined by Refsum includes:
\begin{itemize}
\item retinitis pigmentosa, cerebellar ataxia and chronic polyneuropathy
\end{itemize}
\item rarely seen in single patients with ARD
\end{itemize}
\end{frame}

\begin{frame}[label={sec:orgd50c6f1}]{Metabolic derangement}
\begin{itemize}
\item Phytanoyl-CoA hydroxylase is deficient in ARD
\item required for \(\alpha\)-oxidation of phytanic acid
\item \(\to\) accumulation of phytanic acid
\end{itemize}

\begin{figure}[htbp]
\centering
\includegraphics[width=0.3\textwidth]{./figures/alpha.png}
\caption[oxidation of phytanic]{\label{fig:org9635189}
Oxidation of Phytanic Acid}
\end{figure}
\end{frame}

\begin{frame}[label={sec:orge4268e0}]{Genetics}
\begin{itemize}
\item ARD is an autosomal recessive disorder caused by mutations in PHYH.
\item A large number of often private mutations has been identified
\end{itemize}
\end{frame}

\begin{frame}[label={sec:org676d4fa}]{Diagnostic Tests}
\begin{itemize}
\item \(\uparrow\) \(\uparrow\) \(\uparrow\) plasma phytanic acid
\item \(\uparrow\) phytanic acid in ZS
\begin{itemize}
\item initially called infantile Refsum
\end{itemize}
\end{itemize}
\end{frame}

\begin{frame}[label={sec:orgaab0fe6}]{Treatment}
\begin{itemize}
\item Dietary restriction of phytanic acid 
\begin{itemize}
\item critical to minimize ongoing tissue accumulation.
\end{itemize}
\item The largest sources of phytanic acid and its metabolic precursor phytol are:
\begin{itemize}
\item dairy products, meats and certain fish
\end{itemize}
\item vegetables do not need to be restricted
\begin{itemize}
\item phytanic acid is not released from chlorophyll
\end{itemize}
\item avoid rapid weight loss
\begin{itemize}
\item may mobilize phytanic acid from adipose tissue
\end{itemize}
\item Can halt progression of symptoms and some functional recovery if the
disease is recognized early and dietary restriction and regular
lipid apheresis are maintained life-long.
\end{itemize}
\end{frame}

\section{Sj\"{o}gren Larsson Syndrome}
\label{sec:org73c6c2f}
\begin{frame}[label={sec:org31f06ec}]{Clinical Presentation}
\begin{itemize}
\item Classical tetrad of abnormalities in SLS includes:
\begin{enumerate}
\item ichthyosis
\item spasticity
\item ophthalmological abnormalities
\item intellectual disability
\end{enumerate}
\item full-blown phenotype of SLS is not observed in all patients
\item manifests later on in childhood \textgreater{} 3 years of age.
\end{itemize}
\end{frame}

\begin{frame}[label={sec:org8a926f6}]{Metabolic Derangement}
\begin{itemize}
\item Enzyme deficient in SLS is fatty aldehyde dehydrogenase (FALDH)
\item degradation of long-chain fatty alcohols and leukotriene B4
\end{itemize}
\end{frame}

\begin{frame}[label={sec:orgc97bd4e}]{Genetics}
\begin{itemize}
\item SLS is an autosomal recessive disorder caused by mutations in
ALD-H3A2
\item a range of different mutations including missense, nonsense,
splice-site and deletions has been reported.
\end{itemize}
\end{frame}

\begin{frame}[label={sec:org49848e5}]{Diagnostic Tests}
\begin{itemize}
\item \(\uparrow\) long-chain fatty alcohols in plasma
\item \(\uparrow\) LTB4 metabolites in urine.
\begin{itemize}
\item No easy methods have been described to measure these metabolites
\end{itemize}
\item enzymatic analysis is the method of choice
\begin{itemize}
\item can be done in polymorphonuclear lymphocytes using pyrenedecanal as substrate
\item identification of FALDH-deficiency in candidate patients.
\end{itemize}
\end{itemize}
\end{frame}

\begin{frame}[label={sec:orgc889018}]{Treatment}
\begin{itemize}
\item Treatment of SLS patients is focused on the spasticity and prevention of contracture development.
\item One of the key problems in SLS patients is the striking pruritus
\begin{itemize}
\item may originate from LTB4 accumulation.
\end{itemize}
\item Zileuton, inhibits leukotriene formation by blocking its biosynthesis
\begin{itemize}
\item effective in managing chronic (severe) asthma.
\end{itemize}
\item improvement of pruritus
\begin{itemize}
\item \(\downarrow\) urinary LTB4
\item \(\downarrow\) lipid peak on MRS.
\end{itemize}
\item A double-blind placebo controlled trial is currently underway
\end{itemize}
\end{frame}
\end{document}
% Created 2018-04-25 Wed 10:32
\documentclass[presentation, smaller]{beamer}
\usepackage[utf8]{inputenc}
\usepackage[T1]{fontenc}
\usepackage{fixltx2e}
\usepackage{graphicx}
\usepackage{grffile}
\usepackage{longtable}
\usepackage{wrapfig}
\usepackage{rotating}
\usepackage[normalem]{ulem}
\usepackage{amsmath}
\usepackage{textcomp}
\usepackage{amssymb}
\usepackage{capt-of}
\usepackage{hyperref}
\hypersetup{colorlinks,linkcolor=gray,urlcolor=blue}
\usepackage{textpos}
\usepackage{textgreek}
\usepackage[version=4]{mhchem}
\usepackage{chemfig}
\usepackage{siunitx}
\usepackage{gensymb}
\usepackage[usenames,dvipsnames]{xcolor}
\usepackage[T1]{fontenc}
\usepackage{lmodern}
\usepackage{verbatim}
\usepackage{tikz}
\usetikzlibrary{shapes.geometric,arrows,decorations.pathmorphing,backgrounds,positioning,fit,petri}
\AtBeginSection[]{\begin{frame}\tableofcontents[currentsection] \end{frame}}
\author{Matthew Henderson, PhD, FCACB}
\date{\today}
\title{Multiple Carboxylase Deficiency}
\hypersetup{
 pdfauthor={Matthew Henderson, PhD, FCACB},
 pdftitle={Multiple Carboxylase Deficiency},
 pdfkeywords={},
 pdfsubject={},
 pdfcreator={Emacs 25.2.1 (Org mode 8.3.4)}, 
 pdflang={English}}
\begin{document}

\maketitle
\tableofcontents


\logo{\includegraphics[width=1cm,height=1cm,keepaspectratio]{../logos/NSO_logo_small.pdf}}

\vspace{220pt}
\beamertemplatenavigationsymbolsempty
\setbeamertemplate{caption}[numbered]
\setbeamerfont{caption}{size=\tiny}
% \addtobeamertemplate{frametitle}{}{%
% \begin{textblock*}{100mm}(.85\textwidth,-1cm)
% \includegraphics[height=1cm,width=2cm]{cat}
% \end{textblock*}}


\tikzstyle{chemical} = [rectangle, rounded corners, text width=5em, minimum height=1em,text centered, draw=black, fill=none]
\tikzstyle{hardware} = [rectangle, rounded corners, text width=5em, minimum height=1em,text centered, draw=black, fill=gray!30]
\tikzstyle{ms} = [rectangle, rounded corners, text width=5em, minimum height=1em,text centered, draw=orange, fill=none]
\tikzstyle{msw} = [rectangle, rounded corners, text width=7em, minimum height=1em,text centered, draw=orange, fill=none]
\tikzstyle{label} = [rectangle,text width=8em, minimum height=1em, text centered, draw=none, fill=none]
\tikzstyle{hl} = [rectangle, rounded corners, text width=5em, minimum height=1em,text centered, draw=black, fill=red!30]
\tikzstyle{box} = [rectangle, rounded corners, text width=5em, minimum height=5em,text centered, draw=black, fill=none]
\tikzstyle{arrow} = [thick,->,>=stealth]
\tikzstyle{hl-arrow} = [ultra thick,->,>=stealth,draw=red]

\section{Introduction}
\label{sec:orgheadline9}

\subsection{Multiple Carboxylase Deficiency}
\label{sec:orgheadline1}
\begin{itemize}
\item Four carboxylases:
\begin{itemize}
\item pyruvate carboxylase
\item acetyl-CoA carboxylase
\item propionyl carboxylase
\item 3-methylcrotonyl carboxylase
\end{itemize}
\end{itemize}

\subsection{Carboxylases}
\label{sec:orgheadline2}


\includegraphics[width=.9\linewidth]{./figures/carboxylases.png}

\subsection{Causes of Multiple Carboxylase Deficiency}
\label{sec:orgheadline3}
\begin{itemize}
\item Biotin Deficiency
\item Holocarboxylase Synthetase Deficiency
\item Biotinidase Deficiency
\end{itemize}

\subsection{Holocarboxylase Synthetase}
\label{sec:orgheadline4}
\begin{itemize}
\item HCS activates biotin to D-biotin-5'-adenylate
\item catalyzes attachment to an apocarboxylase
\begin{itemize}
\item lysine \(\epsilon\)-amino group
\end{itemize}
\end{itemize}

\includegraphics[width=.9\linewidth]{./figures/hcs.png}

\subsection{Holocarboxylase Synthetase Deficiency}
\label{sec:orgheadline5}
\begin{itemize}
\item Increased Km for Biotin
\begin{itemize}
\item Normally 1-6 nmol/L, patients 9-12 nmol/L
\end{itemize}
\end{itemize}

\begin{center}
\includegraphics[width=0.7\textwidth]{./figures/kinetics.png}
\end{center}
\subsection{Biotinidase}
\label{sec:orgheadline6}

\includegraphics[width=.9\linewidth]{./figures/biot.png}

\subsection{Biotin and HCS}
\label{sec:orgheadline7}

\includegraphics[width=.9\linewidth]{./figures/biotHCS.png}


\subsection{Treatment}
\label{sec:orgheadline8}
\begin{itemize}
\item All symptoms and biochemical abnormalities treated with biotin.
\begin{itemize}
\item Except optic and auditory nerve atrophy
\item Biocytin?
\end{itemize}
\end{itemize}
\end{document}

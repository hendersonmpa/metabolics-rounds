% Created 2018-05-22 Tue 16:09
\documentclass[presentation, smaller]{beamer}
\usepackage[utf8]{inputenc}
\usepackage[T1]{fontenc}
\usepackage{fixltx2e}
\usepackage{graphicx}
\usepackage{grffile}
\usepackage{longtable}
\usepackage{wrapfig}
\usepackage{rotating}
\usepackage[normalem]{ulem}
\usepackage{amsmath}
\usepackage{textcomp}
\usepackage{amssymb}
\usepackage{capt-of}
\usepackage{hyperref}
\hypersetup{colorlinks,linkcolor=gray,urlcolor=blue}
\usepackage{textpos}
\usepackage{textgreek}
\usepackage[version=4]{mhchem}
\usepackage{chemfig}
\usepackage{siunitx}
\usepackage{gensymb}
\usepackage[usenames,dvipsnames]{xcolor}
\usepackage[T1]{fontenc}
\usepackage{lmodern}
\usepackage{verbatim}
\usepackage{tikz}
\usetikzlibrary{shapes.geometric,arrows,decorations.pathmorphing,backgrounds,positioning,fit,petri}
\usetheme[height=20pt]{Boadilla}
\usecolortheme[RGB={170,160,80}]{{structure}}
\author{Matthew Henderson, PhD, FCACB}
\date{\today}
\title{Fasting Urine Organic Acids}
\institute[NSO]{Newborn Screening Ontario}
\titlegraphic{\includegraphics[height=1cm,keepaspectratio]{../logos/NSO_logo.pdf}\includegraphics[height=1cm,keepaspectratio]{../logos/cheo-logo.png} \includegraphics[height=1cm,keepaspectratio]{../logos/UOlogoBW.eps}}
\hypersetup{
 pdfauthor={Matthew Henderson, PhD, FCACB},
 pdftitle={Fasting Urine Organic Acids},
 pdfkeywords={},
 pdfsubject={},
 pdfcreator={Emacs 25.2.1 (Org mode 8.3.4)}, 
 pdflang={English}}
\begin{document}

\maketitle

\logo{\includegraphics[width=1cm,height=1cm,keepaspectratio]{../logos/NSO_logo_small.pdf}}

\vspace{220pt}
\beamertemplatenavigationsymbolsempty
\setbeamertemplate{caption}[numbered]
\setbeamerfont{caption}{size=\tiny}
% \addtobeamertemplate{frametitle}{}{%
% \begin{textblock*}{100mm}(.85\textwidth,-1cm)
% \includegraphics[height=1cm,width=2cm]{cat}
% \end{textblock*}}


\tikzstyle{chemical} = [rectangle, rounded corners, text width=5em, minimum height=1em,text centered, draw=black, fill=none]
\tikzstyle{hardware} = [rectangle, rounded corners, text width=5em, minimum height=1em,text centered, draw=black, fill=gray!30]
\tikzstyle{ms} = [rectangle, rounded corners, text width=5em, minimum height=1em,text centered, draw=orange, fill=none]
\tikzstyle{msw} = [rectangle, rounded corners, text width=7em, minimum height=1em,text centered, draw=orange, fill=none]
\tikzstyle{label} = [rectangle,text width=8em, minimum height=1em, text centered, draw=none, fill=none]
\tikzstyle{hl} = [rectangle, rounded corners, text width=5em, minimum height=1em,text centered, draw=black, fill=red!30]
\tikzstyle{box} = [rectangle, rounded corners, text width=5em, minimum height=5em,text centered, draw=black, fill=none]
\tikzstyle{arrow} = [thick,->,>=stealth]
\tikzstyle{hl-arrow} = [ultra thick,->,>=stealth,draw=red]

\section{Background}
\label{sec:orgheadline10}
\begin{frame}[label={sec:orgheadline1}]{Critical Sample}
\begin{itemize}
\item Abnormal urinary organic acid profiles of patients with FAOD may
occur only during acute episodes of hypoglycemia and/or fasting.

\begin{itemize}
\item plasma concentration of fatty acids increases, leading to an
elevated urinary excretion of medium-chain dicarboxylic acids.
\end{itemize}

\item Clinical notes refer to "Ketotic Hypoglycemia"

\item These medium-chain saturated metabolites usually include adipic
(DC6), suberic (DC8), and sebacic (DC10) acids?
\end{itemize}
\end{frame}

\begin{frame}[label={sec:orgheadline2}]{Fasting Urine Organic Acids}
\begin{itemize}
\item Many sample are sent from the ED or ICU in cases or prolonged
fasting, vomiting, dehydration.
\begin{itemize}
\item Lactate
\item urea
\item Ketones
\item BCAA intermediate metabolites
\item dicarboxylic acids
\item NSAIDs
\end{itemize}
\end{itemize}
\end{frame}

\begin{frame}[label={sec:orgheadline3}]{Glucose Sources with Fasting}
\includegraphics[width=.9\linewidth]{./figures/fasting_graph1.pdf}

\tiny
courtesy of SIMD-NAMA
\end{frame}

\begin{frame}[label={sec:orgheadline4}]{Energy Sources with Fasting}
\includegraphics[width=.9\linewidth]{./figures/fasting_graph2.pdf}

\tiny
courtesy of SIMD-NAMA
\end{frame}

\begin{frame}[label={sec:orgheadline5}]{Ketonuria}
\includegraphics[width=.9\linewidth]{./figures/ketones.png}

\begin{itemize}
\item Ketone bodies
\begin{itemize}
\item \(\beta\)-hydroxybutyric acid
\item acetoacetic acid
\end{itemize}
\end{itemize}
\end{frame}


\begin{frame}[label={sec:orgheadline6}]{Ketotic Hypoglycemia in IMD}
\begin{itemize}
\item Other conditions which can present with ketotic hypoglycaemia are:
\begin{itemize}
\item FAOD: VLCADD, MCADD 
\begin{itemize}
\item inappropriately low ketones
\end{itemize}
\end{itemize}
\item intermediate form of MSUD
\item organic acidaemia, particularly methylmalonic and isovaleric acidaemia
\item multiple carboxylase deficiency
\item defects of gluconeogenesis, such as glycogen storage disease type
1a where hepatomegaly is present
\end{itemize}
\end{frame}

\begin{frame}[label={sec:orgheadline7}]{Dicarboxylic Aciduria}
\begin{block}{Ketotic Dicarboxylic Aciduria.}
\begin{itemize}
\item Physiological response to hypoglycemia
\begin{itemize}
\item DC6 > DC8 > DC10
\end{itemize}
\end{itemize}
\end{block}

\begin{block}{Non-Ketotic Dicarboxylic Aciduria.}
\begin{itemize}
\item FAOD:
\begin{itemize}
\item DC6 > DC8 > DC10
\end{itemize}
\item MCAD + MCT
\begin{itemize}
\item DC10 > DC8 > DC6
\item or DC8 > DC10 > DC6 in some cases.
\end{itemize}

\item All fatty acid oxidation defects will permit production of ketones.
\end{itemize}
\end{block}
\end{frame}

\begin{frame}[label={sec:orgheadline8}]{BCAA metabolites}
\begin{itemize}
\item 2-ketoisocaproic
\item 3-keto-3-methylvalerate
\begin{itemize}
\item 2-methyl-3-hydroxybutyric
\end{itemize}
\item 2-ketoisovaleric
\begin{itemize}
\item 3-hydroxyisobutyric
\end{itemize}
\end{itemize}
\end{frame}



\begin{frame}[label={sec:orgheadline9}]{Dehydration}
\begin{itemize}
\item Urine Creatinine > 4,000 umol/L
\item 3-hydroxypropionic
\item 3-hydroxyisovaleric
\item 3-methylglutaconic
\end{itemize}
\end{frame}
\end{document}

% Created 2017-12-14 Thu 12:55
\documentclass[presentation, smaller]{beamer}
\usepackage[utf8]{inputenc}
\usepackage[T1]{fontenc}
\usepackage{fixltx2e}
\usepackage{graphicx}
\usepackage{grffile}
\usepackage{longtable}
\usepackage{wrapfig}
\usepackage{rotating}
\usepackage[normalem]{ulem}
\usepackage{amsmath}
\usepackage{textcomp}
\usepackage{amssymb}
\usepackage{capt-of}
\usepackage{hyperref}
\hypersetup{colorlinks,linkcolor=white,urlcolor=blue}
\usepackage{textpos}
\usepackage{textgreek}
\usepackage[version=4]{mhchem}
\usepackage{chemfig}
\usepackage{siunitx}
\usepackage[usenames,dvipsnames]{xcolor}
\usepackage[T1]{fontenc}
\usepackage{lmodern}
\usepackage{verbatim}
\usepackage{tikz}
\usetikzlibrary{shapes.geometric,arrows,decorations.pathmorphing,backgrounds,positioning,fit,petri}
\usetheme{Hannover}
\usecolortheme{whale}
\author{Matthew Henderson, PhD, FCACB}
\date{\today}
\title{Abnormal Routine Biochemistry Results in Mitochondrial FAO Defects}
\institute[NSO]{Newborn Screening Ontario | The University of Ottawa}
\titlegraphic{\includegraphics[height=1cm,keepaspectratio]{../logos/NSO_logo.pdf}\includegraphics[height=1cm,keepaspectratio]{../logos/cheo-logo.png} \includegraphics[height=1cm,keepaspectratio]{../logos/UOlogoBW.eps}}
\hypersetup{
 pdfauthor={Matthew Henderson, PhD, FCACB},
 pdftitle={Abnormal Routine Biochemistry Results in Mitochondrial FAO Defects},
 pdfkeywords={},
 pdfsubject={},
 pdfcreator={Emacs 25.2.1 (Org mode 8.3.4)}, 
 pdflang={English}}
\begin{document}

\maketitle
%\logo{\includegraphics[width=1cm,height=1cm,keepaspectratio]{../logos/NSO_logo_small.pdf}~%
%    \includegraphics[width=1cm,height=1cm,keepaspectratio]{../logos/UOlogoBW.eps}%
%}

\vspace{220pt}
\beamertemplatenavigationsymbolsempty
\setbeamertemplate{caption}[numbered]
\setbeamerfont{caption}{size=\tiny}
% \addtobeamertemplate{frametitle}{}{%
% \begin{textblock*}{100mm}(.85\textwidth,-1cm)
% \includegraphics[height=1cm,width=2cm]{cat}
% \end{textblock*}}

\tikzstyle{chemical} = [rectangle, rounded corners, text width=5em, minimum height=1em,text centered, draw=black, fill=none]
\tikzstyle{hardware} = [rectangle, rounded corners, text width=5em, minimum height=1em,text centered, draw=black, fill=gray!30]
\tikzstyle{ms} = [rectangle, rounded corners, text width=5em, minimum height=1em,text centered, draw=orange, fill=none]
\tikzstyle{msw} = [rectangle, rounded corners, text width=7em, minimum height=1em,text centered, draw=orange, fill=none]
\tikzstyle{label} = [rectangle,text width=5em, minimum height=1em, text centered, draw=none, fill=none]
\tikzstyle{hl} = [rectangle, rounded corners, text width=5em, minimum height=1em,text centered, draw=black, fill=red!30]
\tikzstyle{arrow} = [thick,->,>=stealth]
\tikzstyle{hl-arrow} = [ultra thick,->,>=stealth,draw=red]

\section{Follow-up}
\label{sec:orgheadline8}
\begin{frame}[label={sec:orgheadline1}]{Follow-up}
\begin{itemize}
\item Address a few outstanding questions from previous FAO talks
\end{itemize}
\end{frame}

\begin{frame}[label={sec:orgheadline2}]{Carnitine Transport Defect}
\begin{itemize}
\item Organic cation/carnitine transporter(OCTN2) responsible for
carnitine uptake across the plasma membrane
\begin{itemize}
\item heart, muscle and kidney.
\end{itemize}

\item \textbf{What about liver?}

\begin{itemize}
\item Analysis of carnitine transport in different tissues suggests the
presence of heterogeneous transporters.\footnote{Scaglia, F., Wang, Y., \& Longo, N. (1999). Functional
characterization of the carnitine transporter defective in primary
carnitine deficiency. Archives of Biochemistry and Biophysics,
364(1), 99–106.}
\item Liver and brain have a low-affinity (Km 2–10 uM), high-capacity transporter
\item Fibroblast, muscle, and heart cells have a high-affinity (Km 5–10 uM), low-capacity system.
\end{itemize}
\end{itemize}
\end{frame}

\begin{frame}[label={sec:orgheadline3}]{Models for Fatty Acid Transport Across Cell Membranes}
\includegraphics[width=.9\linewidth]{./figures/FA_transport.jpg}

\tiny
\begin{itemize}
\item CD35 = FAT
\end{itemize}
\end{frame}
\begin{frame}[label={sec:orgheadline4}]{Oxidation of Medium-Chain Length Fatty Acids}
\begin{itemize}
\item \(\uparrow\) solubility
\item \(\to\) mito matrix via the monocarboxylate transporter
\item \textbf{Are there any reports of disease causing MCT mutations?}
\end{itemize}
\end{frame}

\begin{frame}[label={sec:orgheadline5}]{Oxidation of Medium-Chain Length Fatty Acids}
\begin{itemize}
\item "There is a general consensus that short-chain and medium-chain fatty
acids (C4 to C12) diffuse freely across plasma and mitochondrial
membranes" \footnote{Odaib, A. A., Shneider, B. L., Bennett, M. J., Pober, B. R.,
Reyes-Mugica, M., Friedman, A. L., … Rinaldo, P. (1998). A defect in
the transport of long-chain fatty acids associated with acute liver
failure. N Engl J Med, 339(24), 1752–1757.}

\item Butyrate is taken up by enterocytes, presumably by means of the
monocarboxylate transporter 1 (MCT-1) and the sodium-coupled
monocarboxylate transporter 1 (SMCT-1) \footnote{Butyrate is used by these cells mostly as fuel.  Schönfeld, P., \&
Wojtczak, L. (2016). Short- and medium-chain fatty acids in energy
metabolism: the cellular perspective. Journal of Lipid Research,
57(6), 943–954.}
\end{itemize}
\end{frame}


\begin{frame}[label={sec:orgheadline6}]{Monocarboxylate Transporter 1}
\begin{itemize}
\item Out of 14 known mammalian MCTs, six isoforms have been functionally
characterized to transport monocarboxylates and short chain fatty
acids (MCT1-4), thyroid hormones (MCT8-10) and aromatic amino
acids (MCT10)

\item MCT1 mediates the movement of lactate and pyruvate across cell
membranes.
\begin{itemize}
\item erythrocytes, muscle, intestine, liver and kidney
\end{itemize}
\end{itemize}

\begin{center}
\begin{tabular}{ll}
Phenotype & Inheritance\\
\hline
Erythrocyte lactate transporter defect & AD\\
Hyperinsulinemic hypoglycemia, familial, 7\footnotemark & AD\\
Monocarboxylate transporter 1 deficiency & AR, AD\\
 & \\
\end{tabular}
\end{center}\footnotetext[4]{promoter-activating mutations in patients with hyperinsulinemic
hypoglycemia induce SLC16A1 expression in beta cells, where this
gene is not usually transcribed, permitting pyruvate uptake and
pyruvate-stimulated insulin release despite ensuing hypoglycemia}
\end{frame}

\begin{frame}[label={sec:orgheadline7}]{ACAD9 in VLCADD}
\begin{itemize}
\item ACAD9 is responsible for production of C14:1 in VLCADD \footnote{Nouws, J., Te brinke, H., Nijtmans, L. G., \& Houten,
S. M. (2014). ACAD9, a complex i assembly factor with a moonlighting
function in fatty acid oxidation deficiencies. Human Molecular
Genetics, 23(5), 1311–1319.}
\item VLCAD\(^{\text{-/-}}\) cell lines accumulate C14:1
\begin{itemize}
\item VLCAD\(^{\text{-/-}}\)/ACAD9\(^{\text{-/-}}\) cell lines accumulate C18:1
\end{itemize}
\end{itemize}
\end{frame}

\section{Routine Biochemistry}
\label{sec:orgheadline21}
\begin{frame}[label={sec:orgheadline9}]{Routine Biochemistry}
\begin{itemize}
\item Glucose
\item Ketones
\item Lactate
\item Ammonia
\item Uric Acid
\item Creatine Kinase (CK)
\item Acute Fatty Liver of Pregnancy (AFLP) \& Hemolysis, Elevated Liver enzymes Low Platlets (HELLP)
\begin{itemize}
\item Mothers who are heterozygous for LCHAD or MTP deficiency when
carrying an affected fetus
\end{itemize}
\end{itemize}
\end{frame}
\begin{frame}[label={sec:orgheadline10}]{Fasting Hypoglycaemia  in FAODs}
\begin{itemize}
\item Fasting hypoglycaemia is the classic metabolic disturbance in FAODs
\begin{itemize}
\item primarily due to increased peripheral glucose consumption
\end{itemize}
\begin{itemize}
\item The hypoglycaemia is hypoketotic.
\begin{itemize}
\item Ketone bodies can be synthesised (medium-or short-chain FAODs or
if there is high residual enzyme activity)
\item plasma concentrations are lower than expected for hypoglycaemia or
the plasma free fatty acid concentrations.
\end{itemize}
\end{itemize}
\end{itemize}
\end{frame}

\begin{frame}[label={sec:orgheadline11}]{Randle Cycle \footnote{Hue, L., Taegtmeyer, H., Randle, P., Garland, P., \& Hales,
N. (2009). The Randle cycle revisited. Am J Physiol Endocrinol Metab,
297, 578–591.}}
\begin{itemize}
\item Fasted state:
\begin{itemize}
\item Fatty Acids inhibit glucose oxidation at pyruvate dehydrogenase
(PDH)
\item Long-chain acyl-CoA derivatives directly inhibit glucokinase
\end{itemize}
\item Fed state:
\begin{itemize}
\item Insulin inhibits lipolysis
\item Malonyl-CoA inhibits CPT1
\begin{itemize}
\item \(\uparrow\) esterification of FA
\end{itemize}
\end{itemize}
\end{itemize}
\end{frame}

\begin{frame}[label={sec:orgheadline12}]{Randle Cycle}
\includegraphics[width=.9\linewidth]{./figures/randle.png}
\end{frame}

\begin{frame}[label={sec:orgheadline13}]{Inhibition of Glucose Utilization}
\includegraphics[width=.9\linewidth]{./figures/glucose_oxidation_inhibition.png}
\end{frame}

\begin{frame}[label={sec:orgheadline14}]{Inhibition of Fatty Acid Utilization}
\includegraphics[width=.9\linewidth]{./figures/FAO_inhibition.png}
\end{frame}

\begin{frame}[label={sec:orgheadline15}]{Lactic Acidemia in FAODs}
\begin{itemize}
\item Lactic acidaemia is seen in long-chain FAODs (VLCAD, LCHAD and MTP deficiencies)
\item Long-chain acyl-CoA esters have been found to inhibit a large
variety of enzymes including the mitochondrial ATP/ADP carrier.

\item \emph{In vivo} the ATP/ADP carrier catalyses the transport of ATP
synthesized in the mitochondrion by the F\(_{\text{1}}\) F\(_{\text{0}}\)-ATPase reaction to the
extra-mitochondrial space in exchange for cytosolic ADP
\begin{itemize}
\item \(\uparrow\) intra-mitochondrial ATP/ADP ratio and \(\downarrow\) oxidation of
NADH \(\to\) NAD\(^{\text{+}}\)
\end{itemize}
\item \(\uparrow\) ATP/ADP and NADH/NAD\(^{\text{+}}\) ratios will activate PDK
\begin{itemize}
\item PDK inactivates PDH by phosphorylation.
\end{itemize}
\end{itemize}
\end{frame}

\begin{frame}[label={sec:orgheadline16}]{Hyperammonaemia in FAODs}
\begin{itemize}
\item Hyperammonaemia occurs in some severe defects,
\begin{itemize}
\item with normal or low glutamine concentrations;
\item decreased acetyl-CoA production reducing the synthesis of N-acetylglutamate \footnote{Haberle, J. (2011). Clinical practice: The management of
hyperammonemia. European Journal of Pediatrics, 170(1), 21–34.}
\end{itemize}
\end{itemize}
\end{frame}

\begin{frame}[label={sec:orgheadline17}]{Hyperammonaemia in FAODs}
\includegraphics[width=.9\linewidth]{./figures/2nd_ammonemia.png}

\tiny
Influence of metabolic disorders on function of urea cycle leading to secondary hyperammonemia
\end{frame}

\begin{frame}[label={sec:orgheadline18}]{Moderate hyperuricaemia , elevated CK}
\begin{itemize}
\item Seen in acute attacks 
\begin{itemize}
\item The association suggests that the mechanism is a breakdown of
cells, particularly muscle.
\item ATP depletion \(\to\) \(\downarrow\) Na/K-ATPase
\end{itemize}
\item Release of CK
\item The uric acid excess is the product of nucleic acid and nucleotide catabolism.
\end{itemize}
\end{frame}

\begin{frame}[label={sec:orgheadline19}]{AFLP/HELLP}
\begin{itemize}
\item Mothers who are heterozygous for LCHAD or MTP deficiency when
carrying an affected fetus
\item A woman whose affected fetus has the Glu474Gln mutation on one or
both alleles of the \(\alpha\)-subunit of the trifunctional protein is
likely to have acute fatty liver of pregnancy or the HELLP syndrome \footnote{Odaib, A. A., Shneider, B. L., Bennett, M. J., Pober, B. R.,
Reyes-Mugica, M., Friedman, A. L., … Rinaldo, P. (1998). A defect in
the transport of long-chain fatty acids associated with acute liver
failure. N Engl J Med, 339(24), 1752–1757.}
\begin{itemize}
\item Long-chain 3-hydroxyacyl metabolites produced by the fetus or
placenta accumulate in the mother and are highly toxic to the liver
\item Exaggerated by the decreased metabolic utilization of fatty acids
during pregnancy.
\end{itemize}
\end{itemize}
\end{frame}



\begin{frame}[label={sec:orgheadline20}]{Next time}
\begin{itemize}
\item Diagnostic Testing for FAODs
\begin{itemize}
\item Biochemical
\begin{itemize}
\item LC-MS/MS
\item enzymatic
\item cell based assays
\end{itemize}
\item Molecular
\end{itemize}

\item Methods for Quantitation of Acylcarnitines
\begin{itemize}
\item LC-MS/MS
\item FIA-MS/MS
\end{itemize}
\end{itemize}
\end{frame}
\end{document}

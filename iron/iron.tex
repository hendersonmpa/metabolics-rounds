% Created 2019-06-13 Thu 12:39
% Intended LaTeX compiler: pdflatex
\documentclass[presentation, smaller]{beamer}
\usepackage[utf8]{inputenc}
\usepackage[T1]{fontenc}
\usepackage{graphicx}
\usepackage{grffile}
\usepackage{longtable}
\usepackage{wrapfig}
\usepackage{rotating}
\usepackage[normalem]{ulem}
\usepackage{amsmath}
\usepackage{textcomp}
\usepackage{amssymb}
\usepackage{capt-of}
\usepackage{hyperref}
\hypersetup{colorlinks,linkcolor=white,urlcolor=blue}
\usepackage{textpos}
\usepackage{textgreek}
\usepackage[version=4]{mhchem}
\usepackage{chemfig}
\usepackage{siunitx}
\usepackage{gensymb}
\usepackage[usenames,dvipsnames]{xcolor}
\usepackage[T1]{fontenc}
\usepackage{lmodern}
\usepackage{verbatim}
\usepackage{tikz}
\usepackage{wasysym}
\usetikzlibrary{shapes.geometric,arrows,decorations.pathmorphing,backgrounds,positioning,fit,petri}
\usetheme{Hannover}
\usecolortheme{whale}
\author{Matthew Henderson, PhD, FCACB}
\date{\today}
\title{Iron Transport Disorders}
\institute[NSO]{Newborn Screening Ontario | The University of Ottawa}
\titlegraphic{\includegraphics[height=1cm,keepaspectratio]{../logos/NSO_logo.pdf}\includegraphics[height=1cm,keepaspectratio]{../logos/cheo-logo.png} \includegraphics[height=1cm,keepaspectratio]{../logos/UOlogoBW.eps}}
\hypersetup{
 pdfauthor={Matthew Henderson, PhD, FCACB},
 pdftitle={Iron Transport Disorders},
 pdfkeywords={},
 pdfsubject={},
 pdfcreator={Emacs 26.1 (Org mode 9.1.9)}, 
 pdflang={English}}
\begin{document}

\maketitle

%\logo{\includegraphics[width=1cm,height=1cm,keepaspectratio]{../logos/NSO_logo_small.pdf}~%
%    \includegraphics[width=1cm,height=1cm,keepaspectratio]{../logos/UOlogoBW.eps}%
%}

\vspace{220pt}
\beamertemplatenavigationsymbolsempty
\setbeamertemplate{caption}[numbered]
\setbeamerfont{caption}{size=\tiny}
% \addtobeamertemplate{frametitle}{}{%
% \begin{textblock*}{100mm}(.85\textwidth,-1cm)
% \includegraphics[height=1cm,width=2cm]{cat}
% \end{textblock*}}

\section{Introduction}
\label{sec:orgfa67745}
\begin{frame}[label={sec:org5656c8b}]{Introduction}
\begin{itemize}
\item Iron is essential for the synthesis of haem and other
metalloproteins.
\item Among these metalloproteins, the iron sulfur protein cluster is very important
\begin{itemize}
\item crucial role in mitochondrial metabolism
\end{itemize}
\item \textgreater{} 20 mg of iron per day is required, only 1–2 mg from intestinal absorption,
\begin{itemize}
\item the remainder is re-used
\end{itemize}
\item Not actively secreted from the body
\item \(\uparrow\) [iron] \(\to\) \(\uparrow\) [circulating free iron]
\begin{itemize}
\item primarily taken up by the liver, the pancreas and the heart.
\item syndromes manifest with cirrhosis, diabetes and cardiomyopathy.
\end{itemize}
\end{itemize}
\end{frame}

\begin{frame}[label={sec:org82411e3}]{Introduction}
\begin{itemize}
\item Absorption of iron occurs primarily in the duodenum via the
divalent-metal transporter(DMT1)
\item The major recycling route for iron is removal from
erythrocytic haem by haemoxygenase
\begin{itemize}
\item in macrophages and enterocytes.
\end{itemize}
\item Circulating free iron binds to to apo-transferrin forming holo-transferrin
\item Transferrin can only bind iron in the ferric state
\item Ceruloplasmin catalyses the oxidization of Fe\(^{\text{2+}}\) into Fe\(^{\text{3+}}\)
\item The transferrin receptor mediates the uptake of transferrin
\item Iron is released from intracellular transferrin by a specific isoform of DMT1.
\item In several cell types, including macrophages, iron can be stored
bound to ferritin until needed.
\end{itemize}
\end{frame}

\begin{frame}[label={sec:orgb0857cf}]{Iron Metabolism}
\begin{figure}[htbp]
\centering
\includegraphics[width=0.9\textwidth]{./figures/iron_met.png}
\caption[iron]{\label{fig:org9dd472d}
Iron Metabolism}
\end{figure}
\end{frame}

\begin{frame}[label={sec:org5bb9047}]{Iron Sulfur Cluster Proteins}
\begin{figure}[htbp]
\centering
\includegraphics[width=0.9\textwidth]{./figures/fes.png}
\caption[fes]{\label{fig:orge50a343}
Iron Sulfur Cluster Proteins}
\end{figure}
\end{frame}


\section{Classic Hereditary Haemochromatosis}
\label{sec:org8c979f2}

\begin{frame}[label={sec:orga2fb1c5}]{Classic Hereditary Haemochromatosis: Clinical Presentation}
\begin{itemize}
\item Also called Type 1 or HFE related haemochromatosis
\item An autosomal recessive disorder
\item Slow but progressive accumulation of iron in various organs
\item Clinically apparent by the fourth or fifth decade of life
\item Initial symptoms are nonspecific and include:
\begin{itemize}
\item fatigue, weakness, abdominal pain, weight loss and arthralgia.
\end{itemize}
\item Increased awareness, and improved diagnostic testing
\begin{itemize}
\item Classic symptoms of full-blown haemochromatosis are rarely seen
\begin{itemize}
\item liver fibrosis and cirrhosis, diabetes, cardiomyopathy,
hypogonadotrophic hypogonadism, arthropathy and skin
pigmentation
\end{itemize}
\end{itemize}
\end{itemize}
\end{frame}


\begin{frame}[label={sec:org15a3c4d}]{Classic Hereditary Haemochromatosis: Metabolic Derangement}
\begin{itemize}
\item Caused by a disturbance in iron homeostasis associated with hepcidin
deficiency and systemic accumulation of iron.
\item The exact role of HFE, is unclear at present.
\item Most probably it is essential for sensing iron levels and thus
indirectly for regulating hepcidin synthesis.
\end{itemize}
\end{frame}

\begin{frame}[label={sec:org1623de9}]{Classic Hereditary Haemochromatosis}
\begin{figure}[htbp]
\centering
\includegraphics[width=0.9\textwidth]{./figures/iron_met_HFE.png}
\label{fig:org441e3f2}
\end{figure}
\end{frame}


\begin{frame}[label={sec:org0798f62}]{Classic Hereditary Haemochromatosis: Genetics}
\begin{itemize}
\item As many as 0.5\% of the Northern European population are homozygous
for the C282Y mutation in HFE,

\begin{itemize}
\item only 5\% of male and <1\% of female C282Y homozygotes eventually
develop liver fibrosis or cirrhosis.
\end{itemize}

\item Other mutations in HFE are also described, e.g. H63D,

\begin{itemize}
\item compound heterozygosity for H63D and C282 is associated with iron overload
\end{itemize}
\end{itemize}
\end{frame}

\begin{frame}[label={sec:org84ff796}]{Classic Hereditary Haemochromatosis:Diagnostic Tests}
\begin{itemize}
\item transferrin saturation initially \(\uparrow\)
\item followed by \(\uparrow\) serum ferritin
\begin{itemize}
\item reflects increasing iron overload.
\end{itemize}
\item Genetic testing of HFE should be performed when:
\begin{itemize}
\item transferrin saturation is above 45\%
\item serum ferritin is elevated:
\begin{itemize}
\item >200 ng/ml in adult females
\item >300 ng/ml in adult males
\end{itemize}
\end{itemize}
\end{itemize}
\end{frame}

\begin{frame}[label={sec:org550cff5}]{Classic Hereditary Haemochromatosis: Treatment and Prognosis}
\begin{itemize}
\item At least half of all male and female C282Y homozygotes have normal
serum ferritin levels and may never require therapy.
\item Many have moderately elevated serum ferritin levels  (200-1000 ng/ml)
\begin{itemize}
\item it is unclear at present whether all should have regular
phlebotomies to reduce systemic iron load.
\end{itemize}
\item serum ferritin levels exceeding 1000 ng/ml a phlebotomy regimen is clearly
necessary.
\begin{itemize}
\item In adults initially 500 ml blood is removed weekly or bi-weekly.
\item Phlebotomy frequency is usually reduced to once every 3-6 months
when serum ferritin levels are below 50 ng/ml.
\end{itemize}
\end{itemize}
\end{frame}


\section{Systemic Iron Overload Syndromes}
\label{sec:org925ece4}

\begin{frame}[label={sec:org90c2c31}]{Juvenile Hereditary Haemochromatosis (Type 2)}
\begin{itemize}
\item the most severe type of hereditary haemochromatosis
\begin{itemize}
\item probably because hepcidin deficiency is more pronounced
\end{itemize}
\item Patients present in the 2nd and 3rd decade
\begin{itemize}
\item mostly w hypogonadotropic hypogonadism and cardiomyopathy due to
iron overload.
\end{itemize}
\item Type 2A is caused by mutations in the HJV gene encoding for hemojuvelin
\begin{itemize}
\item necessary for proper hepcidin synthesis
\end{itemize}
\item Type 2B from mutations in the HAMP gene encoding hepcidin.
\item Serum ferritin is high and transferrin iron saturation elevated, as in classic
HFE-related haemochromatosis.
\item A final diagnosis is made by mutation analysis
\item Phlebotomy is the treatment of choice and may prevent organ damage
if initiated early.
\end{itemize}
\end{frame}


\begin{frame}[label={sec:org4833391}]{Juvenile Hereditary Haemochromatosis (Type 2)}
\begin{figure}[htbp]
\centering
\includegraphics[width=0.9\textwidth]{./figures/iron_met_HJV.png}
\label{fig:orgfc022b3}
\end{figure}
\end{frame}



\begin{frame}[label={sec:orgbe3cdd6}]{TFR2-Related Hereditary Haemochromatosis (Type 3)}
\begin{itemize}
\item Transferrin Receptor 2 (TFR2 gene) is important for sensing the
intracellular iron status (e.g erythroid cells)
\item Mutations result in iron overload phenotype which resembles classic, HFE-related haemochromatosis
\begin{itemize}
\item patients are generally younger
\end{itemize}
\item Low hepcidin levels along with elevated transferrin iron saturation,
elevated ferritin and high liver iron content are present.
\item Mutation analysis leads to the correct diagnosis in the absence of
the classic haemochromatosis genotype.
\item Phlebotomy is the treatment of choice.
\end{itemize}
\end{frame}

\begin{frame}[label={sec:org8239d4b}]{TFR2-Related Hereditary Haemochromatosis (Type 3)}
\begin{figure}[htbp]
\centering
\includegraphics[width=0.9\textwidth]{./figures/iron_met_TFR.png}
\label{fig:orgb36d084}
\end{figure}
\end{frame}



\begin{frame}[label={sec:orgbea8f45}]{Ferroportin Related Hereditary Haemochromatosis (Type 4)}
\begin{itemize}
\item Differs in several ways from the other three subtypes of haemochromatosis.
\item AD inheritance and caused by mutations in SLC40A1, encoding ferroportin
\item Expressed at the enterocyte and plasma membrane of macrophages.
\item Loss of function mutations impair the export of iron from macrophages causing an iron
deficiency in erythrocytic precursors.
\item Patients present with a combination of mild microcytic anaemia with
low transferrin saturation
\begin{itemize}
\item iron overload predominantly in macrophages.
\end{itemize}
\item Tolerance to phlebotomy is limited by the concurrent anaemia.
\item In contrast, gain of function mutations cause resistance to feedback
inhibition by hepcidin.
\begin{itemize}
\item These patients present with a more classic hepatic iron overload
haemochromatosis phenotype.
\end{itemize}
\end{itemize}
\end{frame}

\begin{frame}[label={sec:org0da7dc4}]{Ferroportin Related Hereditary Haemochromatosis (Type 4)}
\begin{figure}[htbp]
\centering
\includegraphics[width=0.9\textwidth]{./figures/iron_met_FP.png}
\label{fig:orga70e841}
\end{figure}
\end{frame}


\begin{frame}[label={sec:org563bf33}]{Neonatal Haemochromatosis}
\begin{itemize}
\item Once thought to be an AR inherited disorder, now recognized as acquired
\begin{itemize}
\item Any disease state that chronically prevents the synthesis or
activity of hepcidin will lead to haemochromatosis.
\end{itemize}
\item Patients present in the first few weeks of life with severe liver
failure.
\item Caused by a maternal allo-immune reaction to the infant liver
\begin{itemize}
\item starts \emph{in utero}.
\end{itemize}
\item Liver injury leads to a decrease in hepcidin
\begin{itemize}
\item manifests as severe siderosis of both liver and extrahepatic organs.
\end{itemize}
\item The diagnosis is made in any child with neonatal liver failure in
combination with high serum ferritin and extrahepatic siderosis,
\begin{itemize}
\item shown by MRI and/or oral mucosal biopsy
\item iron deposits in minor salivary glands in patients with NH.
\end{itemize}
\item Therapy is by exchange transfusion in combination with IVIGs to remove/bind maternally derived IgG
\item May a role for simultaneous antioxidant therapy
\item The risk of recurrence in a subsequent pregnancy from a mother who
has given birth to an affected child is as high as 90\%.
\begin{itemize}
\item Recurrence risk reduced by IVIGs during pregnancy
\end{itemize}
\end{itemize}
\end{frame}

\section{Iron Deficiency and Distribution Disorders}
\label{sec:orgcbe1ae7}

\begin{frame}[label={sec:orge22486c}]{Iron-Refractory Iron Deficiency Anaemia (IRIDA)}
\begin{itemize}
\item This disease is caused by a deficiency of matriptase-2, encoded by TMPRSS6.
\item If a mutation in both copies of this gene is present the normal cleavage of haemojuvelin is interrupted,
\begin{itemize}
\item results in high hepcidin levels.
\begin{itemize}
\item This will result in iron deficiency, low transferrin saturation
(<10\%) and microcytic anaemia at a young age [41].
\end{itemize}
\end{itemize}
\item Oral iron supplementation is not effective, as high hepcidin
levels will prevent iron release from the enterocytes
\begin{itemize}
\item requires intravenous iron therapy
\end{itemize}
\end{itemize}
\end{frame}

\begin{frame}[label={sec:org44de9c8}]{Iron-Refractory Iron Deficiency Anaemia (IRIDA)}
\begin{figure}[htbp]
\centering
\includegraphics[width=0.9\textwidth]{./figures/iron_met_IRDA.png}
\label{fig:org5102803}
\end{figure}
\end{frame}

\begin{frame}[label={sec:org225a5c0}]{Atransferrinaemia}
\begin{itemize}
\item First described in 1961, very few cases of atransferrinaemia have thus far been described.
\item AR disorder, mutations in TF,
\item present with moderate to severe hypochromic microcytic anaemia and growth retardation along with signs of haemochromatosis.
\item \downarrown serum transferrin
\item \(\uparrow\) serum ferritin
\item Plasma infusions to increase the transferrin pool, represent an
effective treatment
\end{itemize}
\end{frame}

\begin{frame}[label={sec:orgbae116c}]{Atransferrinaemia}
\begin{figure}[htbp]
\centering
\includegraphics[width=0.9\textwidth]{./figures/iron_met_TR.png}
\label{fig:org99f6424}
\end{figure}
\end{frame}

\begin{frame}[label={sec:org8612c05}]{Hypochromic Microcytic Anaemia with Iron Overload Type 1}
\begin{itemize}
\item Hypochromic microcytic anaemia with iron overload type 1 is caused
by mutations in SLC11A2 , encoding DMT1.
\item One of the isoforms of DMT1 is responsible for removing iron from
absorbed transferrin in erythroid cells.
\item Patients present at a young age with microcytic anaemia in combination with
mild hepatic iron overload.
\item Transferrin saturation and serum ferritin levels are elevated.
\item With erythropoietin (EPO) treatment regular transfusions can often be avoided
\end{itemize}
\end{frame}


\begin{frame}[label={sec:org6f03d0a}]{Hypochromic Microcytic Anaemia with Iron Overload Type 1}
\begin{figure}[htbp]
\centering
\includegraphics[width=0.9\textwidth]{./figures/iron_met_DMT.png}
\label{fig:orgf687d9d}
\end{figure}
\end{frame}


\begin{frame}[label={sec:orgab1c567}]{Hypochromic Microcytic Anaemia with Iron Overload Type 2}
\begin{itemize}
\item This subtype is caused by mutations in STEAP3.
\item STEAP3, is an endosomal ferrireductase which facilitates the
transferrin mediated uptake of iron.
\item In the 3 siblings reported thus far, anaemia was present from early
childhood,
\item While patients became transfusion dependent several years
\item Later, usually in late childhood.
\item High ferritin levels, together with low transferrin and increased
transferrin saturation were found.
\item The degree of liver iron overload varied, all 3 had hypogonadism.
\end{itemize}
\end{frame}
\end{document}
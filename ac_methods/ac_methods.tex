% Created 2018-01-11 Thu 12:45
\documentclass[presentation, smaller]{beamer}
\usepackage[utf8]{inputenc}
\usepackage[T1]{fontenc}
\usepackage{fixltx2e}
\usepackage{graphicx}
\usepackage{grffile}
\usepackage{longtable}
\usepackage{wrapfig}
\usepackage{rotating}
\usepackage[normalem]{ulem}
\usepackage{amsmath}
\usepackage{textcomp}
\usepackage{amssymb}
\usepackage{capt-of}
\usepackage{hyperref}
\hypersetup{colorlinks,linkcolor=white,urlcolor=blue}
\usepackage{textpos}
\usepackage{textgreek}
\usepackage[version=4]{mhchem}
\usepackage{chemfig}
\usepackage{siunitx}
\usepackage{gensymb}
\usepackage[usenames,dvipsnames]{xcolor}
\usepackage[T1]{fontenc}
\usepackage{lmodern}
\usepackage{verbatim}
\usepackage{tikz}
\usetikzlibrary{shapes.geometric,arrows,decorations.pathmorphing,backgrounds,positioning,fit,petri}
\usetheme{Hannover}
\usecolortheme{whale}
\author{Matthew Henderson, PhD, FCACB}
\date{\today}
\title{Laboratory Methods for Quantitative Acylcarnitines in Biological Matrices}
\institute[NSO]{Newborn Screening Ontario | The University of Ottawa}
\titlegraphic{\includegraphics[height=1cm,keepaspectratio]{../logos/NSO_logo.pdf}\includegraphics[height=1cm,keepaspectratio]{../logos/cheo-logo.png} \includegraphics[height=1cm,keepaspectratio]{../logos/UOlogoBW.eps}}
\hypersetup{
 pdfauthor={Matthew Henderson, PhD, FCACB},
 pdftitle={Laboratory Methods for Quantitative Acylcarnitines in Biological Matrices},
 pdfkeywords={},
 pdfsubject={},
 pdfcreator={Emacs 25.2.1 (Org mode 8.3.4)}, 
 pdflang={English}}
\begin{document}

\maketitle
%\logo{\includegraphics[width=1cm,height=1cm,keepaspectratio]{../logos/NSO_logo_small.pdf}~%
%    \includegraphics[width=1cm,height=1cm,keepaspectratio]{../logos/UOlogoBW.eps}%
%}

\vspace{220pt}
\beamertemplatenavigationsymbolsempty
\setbeamertemplate{caption}[numbered]
\setbeamerfont{caption}{size=\tiny}
% \addtobeamertemplate{frametitle}{}{%
% \begin{textblock*}{100mm}(.85\textwidth,-1cm)
% \includegraphics[height=1cm,width=2cm]{cat}
% \end{textblock*}}

\tikzstyle{chemical} = [rectangle, rounded corners, text width=5em, minimum height=1em,text centered, draw=black, fill=none]
\tikzstyle{hardware} = [rectangle, rounded corners, text width=5em, minimum height=1em,text centered, draw=black, fill=gray!30]
\tikzstyle{ms} = [rectangle, rounded corners, text width=5em, minimum height=1em,text centered, draw=orange, fill=none]
\tikzstyle{msw} = [rectangle, rounded corners, text width=7em, minimum height=1em,text centered, draw=orange, fill=none]
\tikzstyle{label} = [rectangle,text width=8em, minimum height=1em, text centered, draw=none, fill=none]
\tikzstyle{hl} = [rectangle, rounded corners, text width=5em, minimum height=1em,text centered, draw=black, fill=red!30]
\tikzstyle{box} = [rectangle, rounded corners, text width=5em, minimum height=5em,text centered, draw=black, fill=none]
\tikzstyle{arrow} = [thick,->,>=stealth]
\tikzstyle{hl-arrow} = [ultra thick,->,>=stealth,draw=red]

\section{Introduction}
\label{sec:orgheadline7}
\begin{frame}[label={sec:orgheadline1}]{Carnitine and Acylcarnitines}
\begin{itemize}
\item Carnitine (\(\beta\)-hydroxy-\(\gamma\)-N-trimethylaminobutyric acid) is
an endogenous quaternary ammonium compound synthesized from lysine
and methionine.

\item Primary function is to shuttle long chain fatty acids to the
mitochondrial matrix, for \(\beta\)-oxidation.
\item Acylcarnitines are markers for FAODs and OAs
\end{itemize}
\vspace{2em}
\centering
\chemname{\chemfig[][scale=.5]{H3C-N^{+}([2]-CH3)([6]-CH3)-CH2-C([2]-H)([6]-OH)-CH_2-C([1]=O)([7]-O^{-})}}{\tiny Carnitine}
\hspace{3em}
\chemname{\chemfig[][scale=.5]{H3C-N^{+}([2]-CH3)([6]-CH3)-CH2-C([2]-H)([6]-O-C([0]=O)-{\color{red}R})-CH_2-C([1]=O)([7]-O^{-})}}{\tiny Acylcarnitine}
%\chemname{\chemfig[][scale=.5]{H3C-N^{+}([2]-CH3)([6]-CH3)-CH2-C([2]-H)([6]-O-C([0]=O)-{\color{red}R})-CH_2-C([2]=O)-O-CH_2-CH_2-CH_2-CH_3}}{\tiny Acylcarnitine, butyl ester}
\end{frame}

\begin{frame}[label={sec:orgheadline2}]{Sample Type}
\begin{itemize}
\item Plasma
\begin{itemize}
\item Diagnostic testing for FAODs and OAs
\item Monitoring
\end{itemize}
\item Dried blood spot
\begin{itemize}
\item Newborn Screening for FAODs and OAs
\end{itemize}
\item Urine
\begin{itemize}
\item Diagnosis of CUD
\end{itemize}
\end{itemize}
\end{frame}
\begin{frame}[label={sec:orgheadline3}]{ERNDIM Plasma Acylcarnitines Survey}
\begin{figure}[htb]
\centering
\includegraphics[height=0.8\textheight]{./figures/free_carnitine_erndim.png}
\caption{Free Carnitine}
\end{figure}
\end{frame}

\begin{frame}[label={sec:orgheadline4}]{ERNDIM Plasma Acylcarnitines Survey}
\begin{figure}[htb]
\centering
\includegraphics[height=0.8\textheight]{./figures/acetylcarnitine_erndim.png}
\caption{Acetylcarnitine}
\end{figure}
\end{frame}

\begin{frame}[label={sec:orgheadline5}]{Overestimation of Free Carnitine}
\begin{itemize}
\item Butylated acylcarnitines are converted to butylated carnitine in
n-butanol-3M HCl at 65\degree{}C. \footnote{Johnson, D. W. (1999). Inaccurate measurement of free
carnitine by the electrospray tandem mass spectrometry screening
method for blood spots. Journal of Inherited Metabolic Disease, 22(2),
201–202.}
\end{itemize}

\begin{center}
\begin{tabular}{lr}
Acyl Carnitine & Half-life (min)\\
\hline
C2 & 31\\
C10 & 125\\
C18 & 172\\
\end{tabular}
\end{center}

\begin{itemize}
\item 65\degree{}C for 15 min.
\item NSO uses 60\degree{}C for 20 minutes.
\item IMD uses 55\degree{}C for 20 minutes.

\item In a sample with low free carnitine and high acetylcarnitine.
\begin{itemize}
\item 30\% of the acetylcarnitine and smaller amounts of higher
molecular mass acylcarnitines are converted to carnitine
\item a low carnitine sample could appear to be normal.
\end{itemize}
\item "The free carnitine results obtained by this screening method on
blood spots with high levels of acylcarnitines should therefore be
used with caution." \footnotemark[1]{}
\end{itemize}
\end{frame}
\begin{frame}[label={sec:orgheadline6}]{Free and Total Carnitine}
\begin{block}{Fractional Tubular Re-absorption of Carnitine}
\begin{equation*}
FTR_{carnitine}\% = \left( 1 -  \frac{carnitine_{urine} \cdot creatinine_{plasma}}{carnitine_{plasma} \cdot creatinine_{urine}}\right) \cdot 100
\end{equation*}

\begin{itemize}
\item normally >98\%, \(\Downarrow\) in CUD
\end{itemize}
\end{block}

\begin{block}{Free/Total Carnitine}
\[
\frac{Free_{carnitine}}{Total_{carnitine}} = \frac{C_0}{\sum_{0}^{18} C_n}
\]

\begin{itemize}
\item \(\Downarrow\) in CUD, < 5-10\% of normal
\end{itemize}
\end{block}
\end{frame}
\section{Screening FIA-MS/MS}
\label{sec:orgheadline14}
\begin{frame}[label={sec:orgheadline8}]{FIA-MS/MS schematic}
\begin{center}
\begin{tikzpicture}[node distance=7em]
% nodespp
\node(ms1)[ms]{MS1: Mass Filter};
\node(cc)[ms, right of=ms1]{Collision cell};
\node(ms2)[ms, right of=cc]{MS2: Mass Filter};
\node(ion)[ms, below of=ms1,yshift=3em]{Ionization};
\node(lc)[msw, below of=ion,yshift=3em]{Injection};
\node(detector)[ms, below of=ms2, yshift=3em]{Detector};
% arrows
\draw[arrow](lc) -- (ion);
\draw[arrow](ion) -- (ms1);
\draw[arrow](ms1) -- (cc);
\draw[arrow](cc) -- (ms2);
\draw[arrow](ms2) -- (detector);
\end{tikzpicture}
\end{center}

\begin{itemize}
\item Solvent delivery is via HPLC with no chromatography
\item 10 \textmu{}L of sample extract is injected into a flowing stream operating at \textasciitilde{}0.15 ml/min.

\item Typical injection rates between samples are 2 min, giving a potential 400
to 600 sample capacity per instrument per day.
\begin{itemize}
\item volume is typically 200-400 specimens per instrument per day
\item maintenance issues, repeat specimen analysis.
\end{itemize}
\end{itemize}
\end{frame}
\begin{frame}[label={sec:orgheadline9}]{FIA-MS/MS sample}
\begin{itemize}
\item Acylcarnitines in the DBS eluate are esterified as butyl esters with butanol-3M HCL
\end{itemize}

\definesubmol{x}{-[1,.6]-[7,.6]}
\definesubmol{y}{-[7,.6]-[1,.6]}
\definesubmol{d}{!y!y-[7,.6]{\color{red}COOH}}
\definesubmol{e}{!y!y}
\centering
\schemedebug{false}
\schemestart
\chemname{\chemfig[][scale=.33]{-N^{+}([2]-)([6]-)-[1]-[7]([6]-O-([5]=O)!e)-[1]-[7]([7]=O)([1]-O^{-})}}{\tiny C5-carnitine}
\+
\chemname{\chemfig[][scale=.33]{HO!x!x}}{\tiny n-butanol}
\arrow{-U>[][{\tiny \ce{H2O}}]}
\chemname{\chemfig[][scale=.33]{-N^{+}([2]-)([6]-)-[1]-[7]([6]-O-([5]=O)!e)-[1]-[7]([6]=O)-[1,.6]O!y!y}}{\tiny C5-carnitine, butyl ester}
\schemestop
\vspace{2em}
\schemedebug{false}
\schemestart
\chemname{\chemfig[][scale=.33]{-N^{+}([2]-)([6]-)-[1]-[7]([6]-O-([5]=O)!d)-[1]-[7]([7]=O)([1]-O^{-})}}{\tiny C6DC-carnitine}
\+
\chemname{\chemfig[][scale=.33]{HO!x!x}}{\tiny n-butanol}
\arrow{-U>[][{\tiny \ce{2H2O}}]}
\chemname{\chemfig[][scale=.33]{-N^{+}([2]-)([6]-)-[1]-[7]([6]-O-([5]=O)!e-[7,.6]O!x!x)-[1]-[7]([6]=O)-[1,.6]O!y!y}}{\tiny C6DC-carnitine, butyl ester}
\schemestop 
\end{frame}

\begin{frame}[label={sec:orgheadline10}]{FIA-MS/MS Method}
\begin{itemize}
\item Electrospray ionization in positive mode
\item Butylated acylcarnitines fragment to produce a characteristic ion with mass of 85 Da.
\item A precursor ion scan is used to identify molecules that fragment to form a 85 m/z molecule.
\end{itemize}

\begin{block}{Precursor Ion Scan}
\begin{center}
\begin{tikzpicture}[]
\node[box](ms1)[]{};
\node[label](ms1u)[above=of ms1,yshift=-3em]{MS1};
\node[label](ms1l)[below=of ms1,yshift=3em]{scanning};
\node[box](cc)[right= of ms1]{};
\node[label](ccu)[above=of cc,yshift=-3em]{Collision cell};
\node[label](ccl)[below=of cc,yshift=3em]{fragmentation};
\node[box](ms2)[right= of cc]{};
\node[label](ms2u)[above=of ms2,yshift=-3em]{MS2};
\node[label](ms2l)[below=of ms2,yshift=3em]{85 m/z};
\draw[->](ms1) -- (cc);
\draw[->](cc) -- (ms2);

%ms1
\draw [gray,->, decorate,decoration=snake] (-.8,0.5) -- (.8,0.5);
\draw [gray,->, decorate,decoration=snake] (-.8,0.25) -- (.8,0.25);
\draw [blue, ->,decorate,decoration=snake] (-.8, 0) -- (.8,0);
\draw [gray,->, decorate,decoration=snake] (-.8,-0.25) -- (.8,-0.25);
\draw [gray,->,decorate,decoration=snake] (-.8,-0.5) -- (.8,-0.5);

%cc
\draw [blue,->,decorate,decoration=snake] (2.1, 0) -- (2.4,0);
\fill (2.6,0) circle (0.1); 
\draw [gray,->,decorate,decoration=snake] (2.8, 0) -- (3.8,0.5);
\draw [red, ->,decorate,decoration=snake] (2.8, 0) -- (3.8,0);
\draw [gray,->,decorate,decoration=snake] (2.8, 0) -- (3.8,-0.5);

%ms2
\draw [red,->,decorate,decoration=snake] (5.1, 0) -- (6.8,0);
\end{tikzpicture}
\end{center}
\end{block}
\end{frame}

\begin{frame}[label={sec:orgheadline11}]{Fragmentation}
\definesubmol{x}{-[1,.6]-[7,.6]}
\centering
 \chemname{\chemfig[][scale=.33]{H_{3}C-N^{+}([2]-CH_3)([6]-CH_{3})-CH_2-C([2]-H)([6]-O-C([0]=O)-{\color{red}R})-CH_2-C([2]=O)-O-CH_2-CH_2-CH_2-CH_3}}{\tiny acylcarnitine, butyl ester}

\vspace{2.5em}

 \chemname{\chemfig[][scale=.33]{H_{3}C-N([1]-CH_3)([7]-CH_3)}}{\tiny trimethylamine}
\hspace{2em}
\chemname{\chemfig[][scale=.33]{{\color{red}R}-C([1]=O)([7]-OH)}}{\tiny carboxylic acid}
\hspace{2em}
 \chemname{\chemfig[][scale=.33]{H!x!x}}{\tiny butyl group}
\hspace{2em}
 \chemname{\chemfig[][scale=.33]{H_{2}C^{+}-HC=CH-C([1]=O)([7]-OH)}}{\tiny 85 m/z}
\end{frame}

\begin{frame}[label={sec:orgheadline12}]{MRM is used to detected dicarboxylic acylcarnitines}
\begin{itemize}
\item C0-Bu 218.1 Da \(\to\) 103 Da transition is optimal
\item All others benefit from the added sensitivity of MRM mode as compared to parent ion scan
\end{itemize}

\small
\begin{center}
\begin{tabular}{ll}
Compound & Reaction\\
\hline
C0 & 218.10 > 103.00\\
C0 IS & 227.10 > 103.00\\
C2 & 260.20 > 85.00\\
C2 IS & 263.20 > 85.00\\
C3 & 274.20 > 85.00\\
C3 IS & 277.20 > 85.00\\
C3DC & 360.30 > 85.00\\
C4DC & 374.30 > 85.00\\
C5DC & 388.35 > 85.00\\
C5DC IS & 391.35 > 85.00\\
C6DC & 402.45 > 85.00\\
C8DC & 430.45 > 85.00\\
\end{tabular}
\end{center}
\end{frame}

\begin{frame}[label={sec:orgheadline13}]{FIA-MS/MS Acylcarnitine Scan}
\begin{block}{Quantified Acylcarnitines}
\begin{columns}
\begin{column}{0.3\columnwidth}
\begin{itemize}
\item C0
\item C2
\item C3
\item C3DC
\item C4
\item C4DC
\item C5
\item C5:1
\item C5DC
\item C5-OH
\item C6
\item C6DC
\end{itemize}
\end{column}
\begin{column}{0.3\columnwidth}
\begin{itemize}
\item C8
\item C8:1
\item C10
\item C10:1
\item C12
\item C12:1
\item C14
\item C14:1
\item C14:2
\item C14-OH
\end{itemize}
\end{column}
\begin{column}{0.3\columnwidth}
\begin{itemize}
\item C16
\item C16:1
\item C16:1-OH
\item C16-OH
\item C18
\item C18:1
\item C18:1-OH
\item C18:2
\item C18-OH
\end{itemize}
\end{column}
\end{columns}
\end{block}
\end{frame}
\section{Diagnostic FIA-MS/MS}
\label{sec:orgheadline20}
\begin{frame}[label={sec:orgheadline15}]{Diagnostic FIA-MS/MS schematic}
\begin{center}
\begin{tikzpicture}[node distance=7em]
% nodes
\node(ms1)[ms]{MS1: Mass Filter};
\node(cc)[ms, right of=ms1]{Collision cell};
\node(ms2)[ms, right of=cc]{MS2: Mass Filter};
\node(ion)[ms, below of=ms1,yshift=3em]{Ionization};
\node(lc)[msw, below of=ion,yshift=3em]{Fused silica};
\node(detector)[ms, below of=ms2, yshift=3em]{Detector};
% arrows
\draw[arrow](lc) -- (ion);
\draw[arrow](ion) -- (ms1);
\draw[arrow](ms1) -- (cc);
\draw[arrow](cc) -- (ms2);
\draw[arrow](ms2) -- (detector);
\end{tikzpicture}
\end{center}
\begin{itemize}
\item ESI in positive mode
\end{itemize}
\end{frame}
\begin{frame}[label={sec:orgheadline16}]{FIA-MS/MS sample prep}
\begin{itemize}
\item 20 \textmu{}L of sample is mixed with 400 \textmu{}L of IS in Methanol centrifuge to deproteinize.
\item supernatant is removed and 100 \textmu{}L of n-butanol-3M HCL is added
\item dried down
\item reconstituted with 200 \textmu{}L 80\% acetonitrile.
\item 7.5 \textmu{}L injection.
\end{itemize}

\includegraphics[width=0.7\textwidth]{./figures/outletmethod.pdf}
\end{frame}

\begin{frame}[label={sec:orgheadline17}]{FIA-MS/MS transitions}
\begin{block}{Quantified acetylcarnitines}
\tiny

\begin{columns}
\begin{column}{0.33\columnwidth}
\begin{itemize}
\item C2 (ACETYL)
\item C3:1 (PROPENYL)
\item C3 (PROPIONYL)
\item C4 (BUTYRYL)
\item C5:1 (TIGLYL)
\item C5 (ISOVALERYL)
\item C4-OH (3OH-BUTYRYL)
\item C6 (HEXANOYL)
\item C5-OH/2ME3-OH BUTYRYL
\item BENZOYL
\item C6-OH (3OH-HEXANOYL)
\item PHENYLACETYL
\item C8:1 (OCTENOYL)
\item C8 (OCTANOYL)
\end{itemize}
\end{column}

\begin{column}{0.33\columnwidth}
\begin{itemize}
\item C3DC (MALONYL)
\item C10:3 (DECATRIENOYL)
\item C10:2 (DECADIENOYL)
\item C10:1 (DECENOYL)
\item C10 (DECANOYL)
\item C4DC (MEMALONYL/SUCCINYL)
\item C5DC (GLUTARYL)/C10-OH
\item C12:1 (DODECENOYL)
\item C12 (DODECANOYL)
\item C6:DC/3 MEGLUTARYL
\item C12OH (3 OH DODECANOYL)
\item C14:2 (TETRADECADIENOYL)
\item C14:1 (TETRADECENOYL)
\item C14 (TETRADECANOYL)
\item C8DC
\end{itemize}
\end{column}

\begin{column}{0.33\columnwidth}
\begin{itemize}
\item C14-1OH (3OH TETRADECENYL)
\item C14-OH (3OH TETRADECANOYL)
\item C16:1 (PALMITOLEYL)
\item C16 (PALMITOYL)
\item C10DC (SEBACYL)
\item C16-1OH (3OH PALMITOLEYL)
\item C16OH (3OH PALMITOYL)
\item C18:2 (LINOLEYL)
\item C18:1 (OLEOYL)
\item C18 (STEAROYL)
\item C18:2OH (3OH LINOLEYL)
\item C18:1OH (3OH OLELYL)
\item C18OH (3OH STEAROYL)
\item C16DC
\item C18:1DC
\end{itemize}
\end{column}
\end{columns}
\end{block}
\end{frame}

\begin{frame}[label={sec:orgheadline18}]{Pros and Cons of Butanol  FIA-MSMS for Aceylcarnitines}
\begin{block}{Pros}
\begin{itemize}
\item Speed
\item Sensitivity
\item Expertise
\item Amino acid measurement
\end{itemize}
\end{block}
\begin{block}{Cons}
\begin{itemize}
\item Isobaric compounds
\begin{itemize}
\item C5DC and C10-OH
\end{itemize}
\item Overestimation of CO due to hydrolysis
\end{itemize}
\end{block}
\end{frame}

\begin{frame}[label={sec:orgheadline19}]{Why derivatize?}
\includegraphics[width=.9\linewidth]{./figures/ionization.png}
\end{frame}
\end{document}

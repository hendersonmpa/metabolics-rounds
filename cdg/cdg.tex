% Created 2019-11-01 Fri 16:22
% Intended LaTeX compiler: pdflatex
\documentclass[presentation, smaller]{beamer}
\usepackage[utf8]{inputenc}
\usepackage[T1]{fontenc}
\usepackage{graphicx}
\usepackage{grffile}
\usepackage{longtable}
\usepackage{wrapfig}
\usepackage{rotating}
\usepackage[normalem]{ulem}
\usepackage{amsmath}
\usepackage{textcomp}
\usepackage{amssymb}
\usepackage{capt-of}
\usepackage{hyperref}
\hypersetup{colorlinks,linkcolor=white,urlcolor=blue}
\usepackage{textpos}
\usepackage{textgreek}
\usepackage[version=4]{mhchem}
\usepackage{chemfig}
\usepackage{siunitx}
\usepackage{gensymb}
\usepackage[usenames,dvipsnames]{xcolor}
\usepackage[T1]{fontenc}
\usepackage{lmodern}
\usepackage{verbatim}
\usepackage{tikz}
\usepackage{wasysym}
\usetikzlibrary{shapes.geometric,arrows,decorations.pathmorphing,backgrounds,positioning,fit,petri}
\usetheme{Hannover}
\usecolortheme{whale}
\author{Matthew Henderson, PhD, FCACB}
\date{\today}
\title{Congenital Disorders of Glycosylation}
\institute[NSO]{Newborn Screening Ontario | The University of Ottawa}
\titlegraphic{\includegraphics[height=1cm,keepaspectratio]{../logos/NSO_logo.pdf}\includegraphics[height=1cm,keepaspectratio]{../logos/cheo-logo.png} \includegraphics[height=1cm,keepaspectratio]{../logos/UOlogoBW.eps}}
\hypersetup{
 pdfauthor={Matthew Henderson, PhD, FCACB},
 pdftitle={Congenital Disorders of Glycosylation},
 pdfkeywords={},
 pdfsubject={},
 pdfcreator={Emacs 26.1 (Org mode 9.1.9)}, 
 pdflang={English}}
\begin{document}

\maketitle

%\logo{\includegraphics[width=1cm,height=1cm,keepaspectratio]{../logos/NSO_logo_small.pdf}~%
%    \includegraphics[width=1cm,height=1cm,keepaspectratio]{../logos/UOlogoBW.eps}%
%}

\vspace{220pt}
\beamertemplatenavigationsymbolsempty
\setbeamertemplate{caption}[numbered]
\setbeamerfont{caption}{size=\tiny}
% \addtobeamertemplate{frametitle}{}{%
% \begin{textblock*}{100mm}(.85\textwidth,-1cm)
% \includegraphics[height=1cm,width=2cm]{cat}
% \end{textblock*}}

\section{Introduction}
\label{sec:orgbc5cdfa}

\begin{frame}[label={sec:orgc2425cd}]{Introduction}
\begin{itemize}
\item Most extracellular proteins, such as serum proteins, most membrane proteins and several intracellular proteins (such as lysosomal enzymes), are glycoproteins.
\item The glycans are defined by their linkage to the protein:
\begin{itemize}
\item N-glycans are linked to the amide group of asparagine
\item O-glycans are linked to the hydroxyl group of serine or
threonine.
\end{itemize}

\item Glycosylphosphatidylinositol (GPI) anchors are glycolipids that
tether more than 150 proteins to the outer leaflet of plasma
membranes.

\item Congenital disorders of glycosylation are due to defects in the
synthesis of glycans and in the attachment of glycans to proteins
and lipids.
\item Rapidly growing disease family
\end{itemize}
\end{frame}

\begin{frame}[label={sec:org22880d0}]{Synthesis of N-glycans}
\begin{enumerate}
\item Formation in the cytosol of nucleotide-linked sugars, mainly
GDP-Man, and also UDP-Glc and UDP-GlcNAc followed by attachment of
GlcNAc and Man units to dolichol phosphate, and flipping of the
nascent oligosaccharide structure into the ER.
\item Stepwise assembly in the ER, by addition of Man and Glc resulting
in the 14-unit oligosaccharide precuror:
\begin{itemize}
\item dolichol pyrophosphate-N-acetyl-glucosamine\(_{\text{2}}\) -mannose\(_{\text{9}}\) -glucose\(_{\text{3}}\)
\end{itemize}
\item Transfer of this precursor onto the nascent protein, followed by
final processing of the glycan in the Golgi apparatus by trimming
and attachment of various sugar units.
\end{enumerate}
\end{frame}

\begin{frame}[label={sec:org92471c6}]{N-glycan Assembly}
\begin{figure}[htbp]
\centering
\includegraphics[width=0.9\textwidth]{./figures/ngassembly.png}
\caption{\label{fig:orgae37c65}
N-glycan assembly}
\end{figure}
\end{frame}

\begin{frame}[label={sec:org3d8ee96}]{N-glycan Remodelling}
\begin{figure}[htbp]
\centering
\includegraphics[width=0.9\textwidth]{./figures/ngremodel.png}
\caption{\label{fig:orgbd5f0e2}
N-glycan remodelling}
\end{figure}
\end{frame}


\section{CDG-1a}
\label{sec:orga3c9e3d}
CDG-Ia accounts for 70\% of the congenital disorders of glycosylation, which combined affect 1 in every 50,000 to 100,000 births. Cases of CDG-Ia have been reported worldwide, with about half coming from Scandinavian countries
\end{document}
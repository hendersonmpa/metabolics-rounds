% Created 2019-11-05 Tue 16:41
% Intended LaTeX compiler: pdflatex
\documentclass[presentation, smaller]{beamer}
\usepackage[utf8]{inputenc}
\usepackage[T1]{fontenc}
\usepackage{graphicx}
\usepackage{grffile}
\usepackage{longtable}
\usepackage{wrapfig}
\usepackage{rotating}
\usepackage[normalem]{ulem}
\usepackage{amsmath}
\usepackage{textcomp}
\usepackage{amssymb}
\usepackage{capt-of}
\usepackage{hyperref}
\hypersetup{colorlinks,linkcolor=white,urlcolor=blue}
\usepackage{textpos}
\usepackage{textgreek}
\usepackage[version=4]{mhchem}
\usepackage{chemfig}
\usepackage{siunitx}
\usepackage{gensymb}
\usepackage[usenames,dvipsnames]{xcolor}
\usepackage[T1]{fontenc}
\usepackage{lmodern}
\usepackage{verbatim}
\usepackage{tikz}
\usepackage{wasysym}
\usetikzlibrary{shapes.geometric,arrows,decorations.pathmorphing,backgrounds,positioning,fit,petri}
\usetheme{Hannover}
\usecolortheme{whale}
\author{Matthew Henderson, PhD, FCACB}
\date{\today}
\title{Congenital Disorders of N-Linked Glycosylation}
\institute[NSO]{Newborn Screening Ontario | The University of Ottawa}
\titlegraphic{\includegraphics[height=1cm,keepaspectratio]{../logos/NSO_logo.pdf}\includegraphics[height=1cm,keepaspectratio]{../logos/cheo-logo.png} \includegraphics[height=1cm,keepaspectratio]{../logos/UOlogoBW.eps}}
\hypersetup{
 pdfauthor={Matthew Henderson, PhD, FCACB},
 pdftitle={Congenital Disorders of N-Linked Glycosylation},
 pdfkeywords={},
 pdfsubject={},
 pdfcreator={Emacs 26.1 (Org mode 9.1.9)}, 
 pdflang={English}}
\begin{document}

\maketitle

%\logo{\includegraphics[width=1cm,height=1cm,keepaspectratio]{../logos/NSO_logo_small.pdf}~%
%    \includegraphics[width=1cm,height=1cm,keepaspectratio]{../logos/UOlogoBW.eps}%
%}

\vspace{220pt}
\beamertemplatenavigationsymbolsempty
\setbeamertemplate{caption}[numbered]
\setbeamerfont{caption}{size=\tiny}
% \addtobeamertemplate{frametitle}{}{%
% \begin{textblock*}{100mm}(.85\textwidth,-1cm)
% \includegraphics[height=1cm,width=2cm]{cat}
% \end{textblock*}}

\section{Disorders of Protein N-Glycosylation}
\label{sec:org4d58de1}
\begin{frame}[label={sec:org0bf6592}]{Introduction}
\begin{itemize}
\item Most extracellular proteins, membrane proteins and several
intracellular proteins (lysosomal enzymes), are glycoproteins.

\item The glycans are defined by their linkage to the protein:
\begin{itemize}
\item N-glycans are linked to the amide group of asparagine
\item O-glycans are linked to the hydroxyl group of serine or
threonine.
\end{itemize}

\item Congenital disorders of glycosylation are due to defects in the
synthesis of glycans and in the attachment of glycans to proteins
and lipids.
\item Rapidly growing disease family (N-,O-,GPI) \textasciitilde{}60 (listed on NORD)
\begin{itemize}
\item 1\% of HG involved in glycosylation
\end{itemize}
\end{itemize}
\end{frame}

\begin{frame}[label={sec:org34c445a}]{Synthesis of N-glycans}
\begin{enumerate}
\item Formation in the cytosol of nucleotide-linked sugars
\begin{itemize}
\item mainly GDP-Man, UDP-Glc and UDP-GlcNAc
\item attachment of GlcNAc and Man units to dolichol phosphate
\item flipping into the ER
\end{itemize}
\item Stepwise assembly in the ER
\begin{itemize}
\item addition of Man and Glc \(\to\) 14-unit oligosaccharide precuror:
\begin{itemize}
\item dolichol pyrophosphate-N-acetyl-glucosamine\(_{\text{2}}\)-mannose\(_{\text{9}}\)-glucose\(_{\text{3}}\)
\end{itemize}
\end{itemize}
\item Transfer of this precursor onto the nascent protein
\begin{itemize}
\item processing of the glycan in the Golgi apparatus
\begin{itemize}
\item trimming and attachment of various sugar units
\end{itemize}
\end{itemize}
\end{enumerate}
\end{frame}

\begin{frame}[label={sec:org2140e16}]{N-glycan Assembly}
\begin{figure}[htbp]
\centering
\includegraphics[width=0.9\textwidth]{./figures/glyc.png}
\caption{\label{fig:org5cc62ed}
N-glycan assembly}
\end{figure}
\end{frame}


\begin{frame}[label={sec:orge067685}]{N-glycan Assembly}
\begin{figure}[htbp]
\centering
\includegraphics[width=0.9\textwidth]{./figures/ngassembly.png}
\caption{\label{fig:orgfe7d647}
N-glycan assembly}
\end{figure}
\end{frame}

\begin{frame}[label={sec:orgbe60255}]{N-glycan Remodelling}
\begin{figure}[htbp]
\centering
\includegraphics[width=0.9\textwidth]{./figures/ngremodel.png}
\caption{\label{fig:org95ac5d1}
N-glycan remodelling}
\end{figure}
\end{frame}

\begin{frame}[label={sec:orgfe39b41}]{CDG}
\begin{itemize}
\item very broad spectrum of clinical manifestations
\item considered in any unexplained clinical condition
\begin{itemize}
\item particularly in multi-organ disease with neurological involvement
\item non-specific developmental disability is the only presenting sign
\end{itemize}
\item Incidence 1:50,000 to 100,000 births
\end{itemize}
\end{frame}

\begin{frame}[label={sec:orgc15db78}]{CDG classification}
\begin{itemize}
\item Each CDG type is defined by a specific enzyme defect and the mutation in its underlying gene
\item Most CDG mutations are hypomorphic and allow some glycan synthesis
\item Type 1: ER defects
\item Type 2: Golgi defects
\item \textasciitilde{}1500+ cases worldwide
\item \textasciitilde{}300+ cases in the US
\begin{itemize}
\item 70\% PMM2-CDG
\end{itemize}
\end{itemize}
\end{frame}

\begin{frame}[label={sec:orga6ccf26}]{Transferrin IEF}
\begin{itemize}
\item serum transferrin IEF is still the screening method of choice
\begin{itemize}
\item able to detect only a limited number of CDGs
\item N-glycosylation disorders associated with sialic acid deficiency
\end{itemize}

\item Normal serum transferrin is mainly composed of tetrasialotransferrin
and small amounts of mono-, di-, tri-, penta- and
hex-asialotransferrins.

\item Partial deficiency of sialic acid (-ve charge) causes a
cathodal shift.

\item Two main types of cathodal shift can be recognized:
\begin{itemize}
\item Type 1 or 2 patterns
\end{itemize}
\end{itemize}
\end{frame}

\begin{frame}[label={sec:org72bc610}]{Transferrin IEF}
\begin{itemize}
\item A Type 1 pattern is characterized by an increase of both disialo- and
asialotransferrin and a decrease of tetra-, penta-and
hexasialotransferrins;

\begin{itemize}
\item indicates an assembly disorder, and PMM2-CDG or MPI-CDG should be
considered first.
\end{itemize}

\item A Type 1 pattern is also seen in secondary glycosylation disorders
such as chronic alcoholism, hereditary fructose intolerance and
galactosaemia.

\item In a Type 2 pattern there is also an increase of the tri- \textpm{}
monosialotransferrin bands.
\begin{itemize}
\item indicates a disorder of processing
\end{itemize}

\item A shift due to a transferrin protein variant has first to be excluded
\begin{itemize}
\item IEF after neuraminidase treatment, study of another glycoprotein
or, investigation of the parents.
\end{itemize}
\end{itemize}
\end{frame}

\begin{frame}[label={sec:org97f57ce}]{Additional Laboratory Investigations}
\begin{itemize}
\item Protein-linked glycan analysis can be performed to identify the defective step
\begin{itemize}
\item MALDI-TOF analysis of released N-linked oligosaccharides
\end{itemize}
\item CDG gene panel analysis or WES.

\item Capillary zone electrophoresis of total serum is a rapid screening
test for CDG.
\begin{itemize}
\item An abnormal result should be further investigated by serum
transferrin IEF.
\end{itemize}
\end{itemize}
\end{frame}

\begin{frame}[label={sec:org96e6929}]{Transferrin IEF}
\begin{figure}[htbp]
\centering
\includegraphics[width=0.9\textwidth]{./figures/transferrin_ief.png}
\caption{\label{fig:org7f9b69b}
Transferrin IEF}
\end{figure}
\end{frame}


\begin{frame}[label={sec:orga513b53}]{CDG diagnosis}
\begin{figure}[htbp]
\centering
\includegraphics[width=0.9\textwidth]{./figures/cdg_diag.png}
\caption{\label{fig:orgbd672c5}
CDG diagnosis}
\end{figure}
\end{frame}


\section{PMM2-CDG (CDG-1a)}
\label{sec:org6cc5ee0}

\begin{frame}[label={sec:org2f712cb}]{Clinical Presenation}
\begin{itemize}
\item CDG-Ia accounts for 70\% CDGs
\item The nervous system is affected in all patients
\item most other organs are involved in a variable way
\item neurological symptoms include:
\begin{itemize}
\item alternating internal strabismus and other abnormal eye movements
\item axial hypotonia, psychomotor disability, ataxia and hyporeflexia.
\end{itemize}
\item After infancy, symptoms include retinitis pigmentosa, stroke-like episodes, \textpm{} epilepsy.
\item 1st year variable feeding problems anorexia, vomiting, diarrhoea \(\to\) failure to thrive.
\item Other features are:
\begin{itemize}
\item variable dysmorphism, which may include large ears, abnormal
subcutaneous adipose tissue distribution, inverted nipples,
\item mild to moderate hepatomegaly, skeletal abnormalities and hypogonadism.
\end{itemize}
\item Some infants develop a pericardial effusion \textpm{} cardiomyopathy.
\item At the other end of the clinical spectrum are patients with a very
mild phenotype - no dysmorphic features, slight intellectual disability.
\end{itemize}
\end{frame}

\begin{frame}[label={sec:org8ab4949}]{Metabolic Derangement}
\begin{itemize}
\item PMM2-CDG is due to the deficiency of PMM2
\begin{itemize}
\item principal isozyme of PMM
\end{itemize}
\item Phosphomannomutase (PMM) 2 catalyses the second committed step in the synthesis of GDP-mannose
\begin{itemize}
\item mannose-6-phosphate \(\to\) mannose-1-phosphate, occurs in the cytosol
\end{itemize}
\item GDP-mannose is the donor of the mannose units used in the ER to
assemble the dolichol-pyrophosphate oligosaccharide precursor
\item defect \(\to\) hypoglycosylation
\item deficiency and/or dysfunction of numerous glycoproteins, including:
\begin{itemize}
\item serum proteins thyroxin-binding globulin, haptoglobin, clotting factor XI, antithrombin III, cholinesterase
\item lysosomal enzymes
\item membranous glycoproteins
\end{itemize}
\end{itemize}
\end{frame}

\begin{frame}[label={sec:orge1f068e}]{Metabolic Derangement}
\begin{figure}[htbp]
\centering
\includegraphics[width=0.9\textwidth]{./figures/pmm.png}
\caption{\label{fig:org5960753}
PMM2}
\end{figure}
\end{frame}




\begin{frame}[label={sec:orgfbb97f5}]{Genetics}
\begin{itemize}
\item AR, PMM2
\item \(\ge\) 107 mutations identified
\item The most frequent mutation (c.422G>A) causes an R141H substitution
\begin{itemize}
\item present in 75\% of Caucasian patients
\item not compatible with life in the homozygous state.
\item frequency in Belgian as high as 1 in 50.
\end{itemize}
\item The incidence of PMM2 deficiency is not known;
\begin{itemize}
\item in Sweden it has been estimated at 1 in 40,000.
\end{itemize}
\end{itemize}
\end{frame}

\begin{frame}[label={sec:orgc730f40}]{Diagnostic Tests}
\begin{itemize}
\item elevation of serum transaminase levels, hypoalbuminaemia, hypocholesterolaemia, and tubular proteinuria.
\item Transferrin IEF
\item To confirm the diagnosis, the activity of PMM should be measured in
leukocytes or fibroblasts.
\begin{itemize}
\item\relax [2-H\(^{\text{3}}\)]mannose-6-phosphate
\end{itemize}
\item PMM activity in fibroblasts can be normal
\end{itemize}
\end{frame}

\begin{frame}[label={sec:orgf594275}]{Treatment}
\begin{itemize}
\item No effective treatment is available.
\item The promising finding that mannose is able to correct glycosylation
in fibroblasts with PMM2 deficiency could not be substantiated in patients.
\end{itemize}
\end{frame}
\end{document}
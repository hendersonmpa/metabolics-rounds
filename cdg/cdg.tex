% Created 2019-11-07 Thu 12:43
% Intended LaTeX compiler: pdflatex
\documentclass[presentation, smaller]{beamer}
\usepackage[utf8]{inputenc}
\usepackage[T1]{fontenc}
\usepackage{graphicx}
\usepackage{grffile}
\usepackage{longtable}
\usepackage{wrapfig}
\usepackage{rotating}
\usepackage[normalem]{ulem}
\usepackage{amsmath}
\usepackage{textcomp}
\usepackage{amssymb}
\usepackage{capt-of}
\usepackage{hyperref}
\hypersetup{colorlinks,linkcolor=white,urlcolor=blue}
\usepackage{textpos}
\usepackage{textgreek}
\usepackage[version=4]{mhchem}
\usepackage{chemfig}
\usepackage{siunitx}
\usepackage{gensymb}
\usepackage[usenames,dvipsnames]{xcolor}
\usepackage[T1]{fontenc}
\usepackage{lmodern}
\usepackage{verbatim}
\usepackage{tikz}
\usepackage{wasysym}
\usetikzlibrary{shapes.geometric,arrows,decorations.pathmorphing,backgrounds,positioning,fit,petri}
\usetheme{Hannover}
\usecolortheme{whale}
\author{Matthew Henderson, PhD, FCACB}
\date{\today}
\title{Congenital Disorders of N-Linked Glycosylation}
\institute[NSO]{Newborn Screening Ontario | The University of Ottawa}
\titlegraphic{\includegraphics[height=1cm,keepaspectratio]{../logos/NSO_logo.pdf}\includegraphics[height=1cm,keepaspectratio]{../logos/cheo-logo.png} \includegraphics[height=1cm,keepaspectratio]{../logos/UOlogoBW.eps}}
\hypersetup{
 pdfauthor={Matthew Henderson, PhD, FCACB},
 pdftitle={Congenital Disorders of N-Linked Glycosylation},
 pdfkeywords={},
 pdfsubject={},
 pdfcreator={Emacs 26.1 (Org mode 9.1.9)}, 
 pdflang={English}}
\begin{document}

\maketitle

%\logo{\includegraphics[width=1cm,height=1cm,keepaspectratio]{../logos/NSO_logo_small.pdf}~%
%    \includegraphics[width=1cm,height=1cm,keepaspectratio]{../logos/UOlogoBW.eps}%
%}

\vspace{220pt}
\beamertemplatenavigationsymbolsempty
\setbeamertemplate{caption}[numbered]
\setbeamerfont{caption}{size=\tiny}
% \addtobeamertemplate{frametitle}{}{%
% \begin{textblock*}{100mm}(.85\textwidth,-1cm)
% \includegraphics[height=1cm,width=2cm]{cat}
% \end{textblock*}}

\section{Disorders of Protein N-Glycosylation}
\label{sec:org666aace}
\begin{frame}[label={sec:orgb3df3ea}]{Introduction}
\begin{itemize}
\item Most extracellular proteins, membrane proteins and several
intracellular proteins (lysosomal enzymes), are glycoproteins.

\item glycans are defined by their linkage to the protein:
\begin{itemize}
\item N-glycans are linked to the amide group of asparagine
\item O-glycans are linked to the hydroxyl group of serine or
threonine.
\end{itemize}

\item Congenital disorders of glycosylation are due to defects in the
synthesis of glycans and in the attachment of glycans to proteins
and lipids.
\item Rapidly growing disease family (N-,O-,GPI)
\begin{itemize}
\item \textasciitilde{}60 listed on NORD
\item 1\% of HG involved in glycosylation
\end{itemize}
\end{itemize}
\end{frame}

\begin{frame}[label={sec:orgd2aa442}]{Synthesis of N-glycans}
\begin{enumerate}
\item Formation in the cytosol of nucleotide-linked sugars
\begin{itemize}
\item mainly GDP-Man, UDP-Glc and UDP-GlcNAc
\item attachment of GlcNAc and Man units to dolichol phosphate
\item flipping into the ER
\end{itemize}
\item Stepwise assembly in the ER
\begin{itemize}
\item addition of Man and Glc \(\to\) 14-unit oligosaccharide precuror:
\item dolichol pyrophosphate-GlcNac\(_{\text{2}}\)-Man\(_{\text{9}}\)-Glu\(_{\text{3}}\)
\end{itemize}
\item Transfer of this precursor onto the nascent protein by OST
\begin{itemize}
\item processing of the glycan in the Golgi apparatus
\begin{itemize}
\item trimming and attachment of various sugar units
\end{itemize}
\end{itemize}
\end{enumerate}
\end{frame}

\begin{frame}[label={sec:org35a63f3}]{N-glycan Assembly}
\begin{figure}[htbp]
\centering
\includegraphics[width=0.9\textwidth]{./figures/glyc.png}
\caption{\label{fig:orgfa81ad7}
N-glycan assembly}
\end{figure}
\end{frame}


\begin{frame}[label={sec:orgdb87a9e}]{N-glycan Assembly}
\begin{figure}[htbp]
\centering
\includegraphics[width=0.9\textwidth]{./figures/ngassembly.png}
\caption{\label{fig:org81898e1}
N-glycan assembly}
\end{figure}
\end{frame}

\begin{frame}[label={sec:orgfdf7ad6}]{N-glycan Remodelling}
\begin{figure}[htbp]
\centering
\includegraphics[width=0.9\textwidth]{./figures/ngremodel.png}
\caption{\label{fig:org050533c}
N-glycan remodelling}
\end{figure}
\end{frame}

\begin{frame}[label={sec:org64a36b2}]{Congenital Disorders of Glycosylation}
\begin{itemize}
\item very broad spectrum of clinical manifestations
\item consider in any unexplained clinical condition
\begin{itemize}
\item particularly in multi-organ disease with neurological involvement
\item when non-specific developmental disability is the only presenting sign
\end{itemize}
\item Incidence 1:50,000 to 100,000 births
\end{itemize}
\end{frame}

\begin{frame}[label={sec:org9f18227}]{CDG classification}
\begin{itemize}
\item Each CDG type is defined by a specific enzyme defect and the mutation in its underlying gene
\item Most CDG mutations are hypomorphic - allow some glycan synthesis
\item N-Glycan CDGs
\begin{itemize}
\item Type 1: ER defects
\item Type 2: Golgi defects
\item \textasciitilde{}1500+ cases worldwide
\item \textasciitilde{}300+ cases in the US
\begin{itemize}
\item 70\% PMM2-CDG
\end{itemize}
\end{itemize}
\end{itemize}
\end{frame}

\begin{frame}[label={sec:org81032ba}]{Transferrin IEF}
\begin{itemize}
\item serum transferrin IEF is the screening method of choice

\begin{itemize}
\item can detect nearly all known CDG-I types as well as most CDG-II types and many CDG-X cases.
\item N-glycosylation disorders associated with sialic acid deficiency
\end{itemize}

\item Normal serum transferrin is mainly composed of:
\begin{itemize}
\item tetrasialotransferrin and small amounts of mono-, di-, tri-,
penta- and hex-asialotransferrins
\end{itemize}

\item Partial deficiency of sialic acid (-ve charge) causes a
cathodal shift.

\item Two main types of cathodal shift can be recognized:
\begin{itemize}
\item Type 1 or 2 patterns
\end{itemize}
\end{itemize}
\end{frame}

\begin{frame}[label={sec:orgc023a8a}]{Transferrin IEF}
\begin{itemize}
\item Type 1 pattern

\begin{itemize}
\item \(\uparrow\) disialo- and asialotransferrin

\item \(\downarrow\)  tetra-, penta-and hexasialotransferrins

\item defects in the assembly of the dolichol lipid-linked
oligosaccharide chain and transfer to the nascent protein
\item PMM2-CDG or MPI-CDG should be considered first

\item also seen in secondary glycosylation disorders such as:
\begin{itemize}
\item chronic alcoholism, hereditary fructose intolerance and galactosaemia
\end{itemize}
\end{itemize}

\item Type 2 pattern

\begin{itemize}
\item Type 1 pattern with additional \(\uparrow\) tri- \textpm{}
monosialotransferrin bands.

\item defects in the trimming and processing of the protein-bound
glycans either late in the endoplasmic reticulum or the Golgi
compartments.
\end{itemize}
\end{itemize}
\end{frame}

\begin{frame}[label={sec:orgd1bc932}]{Transferrin IEF limitations}
\begin{itemize}
\item deficiencies of ER-glucosidase I (CDG-IIb) and Golgi GDP-fucose
transporter (CDG-IIc) are missed.
\item prenatal diagnostics by IEF analysis from fetal blood is not
reliable
\item IEF of serum from children \textless{} 2 weeks may be false-positive
\item Heavy alcohol consumption can also result in serum transferrin
deficiency in carbohydrate moieties, leading to an abnormal
IEF-pattern.
\item Mutations in the protein backbone of transferrin
\begin{itemize}
\item desialylation of transferrin by neuraminidase treatment or IEF of
an alternative glycoprotein like \(\alpha\) 1-antitrypsin should be
performed.
\end{itemize}
\end{itemize}
\end{frame}

\begin{frame}[label={sec:org1c45496}]{Additional Laboratory Investigations}
\begin{itemize}
\item Protein-linked glycan analysis can be performed to identify the defective step
\begin{itemize}
\item MALDI-TOF analysis of released N-linked oligosaccharides
\end{itemize}
\item CDG gene panel analysis or WES.

\item Capillary zone electrophoresis of total serum is a rapid screening
test for CDG.
\begin{itemize}
\item An abnormal result should be further investigated by serum
transferrin IEF.
\end{itemize}

\item HPLC-UV/Vis @ Sickkids
\end{itemize}
\end{frame}

\begin{frame}[label={sec:org60de3ad}]{Transferrin IEF}
\begin{figure}[htbp]
\centering
\includegraphics[width=0.9\textwidth]{./figures/transferrin_ief.png}
\caption{\label{fig:org60cb1fe}
Transferrin IEF}
\end{figure}
\end{frame}



\section{PMM2-CDG (CDG-Ia)}
\label{sec:org7eb4f33}

\begin{frame}[label={sec:org8ba946b}]{Clinical Presentation}
\begin{itemize}
\item \textasciitilde{}70\% CDGs
\item The nervous system is affected in all patients
\begin{itemize}
\item alternating internal strabismus and other abnormal eye movements
\item axial hypotonia, psychomotor disability, ataxia and hyporeflexia
\end{itemize}
\item Other features are:
\begin{itemize}
\item variable dysmorphism, which may include large ears, abnormal
subcutaneous adipose tissue distribution, inverted nipples,
\item mild to moderate hepatomegaly, skeletal abnormalities and hypogonadism
\end{itemize}
\item After infancy, symptoms include retinitis pigmentosa, stroke-like episodes, \textpm{} epilepsy
\item 1st year variable feeding problems anorexia, vomiting, diarrhoea \(\to\) failure to thrive
\item Some infants develop a pericardial effusion \textpm{} cardiomyopathy
\item At the other end of the clinical spectrum are patients with a very
mild phenotype - no dysmorphic features, slight intellectual disability
\end{itemize}
\end{frame}

\begin{frame}[label={sec:orgaed9bd1}]{Metabolic Derangement}
\begin{itemize}
\item Deficiency of PMM2, principal isozyme of PMM
\item Phosphomannomutase 2 catalyses the second committed step in the synthesis of GDP-mannose
\begin{itemize}
\item Man-6-P \ce{<=>} Man-1-P, occurs in the cytosol
\end{itemize}
\item GDP-mannose is used in the ER to assemble the dolichol-pyrophosphate
oligosaccharide precursor
\item defect \(\to\) hypoglycosylation
\item deficiency and/or dysfunction of numerous glycoproteins, including:
\begin{itemize}
\item serum proteins thyroxin-binding globulin, haptoglobin, clotting
factor XI, antithrombin III, cholinesterase
\item lysosomal enzymes
\item membranous glycoproteins
\end{itemize}
\end{itemize}
\end{frame}

\begin{frame}[label={sec:orgdc602f3}]{Genetics}
\begin{itemize}
\item AR, PMM2
\item \(\ge\) 107 mutations identified
\item The most frequent mutation (c.422G>A) causes an R141H substitution
\begin{itemize}
\item present in 75\% of Caucasian patients
\item not compatible with life in the homozygous state
\item frequency in Belgian as high as 1 in 50
\end{itemize}
\item The incidence of PMM2 deficiency is not known
\begin{itemize}
\item in Sweden it has been estimated at 1 in 40,000
\end{itemize}
\end{itemize}
\end{frame}

\begin{frame}[label={sec:orge0b0d9f}]{Diagnostic Tests}
\begin{itemize}
\item \(\uparrow\) transaminases, hypoalbuminaemia, hypocholesterolaemia, and
tubular proteinuria
\item transferrin IEF
\item To confirm the diagnosis, the activity of PMM should be measured in
leukocytes or fibroblasts
\begin{itemize}
\item\relax [2-H\(^{\text{3}}\)]mannose-6-phosphate
\end{itemize}
\item PMM activity in fibroblasts can be normal
\end{itemize}
\end{frame}

\begin{frame}[label={sec:orgbc211f6}]{Treatment}
\begin{itemize}
\item No effective treatment is available
\item The promising finding that mannose is able to correct glycosylation
in fibroblasts with PMM2 deficiency could not be substantiated in
patients
\end{itemize}
\end{frame}


\begin{frame}[label={sec:org13b6b89}]{MPI-CDG (CDG-1b)}
\begin{itemize}
\item Mannose-6 phosphate isomerase deficiency
\item F-6-P \ce{<=>} M-6-P

\item Prevalence: \textless{} 1/1,000,000
\item AR, MPI
\item onset in infancy, neonatal

\item cyclic vomiting, profound hypoglycemia, failure to thrive, liver
fibrosis, gastrointestinal complications
\begin{itemize}
\item protein-losing enteropathy with hypoalbuminaemia, life-threatening
intestinal bleeding of diffuse origin
\end{itemize}
\item thrombotic events protein C and S deficiency, low anti-thrombine III levels
\item neurological development and cognitive capacity is usually normal
\item \alert{treated effectively with oral mannose supplementation}
\item can be fatal if untreated
\item Saquenay-Lac Saint-Jean syndrome
\item Type I pattern, \(\downarrow\) MPI activity WBC, Fib
\end{itemize}
\end{frame}

\section{Type II}
\label{sec:org9d59fe9}

\begin{frame}[label={sec:org78cd301}]{MGAT2-CDG (CDG-IIa)}
\begin{itemize}
\item Golgi N-acetylglucosaminyltransferase II deficiency
\begin{itemize}
\item transfer GlcNAc \(\to\) free terminal mannose of core N-linked glycan chain
\item \(\to\) second branch in complex glycans
\end{itemize}
\item AR, MGAT2
\item Prevalence \textless{}1/1,000,000
\item onset in infancy, neonatal
\item facial dysmorphism: large, posteriorly rotated ears with prominent
antihelices, convex nasal ridge, open mouth, large and crowded
teeth
\item stereotypic hand movements, seizures, and varying degrees of
developmental delay.
\item A bleeding tendency is also observed due to diminished platelet
aggregation.
\item Type II pattern, \(\downarrow\) GnT II activity WBC, Fib
\end{itemize}
\end{frame}

\begin{frame}[label={sec:org5ba4c72}]{SLC35C1-CDG (CDG-IIc)}
\begin{itemize}
\item GDP-fucose transporter 1	defect
\item AR
\item Normal transferrin IEF
\item severe mental retardation, microcephaly, cortical atrophy, seizures,
hypotonia, rhizomelic short stature, and recurrent infections with
neutrophilia.
\item \alert{fucose has been used to treat}, thought that:
\begin{itemize}
\item K\(_{\text{M}}\) mutants - treatable
\item V\(_{\text{max}}\) mutants - not treatable
\end{itemize}
\end{itemize}

\begin{figure}[htbp]
\centering
\includegraphics[width=0.4\textwidth]{./figures/Bombay.png}
\caption[Hh]{\label{fig:orgec244c0}
Hh Blood Group}
\end{figure}
\end{frame}



\begin{frame}[label={sec:orgb9feeac}]{CDG diagnosis}
\begin{figure}[htbp]
\centering
\includegraphics[width=0.9\textwidth]{./figures/cdg_diag.png}
\caption{\label{fig:org0f7d76c}
CDG diagnosis}
\end{figure}
\end{frame}
\end{document}
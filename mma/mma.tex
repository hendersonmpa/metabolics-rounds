% Created 2018-03-01 Thu 09:38
\documentclass[presentation, smaller]{beamer}
\usepackage[utf8]{inputenc}
\usepackage[T1]{fontenc}
\usepackage{fixltx2e}
\usepackage{graphicx}
\usepackage{grffile}
\usepackage{longtable}
\usepackage{wrapfig}
\usepackage{rotating}
\usepackage[normalem]{ulem}
\usepackage{amsmath}
\usepackage{textcomp}
\usepackage{amssymb}
\usepackage{capt-of}
\usepackage{hyperref}
\hypersetup{colorlinks,linkcolor=white,urlcolor=blue}
\usepackage{textpos}
\usepackage{textgreek}
\usepackage[version=4]{mhchem}
\usepackage{chemfig}
\usepackage{siunitx}
\usepackage{gensymb}
\usepackage[usenames,dvipsnames]{xcolor}
\usepackage[T1]{fontenc}
\usepackage{lmodern}
\usepackage{verbatim}
\usepackage{tikz}
\usetikzlibrary{shapes.geometric,arrows,decorations.pathmorphing,backgrounds,positioning,fit,petri}
\usetheme{Hannover}
\usecolortheme{whale}
\author{Matthew Henderson, PhD, FCACB}
\date{\today}
\title{Methylmalonic Acidemia}
\institute[NSO]{Newborn Screening Ontario | The University of Ottawa}
\titlegraphic{\includegraphics[height=1cm,keepaspectratio]{../logos/NSO_logo.pdf}\includegraphics[height=1cm,keepaspectratio]{../logos/cheo-logo.png} \includegraphics[height=1cm,keepaspectratio]{../logos/UOlogoBW.eps}}
\hypersetup{
 pdfauthor={Matthew Henderson, PhD, FCACB},
 pdftitle={Methylmalonic Acidemia},
 pdfkeywords={},
 pdfsubject={},
 pdfcreator={Emacs 25.2.1 (Org mode 8.3.4)}, 
 pdflang={English}}
\begin{document}

\maketitle
%\logo{\includegraphics[width=1cm,height=1cm,keepaspectratio]{../logos/NSO_logo_small.pdf}~%
%    \includegraphics[width=1cm,height=1cm,keepaspectratio]{../logos/UOlogoBW.eps}%
%}

\vspace{220pt}
\beamertemplatenavigationsymbolsempty
\setbeamertemplate{caption}[numbered]
\setbeamerfont{caption}{size=\tiny}
% \addtobeamertemplate{frametitle}{}{%
% \begin{textblock*}{100mm}(.85\textwidth,-1cm)
% \includegraphics[height=1cm,width=2cm]{cat}
% \end{textblock*}}

\tikzstyle{chemical} = [rectangle, rounded corners, text width=5em, minimum height=1em,text centered, draw=black, fill=none]
\tikzstyle{hardware} = [rectangle, rounded corners, text width=5em, minimum height=1em,text centered, draw=black, fill=gray!30]
\tikzstyle{ms} = [rectangle, rounded corners, text width=5em, minimum height=1em,text centered, draw=orange, fill=none]
\tikzstyle{msw} = [rectangle, rounded corners, text width=7em, minimum height=1em,text centered, draw=orange, fill=none]
\tikzstyle{label} = [rectangle,text width=8em, minimum height=1em, text centered, draw=none, fill=none]
\tikzstyle{hl} = [rectangle, rounded corners, text width=5em, minimum height=1em,text centered, draw=black, fill=red!30]
\tikzstyle{box} = [rectangle, rounded corners, text width=5em, minimum height=5em,text centered, draw=black, fill=none]
\tikzstyle{arrow} = [thick,->,>=stealth]
\tikzstyle{hl-arrow} = [ultra thick,->,>=stealth,draw=red]

\section{Introduction}
\label{sec:orgheadline8}
\begin{frame}[label={sec:orgheadline1}]{History}
\begin{itemize}
\item First reported in 1967 by Oberholzer and Stokke
\item Methylmalonic mutase
\begin{itemize}
\item catalyzes the isomerization of methylmalonyl-CoA to succinyl-CoA
\end{itemize}
\item Requires adenosylcobalamin as a cofactor
\begin{itemize}
\item derived from vitamin B\(_{\text{12}}\)
\end{itemize}
\item Genetic heterogeneity was observed early
\begin{itemize}
\item response to large doses of B\(_{\text{12}}\)
\end{itemize}
\item B\(_{\text{12}}\) responsive have defects in adenosylcobalamin synthesis
\item B\(_{\text{12}}\) unresponsive have defects in the apoenzyme methylmalonyl mutase
\begin{description}
\item[{mut\^{}-}] little activity
\item[{mut\(^{\text{0}}\)}] no activity
\end{description}
\end{itemize}
\end{frame}

\begin{frame}[label={sec:orgheadline2}]{Methylmalonyl CoA Mutase}
\begin{itemize}
\item Upon entry to the mitochondria, the 32 amino acid mitochondrial
leader sequence at the N-terminus of the protein is cleaved, forming
the fully processed monomer.
\item The monomers associate into homodimers, and bind AdoCbl (one
for each monomer active site) to form the active holoenzyme
\end{itemize}

\includegraphics[width=.9\linewidth]{./figures/mut.pdf}
\end{frame}

\begin{frame}[label={sec:orgheadline3}]{Genetics}
\begin{itemize}
\item autosomal recessive
\item 1/50000
\begin{itemize}
\item 1/2-2/3 have mutase apoenzyme defect
\end{itemize}
\end{itemize}
\begin{block}{Methylmalonyl CoA Pathway}
\begin{itemize}
\item Methylmalonyl CoA Epimerase deficiency
\item Methylmalonyl CoA mutase deficiency
\begin{itemize}
\item mut\(^{\text{0}}\), mut\(^{\text{-}}\)
\item >200 mutations in MUT locus identified
\end{itemize}
\item Succinyl CoA ligase deficiency (SUCLG1, SUCLA1)
\end{itemize}
\end{block}

\begin{block}{Cobalamin Pathway}
\begin{itemize}
\item Adenosyltransferase deficiency (Cbl B)
\item Cbl A
\item Homocystinuria w MMA (Cbl C, D)
\item B\(_{\text{12}}\) deficiency
\begin{itemize}
\item Dietary (vegan)
\item Pernicious anemia
\end{itemize}
\item Transcobalamin II deficiency
\item B\(_{\text{12}}\) transport from lysosome defect (Cbl F)
\end{itemize}
\end{block}
\end{frame}

\begin{frame}[label={sec:orgheadline4}]{Different Types of MMA}
\begin{itemize}
\item Methylmalonyl CoA mutase deficiency
\begin{itemize}
\item mut\(^{\text{0}}\), mut\(^{\text{-}}\)
\end{itemize}
\item Adenosyltransferase deficiency (Cbl B)
\item Cbl A
\item Homocystinuria w MMA (Cbl C, D)
\item Methylmalonyl CoA Epimerase deficiency
\item B\(_{\text{12}}\) deficiency
\begin{itemize}
\item Dietary (vegan)
\item Pernicious anemia
\end{itemize}
\item Transcobalamin II deficiency
\item B\(_{\text{12}}\) transport from lysosome defect (Cbl F)
\item Succinyl CoA ligase deficiency (SUCLG1, SUCLA1)
\end{itemize}
\end{frame}

\begin{frame}[label={sec:orgheadline5}]{Methylmalonic  Acidemia Pathway}
\centering
\includegraphics[height=0.85\textheight]{./figures/expanded_mma_path.png}
\end{frame}

\begin{frame}[label={sec:orgheadline6}]{Cobalamin Transport and Metabolism}
\includegraphics[width=.9\linewidth]{./figures/cbl_path.png}
\end{frame}

\begin{frame}[label={sec:orgheadline7}]{Sources of Methylmalonic Acid}
\begin{itemize}
\item AA catabolism
\begin{itemize}
\item isoleucine
\item valine
\item threonine
\item methionine
\end{itemize}
\item FA
\begin{itemize}
\item \(\downarrow\) contribution to urine MMA
\end{itemize}
\end{itemize}

\href{http://www.genome.jp/kegg-bin/show_pathway?org_name=hsa&mapno=00640&mapscale=&show_description=hide}{propionate metabolism - KEGG}
\end{frame}


\section{Pathogenesis}
\label{sec:orgheadline11}


\begin{frame}[label={sec:orgheadline9}]{Methylmalonyl-CoA}
\begin{block}{PA}
\begin{itemize}
\item inhibits PCC \(\to\) \(\uparrow\) PA
\end{itemize}
\end{block}
\begin{block}{Hyperglycinemia}
\begin{itemize}
\item inhibits synthesis of glycine cleaving enzyme.
\item inhibits transport of malate, 2-ketoglutarate and isocitrate
\end{itemize}
\end{block}
\begin{block}{CBC}
\begin{itemize}
\item megaloblastosis in cblC defects
\begin{itemize}
\item \(\downarrow\) methylcobalamin
\end{itemize}
\end{itemize}
\end{block}

\begin{block}{Hyperammonemia}
\begin{itemize}
\item PA inhibits carbamylphosphate synthetase
\end{itemize}
\end{block}
\begin{block}{Ketosis}
\begin{itemize}
\item PA inhibits mitochondrial oxidation of succinic acid and 2-ketoglutaric acid
\end{itemize}
\end{block}
\end{frame}


\begin{frame}[label={sec:orgheadline10}]{Biochemical Markers}
\begin{itemize}
\item methylmalonic acid
\item 3-hydroxypropionic acid
\item methylcitrate
\begin{itemize}
\item condensation of propionyl-CoA \& oxaloacetic acid
\item metabolic end product and very stable
\end{itemize}
\end{itemize}
\end{frame}

\section{Laboratory Investigations}
\label{sec:orgheadline17}
\begin{frame}[label={sec:orgheadline12}]{NSO PA/MMA Screening Logic}
\begin{block}{Inital positive \textless{} 7 days}
(C3/C2 \(\ge\) 0.21 AND C3 \(\ge\) 4.0)
OR
(C3/C2 \(\ge\) 0.23 AND C3 \(\ge\) 3.5)
\end{block}
\begin{block}{Inital positive \textgreater{} 7 days}
(C3/C2 \(\ge\) 0.21 AND C3 \(\ge\) 2.6)
OR
(C3/C2 \(\ge\) 0.23 AND C3 \(\ge\) 2.4)
\begin{itemize}
\item Repeat overnight
\item No weekend reporting
\end{itemize}
\end{block}
\begin{block}{Alert}
C3/C2 \(\ge\) 0.3 AND C3 \(\ge\) 9.0
\begin{itemize}
\item Repeat same day
\item Weekend reporting
\end{itemize}
\end{block}
\begin{block}{Confirmation}
C3/C2 \(\ge\) 0.23 AND MCA \(\ge\) 0.5
\end{block}
\end{frame}

\begin{frame}[label={sec:orgheadline13}]{Clinical Chemistry}
\begin{itemize}
\item Acidosis in acute episodes
\begin{itemize}
\item accumulation of \(\beta\)-hydroxybutyrate and acetoacetate
\item Arterial pH as low as 6.9
\item Bicarb as low as 5 mEq/L
\end{itemize}
\item \(\uparrow\) lactic acid
\item Hypoglycemia
\item Hyperammonemia
\item Measure B12
\end{itemize}
\end{frame}

\begin{frame}[label={sec:orgheadline14}]{Biochemical Genetics}
\begin{block}{Plasma Amino Acids}
\begin{itemize}
\item \textpm{} glycine
\item \(\uparrow\) glutamine when hyperammonemia
\end{itemize}
\end{block}
\begin{block}{Plasma Acylcarnitines}
\begin{itemize}
\item \(\uparrow\) propionyl carnitine (C3)
\end{itemize}
\end{block}
\begin{block}{Urine Organic Acids}
\begin{itemize}
\item methylmalonic acid
\item 3-hydroxypropionic acid
\item methylcitric acid
\item tiglic acid / tiglyglycine
\item Ketones
\begin{itemize}
\item BHB
\item acetoacetate
\end{itemize}
\item lactic acid
\end{itemize}
\end{block}
\end{frame}

\begin{frame}[label={sec:orgheadline15}]{Urine Organic Acids}
\includegraphics[width=.9\linewidth]{./figures/mma_uoa.png}
\end{frame}

\begin{frame}[label={sec:orgheadline16}]{Typical Urine Methylmalonic Acid Values}
\begin{center}
\begin{tabular}{lr}
Clinical Status & mmol/mol creatinine\\
\hline
normal & 0-2\\
Mut\(^{\text{0}}\);presentation & 3000-13000\\
Mut\(^{\text{0}}\);steady state & 200-2000\\
B\(_{\text{12}}\) responsive;presentation & 2000\\
B\(_{\text{12}}\) responsive;steady-state & 90-300\\
B\(_{\text{12}}\) deficient infanct & 4500-5700\\
Transcobalamin II deficiency & 600\\
Cobalamin C,D & 270\\
Atypical-normal mutase & 200\\
Succinyl CoA ligase & 80-120\\
Methylmalonyl CoA epimerase deficiency & 30-300\\
\end{tabular}
\end{center}
\end{frame}

\section{Clinical Findings}
\label{sec:orgheadline23}
\begin{frame}[label={sec:orgheadline18}]{Acute presentation}
\begin{itemize}
\item Life-threatening illness early in life
\begin{itemize}
\item ketonuria
\begin{itemize}
\item acidosis
\item dehydration
\end{itemize}
\item vomiting
\item lethargy \(\to\) coma
\end{itemize}
\end{itemize}
\end{frame}

\begin{frame}[label={sec:orgheadline19}]{Recurrent Symptoms}
\begin{itemize}
\item ketotic episodes
\item infection
\item protein intolerance
\end{itemize}
\end{frame}

\begin{frame}[label={sec:orgheadline20}]{Long term}
\begin{itemize}
\item Variable developmental/cognitive outcomes
\begin{itemize}
\item appears linked to incidence of illness
\end{itemize}
\item hypotonic
\begin{itemize}
\item developmental delay
\end{itemize}
\item A subset with exclusively neurological presentation
\begin{itemize}
\item \textpm{} ketoacidosis
\item hypotonia \(\to\) hypertonia
\end{itemize}
\item Propionyl-CoA toxic to bone marrow
\begin{itemize}
\item neutropenia
\item transient thrombocytopenia in infancy
\end{itemize}
\item Osteoporosis
\item Pancreatitis
\item Cardiomyopathy
\item Fatty infiltration of liver on PM
\end{itemize}
\end{frame}

\begin{frame}[label={sec:orgheadline21}]{Neurological Findings}
\begin{columns}
\begin{column}{.5\columnwidth}
\begin{itemize}
\item Neonatal death
\begin{itemize}
\item spongy degeneration of white matter
\end{itemize}
\item Later death
\begin{itemize}
\item shrinkage and marbling in basal ganglia
\item neuronal loss
\item gliosis
\end{itemize}
\end{itemize}
\end{column}
\begin{column}{.5\columnwidth}
\includegraphics[width=.9\linewidth]{./figures/pa_mri.png}
\end{column}
\end{columns}
\end{frame}

\begin{frame}[label={sec:orgheadline22}]{Long-term Treatment}
\begin{block}{Diet}
\begin{itemize}
\item Limit Val, Ile, Thr, Met
\begin{itemize}
\item Monitor urine metabolites, plasma amino acids
\item Urine ketones (daily in infancy)
\item Monitor weight, nitrogen balance
\end{itemize}
\item Avoid fasting
\begin{itemize}
\item Catabolism
\item Propionate release from lipids
\end{itemize}
\end{itemize}
\end{block}

\begin{block}{Supplementation}
\begin{itemize}
\item Carnitine
\begin{itemize}
\item excretion of carnitine esters \(\to\) detoxification
\item Daily dose 60 to 100 mg/kg
\end{itemize}
\item Biotin
\begin{itemize}
\item conflicting information
\end{itemize}
\end{itemize}
\end{block}
\end{frame}
\end{document}

% Created 2018-03-29 Thu 12:53
\documentclass[presentation, smaller]{beamer}
\usepackage[utf8]{inputenc}
\usepackage[T1]{fontenc}
\usepackage{fixltx2e}
\usepackage{graphicx}
\usepackage{grffile}
\usepackage{longtable}
\usepackage{wrapfig}
\usepackage{rotating}
\usepackage[normalem]{ulem}
\usepackage{amsmath}
\usepackage{textcomp}
\usepackage{amssymb}
\usepackage{capt-of}
\usepackage{hyperref}
\hypersetup{colorlinks,linkcolor=white,urlcolor=blue}
\usepackage{textpos}
\usepackage{textgreek}
\usepackage[version=4]{mhchem}
\usepackage{chemfig}
\usepackage{siunitx}
\usepackage{gensymb}
\usepackage[usenames,dvipsnames]{xcolor}
\usepackage[T1]{fontenc}
\usepackage{lmodern}
\usepackage{verbatim}
\usepackage{tikz}
\usetikzlibrary{shapes.geometric,arrows,decorations.pathmorphing,backgrounds,positioning,fit,petri}
\usetheme{Hannover}
\usecolortheme{whale}
\author{Matthew Henderson, PhD, FCACB}
\date{\today}
\title{Methylmalonic Acidemia}
\institute[NSO]{Newborn Screening Ontario | The University of Ottawa}
\titlegraphic{\includegraphics[height=1cm,keepaspectratio]{../logos/NSO_logo.pdf}\includegraphics[height=1cm,keepaspectratio]{../logos/cheo-logo.png} \includegraphics[height=1cm,keepaspectratio]{../logos/UOlogoBW.eps}}
\hypersetup{
 pdfauthor={Matthew Henderson, PhD, FCACB},
 pdftitle={Methylmalonic Acidemia},
 pdfkeywords={},
 pdfsubject={},
 pdfcreator={Emacs 25.2.1 (Org mode 8.3.4)}, 
 pdflang={English}}
\begin{document}

\maketitle
%\logo{\includegraphics[width=1cm,height=1cm,keepaspectratio]{../logos/NSO_logo_small.pdf}~%
%    \includegraphics[width=1cm,height=1cm,keepaspectratio]{../logos/UOlogoBW.eps}%
%}

\vspace{220pt}
\beamertemplatenavigationsymbolsempty
\setbeamertemplate{caption}[numbered]
\setbeamerfont{caption}{size=\tiny}
% \addtobeamertemplate{frametitle}{}{%
% \begin{textblock*}{100mm}(.85\textwidth,-1cm)
% \includegraphics[height=1cm,width=2cm]{cat}
% \end{textblock*}}

\tikzstyle{chemical} = [rectangle, rounded corners, text width=5em, minimum height=1em,text centered, draw=black, fill=none]
\tikzstyle{hardware} = [rectangle, rounded corners, text width=5em, minimum height=1em,text centered, draw=black, fill=gray!30]
\tikzstyle{ms} = [rectangle, rounded corners, text width=5em, minimum height=1em,text centered, draw=orange, fill=none]
\tikzstyle{msw} = [rectangle, rounded corners, text width=7em, minimum height=1em,text centered, draw=orange, fill=none]
\tikzstyle{label} = [rectangle,text width=8em, minimum height=1em, text centered, draw=none, fill=none]
\tikzstyle{hl} = [rectangle, rounded corners, text width=5em, minimum height=1em,text centered, draw=black, fill=red!30]
\tikzstyle{box} = [rectangle, rounded corners, text width=5em, minimum height=5em,text centered, draw=black, fill=none]
\tikzstyle{arrow} = [thick,->,>=stealth]
\tikzstyle{hl-arrow} = [ultra thick,->,>=stealth,draw=red]

\section{Introduction}
\label{sec:orgheadline11}
\begin{frame}[label={sec:orgheadline1}]{History}
\begin{itemize}
\item First reported in 1967 by Oberholzer and Stokke
\item Defect in methylmalonic mutase
\begin{itemize}
\item catalyzes the isomerization of methylmalonyl-CoA to succinyl-CoA
\end{itemize}
\item Requires adenosylcobalamin as a cofactor
\begin{itemize}
\item derived from vitamin B\(_{\text{12}}\)
\end{itemize}
\item Genetic heterogeneity was observed early
\begin{itemize}
\item response to large doses of B\(_{\text{12}}\)
\end{itemize}
\item B\(_{\text{12}}\) responsive have defects in adenosylcobalamin synthesis
\item B\(_{\text{12}}\) unresponsive have defects in the apoenzyme methylmalonyl mutase
\begin{description}
\item[{mut\^{}-}] little activity
\item[{mut\(^{\text{0}}\)}] no activity
\end{description}
\end{itemize}
\end{frame}

\begin{frame}[label={sec:orgheadline2}]{Methylmalonyl CoA Mutase}
\begin{itemize}
\item Upon entry to the mitochondria, the 32 amino acid mitochondrial
leader sequence at the N-terminus of the protein is cleaved, forming
the fully processed monomer.
\item The monomers associate into homodimers, and bind AdoCbl (one
for each monomer active site) to form the active holoenzyme
\end{itemize}

\includegraphics[width=.9\linewidth]{./figures/mut.pdf}
\end{frame}

\begin{frame}[label={sec:orgheadline3}]{Genetics}
\begin{itemize}
\item autosomal recessive
\item 1/50000
\begin{itemize}
\item 1/2-2/3 have mutase apoenzyme defect
\end{itemize}
\item Methymalonyl pathway
\item Cobalamin transport and metabolism
\end{itemize}
\end{frame}

\begin{frame}[label={sec:orgheadline4}]{Methylmalonyl CoA Pathway}
\begin{itemize}
\item Methylmalonyl CoA Epimerase deficiency
\item Methylmalonyl CoA mutase deficiency
\begin{itemize}
\item mut\(^{\text{0}}\), mut\(^{\text{-}}\)
\item >200 mutations in MUT locus identified
\end{itemize}
\item Succinyl CoA ligase deficiency (SUCLG1, SUCLA1)
\end{itemize}
\end{frame}

\begin{frame}[label={sec:orgheadline5}]{Methylmalonyl-CoA  Pathway}
\centering
\includegraphics[height=0.85\textheight]{./figures/expanded_mma_path.png}
\end{frame}

\begin{frame}[label={sec:orgheadline6}]{Cobalamin Transport and Metabolism}
\begin{itemize}
\item Adenosyltransferase deficiency (Cbl B)
\item Cbl A
\item Homocystinuria w MMA (Cbl C, D)
\item B\(_{\text{12}}\) deficiency
\begin{itemize}
\item Dietary (vegan)
\item Pernicious anemia
\end{itemize}
\item Transcobalamin II deficiency
\item B\(_{\text{12}}\) transport from lysosome defect (Cbl F)
\end{itemize}
\end{frame}

\begin{frame}[label={sec:orgheadline7}]{Cobalamin Transport and Metabolism}
\includegraphics[width=.9\linewidth]{./figures/cbl_path.png}
\end{frame}

\begin{frame}[label={sec:orgheadline8}]{Methylmalonic Acid and Derivatives}
\centering
\vspace{6em}
\chemname{\chemfig[][scale=.5]{OH-[1]([2]=O)-[7]([6]<)-[1]([2]=O)-[7]S-CoA}}{\tiny S-methylmalonyl-CoA}
\hspace{2em}
\chemname{\chemfig[][scale=.5]{OH-[1]([2]=O)-[7]([6]<:)-[1]([2]=O)-[7]S-CoA}}{\tiny R-methylmalonyl-CoA}
\hspace{2em}
\chemname{\chemfig[][scale=.5]{OH-[1]([2]=O)-[7]([6]-)-[1]([2]=O)-[7]OH}}{\tiny methylmalonic acid}
\end{frame}

\begin{frame}[label={sec:orgheadline9}]{Sources of Methylmalonic Acid}
\begin{itemize}
\item AA catabolism
\begin{itemize}
\item isoleucine
\item valine
\item threonine
\item methionine
\end{itemize}
\item FA
\begin{itemize}
\item \(\downarrow\) contribution to urine MMA
\end{itemize}
\end{itemize}

\href{http://www.genome.jp/kegg-bin/show_pathway?scale=1.0&query=methylmalonyl-CoA&map=hsa00640&scale=&auto_image=&show_description=hide&multi_query=}{propionate metabolism - KEGG}
\end{frame}

\begin{frame}[label={sec:orgheadline10}]{Methylmalonyl-CoA}
\begin{itemize}
\item inhibits PCC \(\to\) \(\uparrow\) PA
\end{itemize}
\begin{block}{Hyperglycinemia}
\begin{itemize}
\item inhibits synthesis of glycine cleaving enzyme.
\item inhibits transport of malate, 2-ketoglutarate and isocitrate
\end{itemize}
\end{block}
\begin{block}{CBC}
\begin{itemize}
\item megaloblastosis in cblC defects
\begin{itemize}
\item \(\downarrow\) methylcobalamin
\end{itemize}
\end{itemize}
\end{block}

\begin{block}{Hyperammonemia}
\begin{itemize}
\item PA inhibits carbamylphosphate synthetase
\end{itemize}
\end{block}
\begin{block}{Ketosis}
\begin{itemize}
\item PA inhibits mitochondrial oxidation of succinic acid and 2-ketoglutaric acid
\end{itemize}
\end{block}
\end{frame}

\section{Laboratory Investigations}
\label{sec:orgheadline17}
\begin{frame}[label={sec:orgheadline12}]{NSO PA/MMA Screening Logic}
\begin{block}{Inital positive \textless{} 7 days}
(C3/C2 \(\ge\) 0.21 AND C3 \(\ge\) 4.0)
OR
(C3/C2 \(\ge\) 0.23 AND C3 \(\ge\) 3.5)
\end{block}
\begin{block}{Inital positive \textgreater{} 7 days}
(C3/C2 \(\ge\) 0.21 AND C3 \(\ge\) 2.6)
OR
(C3/C2 \(\ge\) 0.23 AND C3 \(\ge\) 2.4)
\begin{itemize}
\item Repeat overnight
\item No weekend reporting
\end{itemize}
\end{block}
\begin{block}{Alert}
C3/C2 \(\ge\) 0.3 AND C3 \(\ge\) 9.0
\begin{itemize}
\item Repeat same day
\item Weekend reporting
\end{itemize}
\end{block}
\begin{block}{Confirmation}
C3/C2 \(\ge\) 0.23 AND MCA \(\ge\) 0.5
\end{block}
\end{frame}

\begin{frame}[label={sec:orgheadline13}]{Clinical Chemistry}
\begin{itemize}
\item Acidosis in acute episodes
\begin{itemize}
\item accumulation of \(\beta\)-hydroxybutyrate and acetoacetate
\item Arterial pH as low as 6.9
\item Bicarb as low as 5 mEq/L
\end{itemize}
\item \(\uparrow\) lactic acid
\item Hypoglycemia
\item Hyperammonemia
\item Measure B12
\end{itemize}
\end{frame}

\begin{frame}[label={sec:orgheadline14}]{Biochemical Genetics}
\begin{block}{Plasma Amino Acids}
\begin{itemize}
\item \textpm{} glycine
\item \(\uparrow\) glutamine when hyperammonemia
\end{itemize}
\end{block}
\begin{block}{Plasma Acylcarnitines}
\begin{itemize}
\item \(\uparrow\) propionyl carnitine (C3)
\item \(\uparrow\) methylmalonyl carnitine (C4DC)
\end{itemize}
\end{block}

\begin{block}{Urine Organic Acids}
\begin{itemize}
\item methylmalonic acid
\item 3-hydroxypropionic acid
\item methylcitric acid
\item tiglic acid / tiglyglycine
\item Ketones
\begin{itemize}
\item BHB
\item acetoacetate
\end{itemize}
\item lactic acid
\end{itemize}
\end{block}
\end{frame}

\begin{frame}[label={sec:orgheadline15}]{Urine Organic Acids}
\includegraphics[width=.9\linewidth]{./figures/mma_uoa.png}
\end{frame}

\begin{frame}[label={sec:orgheadline16}]{Typical Urine Methylmalonic Acid Values}
\begin{center}
\begin{tabular}{lr}
Clinical Status & mmol/mol creatinine\\
\hline
normal & 0-2\\
Mut\(^{\text{0}}\);presentation & 3000-13000\\
Mut\(^{\text{0}}\);steady state & 200-2000\\
B\(_{\text{12}}\) responsive;presentation & 2000\\
B\(_{\text{12}}\) responsive;steady-state & 90-300\\
B\(_{\text{12}}\) deficient infant & 4500-5700\\
Transcobalamin II deficiency & 600\\
Cobalamin C,D & 270\\
Atypical-normal mutase & 200\\
Succinyl CoA ligase & 80-120\\
Methylmalonyl CoA epimerase deficiency & 30-300\\
\end{tabular}
\end{center}
\end{frame}

\section{Clinical Findings}
\label{sec:orgheadline25}
\begin{frame}[label={sec:orgheadline18}]{Initial presentation}
\begin{itemize}
\item Failure to thrive
\begin{itemize}
\item \(\downarrow\) linear growth
\end{itemize}
\item Skin lesions - candidasis
\item Life-threatening illness early in life
\begin{itemize}
\item ketonuria
\begin{itemize}
\item acidosis
\item dehydration
\end{itemize}
\item vomiting
\item lethargy \(\to\) coma
\end{itemize}
\end{itemize}
\end{frame}

\begin{frame}[label={sec:orgheadline19}]{Acute Treatment}
\begin{itemize}
\item aggressive intravenous hydration
\begin{itemize}
\item efficient renal excretion of MMA
\end{itemize}
\item Insulin and glucose \(\to\) anabolism
\item Acute HGH
\end{itemize}
\end{frame}

\begin{frame}[label={sec:orgheadline20}]{Recurrent Symptoms}
\begin{itemize}
\item ketotic episodes
\item infection
\item protein intolerance
\end{itemize}
\end{frame}

\begin{frame}[label={sec:orgheadline21}]{Long term}
\begin{itemize}
\item Variable developmental/cognitive outcomes
\begin{itemize}
\item appears linked to incidence of illness
\end{itemize}
\item severe hypotonia
\item stroke
\item hepatomegaly - normal LFTs
\item \downarrown renal function
\item pancreatitis
\item candidasis
\begin{itemize}
\item MMA inhibits maturation of hematopoietic cells and T cells
\end{itemize}
\end{itemize}
\end{frame}

\begin{frame}[label={sec:orgheadline22}]{Neurological Findings}
\begin{itemize}
\item due to acute episodes
\begin{itemize}
\item \(\downarrow\) cerebral perfusion
\item hypoglycemia
\item hyperammonemia
\end{itemize}
\item more common in apo enzyme defect than Cbl
\item Successful treatment \(\to\) normal IQ
\item 25 of 33 patients :
\begin{itemize}
\item ataxia = lack of coordination
\item dystonia = muscle contraction
\item dyskinesia = involuntary movement
\item dsyarthria = speech
\item chorea = rythmic contractions
\item clonus = jerky movements
\item tremors
\end{itemize}
\end{itemize}
\end{frame}

\begin{frame}[label={sec:orgheadline23}]{Long-term Treatment}
\begin{block}{Diet}
\begin{itemize}
\item Limit Val, Ile, Thr, Met
\begin{itemize}
\item Monitor urine metabolites, plasma amino acids
\item Urine ketones (daily in infancy)
\item Monitor weight, nitrogen balance
\end{itemize}
\item Avoid fasting
\begin{itemize}
\item Catabolism
\item Propionate release from lipids
\end{itemize}
\end{itemize}
\end{block}

\begin{block}{Supplementation}
\begin{itemize}
\item test for B\(_{\text{12}}\) response
\begin{itemize}
\item pharmacologic does of B\(_{\text{12}}\)
\end{itemize}
\item carnitine
\begin{itemize}
\item excretion of carnitine esters \(\to\) detoxification
\item Daily dose 60 to 100 mg/kg
\end{itemize}
\item ?alanine supplementation?
\end{itemize}
\end{block}
\end{frame}

\begin{frame}[label={sec:orgheadline24}]{Long-term Treatment}
\begin{block}{Medication}
\begin{itemize}
\item Production of PA by intestinal bacteria
\begin{itemize}
\item metronidazole
\item neomycin
\end{itemize}
\item HGH \(\to\) anabolism
\end{itemize}
\end{block}

\begin{block}{Transplantation}
\begin{itemize}
\item liver transplantation does not stop progressive neurological symptoms
\item liver \& kidney may be considered
\end{itemize}
\end{block}
\end{frame}
\end{document}

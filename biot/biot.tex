% Created 2018-07-04 Wed 17:10
\documentclass[presentation, smaller]{beamer}
\usepackage[utf8]{inputenc}
\usepackage[T1]{fontenc}
\usepackage{fixltx2e}
\usepackage{graphicx}
\usepackage{grffile}
\usepackage{longtable}
\usepackage{wrapfig}
\usepackage{rotating}
\usepackage[normalem]{ulem}
\usepackage{amsmath}
\usepackage{textcomp}
\usepackage{amssymb}
\usepackage{capt-of}
\usepackage{hyperref}
\hypersetup{colorlinks,linkcolor=gray,urlcolor=blue}
\usepackage{textpos}
\usepackage{textgreek}
\usepackage[version=4]{mhchem}
\usepackage{chemfig}
\usepackage{siunitx}
\usepackage{gensymb}
\usepackage[usenames,dvipsnames]{xcolor}
\usepackage[T1]{fontenc}
\usepackage{lmodern}
\usepackage{verbatim}
\usepackage{tikz}
\usetikzlibrary{shapes.geometric,arrows,decorations.pathmorphing,backgrounds,positioning,fit,petri}
\usetheme[height=20pt]{Boadilla}
\usecolortheme[RGB={170,160,80}]{{structure}}
\author{Matthew Henderson, PhD, FCACB}
\date{\today}
\title{Laboratory diagnosis of biotinidase deficiency: 2017 update}
\institute[NSO]{Newborn Screening Ontario}
\titlegraphic{\includegraphics[height=1cm,keepaspectratio]{../logos/NSO_logo.pdf}\includegraphics[height=1cm,keepaspectratio]{../logos/cheo-logo.png} \includegraphics[height=1cm,keepaspectratio]{../logos/UOlogoBW.eps}}
\hypersetup{
 pdfauthor={Matthew Henderson, PhD, FCACB},
 pdftitle={Laboratory diagnosis of biotinidase deficiency: 2017 update},
 pdfkeywords={},
 pdfsubject={},
 pdfcreator={Emacs 25.2.1 (Org mode 8.3.4)}, 
 pdflang={English}}
\begin{document}

\maketitle

\logo{\includegraphics[width=1cm,height=1cm,keepaspectratio]{../logos/NSO_logo_small.pdf}}

\vspace{220pt}
\beamertemplatenavigationsymbolsempty
\setbeamertemplate{caption}[numbered]
\setbeamerfont{caption}{size=\tiny}
% \addtobeamertemplate{frametitle}{}{%
% \begin{textblock*}{100mm}(.85\textwidth,-1cm)
% \includegraphics[height=1cm,width=2cm]{cat}
% \end{textblock*}}


\tikzstyle{chemical} = [rectangle, rounded corners, text width=5em, minimum height=1em,text centered, draw=black, fill=none]
\tikzstyle{hardware} = [rectangle, rounded corners, text width=5em, minimum height=1em,text centered, draw=black, fill=gray!30]
\tikzstyle{ms} = [rectangle, rounded corners, text width=5em, minimum height=1em,text centered, draw=orange, fill=none]
\tikzstyle{msw} = [rectangle, rounded corners, text width=7em, minimum height=1em,text centered, draw=orange, fill=none]
\tikzstyle{label} = [rectangle,text width=8em, minimum height=1em, text centered, draw=none, fill=none]
\tikzstyle{hl} = [rectangle, rounded corners, text width=5em, minimum height=1em,text centered, draw=black, fill=red!30]
\tikzstyle{box} = [rectangle, rounded corners, text width=5em, minimum height=5em,text centered, draw=black, fill=none]
\tikzstyle{arrow} = [thick,->,>=stealth]
\tikzstyle{hl-arrow} = [ultra thick,->,>=stealth,draw=red]


\section{Background}
\label{sec:orgheadline8}

\begin{frame}[label={sec:orgheadline1}]{Biotinidase Deficiency}
\begin{itemize}
\item autosomal recessively inherited disorder of biotin recycling
\begin{itemize}
\item associated with secondary alterations in amino acid, carbohydrate,
and fatty acid metabolism.
\end{itemize}
\item caused by absent or markedly deficient activity of biotinidase
\begin{itemize}
\item cytosolic enzyme that liberates biotin from biocytin during the
normal proteolytic turnover of holocarboxylases and other biotiny-
lated proteins.
\end{itemize}

\item Based on newborn screening outcome data from 2006, the incidence of
profound biotinidase deficiency in the United States is estimated at
1/80,000 births
\item partial biotinidase deficiency between 1/31,000 and 1/40,000
\end{itemize}
\end{frame}

\begin{frame}[label={sec:orgheadline2}]{Biotinidase Deficiency}
\begin{itemize}
\item Diagnosis of biotinidase deficiency is based on demonstrating
deficient enzyme activity in serum or plasma

\item Patients with profound biotinidase deficiency have less than 10\% of
mean normal serum activity

\item Patients with the partial biotinidase deficiency variant have 10–30\%
of mean normal serum activity
\begin{itemize}
\item are largely asymptomatic
\end{itemize}

\item Confirmation of biotinidase deficiency by DNA analysis, by either
allele-targeted methods or full-gene sequencing, may be useful.
\end{itemize}
\end{frame}


\begin{frame}[label={sec:orgheadline3}]{Biotin Cycle}
\includegraphics[width=.9\linewidth]{./figures/biot_cycle.png}
\end{frame}


\begin{frame}[label={sec:orgheadline4}]{Biotinidase}
\begin{itemize}
\item Biotinidase is a monomeric enzyme encoded by a single gene (BTD) located on chromosome 3p25
\begin{itemize}
\item comprises 543 amino acid residues, including 41 amino acids of a potential signal peptide.
\end{itemize}

\item Three publicly available databases of biotinidase variants:
\begin{itemize}
\item \href{https://grenada.lumc.nl/LOVD2/shared1/home.php?select_db=BTD}{Leiden Open Variation Database}
\item \href{https://www.ncbi.nlm.nih.gov/clinvar/}{ClinVar}
\item \href{http://www.arup.utah.edu/database/BTD/BTD_welcome.php}{ARUP:Biotinidase Deficiency and BTD}
\end{itemize}

\item 204 biotinidase variants are consolidated in the ARUP database,
\begin{itemize}
\item >150 categorized as pathogenic
\item 145 missense changes
\item Four common pathogenic variants cause profound biotinidase deficiency.
\end{itemize}
\end{itemize}
\end{frame}

\begin{frame}[label={sec:orgheadline5}]{Pathogenic Variants}
\begin{itemize}
\item Among children ascertained because of clinical symptoms, the two
most commonly reported variants are:

\begin{itemize}
\item c.98\(_{\text{104delinsTCC}}\) in exon 2
\begin{itemize}
\item seven-base deletion/three-base insertion
\item occurring in at least one allele in approximately 50\% of individuals
\end{itemize}

\item p.Arg538Cys in exon 4
\begin{itemize}
\item occurring at least once in 30\% of children.
\end{itemize}

\item These variants result in complete absence of biotinidase protein.
\end{itemize}

\item Other relatively common variants discovered by newborn screening are:
\begin{itemize}
\item p.Gln456His, associated with profound deficiency

\item p.Asp444His, a substitution that reduces enzymatic activity by about 50\%.

\item The p.Asp444His variant in trans with a severe BTD pathogenic variant is associated with partial biotinidase deficiency,
\item p.Asp444His in cis with p.Ala171Thr (i.e., as the double mutant p.[(Ala171Thr); (Asp444His)]), results in a profound biotinidase deficiency allele.
\end{itemize}
\end{itemize}
\end{frame}


\begin{frame}[label={sec:orgheadline6}]{Partial \& Profound Deficiency}
\begin{block}{Profound Deficiency}
\begin{itemize}
\item Initially, most symptomatic children with biotinidase deficiency were found to have 3\% of mean serum biotinidase activity of normal individuals.
\item Three standard deviations above this mean, corresponding to 10\% of mean normal activity, was taken as the threshold below which individuals were considered to have profound biotinidase deficiency.
\end{itemize}
\end{block}

\begin{block}{Partial Deficiency}
\begin{itemize}
\item With NBS for biotinidase deficiency babies were identified with about 25\% of mean normal activity.
\item Essentially all of these children had the p.Asp444His variant as one of their alleles
\item This variant, together with a variant for profound deficiency on the other allele, results in 10–30\% of mean normal biotinidase activity.
\item These children are considered to have partial biotinidase deficiency.
\end{itemize}
\end{block}
\end{frame}


\begin{frame}[label={sec:orgheadline7}]{NBS for partial deficiency}
\begin{itemize}
\item A retrospective study reviewing clinical histories of
individuals with profound (22) or partial (120) biotinidase
deficiency identified by newborn screening supports the long-term
benefit of treatment and management of both populations. \footnote{Outcomes of individuals with profound and partial
biotinidase deficiency ascertained by newborn screening in Michigan
over 25 years, Genetics In Medicine, 2014/08/21/}
\end{itemize}
\end{frame}



\section{Laboratory}
\label{sec:orgheadline12}

\begin{frame}[label={sec:orgheadline9}]{NBS for Biotinidase Deficiency}
\begin{itemize}
\item Fluorimetric method showed 100\% sensitivity and 97\% specificity. \footnote{Comparison of spectrophotometric and fluorimetric
methods in evaluation of biotinidase deficiency. J Med Biochem 2016;35:123–129.}
\begin{itemize}
\item biotinyl-6-aminoquinoline
\end{itemize}
\item Spectrophotometric method showed 90.5\% sensitivity and 93.7\% specificity.
\begin{itemize}
\item biotin-4-amidobenzoic acid
\end{itemize}
\end{itemize}
\end{frame}


\begin{frame}[label={sec:orgheadline10}]{Prematurity and False Positives}
\begin{center}
\begin{tabular}{lrrrr}
Disorder & Total & \% & Prem & \%\\
\hline
Galactosemia & 480 & 0.16 & 41 & 8.5\\
Biotinidase & 156 & 0.05 & 73 & 46.8\\
\end{tabular}
\end{center}

\begin{itemize}
\item Although approximately half of the infants with false-positive
results for biotinidase deficiency were premature, less than 1\% of
the premature infants had false-positive results for this test.\footnote{Comparison of the effects of season and prematurity on the
enzymatic newborn screening tests for galactosemia and biotinidase
deficiency. Screening 1993}
\end{itemize}
\end{frame}


\begin{frame}[label={sec:orgheadline11}]{Seasonal Variation}
\includegraphics[width=.9\linewidth]{./figures/seasonal.png}
\end{frame}
\end{document}

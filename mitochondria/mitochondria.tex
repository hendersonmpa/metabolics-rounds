% Created 2019-02-11 Mon 21:08
% Intended LaTeX compiler: pdflatex
\documentclass[presentation, smaller]{beamer}
\usepackage[utf8]{inputenc}
\usepackage[T1]{fontenc}
\usepackage{graphicx}
\usepackage{grffile}
\usepackage{longtable}
\usepackage{wrapfig}
\usepackage{rotating}
\usepackage[normalem]{ulem}
\usepackage{amsmath}
\usepackage{textcomp}
\usepackage{amssymb}
\usepackage{capt-of}
\usepackage{hyperref}
\hypersetup{colorlinks,linkcolor=white,urlcolor=blue}
\usepackage{textpos}
\usepackage{textgreek}
\usepackage[version=4]{mhchem}
\usepackage{chemfig}
\usepackage{siunitx}
\usepackage{gensymb}
\usepackage[usenames,dvipsnames]{xcolor}
\usepackage[T1]{fontenc}
\usepackage{lmodern}
\usepackage{verbatim}
\usepackage{tikz}
\usetikzlibrary{shapes.geometric,arrows,decorations.pathmorphing,backgrounds,positioning,fit,petri}
\usetheme{Hannover}
\usecolortheme{whale}
\author{Matthew Henderson, PhD, FCACB}
\date{\today}
\title{Hinterland Who's Who: Mitochondria}
\institute[NSO]{Newborn Screening Ontario | The University of Ottawa}
\titlegraphic{\includegraphics[height=1cm,keepaspectratio]{../logos/NSO_logo.pdf}\includegraphics[height=1cm,keepaspectratio]{../logos/cheo-logo.png} \includegraphics[height=1cm,keepaspectratio]{../logos/UOlogoBW.eps}}
\hypersetup{
 pdfauthor={Matthew Henderson, PhD, FCACB},
 pdftitle={Hinterland Who's Who: Mitochondria},
 pdfkeywords={},
 pdfsubject={},
 pdfcreator={Emacs 26.1 (Org mode 9.1.9)}, 
 pdflang={English}}
\begin{document}

\maketitle

%\logo{\includegraphics[width=1cm,height=1cm,keepaspectratio]{../logos/NSO_logo_small.pdf}~%
%    \includegraphics[width=1cm,height=1cm,keepaspectratio]{../logos/UOlogoBW.eps}%
%}

\vspace{220pt}
\beamertemplatenavigationsymbolsempty
\setbeamertemplate{caption}[numbered]
\setbeamerfont{caption}{size=\tiny}
% \addtobeamertemplate{frametitle}{}{%
% \begin{textblock*}{100mm}(.85\textwidth,-1cm)
% \includegraphics[height=1cm,width=2cm]{cat}
% \end{textblock*}}

\tikzstyle{chemical} = [rectangle, rounded corners, text width=5em, minimum height=1em,text centered, draw=black, fill=none]
\tikzstyle{hardware} = [rectangle, rounded corners, text width=5em, minimum height=1em,text centered, draw=black, fill=gray!30]
\tikzstyle{ms} = [rectangle, rounded corners, text width=5em, minimum height=1em,text centered, draw=orange, fill=none]
\tikzstyle{msw} = [rectangle, rounded corners, text width=7em, minimum height=1em,text centered, draw=orange, fill=none]
\tikzstyle{label} = [rectangle,text width=8em, minimum height=1em, text centered, draw=none, fill=none]
\tikzstyle{hl} = [rectangle, rounded corners, text width=5em, minimum height=1em,text centered, draw=black, fill=red!30]
\tikzstyle{box} = [rectangle, rounded corners, text width=5em, minimum height=5em,text centered, draw=black, fill=none]
\tikzstyle{arrow} = [thick,->,>=stealth]
\tikzstyle{hl-arrow} = [ultra thick,->,>=stealth,draw=red]


\section{Background}
\label{sec:org12f41fc}
\begin{frame}[label={sec:org1e5c08f}]{Mitochondria}
\begin{itemize}
\item A double-membrane-bound organelle found in most eukaryotic organisms
\begin{itemize}
\item liver cells can have more than 2000.
\item red blood cells have no mitochondria
\end{itemize}

\item The organelle is composed of an outer membrane, intermembrane
space, and inner membrane with cristae and matrix.

\item The mitochondrion has its own independent genome that shows
substantial similarity to bacterial genomes

\item Mitochondrial proteins (proteins transcribed from mt DNA)
vary depending on the tissue and the species. In humans, 615
distinct types of protein have been identified from cardiac
mitochondria
\end{itemize}
\end{frame}

\begin{frame}[label={sec:orgfcfdfd0}]{Maternal Inheritance}
\begin{itemize}
\item Mitochondria and therefore the mitochondrial DNA, usually come from
the egg only.
\item The egg cell contains relatively few mitochondria
\begin{itemize}
\item these mitochondria divide to populate the cells
\end{itemize}
\item Sperm mitochondria enter the egg, but do not contribute genetic
information to the embryo.
\begin{itemize}
\item paternal mitochondria are marked with ubiquitin for destruction
inside the embryo.
\end{itemize}
\end{itemize}
\end{frame}

\begin{frame}[label={sec:org964335c}]{Heteroplasmy}
\begin{itemize}
\item Heteroplasmy is the presence of more than one type of organellar
genome (mitochondrial DNA or plastid DNA) within a cell or
individual.

\item It is an important factor in considering the severity of
mitochondrial diseases.

\item Most eukaryotic cells contain many hundreds of mitochondria with
hundreds of copies of mitochondrial DNA, it is common for mutations
to affect only some mitochondria, leaving most unaffected.

\item Although detrimental scenarios are well-studied, heteroplasmy can
also be beneficial. For example, centenarians show a higher than
average degree of heteroplasmy.[1]

\item Microheteroplasmy is present in most individuals. This refers to
hundreds of independent mutations in one organism, with each
mutation found in about 1–2\% of all mitochondrial genomes.[2]
\end{itemize}
\end{frame}

\begin{frame}[label={sec:org19cf295}]{Muller's Ratchet and the Mitochondrial Bottleneck}
\begin{itemize}
\item Entities undergoing uniparental inheritance and with little to no
recombination may be expected to be subject to Muller's ratchet
\begin{itemize}
\item the inexorable accumulation of deleterious mutations until functionality
is lost.
\end{itemize}
\item mitochondria avoid this buildup through a developmental process
known as the mtDNA bottleneck. 
\begin{itemize}
\item Cell-level selection acts to remove those cells with more deleterious mtDNA
\end{itemize}
\end{itemize}

\begin{block}{Fusion}
\begin{itemize}
\item response to cellular stress
\item enabling genetic complementation, fusion of the mitochondria allows
for two mitochondrial genomes with different defects within the same
organelle to individually encode what the other lacks.
\end{itemize}
\end{block}
\end{frame}

\begin{frame}[label={sec:orgdee1a30}]{Replication}
\begin{itemize}
\item Mitochondria divide by binary fission, similar to bacterial cell division

\item mammalian mitochondria replicate their DNA and divide mainly in response
to the energy needs of the cell,rather than in phase with the cell cycle.
\begin{itemize}
\item When the energy needs of a cell are high, mitochondria grow and
divide.
\item When the energy use is low, mitochondria are destroyed
or become inactive.
\end{itemize}

\item mitochondria are apparently randomly distributed to the daughter
cells during the division of the cytoplasm.
\end{itemize}
\end{frame}


\begin{frame}[label={sec:org5c66d09}]{Human Mitochondrial DNA}
\begin{itemize}
\item The human mitochondrial genome is a circular DNA molecule of about
16 kilobases.
\item It encodes 37 genes: 13 for subunits of respiratory complexes I,
III, IV and V,
\item 22 for mitochondrial tRNA (for the 20 standard amino acids, plus an
extra gene for leucine and serine)
\item 2 for rRNA.
\item One mitochondrion can contain two to ten copies of its DNA.
\end{itemize}
\end{frame}

\begin{frame}[label={sec:org9d4764d}]{Alternative genetic code}
\begin{itemize}
\item The mitochondria of many eukaryotes, including most plants, use the
standard code.
\end{itemize}

\begin{table}[htbp]
\caption[mito code]{\label{tab:org468389c}
Exceptions to the standard genetic code in mitochondria}
\centering
\begin{tabular}{llll}
Organism & Codon & Standard & Mitochondria\\
\hline
Mammals & AGA, AGG & Arginine & Stop codon\\
Invertebrates & AGA, AGG & Arginine & Serine\\
Fungi & CUA & Leucine & Threonine\\
All of the above & AUA & Isoleucine & Methionine\\
 & UGA & Stop codon & Tryptophan\\
\end{tabular}
\end{table}

\begin{itemize}
\item the AUA, AUC, and AUU codons are all allowable start codons.
\item Some of these differences should be regarded as pseudo-changes in
the genetic code due to the phenomenon of RNA editing, which is
common in mitochondria.
\end{itemize}
\end{frame}


\begin{frame}[label={sec:orgd3563d6}]{Mitochondrial Disease}
\begin{itemize}
\item Mitochondrial diseases are about 15\% of the time caused by mutations
in the mitochondrial DNA that affect mitochondrial function.
\item Other mitochondrial diseases are caused by mutations in genes of the
nuclear DNA, whose gene products are imported into the mitochondria
as well as acquired mitochondrial conditions.
\end{itemize}
\end{frame}

\section{Biochemical Functions Relevant to IMD}
\label{sec:org6b9c34c}

\begin{frame}[label={sec:org64a9d32}]{Pyruvate and the Tricyclic Acid Cycle}
\begin{itemize}
\item one molecule of glucose breaks down into two molecules of pyruvate
\item Pyruvate is converted into acetyl-coenzyme A, which is the main
input for a series of reactions known as the Krebs cycle
\item Pyruvate is also converted to oxaloacetate by an anaplerotic
reaction, which replenishes Krebs cycle intermediates; also, the
oxaloacetate is used for gluconeogenesis
\end{itemize}

\url{https://en.wikipedia.org/wiki/Citric\_acid\_cycle\#/media/File:Citric\_acid\_cycle\_with\_aconitate\_2.svg}
\end{frame}

\begin{frame}[label={sec:org9953c48}]{Ketogenesis \& Ketolysis}
\begin{itemize}
\item Ketone bodies are produced mainly in the mitochondria of liver cells, and synthesis can occur in response to an unavailability of blood glucose, such as during fasting
\end{itemize}
\url{https://en.wikipedia.org/wiki/Ketogenesis\#/media/File:Ketogenesis.svg}
\end{frame}
\begin{frame}[label={sec:orgfa182fd}]{Electron Transport Chain}
\begin{itemize}
\item Energy obtained through the transfer of electrons down the ETC is used to pump protons from the mitochondrial matrix into the intermembrane space
\item creats an electrochemical proton gradient (\(\delta\) pH) across the IMM.
\item largely responsible for the mitochondrial membrane potential (ΔΨM).
\item allows ATP synthase to use the flow of H+ through the enzyme back into the matrix to generate ATP from ADP and Pi.
\item Complex I (NADH coenzyme Q reductase) accepts electrons from the Krebs cycle electron carrier NADH
\item passes them to CoQ (ubiquinone; labeled Q),
\item CoQ also receives electrons from complex II (succinate dehydrogenase).
\item CoQ passes electrons to complex III (cytochrome bc1 complex; labeled III), which passes them to cytochrome c (cyt c).
\item Cyt c passes electrons to Complex IV (cytochrome c oxidase; labeled IV), which uses the electrons and hydrogen ions to reduce molecular oxygen to water.
\end{itemize}
\end{frame}
\begin{frame}[label={sec:orgcfaaaf8}]{Oxidative phosphorylation}
\begin{center}
\includegraphics[width=.9\linewidth]{./figures/etc.pdf}
\end{center}

The complete breakdown of glucose in the presence of oxygen is called cellular respiration. The last steps of this process occur in mitochondria. The reduced molecules NADH and FADH2 are generated by the Krebs cycle, glycolysis, and pyruvate processing. These molecules pass electrons to an electron transport chain, which uses the energy released to create a proton gradient across the inner mitochondrial membrane. ATP synthase then uses the energy stored in this gradient to make ATP. This process is called oxidative phosphorylation because it uses energy released by the oxidation of NADH and FADH2 to phospolyrize ADP into ATP. 

\begin{itemize}
\item \url{https://en.wikipedia.org/wiki/Electron\_transport\_chain\#/media/File:ATP-Synthase.svg}
\end{itemize}
\end{frame}

\begin{frame}[label={sec:orgd74cc47}]{Other Biochemical Functions Relevant to IMD}
\begin{itemize}
\item Mitochondrial Fatty Acid Oxidation

\item Urea Cycle

\item Heme Biosynthesis
\end{itemize}
\end{frame}
\end{document}

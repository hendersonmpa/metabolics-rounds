% Created 2019-02-14 Thu 10:50
% Intended LaTeX compiler: pdflatex
\documentclass[presentation, smaller]{beamer}
\usepackage[utf8]{inputenc}
\usepackage[T1]{fontenc}
\usepackage{graphicx}
\usepackage{grffile}
\usepackage{longtable}
\usepackage{wrapfig}
\usepackage{rotating}
\usepackage[normalem]{ulem}
\usepackage{amsmath}
\usepackage{textcomp}
\usepackage{amssymb}
\usepackage{capt-of}
\usepackage{hyperref}
\hypersetup{colorlinks,linkcolor=white,urlcolor=blue}
\usepackage{textpos}
\usepackage{textgreek}
\usepackage[version=4]{mhchem}
\usepackage{chemfig}
\usepackage{siunitx}
\usepackage{gensymb}
\usepackage[usenames,dvipsnames]{xcolor}
\usepackage[T1]{fontenc}
\usepackage{lmodern}
\usepackage{verbatim}
\usepackage{tikz}
\usetikzlibrary{shapes.geometric,arrows,decorations.pathmorphing,backgrounds,positioning,fit,petri}
\usetheme{Hannover}
\usecolortheme{whale}
\author{Matthew Henderson, PhD, FCACB}
\date{\today}
\title{Hinterland Who's Who: Mitochondria}
\institute[NSO]{Newborn Screening Ontario | The University of Ottawa}
\titlegraphic{\includegraphics[height=1cm,keepaspectratio]{../logos/NSO_logo.pdf}\includegraphics[height=1cm,keepaspectratio]{../logos/cheo-logo.png} \includegraphics[height=1cm,keepaspectratio]{../logos/UOlogoBW.eps}}
\hypersetup{
 pdfauthor={Matthew Henderson, PhD, FCACB},
 pdftitle={Hinterland Who's Who: Mitochondria},
 pdfkeywords={},
 pdfsubject={},
 pdfcreator={Emacs 26.1 (Org mode 9.1.9)}, 
 pdflang={English}}
\begin{document}

\maketitle

%\logo{\includegraphics[width=1cm,height=1cm,keepaspectratio]{../logos/NSO_logo_small.pdf}~%
%    \includegraphics[width=1cm,height=1cm,keepaspectratio]{../logos/UOlogoBW.eps}%
%}

\vspace{220pt}
\beamertemplatenavigationsymbolsempty
\setbeamertemplate{caption}[numbered]
\setbeamerfont{caption}{size=\tiny}
% \addtobeamertemplate{frametitle}{}{%
% \begin{textblock*}{100mm}(.85\textwidth,-1cm)
% \includegraphics[height=1cm,width=2cm]{cat}
% \end{textblock*}}

\tikzstyle{chemical} = [rectangle, rounded corners, text width=5em, minimum height=1em,text centered, draw=black, fill=none]
\tikzstyle{hardware} = [rectangle, rounded corners, text width=5em, minimum height=1em,text centered, draw=black, fill=gray!30]
\tikzstyle{ms} = [rectangle, rounded corners, text width=5em, minimum height=1em,text centered, draw=orange, fill=none]
\tikzstyle{msw} = [rectangle, rounded corners, text width=7em, minimum height=1em,text centered, draw=orange, fill=none]
\tikzstyle{label} = [rectangle,text width=8em, minimum height=1em, text centered, draw=none, fill=none]
\tikzstyle{hl} = [rectangle, rounded corners, text width=5em, minimum height=1em,text centered, draw=black, fill=red!30]
\tikzstyle{box} = [rectangle, rounded corners, text width=5em, minimum height=5em,text centered, draw=black, fill=none]
\tikzstyle{arrow} = [thick,->,>=stealth]
\tikzstyle{hl-arrow} = [ultra thick,->,>=stealth,draw=red]


\section{Background}
\label{sec:org55c905b}
\begin{frame}[label={sec:orgf16b254}]{Mitochondria}
\begin{itemize}
\item A double-membrane-bound organelle found in most eukaryotic organisms
\begin{itemize}
\item liver cells can have more than 2000
\item red blood cells have no mitochondria
\end{itemize}

\item The organelle is composed of an outer membrane, intermembrane
space, and inner membrane with cristae and matrix.

\item The mitochondrion has its own independent genome that shows
substantial similarity to bacterial genomes

\item Mitochondrial proteins transcribed from mtDNA vary depending on the
tissue and the species.

\item In humans, 615 distinct types of protein have been identified from
cardiac mitochondria.
\end{itemize}
\end{frame}

\begin{frame}[label={sec:org1d20d1b}]{Mitochondria}
\begin{center}
\includegraphics[width=.9\linewidth]{./figures/Mitochondrion_mini.png}
\end{center}
\end{frame}

\begin{frame}[label={sec:org2dc5e91}]{Mitochondria}
\begin{figure}[htbp]
\centering
\includegraphics[width=0.8\textwidth]{./figures/HeLa_mtGFP.jpg}
\caption[hela]{\label{fig:org59cf2ae}
HeLa Cells mtGFP}
\end{figure}
\end{frame}


\begin{frame}[label={sec:org1be77f0}]{Maternal Inheritance}
\begin{itemize}
\item Mitochondria and therefore the mtDNA, usually come from the egg
\begin{itemize}
\item the egg cell contains relatively few mitochondria
\item these mitochondria divide to populate the cells
\end{itemize}
\item paternal mitochondria are marked with ubiquitin for destruction
inside the embryo.
\item mitochondria are randomly distributed to the daughter cells during
the division of the cytoplasm.
\end{itemize}

\begin{figure}[htbp]
\centering
\includegraphics[width=0.8\textwidth]{./figures/Mitochondrial_Bottleneck.png}
\caption[mom]{\label{fig:org67c4d3f}
Maternal Inheritance}
\end{figure}
\end{frame}

\begin{frame}[label={sec:orgcce0268}]{Heteroplasmy}
\begin{itemize}
\item Heteroplasmy is the presence of more than one type of organellar
genome within a cell or individual.

\item It is an important factor in considering the severity of
mitochondrial diseases.
\begin{itemize}
\item can also be beneficial
\end{itemize}

\item Microheteroplasmy is present in most individuals.
\begin{itemize}
\item hundreds of independent mutations, with each mutation found in
about 1–2\% of all mitochondrial genomes.
\end{itemize}
\end{itemize}


\begin{figure}[htbp]
\centering
\includegraphics[width=0.8\textwidth]{./figures/heteroplasmy.png}
\caption[heter]{\label{fig:orgea49b24}
heteroplasmy}
\end{figure}
\end{frame}


\begin{frame}[label={sec:orgabc120f}]{Muller's Ratchet and the Mitochondrial Bottleneck}
\begin{itemize}
\item Applies to entities undergoing uniparental inheritance and with little to no
recombination
\begin{itemize}
\item the inexorable accumulation of deleterious mutations until functionality
is lost.
\end{itemize}
\item mitochondria avoid this buildup through a developmental process
known as the mtDNA bottleneck. 
\begin{itemize}
\item selection acts to remove those cells with more deleterious mtDNA
\end{itemize}
\end{itemize}

\begin{figure}[htbp]
\centering
\includegraphics[width=0.8\textwidth]{./figures/bottle_neck.jpg}
\caption[bottle]{\label{fig:orgbc048b3}
Mitochondrial bottle neck}
\end{figure}
\end{frame}


\begin{frame}[label={sec:orgc0b06e9}]{Fusion}
\begin{columns}
\begin{column}{0.5\columnwidth}
\begin{itemize}
\item response to cellular stress
\begin{itemize}
\item mtDNA damage
\end{itemize}
\item enables genetic complementation
\end{itemize}
\end{column}

\begin{column}{0.5\columnwidth}
\begin{figure}[htbp]
\centering
\includegraphics[width=1\textwidth]{./figures/nrm1125-f1.jpg}
\caption[fusion]{\label{fig:org4b91074}
Mitochondrial fusion}
\end{figure}
\end{column}
\end{columns}
\end{frame}



\begin{frame}[label={sec:org61b741b}]{Replication and Fission}
\begin{itemize}
\item Mitochondria divide by binary fission, similar to bacterial cell division

\item mammalian mitochondria replicate their DNA and divide mainly in response
to the energy needs of the cell, not in phase with the cell cycle.
\begin{itemize}
\item When the energy needs of a cell are high, mitochondria grow and
divide.
\item When the energy use is low, mitochondria are destroyed
or become inactive.
\end{itemize}
\end{itemize}
\end{frame}

\begin{frame}[label={sec:orgd0f2dda}]{Human Mitochondrial DNA}
\begin{columns}
\begin{column}{.7\columnwidth}
\begin{itemize}
\item a circular DNA molecule \textasciitilde{} 16 kb
\item encodes 37 genes
\begin{itemize}
\item 13 for subunits of respiratory complexes I, III, IV and V
\item 22 for mitochondrial tRNA
\begin{itemize}
\item 20 standard amino acids, plus extra gene for leu and ser
\end{itemize}
\item 2 for rRNA.
\end{itemize}
\item One mitochondrion can contain two to ten copies of its DNA.
\end{itemize}
\end{column}

\begin{column}{.3\columnwidth}
\begin{figure}[htbp]
\centering
\includegraphics[width=1\textwidth]{./figures/mitochondrial_genome.png}
\caption[mtdna]{\label{fig:orgc98abfe}
Human mitochondrial genome}
\end{figure}
\end{column}
\end{columns}
\end{frame}

\begin{frame}[label={sec:org0d7f73b}]{Alternative genetic code}
\begin{itemize}
\item The mitochondria of many eukaryotes, including most plants, use the
standard code.
\end{itemize}

\begin{table}[htbp]
\caption[mito code]{\label{tab:org2778280}
Exceptions to the standard genetic code in mamalian mitochondria}
\centering
\begin{tabular}{lll}
Codon & Standard & Mitochondria\\
\hline
AGA, AGG & Arginine & Stop codon\\
AUA & Isoleucine & Methionine\\
UGA & Stop codon & Tryptophan\\
\end{tabular}
\end{table}

\begin{itemize}
\item AUA, AUC, and AUU codons are all allowable start codons.
\item Some of these differences are pseudo-changes in the genetic code due
to the phenomenon of RNA editing, common in mitochondria.
\end{itemize}
\end{frame}


\begin{frame}[label={sec:orgca42263}]{Mitochondrial Disease}
\begin{itemize}
\item About 15\% of mitochondrial disease is caused by mutations in the
mitochondrial DNA that affect mitochondrial function.
\item Other mitochondrial diseases are caused by
\begin{itemize}
\item mutations  in nuclear DNA
\item acquired mitochondrial conditions (drugs, toxins)
\end{itemize}
\end{itemize}
\end{frame}

\section{Biochemical Functions Relevant to IMD}
\label{sec:org25477c5}

\begin{frame}[label={sec:org245cf33}]{Pyruvate and the Tricyclic Acid Cycle}
\begin{center}
\includegraphics[width=0.8\textwidth]{./figures/tca.png}
\end{center}

\begin{itemize}
\item oxidation of pyruvate \(\rightarrow\) 3 NADH, 1 \ce{FADH2}, and 1 GTP
\end{itemize}
\end{frame}
\begin{frame}[label={sec:org11c9bd4}]{Electron Transport Chain}
\begin{itemize}
\item Energy obtained through the transfer of electrons down the ETC is used to pump protons from the mitochondrial matrix into the intermembrane space
\begin{itemize}
\item creates an electrochemical proton gradient (\(\Delta\)pH) across the IMM.
\begin{itemize}
\item largely responsible for the mitochondrial membrane potential (\(\Delta \Psi\)M).
\end{itemize}
\item ATP synthase uses flow of \ce{H+} through the enzyme back into the
matrix to generate ATP from ADP and Pi.
\end{itemize}
\end{itemize}


\begin{center}
\includegraphics[width=.9\linewidth]{./figures/etc.pdf}
\end{center}
\end{frame}

\begin{frame}[label={sec:org3e43b02}]{ATP synthase}
\begin{itemize}
\item formation of ATP from ADP and Pi is energetically unfavorable
\item ATP synthase couples ATP synthesis to an electrochemical gradient (\(\Delta \Psi\)M).
\end{itemize}

\begin{center}
\includegraphics[width=0.5\textwidth]{./figures/atp_synthase.jpg}
\label{orgf9dfba6}
\end{center}

\centering
\ce{ADP + Pi + H+_{out} <=> ATP + H2O + H+_{in}}
\end{frame}

\begin{frame}[label={sec:org910dc18}]{Ketogenesis}
\begin{columns}
\begin{column}{0.5\columnwidth}
\begin{itemize}
\item produced mainly in the mitochondria of liver cells,
\item in response \(\downarrow\) blood glucose
\end{itemize}
\end{column}

\begin{column}{0.5\columnwidth}
\begin{figure}[htbp]
\centering
\includegraphics[width=1\textwidth]{./figures/Ketogenesis.png}
\caption[keto]{\label{fig:orge91861b}
Ketogenesis}
\end{figure}
\end{column}
\end{columns}
\end{frame}


\begin{frame}[label={sec:org4081fa5}]{Ketolysis}
\begin{center}
\includegraphics[width=.9\linewidth]{./figures/keto.pdf}
\end{center}

\begin{itemize}
\item ketone bodies are a way to move energy from the liver to other cells.
\item The liver does not have the succinyl-CoA transferase, to metabolize ketone bodies
\item liver produces ketone bodies, but does not use a significant amount of them.
\end{itemize}
\end{frame}

\begin{frame}[label={sec:org634c5cf}]{Other Biochemical Functions Relevant to IMD}
\begin{itemize}
\item Mitochondrial Fatty Acid Oxidation

\item Urea Cycle

\item Heme Biosynthesis
\end{itemize}
\end{frame}
\end{document}
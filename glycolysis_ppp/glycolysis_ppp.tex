% Created 2019-10-10 Thu 12:52
% Intended LaTeX compiler: pdflatex
\documentclass[presentation, smaller]{beamer}
\usepackage[utf8]{inputenc}
\usepackage[T1]{fontenc}
\usepackage{graphicx}
\usepackage{grffile}
\usepackage{longtable}
\usepackage{wrapfig}
\usepackage{rotating}
\usepackage[normalem]{ulem}
\usepackage{amsmath}
\usepackage{textcomp}
\usepackage{amssymb}
\usepackage{capt-of}
\usepackage{hyperref}
\hypersetup{colorlinks,linkcolor=white,urlcolor=blue}
\usepackage{textpos}
\usepackage{textgreek}
\usepackage[version=4]{mhchem}
\usepackage{chemfig}
\usepackage{siunitx}
\usepackage{gensymb}
\usepackage[usenames,dvipsnames]{xcolor}
\usepackage[T1]{fontenc}
\usepackage{lmodern}
\usepackage{verbatim}
\usepackage{tikz}
\usepackage{wasysym}
\usetikzlibrary{shapes.geometric,arrows,decorations.pathmorphing,backgrounds,positioning,fit,petri}
\usetheme{Hannover}
\usecolortheme{whale}
\author{Matthew Henderson, PhD, FCACB}
\date{\today}
\title{Disorders of Glycolysis and the Pentose Phosphate Pathway}
\institute[NSO]{Newborn Screening Ontario | The University of Ottawa}
\titlegraphic{\includegraphics[height=1cm,keepaspectratio]{../logos/NSO_logo.pdf}\includegraphics[height=1cm,keepaspectratio]{../logos/cheo-logo.png} \includegraphics[height=1cm,keepaspectratio]{../logos/UOlogoBW.eps}}
\hypersetup{
 pdfauthor={Matthew Henderson, PhD, FCACB},
 pdftitle={Disorders of Glycolysis and the Pentose Phosphate Pathway},
 pdfkeywords={},
 pdfsubject={},
 pdfcreator={Emacs 26.1 (Org mode 9.1.9)}, 
 pdflang={English}}
\begin{document}

\maketitle

%\logo{\includegraphics[width=1cm,height=1cm,keepaspectratio]{../logos/NSO_logo_small.pdf}~%
%    \includegraphics[width=1cm,height=1cm,keepaspectratio]{../logos/UOlogoBW.eps}%
%}

\vspace{220pt}
\beamertemplatenavigationsymbolsempty
\setbeamertemplate{caption}[numbered]
\setbeamerfont{caption}{size=\tiny}
% \addtobeamertemplate{frametitle}{}{%
% \begin{textblock*}{100mm}(.85\textwidth,-1cm)
% \includegraphics[height=1cm,width=2cm]{cat}
% \end{textblock*}}

\section{Introduction}
\label{sec:org91062aa}
\begin{frame}[label={sec:org2781182}]{Glycolysis}
\begin{itemize}
\item Glycolysis is an oxygen-independent metabolic pathway
\begin{itemize}
\item converts each molecule of glucose to two of pyruvate.
\item consists of two phases and ten steps.
\end{itemize}
\item The first five steps are the preparatory phase,
\begin{itemize}
\item consumes ATP to convert glucose into two, three-carbon sugar
phosphate molecules.
\end{itemize}
\item In the other five steps ATP and NADH are produced
\end{itemize}
\end{frame}

\begin{frame}[label={sec:orgd2e28da}]{Pentose Phosphate Pathway}
\begin{block}{Oxidative Phase}
\begin{itemize}
\item glucose 6-P \(\to\) NADPH + ribose 5-P
\item Glucose 6-P dehydrogenase catalyses first step
\item NADPH is for reducing reactions
\begin{itemize}
\item NADPH/NADP\(^{\text{+}}\) \textgreater{}\textgreater{}\textgreater{} NADH/NAD\(^{\text{+}}\)
\item NADH is rapidly converted to NAD\(^{\text{+}}\) in the ETC
\end{itemize}
\end{itemize}
\end{block}
\begin{block}{Non-oxidative Phase}
\begin{itemize}
\item reversible rxns
\item convert glycolytic intermediates to 5 carbon sugars
\end{itemize}
\end{block}
\end{frame}
\begin{frame}[label={sec:orgff4675e}]{Pentose Phosphate Pathway}
\begin{itemize}
\item Ribose-5-P required for purine and pyrimidine synthesis
\item NADPH required for detoxification and synthetic reaction
\begin{itemize}
\item Detoxification
\begin{itemize}
\item Reduction of oxidized glutathione
\item Cytochrome p450 monoxygenases
\end{itemize}
\item Synthetic reactions
\begin{itemize}
\item FA synthesis
\item Cholesterol
\item neurotransmitters
\item deoxynucleotide
\item superoxide
\end{itemize}
\end{itemize}
\end{itemize}
\end{frame}

\begin{frame}[label={sec:org847db7a}]{Glycolysis and PPP}
\begin{figure}[htbp]
\centering
\includegraphics[width=1\textwidth]{./figures/glyc_ppp.png}
\caption{\label{fig:org3439777}
Glycolysis and PPP}
\end{figure}
\end{frame}

\section{Disorders}
\label{sec:orga3ac7c7}
\begin{frame}[label={sec:orgc0761d6}]{Glycolysis Disorders}
\begin{itemize}
\item Glycolysis is the most important source of energy in erythrocytes
and in some types of skeletal muscle fibres

\begin{itemize}
\item \(\therefore\) IMD of glycolysis are mainly characterized by hemolytic
anaemia \textpm{} metabolic myopathy.
\end{itemize}

\item Ten inborn errors of the glycolytic pathway are known,
\begin{itemize}
\item 8 inherited as an autosomal recessive trait
\item X-linked phosphoglycerate kinase and glycerol kinase deficiencies.
\end{itemize}

\item Hexokinase (HK), glucose-6-phosphate isomerase (GPI) and pyruvate
kinase (PKD) deficiencies cause severe haemolytic anaemia.

\item Muscle phosphofructokinase (PFKM), aldolase A, triosephosphate
isomerase (TPI) and phosphoglycerate kinase (PGK) deficiencies are
characterized by haemolytic anaemia alone or coupled with
neurological disease and/or myopathy.
\end{itemize}
\end{frame}

\begin{frame}[label={sec:orge2452c9}]{Glycolysis Disorders}
\begin{itemize}
\item Phosphoglycerate mutase (PGAM), enolase and lactate dehydrogenase
(LDH) deficiencies present with a purely myopathic syndrome
\begin{itemize}
\item characterized by exercise induced cramps and myoglobinuria.
\end{itemize}

\item Glycerol kinase deficiency (GKD) is either an isolated condition
with hypoglycaemia and acidosis or part of a contiguous
gene deletion where it also associated with congenital adrenal
hypoplasia and/or Duchenne muscular dystrophy.

\item Glucose-6-phosphate can also be formed by the conversion of the
glycogen derived glucose-1-phosphate, a reaction catalysed by
phosphoglucomutase (PGM). PGM1 deficiency is a CDG
\end{itemize}
\end{frame}

\begin{frame}[label={sec:org1ab88bf}]{PPP Disorders}
\begin{itemize}
\item Four inborn errors in the pentose phosphate pathway (PPP) are known.
\item Glucose-6-phosphate dehydrogenase deficiency is an X-linked defect
in the first, irreversible step of the pathway.
\begin{itemize}
\item An exclusively haematological disorder.
\end{itemize}
\item Ribose-5-phosphate isomerase (RPI) deficiency has been described in one patient
\begin{itemize}
\item presented with developmental delay and a slowly progressive leukoencephalopathy.
\end{itemize}
\item Transaldolase (TALDO) deficiency often presents in the neonatal or
antenatal period with hepatosplenomegaly, \(\downarrow\) liver function,
hepatic fibrosis and anaemia.
\item Transketolase (TKT) deficiency presents with short stature,
developmental delay and congenital heart defects.
\end{itemize}
\end{frame}

\section{The non-ischemic forearm exercise test}
\label{sec:orgd42175b}
\begin{frame}[label={sec:org5d4debd}]{NIET in Myopathy}
\begin{center}
\begin{tabular}{lll}
 & Lactate & Ammonia\\
\hline
GSD I & N & N\\
GSD III (L\&M) & \(\downarrow\) \(\downarrow\) & N/\(\uparrow\)\\
GSD V & \(\downarrow\) \(\downarrow\) & N/\(\uparrow\)\\
\alert{GSD VII (PFK)} & \(\downarrow\) \(\downarrow\) & N/\(\uparrow\)\\
\alert{GSD IX (PGK)} & \(\downarrow\) \(\downarrow\) & N/\(\uparrow\)\\
\alert{GSD X (PGAM)} & \(\downarrow\) & N/\(\uparrow\)\\
Alcoholic myopathy & N & N\\
CFS & N & N\\
Poor effort & N/\(\downarrow\) & N/\(\downarrow\)\\
\end{tabular}
\end{center}
\end{frame}

\begin{frame}[label={sec:orgfa2fc4d}]{NIET Method}
\begin{figure}[htbp]
\centering
\includegraphics[width=0.9\textwidth]{./figures/niet_method.png}
\caption{\label{fig:org5e155f1}
NIET Method}
\end{figure}
\end{frame}


\begin{frame}[label={sec:orgf2c3a80}]{Exercising Muscle: Lactate}
\begin{itemize}
\item Lactate, ammonia and purine compounds are generated by exercising muscle.
\item Exercising muscle generates lactic acid from the anaerobic breakdown
of glycogen to pyruvate
\begin{itemize}
\item \ce{pyruvate \to lactate}
\end{itemize}
\item Lactate enters the circulation and is converted back to pyruvate in the liver.
\end{itemize}

\begin{figure}[htbp]
\centering
\includegraphics[width=0.5\textwidth]{./figures/Lactate_dehydrogenase_mechanism.png}
\caption{\label{fig:org21f9a7c}
LDH}
\end{figure}
\end{frame}

\begin{frame}[label={sec:org8be6a09}]{Exercising Muscle: ATP}
\begin{itemize}
\item Some ATP regeneration is provided by glycolytic metabolism of fuels,
but this is relatively slow
\item Most ATP regeneration relys on creatine kinase catalysed transfer of
phosphate from phosphocreatine.

\begin{itemize}
\item \ce{phosphocreatine + ADP ->[CK] creatine + ATP}
\end{itemize}

\item adenylatekinase transphosphorylates ATP to be regenerated with the formation
of AMP

\begin{itemize}
\item \ce{2ADP ->[ADK] ATP + AMP}
\end{itemize}

\item AMP deaminase
\begin{itemize}
\item \ce{AMP ->[AMPD] IMP + NH4+}
\end{itemize}

\item IMP degraded to hypoxanthine
\item recycled back to AMP in the purine nucleotide cycle.
\end{itemize}
\end{frame}

\begin{frame}[label={sec:org0eb6d23}]{Exercising Muscle: Ammonia}
\begin{itemize}
\item Most ammonia produced by exercising muscle removed by formation of glutamine
\begin{itemize}
\item ultimately excreted as urea
\end{itemize}
\end{itemize}

\begin{figure}[htbp]
\centering
\includegraphics[width=0.6\textwidth]{./figures/nitrogen_glutamine.png}
\caption[gln]{\label{fig:org70c30d3}
Glutamine and Ammonia}
\end{figure}

\begin{itemize}
\item Some ammonia is released by exercising skeletal muscle directly into the circulation
\begin{itemize}
\item removed with a half-life of 20\textpm{}30 min.
\end{itemize}
\item In resting skeletal muscle ammonia is consumed rather than produced
\item \textasciitilde{}50\% of arterial ammonia can be taken up and metabolized by skeletal muscle.
\end{itemize}
\end{frame}

\begin{frame}[label={sec:org35cc650}]{Interpretation}
\begin{figure}[htbp]
\centering
\includegraphics[width=.8\textheight]{./figures/niet_results.png}
\caption[interp]{\label{fig:orgeb8646b}
NIET Results}
\end{figure}
\end{frame}
\end{document}
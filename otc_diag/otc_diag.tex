% Created 2017-09-21 Thu 12:46
\documentclass[presentation, smaller]{beamer}
\usepackage[utf8]{inputenc}
\usepackage[T1]{fontenc}
\usepackage{fixltx2e}
\usepackage{graphicx}
\usepackage{grffile}
\usepackage{longtable}
\usepackage{wrapfig}
\usepackage{rotating}
\usepackage[normalem]{ulem}
\usepackage{amsmath}
\usepackage{textcomp}
\usepackage{amssymb}
\usepackage{capt-of}
\usepackage{hyperref}
\hypersetup{colorlinks,linkcolor=white,urlcolor=blue}
\usepackage{textpos}
\usepackage[version=4]{mhchem}
\usepackage{chemfig}
\usepackage{siunitx}
\usepackage[usenames,dvipsnames]{xcolor}
\usepackage[T1]{fontenc}
\usepackage{lmodern}
\usepackage{verbatim}
\usetheme[height=20pt]{Boadilla}
\usecolortheme[RGB={170,160,80}]{{structure}}
\author{Matthew Henderson, PhD, FCACB}
\date{\today}
\title{Diagnostic Testing for OTC Deficiency}
\institute[NSO]{Newborn Screening Ontario | The University of Ottawa}
\titlegraphic{\includegraphics[height=1cm,keepaspectratio]{../logos/NSO_logo.pdf} \includegraphics[height=1cm,keepaspectratio]{../logos/UOlogoBW.eps}}
\hypersetup{
 pdfauthor={Matthew Henderson, PhD, FCACB},
 pdftitle={Diagnostic Testing for OTC Deficiency},
 pdfkeywords={},
 pdfsubject={},
 pdfcreator={Emacs 25.2.1 (Org mode 8.3.4)}, 
 pdflang={English}}
\begin{document}

\maketitle
\logo{\includegraphics[width=1cm,height=1cm,keepaspectratio]{../logos/NSO_logo_small.pdf}~%
    \includegraphics[width=1cm,height=1cm,keepaspectratio]{../logos/UOlogoBW.eps}%
}

\vspace{220pt}}
\beamertemplatenavigationsymbolsempty
\setbeamertemplate{caption}[numbered]
\setbeamerfont{caption}{size=\tiny}

% \addtobeamertemplate{frametitle}{}{%
% \begin{textblock*}{100mm}(.85\textwidth,-1cm)
% \includegraphics[height=1cm,width=2cm]{cat}
% \end{textblock*}}

\section{Background}
\label{sec:orgheadline10}
\begin{frame}[label={sec:orgheadline1}]{Objectives}
\begin{itemize}
\item Review of the Urea cycle
\begin{itemize}
\item better figure, and groupings
\item ancillary enzymes
\end{itemize}
\item OTC Case Report
\item OTC diagnostic testing
\end{itemize}
\end{frame}

\begin{frame}[label={sec:orgheadline2}]{The Urea Cycle}
\begin{block}{Mitochondrial enzymes:}
\begin{itemize}
\item Carbamoylphosphate synthetase I (CPS1, AR)
\begin{itemize}
\item rate-limiting reaction of the urea cycle
\item N-acetylglutamate is an obligate activator
\end{itemize}
\item N-acetylglutamate synthetase (NAGS, AR)
\item Ornithine transcarbamylase (OTC, X-linked)
\end{itemize}
\end{block}
\begin{block}{Cytoplasmic enzymes:}
\begin{itemize}
\item Argininosuccinic acid synthetase (ASS1, AR)
\item Argininosuccinic acid lyase (ASL, AR)
\item Arginase (ARG1, AR)
\end{itemize}
\end{block}

\begin{block}{Transporters:}
\begin{itemize}
\item Ornithine translocase (SLC25A15, AR)
\end{itemize}
\end{block}
\end{frame}

\begin{frame}[label={sec:orgheadline3}]{The Urea Cycle: Ancillary Enzymes}
\begin{itemize}
\item Carbonic Anhydrase Va (CA5A, AR)
\item Citrin (SLC25A13, AR)
\item \(\Delta\)-pyrroline-5 carboxylate synthetase (ALDH18A1)
\end{itemize}
\end{frame}

\begin{frame}[label={sec:orgheadline4}]{The Urea Cycle}
\centering
\includegraphics[width=0.7\textwidth]{./figures/urea_cycle_crop.png}
\end{frame}

\begin{frame}[label={sec:orgheadline5}]{OTC Deficiency}
\begin{block}{}
\centering
  \ce{ornithine + carbamoyl phosphate ->[{\color{red}OTC}] citrulline}
\end{block}
\begin{itemize}
\item Incidence 1:56,500
\item X-linked inheritance
\end{itemize}
\end{frame}

\begin{frame}[label={sec:orgheadline6}]{OTCD Neonatal Onset}
\begin{itemize}
\item Severe neonatal-onset disease in males (but rarely in females)
\item Males with severe neonatal-onset OTCD are typically normal
at birth
\begin{itemize}
\item become symptomatic on day two to three of life.
\end{itemize}
\item After successful treatment of neonatal hyperammonemic coma these
infants can easily become hyperammonemic again despite appropriate
treatment
\begin{itemize}
\item typically require liver transplant by age six months to improve quality of life.
\end{itemize}
\end{itemize}
\end{frame}

\begin{frame}[label={sec:orgheadline7}]{OTCD Post-Neonatal Onset}
\begin{itemize}
\item Post-neonatal-onset (partial deficiency) disease in males and females.
\item Males and heterozygous females with post-neonatal-onset (partial)
OTC deficiency can present from infancy to later childhood,
adolescence, or adulthood.
\item In all OTCD a hyperammonemic crisis can be precipitated by stress
\item Typical neuropsychological complications include developmental delay, learning disabilities,
intellectual disability, attention deficit hyperactivity disorder
(ADHD), and executive function deficits.
\end{itemize}
\end{frame}

\begin{frame}[label={sec:orgheadline8}]{OTCD Metabolic Derangements}
\begin{itemize}
\item Hyperammonemia due to product inhibition of CPS by CP.
\item Excess CP \(\to\) \(\uparrow\) pyrimidine biosysthesis
\begin{itemize}
\item \(\uparrow\) orotic acid and uracil in urine
\end{itemize}
\item \(\uparrow\) \ce{NH4+} \(\to\) \(\uparrow\) glutamine
\begin{itemize}
\item \(\uparrow\)  glycine, serine, glutamate, alanine
\end{itemize}
\end{itemize}

\begin{columns}
\begin{column}{0.5\columnwidth}
\begin{itemize}
\item Routine Biochemical Testing
\begin{itemize}
\item hyperammonemia
\item absence of hypoglycemia, lactic acidosis, ketonuria
\end{itemize}
\end{itemize}
\end{column}

\begin{column}{0.5\columnwidth}
\begin{itemize}
\item Plasma Amino Acids
\begin{description}
\item[{glutamine}] \textgreater{} 1000 \si{\micro\mol/\liter}
\item[{alanine}] \textgreater{} 600 \si{\micro\mol/\liter}
\item[{citruline}] \textless{} 10 \si{\micro\mol/\liter}
\item[{arginine}] \textless{} 30 \si{\micro\mol/\liter}
\end{description}
\end{itemize}
\end{column}
\end{columns}
\end{frame}

\begin{frame}[label={sec:orgheadline9}]{Female Carriers of OTC}
\begin{itemize}
\item Variable inactivation of the X-chromosome
\item Variable phenotype
\begin{itemize}
\item low residual OTC function
\item asymptomatic
\item long term symptoms consistent with undiagnosed hyperammonemia
\end{itemize}
\end{itemize}
\end{frame}

\section{Case Study}
\label{sec:orgheadline18}
\begin{frame}[label={sec:orgheadline11}]{A Case of Severe Neonatal Hyperammonemia}
\begin{itemize}
\item Roy W.A. Peake and Edward G. Neilan, Clin Chem 2017
\begin{itemize}
\item Department of Laboratory Medicine and Division of Genetics and Metabolism
\item Boston Children’s Hospital, Boston, MA.
\end{itemize}
\end{itemize}
\end{frame}
\begin{frame}[label={sec:orgheadline12}]{Clinical History}
\begin{itemize}
\item Male child delivered by C-section at 39 wks
\item Emerged limp and cyanotic w/o respiratory effort
\begin{itemize}
\item intubated, suctioned, positive pressure ventilation
\item developed spontaneous respiration
\end{itemize}
\item Treated with boluses of saline and glucose for hypotension and hypoglycemia
\item Thrombocytopenia was treated with empiric antibiotics

\item Full enteral feeding by day 4

\item Day 5 developed apnea and seizures

\begin{itemize}
\item intubated, ventilation, phenobarbital, antibiotics
\end{itemize}
\end{itemize}
\end{frame}

\begin{frame}[label={sec:orgheadline13}]{Screening and Diagnostic Testing}
\begin{itemize}
\item Newborn Screening results became available
\begin{itemize}
\item "concerning for low-normal citrulline levels"
\end{itemize}
\item A metabolic disorder was considered
\item Plasma amino acids are useful for diagnosis for OTCD
\end{itemize}

\begin{block}{Lab results}
\begin{description}
\item[{Ammonia}] 2090 \si{\micro\mol/\liter} (RI < 90)
\item Plasma Amino Acids
\begin{description}
\item[{Glutamine}] 1536 \si{\micro\mol/\liter} (RI 330-1080)
\item[{Alanine}] 1160 \si{\micro\mol/\liter} (RI 120-500)
\item[{Citrulline}] 3 \si{\micro\mol/\liter} (RI 2-50)
\end{description}
\end{description}
\end{block}
\end{frame}

\begin{frame}[label={sec:orgheadline14}]{Urine Orotic Acid}
\begin{itemize}
\item OA is a surrogate marker for increased carbamoyl phosphate.
\begin{itemize}
\item \(\uparrow\) CP overwhelms the pyrimidine synthesis pathway
\end{itemize}
\end{itemize}

\begin{block}{}
\centering
\ce{CP ->[aspartate transcarbamylase] carbamoyl aspartate ->[dihydroorotase] dihydroorotate ->[dihydroorotate dehydrogenase] orotic acid}
\end{block}

\begin{itemize}
\item Qualitative urine organic acids have limited value in the diagnosis of UCDs.
\item Methodological issues
\begin{itemize}
\item OA is a charged molecule and is not efficiently extracted
\item OA often coelutes with cis-aconitic acid
\begin{itemize}
\item Presence of OA ion pair 254 and 357 mz should be confirmed.
\end{itemize}
\end{itemize}
\end{itemize}
\end{frame}

\begin{frame}[label={sec:orgheadline15}]{Urine Organic Acids}
\centering
\includegraphics[width=0.85\textwidth]{./figures/F1_large.jpg}
\end{frame}

\begin{frame}[label={sec:orgheadline16}]{Treatment}
\begin{itemize}
\item IV sodium phenylacetate and sodium benzoate (ammonul)
\end{itemize}
\begin{block}{}
\begin{itemize}
\item phenylacetate + CoA + glutamine \(\to\) phenylactylglutamine
\item benzoate + CoA + glycine \(\to\) hippuric acid
\end{itemize}
\end{block}
\begin{itemize}
\item Emergency hemodialysis
\end{itemize}
\begin{itemize}
\item At day 7 ammonia was normal
\begin{itemize}
\item veno-venous hemofiltration
\item ammonul and arginine
\end{itemize}
\end{itemize}
\end{frame}

\begin{frame}[label={sec:orgheadline17}]{Follow-up}
\begin{itemize}
\item Molecular testing of OTC gene
\begin{itemize}
\item hemizygous pathogenic variant (c.596A>G, p.Asn199Ser)
\begin{itemize}
\item ornithine binding site
\end{itemize}
\end{itemize}
\end{itemize}
\url{http://www.uniprot.org/uniprot/P00480#showFeaturesViewer}
\begin{itemize}
\item previously reported in neonatal onset hyperammonemia
\end{itemize}
\begin{itemize}
\item At 9 months of age:
\begin{itemize}
\item delayed by steady developmental progress
\item 1 episode of hyperammonemia ( 250 \si\{\textmu{}\mol/L)
\end{itemize}
\item Liver transplant is being pursued
\end{itemize}
\end{frame}

\section{Local Testing}
\label{sec:orgheadline22}
\begin{frame}[label={sec:orgheadline19}]{Urine Organic Acids}
\begin{itemize}
\item Oximated with 10\% hydroxylamine-HCL
\begin{itemize}
\item avoids multiple TMS species due to keto-enol tautomerism
\end{itemize}
\end{itemize}

\centering
\schemedebug{false}
\schemestart
\chemname{\chemfig[][scale=.5]{R=[1](-[2]OH)-[7]R}}{\tiny enol}
\arrow{<=>}
\chemname{\chemfig[][scale=.5]{R-[1](=[2]O)-[7]R}}{\tiny ketone}
\+
\chemname{\chemfig[][scale=.5]{N(<:[::-160]H)(<[::-120]H)-O-[1]H}}{\tiny hydroxylamine}
\arrow{->}
\chemname{\chemfig[][scale=.5]{R-[1](=[2]N-[1]OH)-[7]R}}{\tiny ketoxime}
\schemestop

\begin{itemize}
\item Acidified and extracted twice with ethyl ether
\item Derivatised with BSTFA (N,O-bis(trimethylsilyl)trifluoroacetamide)
\begin{itemize}
\item forms organic acid TMS ethers
\end{itemize}
\item DB-1 0.25 mm x 30 m x 0.25 \si{\micro\meter} column
\end{itemize}
\end{frame}


\begin{frame}[label={sec:orgheadline20}]{Quantitative Orotic Acid}
\begin{itemize}
\item \ce{1,3 -^15 N2} Orotic acid isotope IDMS
\item Silicic acid SPE
\item Eluted with chloroform: tertiary-amyl alcohol mixture
\item Derivatised with BSTFA (N,O-bis(trimethylsilyl)trifluoroacetamide)
\item DB-1 0.25 mm x 30 m x 0.25 \si{\micro\meter}
\end{itemize}
\end{frame}



\begin{frame}[label={sec:orgheadline21}]{Genetic Testing for UCDs at NSO}
\begin{block}{Targets of Newborn Screening}
\begin{itemize}
\item ASS1
\item ASL
\item ARG1
\end{itemize}
\end{block}
\begin{block}{Mitochrondrial Gene Panel}
\begin{itemize}
\item NAGS
\item CPS1
\item OTC
\item SLC25a13 (Citrin)
\item SLC25a15 (ORNT1)
\end{itemize}
\end{block}

\begin{block}{Not available}
\begin{itemize}
\item ALDH18A1 (PC5S)
\item CA5A - added to list to consider for Mito panel
\end{itemize}
\end{block}
\end{frame}
\end{document}

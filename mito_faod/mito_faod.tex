% Created 2017-11-30 Thu 16:28
\documentclass[presentation, smaller]{beamer}
\usepackage[utf8]{inputenc}
\usepackage[T1]{fontenc}
\usepackage{fixltx2e}
\usepackage{graphicx}
\usepackage{grffile}
\usepackage{longtable}
\usepackage{wrapfig}
\usepackage{rotating}
\usepackage[normalem]{ulem}
\usepackage{amsmath}
\usepackage{textcomp}
\usepackage{amssymb}
\usepackage{capt-of}
\usepackage{hyperref}
\hypersetup{colorlinks,linkcolor=white,urlcolor=blue}
\usepackage{textpos}
\usepackage{textgreek}
\usepackage[version=4]{mhchem}
\usepackage{chemfig}
\usepackage{siunitx}
\usepackage[usenames,dvipsnames]{xcolor}
\usepackage[T1]{fontenc}
\usepackage{lmodern}
\usepackage{verbatim}
\usepackage{tikz}
\usetikzlibrary{shapes.geometric,arrows,decorations.pathmorphing,backgrounds,positioning,fit,petri}
\usetheme{Hannover}
\usecolortheme{whale}
\author{Matthew Henderson, PhD, FCACB}
\date{\today}
\title{Mitochondrial FAO Defects}
\institute[NSO]{Newborn Screening Ontario | The University of Ottawa}
\titlegraphic{\includegraphics[height=1cm,keepaspectratio]{../logos/NSO_logo.pdf}\includegraphics[height=1cm,keepaspectratio]{../logos/cheo-logo.png} \includegraphics[height=1cm,keepaspectratio]{../logos/UOlogoBW.eps}}
\hypersetup{
 pdfauthor={Matthew Henderson, PhD, FCACB},
 pdftitle={Mitochondrial FAO Defects},
 pdfkeywords={},
 pdfsubject={},
 pdfcreator={Emacs 25.2.1 (Org mode 8.3.4)}, 
 pdflang={English}}
\begin{document}

\maketitle

%\logo{\includegraphics[width=1cm,height=1cm,keepaspectratio]{../logos/NSO_logo_small.pdf}~%
%    \includegraphics[width=1cm,height=1cm,keepaspectratio]{../logos/UOlogoBW.eps}%
%}

\vspace{220pt}
\beamertemplatenavigationsymbolsempty
\setbeamertemplate{caption}[numbered]
\setbeamerfont{caption}{size=\tiny}
% \addtobeamertemplate{frametitle}{}{%
% \begin{textblock*}{100mm}(.85\textwidth,-1cm)
% \includegraphics[height=1cm,width=2cm]{cat}
% \end{textblock*}}

\tikzstyle{chemical} = [rectangle, rounded corners, text width=5em, minimum height=1em,text centered, draw=black, fill=none]
\tikzstyle{hardware} = [rectangle, rounded corners, text width=5em, minimum height=1em,text centered, draw=black, fill=gray!30]
\tikzstyle{ms} = [rectangle, rounded corners, text width=5em, minimum height=1em,text centered, draw=orange, fill=none]
\tikzstyle{msw} = [rectangle, rounded corners, text width=7em, minimum height=1em,text centered, draw=orange, fill=none]
\tikzstyle{label} = [rectangle,text width=5em, minimum height=1em, text centered, draw=none, fill=none]
\tikzstyle{hl} = [rectangle, rounded corners, text width=5em, minimum height=1em,text centered, draw=black, fill=red!30]
\tikzstyle{arrow} = [thick,->,>=stealth]
\tikzstyle{hl-arrow} = [ultra thick,->,>=stealth,draw=red]

\section{Introduction}
\label{sec:orgheadline2}
\begin{frame}[label={sec:orgheadline1}]{Overview}
\includegraphics[width=.9\linewidth]{./figures/b_oxidation.png}
\end{frame}

\section{Carnitine Cycle Defects}
\label{sec:orgheadline13}
\begin{frame}[label={sec:orgheadline3}]{Carnitine Transporter Deficiency}
\includegraphics[width=.9\linewidth]{./figures/octn.pdf}
\end{frame}

\begin{frame}[label={sec:orgheadline4}]{Carnitine Transporter Deficiency}
\begin{itemize}
\item Organic cation/carnitine transporter(OCTN2) responsible for
carnitine uptake across the plasma membrane,
\begin{itemize}
\item heart, muscle and kidney.
\end{itemize}
\item Defects \(\to\) primary carnitine deficiency with \(\uparrow\) renal loss of carnitine
\begin{itemize}
\item low plasma concentrations
\item low intracellular concentrations \(\to\) impair fatty acid oxidation
\end{itemize}
\end{itemize}
\end{frame}
\begin{frame}[label={sec:orgheadline5}]{Carnitine Transporter Deficiency}
\begin{itemize}
\item Precipitated by infection, fasting, pregnancy or antibiotics containing pivalate
\begin{itemize}
\item pivalate is excreted bound to carnitine, \(\downarrow\) carnitine concentration
\item isobaric with C5-carnitine
\end{itemize}
\item Some present in infancy with hypoglycaemia, liver dysfunction and hyperammonaemia
\item Other children develop heart failure due to cardiomyopathy,
thickened ventricular walls and reduced contractility.
\begin{itemize}
\item often accompanied by skeletal muscle weakness.
\end{itemize}
\item Adults may suffer fatigue or arrhythmias
\item Screening has shown that many subjects with low plasma carnitine remain asymptomatic
\begin{itemize}
\item Faroe Islands prevalence is 1:300.
\end{itemize}
\end{itemize}
\end{frame}

\begin{frame}[label={sec:orgheadline6}]{Carnitine Palmitoyltransferase I Deficiency}
\includegraphics[width=.9\linewidth]{./figures/cpt1.pdf}
\end{frame}
\begin{frame}[label={sec:orgheadline7}]{Carnitine Palmitoyltransferase I Deficiency}
\begin{itemize}
\item \textbf{CPT Ia} liver and kidney
\item \textbf{CPT Ib}  muscle and heart
\item \textbf{CPT Ic}  brain

\item Only CPT Ia deficiency has been identified.
\item Usually present by the age of 2 years with hypoketotic hypoglycaemia,
\begin{itemize}
\item induced by fasting or illness.
\end{itemize}
\item Accompanied by hepatomegaly, liver dysfunction and occasionally cholestasis
\begin{itemize}
\item may also be transient lipaemia and renal tubular acidosis.
\end{itemize}
\item CPT I deficiency is extremely common in the Inuit population of Canada and Greenland.
\begin{itemize}
\item c.1436C>T, P479L
\end{itemize}
\item A few of these patients present with hypoglycaemia as neonates or young children
\begin{itemize}
\item most remain asymptomatic.
\end{itemize}
\end{itemize}
\end{frame}

\begin{frame}[label={sec:orgheadline8}]{Carnitine Acylcarnitine Translocase Deficiency}
\includegraphics[width=.9\linewidth]{./figures/cact.pdf}
\end{frame}
\begin{frame}[label={sec:orgheadline9}]{Carnitine Acylcarnitine Translocase Deficiency}
\begin{itemize}
\item This rare disorder usually presents in the neonatal period, with
death by 3 months of age
\begin{itemize}
\item severe hypoglycaemia and hyperammonaemia, cardiomyopathy,
atrioventricular block and ventricular arrhythmias.
\end{itemize}
\item A few more mildly affected patients present later with hypoglycaemic
encephalopathy
\begin{itemize}
\item precipitated by fasting or infections.
\end{itemize}
\end{itemize}
\end{frame}

\begin{frame}[label={sec:orgheadline10}]{Carnitine Palmitoyltransferase II Deficiency}
\includegraphics[width=.9\linewidth]{./figures/cpt2.pdf}
\end{frame}
\begin{frame}[label={sec:orgheadline11}]{Carnitine Palmitoyltransferase II Deficiency}
\begin{block}{Neonatal}
\begin{itemize}
\item Severe neonatal onset CPT II deficiency is usually lethal.
\item Patients become comatose within a few days of birth
\begin{itemize}
\item hypoglycaemia and hyperammonaemia.
\item may have cardiomyopathy, arrhythmias and congenital malformations,
principally renal cysts and neuronal migration defects.
\end{itemize}
\item There is also an intermediate form of CPT II deficiency that causes
episodes of hypoglycaemia and liver dysfunction, sometimes
accompanied by cardiomyopathy and arrhythmias
\end{itemize}
\end{block}

\begin{block}{Childhood}
\begin{itemize}
\item Episodes may be brought on by infections or exercise
\item In moderate or severe episodes there is myoglobinuria, \(\uparrow\) CK
\begin{itemize}
\item may lead to acute renal failure
\item CK often normalises between episodes but may remain moderately
elevated
\end{itemize}
\end{itemize}
\end{block}
\end{frame}

\begin{frame}[label={sec:orgheadline12}]{Carnitine Palmitoyltransferase II Deficiency}
\begin{block}{Adolescence,  young adult}
\begin{itemize}
\item Most common form is a partial deficiency that presents with
episodes of rhabdomyolysis.
\begin{itemize}
\item usually precipitated by prolonged exercise
\item particularly in the cold or after fasting
\end{itemize}
\end{itemize}
\end{block}
\end{frame}

\section{\(\beta\)-Oxidation Defects}
\label{sec:orgheadline37}
\begin{frame}[label={sec:orgheadline14}]{Very-Long-Chain Acyl-CoA Dehydrogenase Deficiency}
\includegraphics[width=.9\linewidth]{./figures/vlcad.pdf}
\end{frame}
\begin{frame}[label={sec:orgheadline15}]{Very-Long-Chain Acyl-CoA Dehydrogenase Deficiency}
\scriptsize
\begin{block}{Early infancy}
\begin{itemize}
\item Severely affected patients present in early infancy with
cardiomyopathy, in addition to the problems seen in milder patients.
\end{itemize}
\end{block}

\begin{block}{Childhood}
\begin{itemize}
\item patients present in childhood with hypoglycaemia but suffer exercise
or illness induced rhabdomyolysis or chronic weakness at a later age.
\end{itemize}
\end{block}

\begin{block}{Adolescence, Adult}
\begin{itemize}
\item Mildly affected patients present as adolescents or adults with
exercise-induced rhabdomyolysis.
\end{itemize}
\end{block}

\begin{block}{Screening}
\begin{itemize}
\item Second most common FAOD in Europe and the USA
\item prevalence between 1:50,000 and 1:100,000
\item Much higher than detected clinically
\item likely that many patients diagnosed by screening would remain
asymptomatic without intervention
\end{itemize}
\end{block}
\end{frame}

\begin{frame}[label={sec:orgheadline16}]{Mitochondrial Trifunctional Protein}
\includegraphics[width=.9\linewidth]{./figures/b_oxidation.png}
\end{frame}
\begin{frame}[label={sec:orgheadline17}]{Mitochondrial Trifunctional Protein}
\begin{itemize}
\item MTP a hetero-octamer composed of four \(\alpha\)-subunits and four \(\beta\)-subunits;
\item \(\alpha\)-subunit has long-chain enoyl-CoA hydratase (LCEH) and LCHAD activities
\item \(\beta\)-subunit has long-chain ketoacyl-CoA thiolase (LCKAT) activity.
\item Patients may have isolated LCHAD deficiency or a generalised deficiency of all three enzyme activities.

\item Mothers who are heterozygous for LCHAD or MTP deficiency have a high
risk of illness during pregnancies when carrying an affected fetus
\item The main problems are HELLP syndrome (Haemolysis, Elevated Liver
enzymes and Low Platelets) and acute fatty liver of pregnancy
(AFLP).
\end{itemize}
\end{frame}

\begin{frame}[label={sec:orgheadline18}]{Long-Chain 3-Hydroxyacyl-CoA Dehydrogenase}
\includegraphics[width=.9\linewidth]{./figures/lchad.pdf}
\end{frame}
\begin{frame}[label={sec:orgheadline19}]{Long-Chain 3-Hydroxyacyl-CoA Dehydrogenase}
\begin{itemize}
\item Isolated LCHAD deficiency usually presents acutely before 6 months of age
\begin{itemize}
\item hypoglycaemia, liver dysfunction, lactic acidosis
\item Many have cardiomyopathy, some have hypoparathyroidism or ARDS
\end{itemize}
\item Other patients present with chronic symptoms
\begin{itemize}
\item failure to thrive, hypotonia, occasionally cholestasis or cirrhosis.
\end{itemize}
\item Subsequently, episodes of rhabdomyolysis are common.
\item Many patients develop retinopathy, may start as early as 2 years of age.
\item Granular pigmentation followed by chorioretinal atrophy w deteriorating central vision.
\item Some patients develop cataracts
\end{itemize}
\end{frame}

\begin{frame}[label={sec:orgheadline20}]{Mitochondrial Trifunctional Protein Deficiency}
\includegraphics[width=.9\linewidth]{./figures/mtp.pdf}
\end{frame}
\begin{frame}[label={sec:orgheadline21}]{Mitochondrial Trifunctional Protein Deficiency}
\begin{itemize}
\item Presentation of generalised MTP deficiency is heterogeneous
\item Patients with severe deficiency present as neonates
\begin{itemize}
\item cardiomyopathy, respiratory distress, hypoglycaemia and liver dysfunction
\item most die within a few months, regardless of treatment.
\end{itemize}
\item Other patients resemble those with isolated LCHAD deficiency.
\item A milder neuromyopathic phenotype:
\begin{itemize}
\item exercise induced rhabdomyolysis and progressive peripheral
neuropathy
\item can present at any age from infancy to adulthood.
\end{itemize}
\end{itemize}
\end{frame}


\begin{frame}[label={sec:orgheadline22}]{Long-Chain Acyl-CoA Dehydrogenase Deficiency}
\includegraphics[width=.9\linewidth]{./figures/lcad.pdf}
\end{frame}
\begin{frame}[label={sec:orgheadline23}]{Long-Chain Acyl-CoA Dehydrogenase Deficiency}
\begin{itemize}
\item No human disease-causing mutations have been identified
\item role  in  human  metabolism  is unclear.
\item \emph{In vitro}, the substrate specificity of LCAD overlaps with that of
VLCAD and ACAD9.
\item enzymes have strong activity toward long-chain acyl-CoAs (C14-20)
\item surfactant deficiency and altered lung mechanics in LCAD deficient
mice.
\begin{itemize}
\item postulated that LCAD deficiency in humans may manifest primarily
as a lung disease
\end{itemize}
\end{itemize}
\end{frame}
\begin{frame}[label={sec:orgheadline24}]{Medium-Chain Acyl-CoA Dehydrogenase Deficiency}
\includegraphics[width=.9\linewidth]{./figures/mcad.pdf}
\end{frame}
\begin{frame}[label={sec:orgheadline25}]{Medium-Chain Acyl-CoA Dehydrogenase Deficiency}
\begin{itemize}
\item most common FAOD with an incidence of approximately 1:10,000-20,000
in Europe,USA and Australia.
\item Before NBS, presented 4 months to 4 years
\begin{itemize}
\item acute hypoglycaemic encephalopathy \sout{and liver dysfunction} not always
\item some deteriorated rapidly and died.
\end{itemize}
\item Precipitated by prolonged fasting or infection with vomiting
\item Some babies still present within 72 hours of birth before
newborn screening results are available
\begin{itemize}
\item hypoglycaemia and/or arrhythmias
\item breast-fed babies are at higher risk, due to the small supply of
breast milk at this stage.
\end{itemize}
\end{itemize}
\end{frame}

\begin{frame}[label={sec:orgheadline26}]{Medium-Chain Acyl-CoA Dehydrogenase Deficiency}
\begin{itemize}
\item MCAD deficiency only presents clinically if exposed to an
appropriate environmental stress.
\item Prior to NBS \textasciitilde{} 30-50\% remained asymptomatic
\item NBS and preventative measures, hypoglycaemia is rare.
\begin{itemize}
\item Patients do not develop cardiomyopathy or myopathy and few present
initially as adults.
\end{itemize}
\item Healty MCAD deficient children > 1 year can fast for 12-14 hours without problems.
\item >14 hours \(\to\) non-ketotic (inappropriately low) hypoglycaemia.
\item Shorter fasts may cause problems in infancy
\item Encephalopathy may occur without hypoglycaemia
\begin{itemize}
\item accumulation of FFA acids and carnitine/CoA esters.
\end{itemize}
\end{itemize}
\end{frame}

\begin{frame}[label={sec:orgheadline27}]{Short-Chain Acyl-CoA Dehydrogenase Deficiency}
\includegraphics[width=.9\linewidth]{./figures/scad.pdf}
\end{frame}
\begin{frame}[label={sec:orgheadline28}]{Short-Chain Acyl-CoA Dehydrogenase Deficiency}
\begin{itemize}
\item non-disease?
\begin{itemize}
\item previous association with symptoms due to ascertainment bias?
\end{itemize}
\end{itemize}
\end{frame}

\begin{frame}[label={sec:orgheadline29}]{Short-chain enoyl-CoA hydratase deficiency}
\includegraphics[width=.9\linewidth]{./figures/crotonase.pdf}
\end{frame}
\begin{frame}[label={sec:orgheadline30}]{Short-chain enoyl-CoA hydratase deficiency}
\begin{columns}
\begin{column}{0.5\columnwidth}
\begin{itemize}
\item Reactive intermediates in the valine pathway are likely responsible
for the pathology.
\item Severe neurological problems, lactic acidosis and sometimes
cardiomyopathy
\item One of >75 genes \(\to\) Leigh syndrome
\end{itemize}
\end{column}

\begin{column}{0.5\columnwidth}
\centering
\includegraphics[height=0.85\textheight]{./figures/valine.png}
\end{column}
\end{columns}
\end{frame}
\begin{frame}[label={sec:orgheadline31}]{3-Hydroxyacyl-CoA Dehydrogenase Deficiency}
\includegraphics[width=.9\linewidth]{./figures/hadh.pdf}
\end{frame}
\begin{frame}[label={sec:orgheadline32}]{3-Hydroxyacyl-CoA Dehydrogenase Deficiency}
\begin{itemize}
\item This defect (HADH), previously called SCHAD deficiency, causes
hyperinsulinaemic hypoglycaemia
\item Role in modulation of ATP production inhibition of GDH
\end{itemize}
\end{frame}

\begin{frame}[label={sec:orgheadline33}]{Acyl-CoA dehydrogenase 9}
\includegraphics[width=.9\linewidth]{./figures/acad9.pdf}
\end{frame}
\begin{frame}[label={sec:orgheadline34}]{Acyl-CoA dehydrogenase 9}
\begin{itemize}
\item A complex I assembly factor with a moonlighting function in fatty
acid oxidation deficiencies.
\item ACAD9 is most homologous to VLCAD
\item recombinant ACAD9 displays activity towards long-chain acyl-CoAs,
very similar to VLCAD.
\item Responsible for production of C14:1-carnitine and C12-carnitine in
VLCAD deficiency.
\item Patients with ACAD9 defects present in infancy or childhood with
myopathy or hypertrophic cardiomyopathy and lactic acidaemia;
\item some also have neurological problems.
\item The myopathic patients often respond to treatment with riboflavin
\begin{itemize}
\item FAD is the enzyme-bound prosthetic group of all acyl-CoA
dehydrogenases
\end{itemize}
\end{itemize}
\end{frame}

\begin{frame}[label={sec:orgheadline35}]{2,4-Dienoyl-CoA reductase deficiency}
\begin{columns}
\begin{column}{0.5\columnwidth}
\centering
\includegraphics[height=0.85\textheight]{./figures/dienol.pdf}
\end{column}
\begin{column}{0.5\columnwidth}
\begin{itemize}
\item Oxidation of unsaturated fatty acids
\end{itemize}
\end{column}
\end{columns}
\end{frame}

\begin{frame}[label={sec:orgheadline36}]{2,4-Dienoyl-CoA reductase deficiency}
\begin{itemize}
\item initially described in 1990 based on a single case who presented with persistent hypotonia.
\begin{itemize}
\item elevated lysine, low carnitine
\item abnormal acylcarnitine profile in urine and plasma.
\end{itemize}
\item The abnormal acylcarnitine species was 2-trans,4-cis-decadienoylcarnitine
\begin{itemize}
\item intermediate of linoleic acid metabolism.
\end{itemize}
\item The index case died of respiratory failure at four months of age.
\item Postmortem enzyme analysis on liver and muscle samples revealed
decreased 2,4-dienoyl-CoA reductase activity when compared to normal
controls.
\item a deficiency of this enzyme has been shown to occur
in a patient due to a mutation in the NADK2 gene, a mitochondrial
NAD kinase
\item disruption of NADP synthesis \(\to\) secondary deficiencies of
2,4-dienoyl-coA reductase and \(\alpha\)-aminoadipic semialdehyde
synthase
\end{itemize}
\end{frame}

\section{Electron Transfer Defects}
\label{sec:orgheadline42}
\begin{frame}[label={sec:orgheadline38}]{Multiple acyl-CoA dehydrogenase deficiency}
\includegraphics[width=.9\linewidth]{./figures/mad.pdf}
\end{frame}
\begin{frame}[label={sec:orgheadline39}]{Multiple acyl-CoA dehydrogenase deficiency}
\begin{itemize}
\item Electron transfer flavoprotein (ETF) and ETF ubiquinone
oxidoreductase (ETFQO) carry electrons to the respiratory chain from
multiple FAD-linked dehydrogenases.
\item Includes enzymes of amino acid, choline metabolism and acyl-CoA
dehydrogenases of \(\beta\)-oxidation
\item Defects in ETF or ETFQO lead to multiple acyl-CoA dehydrogenase
(MAD) deficiency (glutaric aciduria type II).
\item MAD deficiency, less often, a result of defects of riboflavin
transport or metabolism
\end{itemize}
\end{frame}

\begin{frame}[label={sec:orgheadline40}]{Multiple acyl-CoA dehydrogenase deficiency}
\begin{itemize}
\item ETF and ETFQO deficiencies \(\to\) wide range of clinical severity.
\item Severely affected patients present in the first few days of life
\begin{itemize}
\item hypoglycaemia, hyperammonaemia and acidosis
\item hypotonia and hepatomegaly.
\end{itemize}
\item There is usually an odour of sweaty feet similar to that in isovaleric acidaemia.
\item Some patients have congenital anomalies
\begin{itemize}
\item Large cystic kidneys, hypospadias and neuronal migration defects and facial dysmorphism
\begin{itemize}
\item low set ears, high forehead and midfacial hypoplasia.
\end{itemize}
\end{itemize}
\item The malformations resemble those seen in CPT II deficiency but the pathogenesis is unknown.
\end{itemize}
\end{frame}

\begin{frame}[label={sec:orgheadline41}]{Multiple acyl-CoA dehydrogenase deficiency}
\begin{itemize}
\item Most patients with neonatal presentation die within a week of birth
\item Others develop cardiomyopathy and die within a few months.
\item Less severe cases can present at any age from infancy to adulthood
\begin{itemize}
\item with hypoglycaemia, liver dysfunction and weakness
\item usually precipitated by an infection
\end{itemize}
\item Cardiomyopathy is common in infants.
\item Rarer problems include stridor and leukodystrophy.
\item Mildly affected children may have recurrent bouts of vomiting.
\item Muscle weakness is the commonest presentation in adolescents and adults.
\begin{itemize}
\item Predominantly affects proximal muscles and may lead to scoliosis,
hypoventilation or an inability to lift the chin off the chest.
\end{itemize}
\item Weakness can worsen rapidly during infection or pregnancy, myoglobinuria is rare.
\begin{itemize}
\item Many milder defects respond to riboflavin
\end{itemize}
\end{itemize}
\end{frame}

\section{Additional Defects}
\label{sec:orgheadline45}
\begin{frame}[label={sec:orgheadline43}]{FA transport}
\begin{itemize}
\item The mechanisms of fatty acid transport across the plasma membrane are still not completely clear.
\item Impaired uptake has been reported in 2 boys who presented with liver failure.
\item The molecular basis was not identified and the diagnoses remains uncertain.
\end{itemize}
\end{frame}

\begin{frame}[label={sec:orgheadline44}]{Potential Defects}
\begin{block}{Medium-chain 3-ketoacyl-CoA thiolase (MCKAT) deficiency}
\begin{itemize}
\item reported in one patient, who died at 13 days of age
\item hypoglycaemia, hyperammonaemia, acidosis and myoglobinuria
\end{itemize}
\end{block}

\begin{block}{Long-chain acyl-CoA dehydrogenase (LCAD)}
\begin{itemize}
\item involved in surfactant metabolism
\item LCAD deficiency has been reported in two cases of sudden infant death
\end{itemize}
\end{block}
\end{frame}


\section{Metabolic Derangement}
\label{sec:orgheadline48}
\begin{frame}[label={sec:orgheadline46}]{Metabolic Derangement}
\begin{itemize}
\item Fasting hypoglycaemia is the classic metabolic disturbance in FAODs
\begin{itemize}
\item primarily due to increased peripheral glucose consumption
\item epatic glucose output is also reduced under some conditions.
\end{itemize}
\item The hypoglycaemia is hypoketotic.
\begin{itemize}
\item Ketone bodies can be synthesised (medium-or short-chain FAODs or if there is high residual enzyme activity)
\item plasma concentrations are lower than expected for hypoglycaemia or the plasma free fatty acid concentrations.
\end{itemize}
\item Hyperammonaemia occurs in some severe defects,
\begin{itemize}
\item with normal or low glutamine concentrations;
\item decreased acetyl-CoA production reducing the synthesis of N-acetylglutamate
\end{itemize}
\item Lactic acidaemia is seen in long-chain FAODs, (LCHAD and MTP deficiencies)
\begin{itemize}
\item inhibitory effects of metabolites on pyruvate metabolism.
\end{itemize}
\end{itemize}
\end{frame}

\begin{frame}[label={sec:orgheadline47}]{Metabolic Derangement}
\begin{itemize}
\item Moderate hyperuricaemia - frequent finding during acute attacks.
\item Secondary hyperprolinaemia occurs in some babies with MAD deficiency.
\item Accumulating long-chain acylcarnitines may be responsible for
arrhythmias and may interfere with surfactant metabolism.
\item In LCHAD and MTP deficiencies, \sout{long chain hydroxy-acylcarnitine}
  \sout{concentrations correlate with the severity of retinopathy and may}
  \sout{cause both this and the peripheral neuropathy}
\end{itemize}
\end{frame}

\section{Summary}
\label{sec:orgheadline51}
\begin{frame}[label={sec:orgheadline49}]{Common manifestations in FAODs}
\includegraphics[width=.9\linewidth]{./figures/Ch101f016.png}

\begin{itemize}
\item Green squares indicate that the feature is frequently seen in the disorder
\item Yellow squares represent an intermediate rate of occurrence
\item Red squares denote that it is uncommon
\end{itemize}
\tiny{Source: Mitochondrial Fatty Acid Oxidation Disorders, The Online Metabolic and Molecular Bases of Inherited Disease}
\end{frame}
\begin{frame}[label={sec:orgheadline50}]{Next time}
\begin{itemize}
\item Diagnostic Testing for FAODs
\begin{itemize}
\item Biochemical
\begin{itemize}
\item LC-MS/MS
\item enzymatic
\item cell based assays
\end{itemize}
\item Molecular
\end{itemize}

\item Methods for Quantitation of Acylcarnitines
\begin{itemize}
\item LC-MS/MS
\item FIA-MS/MS
\end{itemize}
\end{itemize}
\end{frame}
\end{document}

% Created 2018-08-21 Tue 22:18
% Intended LaTeX compiler: pdflatex
\documentclass[presentation, smaller]{beamer}
\usepackage[utf8]{inputenc}
\usepackage[T1]{fontenc}
\usepackage{graphicx}
\usepackage{grffile}
\usepackage{longtable}
\usepackage{wrapfig}
\usepackage{rotating}
\usepackage[normalem]{ulem}
\usepackage{amsmath}
\usepackage{textcomp}
\usepackage{amssymb}
\usepackage{capt-of}
\usepackage{hyperref}
\hypersetup{colorlinks,linkcolor=gray,urlcolor=blue}
\usepackage{textpos}
\usepackage{textgreek}
\usepackage[version=4]{mhchem}
\usepackage{chemfig}
\usepackage{siunitx}
\usepackage{gensymb}
\usepackage[usenames,dvipsnames]{xcolor}
\usepackage[T1]{fontenc}
\usepackage{lmodern}
\usepackage{verbatim}
\usepackage{tikz}
\usetikzlibrary{shapes.geometric,arrows,decorations.pathmorphing,backgrounds,positioning,fit,petri}
\usetheme[height=20pt]{Boadilla}
\usecolortheme[RGB={170,160,80}]{{structure}}
\author{Matthew Henderson, PhD, FCACB}
\date{\today}
\title{Summary of "Sweat and sour: an update on classic galactosemia"}
\institute[NSO]{Newborn Screening Ontario}
\titlegraphic{\includegraphics[height=1cm,keepaspectratio]{../logos/NSO_logo.pdf}\includegraphics[height=1cm,keepaspectratio]{../logos/cheo-logo.png} \includegraphics[height=1cm,keepaspectratio]{../logos/UOlogoBW.eps}}
\hypersetup{
 pdfauthor={Matthew Henderson, PhD, FCACB},
 pdftitle={Summary of "Sweat and sour: an update on classic galactosemia"},
 pdfkeywords={},
 pdfsubject={},
 pdfcreator={Emacs 26.1 (Org mode 9.1.9)}, 
 pdflang={English}}
\begin{document}

\maketitle

\begin{LaTeX}
\logo\{\includegraphics[width=1cm,height=1cm,keepaspectratio]{../logos/NSO_logo_small.pdf}\}

\vspace{220pt}
\beamertemplatenavigationsymbolsempty
\setbeamertemplate{caption}[numbered]
\setbeamerfont{caption}{size=\tiny}
\% \addtobeamertemplate{frametitle}{}\{\%
\% \begin{textblock*}{100mm}(.85\textwidth,-1cm)
\% \includegraphics[height=1cm,width=2cm]{cat}
\% \end{textblock*}\}


\tikzstyle{chemical} = [rectangle, rounded corners, text width=5em, minimum height=1em,text centered, draw=black, fill=none]
\tikzstyle{hardware} = [rectangle, rounded corners, text width=5em, minimum height=1em,text centered, draw=black, fill=gray!30]
\tikzstyle{ms} = [rectangle, rounded corners, text width=5em, minimum height=1em,text centered, draw=orange, fill=none]
\tikzstyle{msw} = [rectangle, rounded corners, text width=7em, minimum height=1em,text centered, draw=orange, fill=none]
\tikzstyle{label} = [rectangle,text width=8em, minimum height=1em, text centered, draw=none, fill=none]
\tikzstyle{hl} = [rectangle, rounded corners, text width=5em, minimum height=1em,text centered, draw=black, fill=red!30]
\tikzstyle{box} = [rectangle, rounded corners, text width=5em, minimum height=5em,text centered, draw=black, fill=none]
\tikzstyle{arrow} = [thick,->,>=stealth]
\tikzstyle{hl-arrow} = [ultra thick,->,>=stealth,draw=red]
\end{LaTeX}




\begin{frame}[label={sec:orgf945789}]{Classical galactosemia}
\begin{itemize}
\item Caused by deficient activity of galactose-1-phosphate uridylyltransferase
\item prevalence is 1:16,000-60,000 live births
\item autosomal recessive
\begin{itemize}
\item 336 mutations is the GALT gene described
\end{itemize}

\item presents in the neonatal period
\begin{itemize}
\item potentially lethal
\end{itemize}

\item Therapy is life long dietary galactose restriction
\begin{itemize}
\item does not prevent long-term complications
\end{itemize}
\end{itemize}
\end{frame}

\begin{frame}[label={sec:org3f8b2bd}]{Galactose}
\begin{itemize}
\item Functions
\begin{itemize}
\item energy source in pre-weaning infants
\item glycosylation
\end{itemize}
\end{itemize}
\end{frame}

\begin{frame}[label={sec:orgb12169a}]{Galactose metabolism}
\begin{center}
\includegraphics[width=0.7\textwidth]{./figures/Fig1.png}
\end{center}
\end{frame}

\begin{frame}[label={sec:org54ccc22}]{GALT Protein}
\begin{itemize}
\item ubiquitous enzyme
\end{itemize}

\alert{crystal structure}
\end{frame}

\begin{frame}[label={sec:org41a94d0}]{GALT catalytic mechanism}
\alert{mechanism}
\end{frame}

\begin{frame}[label={sec:orge06c9f8}]{GALT gene}
\begin{itemize}
\item 9p13, 11 exons, \textasciitilde{}4 kb
\item housekeeping
\end{itemize}
\begin{block}{c.563A>G, p.Q188R}
\begin{itemize}
\item \textasciitilde{}64\% of galactosemic alleles in caucasian pop
\item Irish Travellers 1:430
\item Non-functional variant: Destabilise UMP-GALT
\begin{itemize}
\item no residual RBC GALT activity
\end{itemize}
\end{itemize}
\end{block}

\begin{block}{c.855G>T, p.K285N}
\begin{itemize}
\item slavic origin?
\item no residual RBC GALT activity
\end{itemize}
\end{block}
\end{frame}

\begin{frame}[label={sec:org0cfcc08}]{Duarte and Los Angles variants}
\begin{itemize}
\item Duarte - 50\% RBC GALT enzyme activity
\item LA - elevated RBC GALT enzyme activity
\item Same electrophoretic pattern, both p.N314D
\item Linkage disequilibrium
\begin{description}
\item[{D}] 4bp deletion in promoter
\end{description}
\item p.D314 is the ancestral allele
\item p.N314
\end{itemize}
\end{frame}


\begin{frame}[label={sec:org4fa738a}]{Diagnosis}
\end{frame}
\begin{frame}[label={sec:org796ea05}]{Long-term outcome}
\end{frame}

\begin{frame}[label={sec:orgd0b5144}]{Therapy}
\end{frame}
\end{document}

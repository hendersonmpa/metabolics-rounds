% Created 2019-04-25 Thu 10:39
% Intended LaTeX compiler: pdflatex
\documentclass[presentation, smaller]{beamer}
\usepackage[utf8]{inputenc}
\usepackage[T1]{fontenc}
\usepackage{graphicx}
\usepackage{grffile}
\usepackage{longtable}
\usepackage{wrapfig}
\usepackage{rotating}
\usepackage[normalem]{ulem}
\usepackage{amsmath}
\usepackage{textcomp}
\usepackage{amssymb}
\usepackage{capt-of}
\usepackage{hyperref}
\hypersetup{colorlinks,linkcolor=gray,urlcolor=blue}
\usepackage{textpos}
\usepackage{textgreek}
\usepackage[version=4]{mhchem}
\usepackage{chemfig}
\usepackage{siunitx}
\usepackage{gensymb}
\usepackage[usenames,dvipsnames]{xcolor}
\usepackage[T1]{fontenc}
\usepackage{lmodern}
\usepackage{verbatim}
\usepackage{tikz}
\usetikzlibrary{shapes.geometric,arrows,decorations.pathmorphing,backgrounds,positioning,fit,petri}
\usetheme[height=20pt]{Boadilla}
\usecolortheme[RGB={170,160,80}]{{structure}}
\author{Matthew Henderson, PhD, FCACB}
\date{\today}
\title{Summary of "Sweat and sour: an update on classic galactosemia"}
\institute[NSO]{Newborn Screening Ontario}
\titlegraphic{\includegraphics[height=1cm,keepaspectratio]{../logos/NSO_logo.pdf}\includegraphics[height=1cm,keepaspectratio]{../logos/cheo-logo.png} \includegraphics[height=1cm,keepaspectratio]{../logos/UOlogoBW.eps}}
\hypersetup{
 pdfauthor={Matthew Henderson, PhD, FCACB},
 pdftitle={Summary of "Sweat and sour: an update on classic galactosemia"},
 pdfkeywords={},
 pdfsubject={},
 pdfcreator={Emacs 26.1 (Org mode 9.1.9)}, 
 pdflang={English}}
\begin{document}

\maketitle


\logo{\includegraphics[width=1cm,height=1cm,keepaspectratio]{../logos/NSO_logo_small.pdf}}

\vspace{220pt}
\beamertemplatenavigationsymbolsempty
\setbeamertemplate{caption}[numbered]
\setbeamerfont{caption}{size=\tiny}
% \addtobeamertemplate{frametitle}{}{%
% \begin{textblock*}{100mm}(.85\textwidth,-1cm)
% \includegraphics[height=1cm,width=2cm]{cat}
% \end{textblock*}}


\tikzstyle{chemical} = [rectangle, rounded corners, text width=5em, minimum height=1em,text centered, draw=black, fill=none]
\tikzstyle{hardware} = [rectangle, rounded corners, text width=5em, minimum height=1em,text centered, draw=black, fill=gray!30]
\tikzstyle{ms} = [rectangle, rounded corners, text width=5em, minimum height=1em,text centered, draw=orange, fill=none]
\tikzstyle{msw} = [rectangle, rounded corners, text width=7em, minimum height=1em,text centered, draw=orange, fill=none]
\tikzstyle{label} = [rectangle,text width=8em, minimum height=1em, text centered, draw=none, fill=none]
\tikzstyle{hl} = [rectangle, rounded corners, text width=5em, minimum height=1em,text centered, draw=black, fill=red!30]
\tikzstyle{box} = [rectangle, rounded corners, text width=5em, minimum height=5em,text centered, draw=black, fill=none]
\tikzstyle{arrow} = [thick,->,>=stealth]
\tikzstyle{hl-arrow} = [ultra thick,->,>=stealth,draw=red]


\begin{frame}[label={sec:org00613fc}]{Classical galactosemia}
\begin{itemize}
\item Caused by deficient activity of galactose-1-phosphate uridylyltransferase
\item prevalence is 1:16,000-60,000 live births
\item autosomal recessive
\begin{itemize}
\item 336 mutations is the GALT gene described
\end{itemize}

\item presents in the neonatal period
\begin{itemize}
\item potentially lethal
\end{itemize}

\item Therapy is life long dietary galactose restriction
\begin{itemize}
\item does not prevent long-term complications
\end{itemize}
\end{itemize}
\end{frame}

\begin{frame}[label={sec:org6a6f4f5}]{Galactose}
\begin{itemize}
\item Primary source is dietary lactose
\item Functions
\begin{itemize}
\item energy source in pre-weaning infants
\item glycosylation
\item glycolipid synthesis
\end{itemize}
\end{itemize}


\begin{figure}[htbp]
\centering
\includegraphics[width=0.4\textwidth]{./figures/Beta-D-Lactose.png}
\caption[lactose]{\label{fig:org98f0f30}
Lactose is a disaccharide derived from the condensation of galactose and glucose, which form a \(\beta\) 1 \(\to\) 4 glycosidic linkage.}
\end{figure}
\end{frame}


\begin{frame}[label={sec:org1bf2c01}]{Galactose metabolism}
\begin{figure}[htbp]
\centering
\includegraphics[width=0.8\textwidth]{./figures/Fig1.png}
\caption[met]{\label{fig:org300f288}
Galactose metabolism}
\end{figure}
\end{frame}


\begin{frame}[label={sec:orgcdcfee9}]{GALT Protein}
\begin{itemize}
\item ubiquitous enzyme
\end{itemize}

\begin{figure}[htbp]
\centering
\includegraphics[width=0.8\textwidth]{./figures/Fig2.png}
\caption[structure]{\label{fig:org3147b9a}
Crystal structure}
\end{figure}
\end{frame}


\begin{frame}[label={sec:orgf425d4b}]{GALT catalytic mechanism}
\begin{figure}[htbp]
\centering
\includegraphics[width=0.8\textwidth]{./figures/Fig3.png}
\caption[mechanism]{\label{fig:orgaf5d42f}
Catalytic mechanism}
\end{figure}
\end{frame}


\begin{frame}[label={sec:orgb5fab23}]{GALT gene}
\begin{itemize}
\item 9p13, 11 exons, \textasciitilde{}4 kb
\item housekeeping
\end{itemize}
\begin{block}{c.563A>G, p.Q188R}
\begin{itemize}
\item \textasciitilde{}64\% of galactosemic alleles in caucasian pop
\item Irish Travellers 1:430
\item Non-functional variant: Destabilise UMP-GALT
\begin{itemize}
\item no residual RBC GALT activity
\end{itemize}
\end{itemize}
\end{block}

\begin{block}{c.855G>T, p.K285N}
\begin{itemize}
\item slavic origin?
\item no residual RBC GALT activity
\end{itemize}
\end{block}
\end{frame}

\begin{frame}[label={sec:org5c57b43}]{Duarte and Los Angles variants}
\begin{itemize}
\item Duarte (p.N314D) - 50\% RBC GALT enzyme activity
\item LA (p.N314D) - elevated RBC GALT enzyme activity
\item Same electrophoretic pattern
\item D is in linkage disequilibrium w a 4bp promoter deletion
\item p.D314 is the ancestral allele
\item p.N314 arose early in human evolution
\end{itemize}
\end{frame}


\begin{frame}[label={sec:orga6e2999}]{Biochemical Features}
\begin{itemize}
\item \(\uparrow\)  Galatactose
\item \(\uparrow\) Gal-1-p - pathogenic
\item \(\uparrow\)  galactitol - cataracts
\item \(\uparrow\) galactonate
\item \(\downarrow\) UPD-Gal - disordered glycosylation, glycolipids
\item \(\downarrow\) UPD-Glc
\end{itemize}
\end{frame}

\begin{frame}[label={sec:org00a127d}]{Acute Presentation}
\begin{itemize}
\item asymptomatic at birth
\item with feeding:
\begin{itemize}
\item poor weight gain, vomiting, diarrhea
\item hepatocellular damage, lethargy, and hypotonia
\end{itemize}
\item May progress to Gram negative sepsis, cataracts
\end{itemize}
\end{frame}

\begin{frame}[label={sec:orgd600043}]{NBS}
\begin{itemize}
\item not universal
\begin{itemize}
\item must be identified early
\item \textless{} 5 days is ideal
\end{itemize}
\end{itemize}
\end{frame}

\begin{frame}[label={sec:org5ae68b5}]{Spotcheck Method}
\ce{Gal-1-P + UDP-Glu ->[GALT] Glu-1-P + UDP-Gal}

\ce{Glu-1-P ->[PGluM] Glu-6-P}

\ce{Glu-6-P + NADP ->[G6PD] 6-PG + NADPH}

\ce{NADPH + MTT ->[methoxy PMS] Coloured Formazan + NADP}
\end{frame}

\begin{frame}[label={sec:org0269ad1}]{Diagnosis}
\begin{itemize}
\item reducing substances in urine - not specific or sensitive
\item Gal-1-P, galactose, galactitol in blood or urine
\item RBC Gal-1-P - not specific
\item RBC GALT activity
\begin{itemize}
\item Classic galactosemia - undetectable or 1\% of controls
\end{itemize}
\end{itemize}
\end{frame}

\begin{frame}[label={sec:orged125d5}]{Therapy and Outcome}
\begin{block}{Therapy}
\begin{itemize}
\item life long dietary restriction of galactose.
\end{itemize}
\end{block}

\begin{block}{Outcome}
\begin{itemize}
\item Endogenous galactose synthesis may be responsible for:
\begin{itemize}
\item Cognitive impairment
\item Ovarian insufficiency
\end{itemize}
\item Dairy restrictions
\begin{itemize}
\item Bone health
\end{itemize}
\end{itemize}
\end{block}
\end{frame}
\end{document}